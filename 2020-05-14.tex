\documentclass[11pt]{article}
\usepackage[utf8]{inputenc}
\usepackage{amsmath, amsthm, amssymb, amsfonts, mathtools, tikz-cd, float}
\usepackage[left=2.5cm,right=2.5cm]{geometry}
\usepackage[shortlabels]{enumitem}

\newcommand{\Int}{\text{Int}}
\newcommand{\R}{\mathbb{R}}
\newcommand{\Z}{\mathbb{Z}}
\newcommand{\pd}{\partial}
\renewcommand{\epsilon}{\varepsilon}
\renewcommand{\hat}{\widehat}

\newtheorem{theorem}{Theorem}[section]
\newtheorem{corollary}{Corollary}[theorem]
\newtheorem{lemma}[theorem]{Lemma}

\newtheorem{definition}{Definition}

\pagestyle{myheadings}

\begin{document}

\section{More examples of manifolds, Quotients (May 14)}

\subsection{More Examples}
\begin{enumerate}
\item (The circle) Define $S^1 = \{x^2 + y^2 = 1\} \subset \R^2$. We can define four functions on open sets of $\R$, the collection of which form a set of functions of which $S^1$ is locally the graph. Define an open cover $\{V_1, V_2, V_3, V_4\}$ of $S^1$ by
\begin{alignat*}{2}
V_1 &= S^1 \cap ((0, \infty) \times (-1,1)) \qquad &&\text{"open right half"}\\
V_2 &= S^1 \cap ((\infty, 0) \times (-1,1)) \qquad &&\text{"open left half"} \\
V_3 &= S^1 \cap ((-1, 1) \times (0, \infty)) \qquad &&\text{"open top half"}\\
V_4 &= S^1 \cap ((-1, 1) \times (-\infty, 0))\qquad &&\text{"open bottom half"}
\end{alignat*}
Define $f_1, f_2, f_3, f_4 : (-1, 1) \to \R$ by
\begin{alignat*}{2}
f_1 (y) &= \sqrt{1-y^2} \qquad &&\text{so that } \Gamma_{f_1} = V_1 \\
f_2 (y) &= -\sqrt{1-y^2} \qquad &&\text{so that } \Gamma_{f_2} = V_2  \\
f_3 (x) &= \sqrt{1-x^2} \qquad &&\text{so that } \Gamma_{f_3} = V_3  \\
f_4 (x) &= -\sqrt{1-x^2} \qquad &&\text{so that } \Gamma_{f_4} = V_4 
\end{alignat*}
What are the charts? Define $\phi_1 : V_1 \to (-1,1)$ by $\phi_1(x,y) = y$. This is continuous with continuous inverse $\phi_1^{-1}(y) = (\sqrt{1-y^2}, y)$. The other coordinate systems $\phi_2, \phi_3, \phi_4$ are defined similarly. Consider 
\[
\mathcal{A} = \{ (V_1, \phi_1),(V_2, \phi_2),(V_3, \phi_3),(V_4, \phi_4) \}.
\]
We claim that $\mathcal{A}$ is a smooth atlas on $S^1$. For example, one transition map is $\phi_1 \circ \phi_3^{-1} : \phi_3(V_{13}) \to \phi_1(V_{13})$, which is a map from $(0, 1)$ to itself. It is given by
\[
(\phi_1 \circ \phi_3^{-1})(t) = \phi_1(t, \sqrt{1-t^2}) = \sqrt{1-t^2},
\]
which is a diffeomorphism of $(0, 1)$ with itself. As a similar proposition holds for the other transition maps, we conclude that $(S^1, \mathcal{A})$ is a smooth manifold of dimension $1$.

Let $f : \R^2 \to \R$ be $f(x,y) = x^2+y^2$. Then $S^1 = f^{-1}(1)$ (preimage). We get a collection of $1$-dimensional manifolds covering $\R^2 \setminus \{0\}$; we say that $\{f^{-1}(r) : r > 0\}$ is a \textit{one-dimensional foliation} of $\R^2 \setminus \{0\}$. (More on that in a later lecture.)

\item (Level sets) Consider a smooth map $F : \R^{n+1} \to \R$. Let $c \in \R$ be such that $F^{-1}(c) \neq \emptyset$ and $\nabla F(a) \neq 0$ for each $a \in F^{-1}(c)$. 

For example, if $F(x) = \|x\|^2$, then $S^n = F^{-1}(1)$ and $\left. \nabla F \right|_{F^{-1}(c)} \neq 0$. (We say $\{F^{-1}(r) : r > 0$ is an \textit{$n$-dimensional foliation} of $\R^{n+1} \setminus \{0\}$.)

Choose $a \in F^{-1}(c)$. Then $DF(a) \neq 0$, so there is an $i$ such that $\frac{\pd F}{\pd x_i}(a) \neq 0$. Then the equation $F(x_1, \dots, x_i, \dots, x_{n+1}) = c$ can be solved locally for $x_i$ in terms of the other coordinates, i.e. $F^{-1}(c)$ is the graph of a smooth function near $a$.

Making this precise, the implicit function theorem provides us with a neighbourhood $U$ of $(a_1, \dots, \hat{a_i}, \dots, a_{n+1})$ in $\R^n$ and a smooth function $g : U \to \R$ satisfying
\begin{itemize}
\item $g(a_1, \dots, \hat{a_i}, \dots, a_{n+1}) = a_i$,
\item $F(x_1, \dots, g(x_1, \dots, \hat{x_i}, \dots, x_{n+1}), \dots, x_{n+1}) = c$ for all $(x_1, \dots, \hat{x_i}, \dots, x_{n+1}) \in U$,
\end{itemize}
i.e.
\[
\Gamma_g = \{(x_1, \dots, g(x_1, \dots, \hat{x_i}, \dots, x_{n+1}), \dots, x_{n+1}) \in \R^{n+1} : (x_1, \dots, \hat{x_i}, \dots, x_{n+1}) \in U\} = V \cap F^{-1}(c)
\]
for some neighbourhood $V$ of $a$ in $\R^{n+1}$.

So we conclude that if $\nabla F(a) \neq 0$ for all $a \in F^{-1}(c) \neq \emptyset$, then $F^{-1}(c)$ is locally the graph of a function. What are the charts? $(V \cap F^{-1}(c), \phi)$, where $\phi : V \cap F^{-1}(c) \to U$ is given by $\phi(x_1, \dots, x_{n+1}) = (x_1, \dots, \hat{x_i}, \dots, x_{n+1})$ with the inverse $\phi^{-1}(x_1, \dots, \hat{x_i}, \dots, x_{n+1}) = (x_1, \dots, g(x_1, \dots, \hat{x_i}, \dots, x_{n+1}), \dots, x_{n+1})$. This is clearly a chart.

Now consider the collection of such charts $\mathcal{A} = \{ (V_a \cap F^{-1}(c), \phi_a) \}$. Consider a transition mapping $\phi_a \circ \phi_b^{-1} : \phi_b(V_{ab}) \to \phi_a(V_{ab})$. This is
\begin{align*}
(\phi_a \circ \phi_b^{-1})(x_1, \dots, x_i, \dots, \hat{x_j}, \dots, x_{n+1}) &= \phi_a(x_1, \dots, x_i, \dots, g_b(x_1, \dots, \hat{x_j}, \dots, x_{n+1}), \dots, x_{n+1}) \\
&= (x_1, \dots, \hat{x_i}, \dots, g_b(x_1, \dots, \hat{x_j}, \dots, x_{n+1}), \dots, x_{n+1})
\end{align*}
which is $C^\infty$, and similarly for its inverse. So $\mathcal{A}$ is a $C^\infty$ atlas on $F^{-1}(c)$, making $F^{-1}(c)$ a smooth manifold of dimension $n$.

\item (Products) Consider two smooth manifolds $M$ and $N$ of dimensions $m$ and $n$, respectively. Equip them with smooth atlases $\mathcal{A}_M$ and $\mathcal{A}_N$, respectively. Define
\[
\mathcal{A}_{M \times N} = \{ (U \times V, \phi \times \psi) : (U, \phi) \in \mathcal{A}_M \text{ and } (V, \psi) \in \mathcal{A}_N \}.
\]
$\mathcal{A}_{M \times N}$ is a smooth atlas on $M \times N$, making $M \times N$ a smooth manifold of dimension $m + n$. To see this, note that the sets $U \times V$ certainly cover $M \times N$, and that the products of homeomorphisms are homeomorphisms. If $(U_1 \times V_1, \phi_1 \times \psi_1), (U_2 \times V_2, \phi_2 \times \psi_2) \in \mathcal{A}_{M \times N}$, then the transition map
\[ 
(\phi_1 \times \psi_1) \circ (\phi_2 \times \psi_2)^{-1} : (\phi_2 \times \psi_2)((U_1 \times V_1) \cap (U_2 \times V_2)) \to (\phi_1 \times \psi_1)((U_1 \times V_1) \cap (U_2 \times V_2))
\]
is, by set theory, equal to
\[
(\phi_1 \circ \phi_2^{-1}) \times (\psi_1 \times \psi_2^{-1}) : \phi_2(U_{12}) \times \psi_2({V_{12}}) \to \phi_1(U_{12}) \times \psi_1(V_{12}),
\]
which is clearly a diffeomorphism.

For example, the cylinder $S^1 \times \R$ is a smooth manifold of dimension $2$, and the torus $S^1 \times S^1$ is a smooth manifold of dimension $2$. We also have the higher tori $T^n = S^1 \times \dots \times S^n$, a smooth manifold of dimension $n$. 

(Algebraic topology remark: $T^n \not\cong S^n$, as the former has first fundamental group $\Z^n$, whereas the latter is simply connected for $n \geq 2$.)
\end{enumerate}

\subsection{Gluing Manifolds}
Due to the informal visual nature of this part of the lecture, the examples can only be described in words.

\begin{enumerate}
\item Glue the endpoints of $[0, 1]$ to get the circle. They aren't homeomorphic however, since removing an interior point from $[0, 1]$ disconnects it, whereas the circle will remain connected if a point is removed.

\item
Glue the two vertical sides of $[0,1]^2$ to get a cylinder. (Note: in order to visualize this, we need to go up one dimension.)

\item Glue the two vertical sides of $[0, 1]^2$, but with points identified "by reflecting through the centre $(1/2,1/2)$". This produces a Mobius strip.

\item Glue the opposite sides of $[0,1]^2$ together as in example 2, but with each opposite side glued. This produces a torus.

\item Glue the opposite vertical sides of $[0,1]^2$ together as in example 2, and the opposite horizontal sides together as in example 3. This produces a "Klein bottle", an example of a manifold which cannot be embedded in $\R^3$.
\end{enumerate}

\subsection{The Quotient Topology}

Let $S$ be a topological space and $\sim$ an equivalence relation on $S$. Let $\pi : S \to S /_\sim$ be the projection map $\pi(x) = [x]$. Topologize $S /_\sim$ by declaring $U \subset S /_\sim$ to be open if and only if $\pi^{-1}(U)$ is open in $S$. This topology on $S /_\sim$ is called the \emph{quotient topology} - it is the finest topology on $S /_\sim$ with respect to which $\pi$ is continuous, as is easily seen.

Now consider a function $f : S \to Y$, where $Y$ is a set. Suppose $f$ is constant on the fibres of $\pi$ (i.e. $f$ is constant on every equivalence class of $\sim$). Then $f$ induces a map $\tilde{f} : S /_\sim \to Y$ for which the following diagram is commutative:
\[
\begin{tikzcd}[column sep = large, row sep = large]
S \arrow[rd, "f"] \arrow[d, "\pi"] \\
S /_\sim \arrow[r, dashed, "\tilde{f}"] & Y
\end{tikzcd} 
\]
The function $\tilde{f}$ is defined in the obvious way: $\tilde{f}([x]) = f(x)$. The new function $\tilde{f}$ is well-defined since we assumed $f$ was constant on equivalence classes. We say that $f$ \emph{descends to the quotient}. If $Y$ is a topological space, we have a very useful lemma.
\begin{lemma}
Suppose $f : S \to Y$ is a function of topological spaces, and that $\sim$ is an equivalence relation on $S$ on whose equivalence classes $f$ is constant. Then the induced map $\tilde{f} : S /_\sim \to Y$ is continuous if and only if $f$ is continuous.
\end{lemma}
\begin{proof}
If $\tilde{f}$ is continuous, then $f = \tilde{f} \circ \pi$ is continuous as a composition of continuous maps. If $f$ is continuous, then given $U$ open in $Y$, $f^{-1}(U)$ is open in $S$. But $f^{-1}(U) = \pi^{-1}(\tilde{f}^{-1}(U))$, so by the definition of the quotient topology, $\tilde{f}^{-1}(U)$ is open in $S /_\sim$, proving continuity of $\tilde{f}$.
\end{proof}

Let's discuss the example of gluing the endpoints of the interval. Define $\sim$ on $I = [0,1]$ by $x \sim x$ for $x \in (0, 1)$ and $x \sim y$ for $x,y \in \{0,1\}$. We claim that $I /_\sim \cong S^1$. An explicit homeomorphism can be found by descending to the quotient.

Define $f : I \to S^1$ by $f(t) = (\cos 2\pi t, \sin 2\pi t)$. Then $f(0) = f(1) = (1,0)$, so $f$ is constant on the equivalence classes of $\sim$. Then $f$ descends to a continuous map $\tilde{f} : I /_\sim \to S^1$, given by
\[
\tilde{f}([t]) = \begin{cases} 
(\cos 2\pi t, \sin 2 \pi t), & [t] \neq [0] \\
(1,0), & t = [0] = [1]
\end{cases}
\]
which is bijective. Since $I /_\sim = \pi(I)$ is compact and $S^1$ is Hausdorff, the map $\tilde{f}$ is a homeomorphism of topological spaces. So indeed, $I /_\sim \cong S^1$.

In order to tackle the question of "when is a quotient a manifold", we need to derive some conditions for when the quotient of a space is Hausdorff or second countable. Here's a simple necessary condition.
\begin{lemma}
If $S /_\sim$ is Hausdorff, then equivalence classes are closed in $S$.
\end{lemma}
\begin{proof}
Each $\{[x]\} = \{\pi(x)\}$ is closed in $S /_\sim$ by Hausdorffness, so by continuity $\pi^{-1}(\{\pi(x)\}) = [x]$ is closed in $S$.
\end{proof}
For a simple application of this necessary condition, consider $\R / (0, \infty)$ - the quotient space obtained by identifying all points of $(0, \infty)$. The lemma dictates that $\R / (0, \infty)$ is not Hausdorff because the equivalence class $(0, \infty)$ is not closed in $\R$.

\subsection{Open Equivalence Relations}

\begin{definition}
An equivalence relation $\sim$ on a space $S$ is said to be open if the projection $\pi : S \to S /_\sim$ is an open mapping. Equivalently, $\sim$ is open if and only if 
\[
\pi^{-1}(\pi(U)) = \bigcup_{x \in U}[x]
\]
is open in $S$, for each $U$ open in $S$.
\end{definition}

This definition is worth making, as the projections need not be open in general. Consider $\R / \{-1, 1\}$. The interval $(-2, 0)$ is open, but
\[
\pi^{-1}(\pi((-2, 0))) = \bigcup_{-2 < x < 0}[x] = (-2, 0) \cup \{1\}
\]
is not open in $\R$. Therefore $\sim$ identifying $-1$ and $1$ on $\R$ is not an open equivalence relation. (Note that $\R / \{-1,1\}$ is not a topological manifold, as it is homeomorphic to the symbol $\propto$ with the ends extending infinitely.)

\begin{definition}
The graph of an equivalence relation $\sim$ on $S$ is the set $R = \{(x, y) \in S \times S : x \sim y\}$.
\end{definition}
\begin{theorem}
Suppose $\sim$ is an open equivalence relation on $S$. Then $S /_\sim$ is Hausdorff if and only if the graph $R$ of $\sim$ is closed in $S \times S$.
\end{theorem}
\begin{proof}
Was left as an exercise in class, so here's a solution. We have a sequence of equivalent statements
\begin{align*}
R \text{ is closed} &\iff S \times S \setminus R \text{ is open} \\
&\iff \text{for all } (x, y) \in S \times S \setminus R \text{ there are open sets } U, V \text{ such that } (x,y) \in U \times V \subset S \times S \setminus R \\
&\iff \text{for all } x \not\sim y \text{ in } S \text{ there are open sets } U \ni x, V \ni Y \text{ such that } (U \times V) \cap R = \emptyset \\
&\iff \text{for all } [x] \neq [y] \text{ in } S /_\sim \text{there are open sets } U \ni x, V \ni y \text{ such that } \pi(U) \cap \pi(V) = \emptyset
\end{align*}
This last statement is equivalent to $S /_\sim$ being Hausdorff, which we now prove. If this statement is true, then $\pi(U)$ and $\pi(V)$ are disjoint open (because $\sim$ is open) sets of $S /_\sim$ separating $[x]$ and $[y]$, which shows that $S /_\sim$ is Hausdorff. Conversely, suppose $S /_\sim$ is Hausdorff. Given $[x] \neq [y]$ in $S /_\sim$, we can find disjoint open sets $U \ni [x]$, $V \ni [y]$ of $S /_\sim$. By surjectivity, $U = \pi(\pi^{-1}(U))$ and $V = \pi(\pi^{-1}(V))$, so $\pi^{-1}(U)$ and $\pi^{-1}(V)$ are open sets of $S$ containing $x$ and $y$, respectively, satisfying the condition of the last statement. So the last statement is equivalent to $S /_\sim$ being Hausdorff.
\end{proof}
With it is a corollary - a classic exercise in point-set topology.
\begin{corollary}
$S$ is Hausdorff if and only if $\Delta = \{(x, x) \in S \times S : x \in S\}$ is closed.
\end{corollary}
\begin{proof}
Let $\sim$ be the equivalence relation identifying every point only with itself. Then $\sim$ is an open equivalence relation and $R = \Delta$. The spaces $S$ and $S /_\sim$ are homeomorphic, so the statement follows from the theorem immediately.
\end{proof}

What about second countability?
\begin{theorem}
If $\sim$ is an open equivalence relation on $S$ and $\{B_n\}$ is a countable basis of $S$, then $\{\pi(B_n)\}$ is a countable basis of $S /_\sim$.
\end{theorem}
\begin{proof}
Was left as an exercise in class, so here's a solution. Note that the collection $\{\pi(B_n))\}$ is a collection of open sets because $\pi$ is an open mapping. Let $U \subset S /_\sim$ be open and consider $[x] \in U$. Then $x \in \pi^{-1}(U)$, so we can find a $B_n$ with $x \in B_n \subset \pi^{-1}(U)$. Then $[x] = \pi(x) \subset \pi(B_n) \subset \pi(\pi^{-1}(U)) = U$, proving that $\{B_n\}$ is a basis of $S /_\sim$.
\end{proof}
To summarize,
\begin{itemize}
\item quotient spaces of Hausdorff spaces under open equivalence relations are Hausdorff if and only if the graph of the relation is closed
\item quotient spaces of second-countable spaces under open equivalence relations are second-countable, and bases for the quotient are obtained in the obvious way.
\end{itemize}

\subsection{Real Projective Space}

Define $\sim$ on $\R^{n+1} \setminus \{0\}$ by $x \sim \lambda x$ for $\lambda \neq 0$. The quotient space $(\R^{n+1} \setminus \{0\}) /_\sim$ is denoted $\R P^n$ and is called \emph{real projective space}. It may be thought of as the set of lines passing through the origin.

Each element of $\R P^n$ can be thought of as a pair of antipodal points on $S^n$, which motivates the following
\begin{theorem}
Define $\sim$ on $S^n$ by identifying antipodal points, i.e. $x \sim \pm x$. Define $f : \R^{n+1} \setminus \{0\} \to S^n$ by $f(x) = \frac{x}{\|x\|}$. Then $f$ induces a homeomorphism $\R P^n \xrightarrow{\sim} S^n /_\sim$.
\end{theorem}
The proof will be essentially the proof given in class, but much more complete and explicit about how maps induce other maps.
\begin{proof}
Consider the following diagram:
\[
\begin{tikzcd}[column sep = large, row sep = large]
\R^{n+1} \setminus \{0\} \arrow{r}{f} \arrow{d}{\pi_1} & S^n \arrow{d}{\pi_2} \\
\R P^n & S^n /_\sim
\end{tikzcd}
\]
where $\pi_1$ and $\pi_2$ are the projections to each quotient space as shown in the diagram. The map $\pi_2 \circ f : \R^{n+1} \setminus \{0\} \to S^n /_\sim$ is given by
\[
(\pi_2 \circ f)(x) = \pi_2\left( \frac{x}{\|x\|} \right) = \left\{ -\frac{x}{\|x\|}, \frac{x}{\|x\|} \right\} = [x]_2,
\]
which is continuous and constant on the fibres of $\pi_1$; the lines through the origin. It thus induces a continuous map $\tilde{f} : \R P^n \to S^n /_\sim$ for which the following diagram is commutative:
\[
\begin{tikzcd}[column sep = large, row sep = large]
\R^{n+1} \setminus \{0\} \arrow{r}{f} \arrow{d}{\pi_1} \arrow{rd}{\pi_2 \circ f} & S^n \arrow{d}{\pi_2} \\
\R P^n \arrow[dashed]{r}{\tilde{f}} & S^n /_\sim
\end{tikzcd}
\]
We define a continuous inverse of $\tilde{f}$. Consider $g : S^n \to \R^{n+1} \setminus \{0\}$ given by $g(x) = x$. As before, consider the diagram
\[
\begin{tikzcd}[column sep = large, row sep = large]
\R^{n+1} \setminus \{0\} \arrow{d}{\pi_1} & S^n \arrow{d}{\pi_2} \arrow{l}{g} \\
\R P^n & S^n /_\sim
\end{tikzcd}
\]
The map $\pi_1 \circ g : S^n \to \R P^n$ is given by
\[
(\pi_1 \circ g)(x) = \pi_1(x) = \{ \lambda x : \lambda \neq 0 \} = [x]_1,
\]
which is continuous and constant on the fibres of $\pi_2$; antipodal points on the $n$-sphere. It thus induces a continuous map $\tilde{g} : S^n /_\sim \to \R P^n$ for which the following diagram is commutative:
\[
\begin{tikzcd}[column sep = large, row sep = large]
\R^{n+1} \setminus \{0\} \arrow{d}{\pi_1} & S^n \arrow{d}{\pi_2} \arrow{l}{g} \arrow{ld}{\pi_1 \circ g}\\
\R P^n & S^n /_\sim \arrow[dashed]{l}{\tilde{g}}
\end{tikzcd}
\]
We claim that $\tilde{f}$ and $\tilde{g}$ are inverses to each other, which will show that $\tilde{f}$ is a homeomorphism $\R P^n \xrightarrow{\sim} S^n /_\sim$. We have
\begin{align*}
(\tilde{g} \circ \tilde{f})([x]_1) &= (\tilde{g} \circ \tilde{f} \circ \pi_1)(x) = (\tilde{g} \circ \pi_2 \circ f)(x) = (\pi_1 \circ g \circ f)(x) = \pi_1\left(g\left( \frac{x}{\|x\|} \right)\right) = \pi_1\left(\frac{x}{\|x\|}\right) = [x]_1 \\
(\tilde{f} \circ \tilde{g})([x]_2) &= (\tilde{f} \circ \tilde{g} \circ \pi_2)(x) = (\tilde{f} \circ \pi_1 \circ g)(x) = (\pi_2 \circ f \circ g)(x) = \pi_2(f(x)) = \pi_2 \left( \frac{x}{\|x\|} \right) = [x]_2 
\end{align*}
So $\tilde{f}$ is a homeomorphism $\R P^n \xrightarrow{\sim} S^n /_\sim$.
\end{proof}
In particular, $\R P^n$ is compact! Note that we could have just explicitly defined
\begin{alignat*}{2}
&\tilde{f} : \R P^n \to S^n /_\sim \qquad &&\tilde{f}([x]_1) := \pi_2(f(x)) = \left[ \frac{x}{\|x\|} \right]_2 \\
&\tilde{g} : S^n /_\sim \to \R P^n \qquad &&\tilde{g}([x]_2) := \pi_1(g(x)) = [x]_1
\end{alignat*}
checked for well-definedness and continuity, and then we'd have been done. That's how the proof on page 362 of Tu goes. However, the abuse of tikz diagrams makes it very clear where the homeomorphism and its inverse come from, and that they're continuous (which is basically what Tu is doing anyway).

% may need work
\subsection{Visualizing $\R P^2$}
In order to visualize $\R P^2$ we will consider some homeomorphisms. Define
\begin{align*}
H^2 &= \{(x,y,z) \in \R^3 : x^2 + y^2 + z^2 = 1, z \geq 0\} \\
D^2 &= \{(x,y) \in \R^2 : x^2 + y^2 \leq 1\}.
\end{align*}
Consider the maps
\begin{alignat*}{2}
&\phi : H^2 \to D^2 \qquad &&\phi(x,y,z) = (x,y) \\
&\psi : D^2 \to H^2 \qquad &&\psi(x,y) = (x,y,\sqrt{1-x^2-y^2})
\end{alignat*}
which are continuous inverses of each other. Define equivalence relations on $H^2$ and $D^2$ as follows:
\begin{itemize}
\item On $H^2$: identify antipodal points on the equator, call the projection $\pi_3$
\item On $D^2$: identify antipodal points on the boundary, call the projection $\pi_4$
\end{itemize}
Considering diagrams similar to those in the previous proof, the map $\pi_4 \circ \phi$ induces a continuous map $\tilde{\phi} : H^2 /_\sim \to D^2 /_\sim$ with $\tilde{\phi} \circ \pi_3 = \pi_4 \circ \phi$, and the map $\pi_3 \circ \psi$ induces a continuous map $\tilde{\psi} : D^2 /_\sim \to H^2 /_\sim$ with $\tilde{\psi} \circ \pi_4 = \pi_3 \circ \psi$. The maps $\tilde{\phi}$ and $\tilde{\psi}$ are continuous inverses of each other (which can be seen using just these given compositions), which shows that we have a homeomorphism $H^2 /_\sim \xrightarrow{\sim} D^2 /_\sim$. 

If we accept on faith that there is a homeomorphism $S^2 /_\sim \xrightarrow{\sim} H^2 /_\sim$, then we have a sequence of homeomorphisms
\[
\R P^2 \xrightarrow{\sim} S^2 /_\sim \xrightarrow{\sim} H^2 /_\sim \xrightarrow{\sim} D^2 /_\sim.
\]
Therefore we can visualize the real projective plane $\R P^2$ as a disk with the antipodal boundary points identified. Such a homeomorphism $S^2 /_\sim \xrightarrow{\sim} H^2 /_\sim$ can be shown by a proof similar to the previous quotient space homeomorphisms that we did, by considering the inclusion map $i : H^2 \to S^2$ and its obvious inverse, and working through steps similar to the proofs of the previous homeomorphisms.
 
\end{document}
 