\documentclass[11pt]{article}
\usepackage[utf8]{inputenc}
\usepackage{amsmath, amsthm, amssymb, amsfonts, mathtools, tikz-cd, float}
\usepackage[left=2.5cm,right=2.5cm]{geometry}
\usepackage[shortlabels]{enumitem}

\newcommand{\Int}{\mathrm{Int}}
\newcommand{\R}{\mathbb{R}}
\newcommand{\Z}{\mathbb{Z}}
\newcommand{\pd}{\partial}
\renewcommand{\epsilon}{\varepsilon}
\renewcommand{\hat}{\widehat}
\renewcommand{\tilde}{\widetilde}

\newtheorem{theorem}{Theorem}[section]
\newtheorem{corollary}{Corollary}[theorem]
\newtheorem{lemma}[theorem]{Lemma}
\newtheorem{proposition}{Proposition}[section]

\newtheorem{definition}{Definition}

\pagestyle{myheadings}

\begin{document}

\section{Tangent Spaces and the Differential (May 26)}

\subsection{Derivatives and the Chain Rule on Manifolds}

Having defined the smooth functions on a manifold, in order to proceed with generalizing calculus to manifolds, we must now differentiate them. The notion of the derivative comes from the \emph{differential} or \emph{pushforward} of a smooth map $F : \R^n \to \R^m$ of standard calculus; it is the derivative $DF(p) : T_p \R^n \to T_{F(p)} \R^m$. As we are now working with derivations as our tangent vectors, the definition must be adjusted accordingly.

\begin{definition}
Let $F : N \to M$ be a $C^\infty$ map of smooth manifolds. We define the differential of $F$ at $p$ by the linear map
\[
F_* : T_p N \to T_{F(p)}M, \qquad F_*(X_p)(f) = X_p(f \circ F),
\]
where $X_p \in T_p N$ and $f \in C_{F(p)}^\infty(M)$. The map $F_*$ is sometimes denoted $F_{*, p}$, and is not to be confused with the pullback operation $F^* : C(M) \to C(N)$ on continuous functions. 
\end{definition}
Let's make sure this makes sense; i.e. that $F_*(X_p)$ is actually a derivation at $F(p)$. Linearity follows immediately from linearity of $X_p$ on $C_p^\infty(N)$. If $f, g \in C_{F(p)}^\infty(M)$, then
\begin{alignat*}{2}
F_*(X_p)(fg) &= X_p((fg) \circ F)  &&\text{by definition}\\
&= X_p((f \circ F)(g \circ F)) \\
&= (f \circ F)(p) X_p(g \circ F) + X_p(f \circ F) (g \circ F)(p) \qquad &&\text{$X_p$ a derivation at $p$} \\
&= f(F(p)) F_*(X_p)(g) + F_*(X_p)(f) g(F(p)).
\end{alignat*}
Therefore $F_*$ is indeed a map $T_p N \to T_{F(p)}M$. That it is also linear is obvious.

We need to make sure this properly generalizes the derivative of a $C^\infty$ map $F : \R^n \to \R^m$. Thinking of a tangent vector $v \in T_p \R^n$ as its directional derivative $D_v : C_p^\infty(\R^n) \to \R$, we have, for $f \in C_{F(p)}^\infty(\R^m)$,
\[
F_*(D_v)(f) = D_v(f \circ F) = \nabla f(F(p))( DF(p) \cdot v ) = D_{DF(p)v}(f),
\]
where the latter directional derivative is at $F(p)$. Therefore $F_*(D_v) = D_{DF(p)v}$. If we again identify derivations of germs at $F(p)$ with tangent vectors in $T_{F(p)} \R^m$, we conclude that the differential between manifolds generalizes the derivative from calculus. 

The differential is the same as the derivative, definition wise. Does it hold the same properties. The answer is "yes".
\begin{theorem}
(Chain Rule) Let $F : N \to M$ and $G : M \to P$ be $C^\infty$ maps of smooth manifolds. Then $(G \circ F)_{*, p} = G_{*, F(p)} \circ F_{*, p}$.
\end{theorem}
\begin{proof}
If $X_p \in T_p N$ and $f \in C_{G(F(p))}^\infty P$, then
\[
(G_{*, F(p)} \circ F_{*, p})(X_p)(f) = G_{*, F(p)}(F_{*,p}(X_p))(f) = F_{*,p}(X_p)(f \circ G) = X_p(f \circ G \circ F) = (G \circ F)_{*, p}(X_p)(f)
\]
\end{proof}
If the "base point" is understood, then we will often omit it and simply write $F_*$ and $G_*$, in which case the chain rule reads as $(G \circ F)_* =G^* \circ F_*$.

\subsection{Dimension of Tangent Spaces}

We present some very useful corollaries of the chain rule.
\begin{corollary}
The differential of the identity map $\mathrm{Id} : M \to M$ is the identity map $\mathrm{Id}_* : T_p M \to T_p M$.
\end{corollary}
\begin{proof}
If $X_p \in T_p M$ and $f \in C_p^\infty(M)$ then
\[
\mathrm{Id}_*(X_p)(f) = X_p(f \circ \mathrm{Id}) = X_p(f),
\] 
so $\mathrm{Id}_*(X_p) = X_p$.
\end{proof}
\begin{corollary}
If $F : N \to M$ is a diffeomorphism of smooth manifolds and $p \in N$, then $F_* : T_p N \to T_{F(p)} M$ is an isomorphism of vector spaces.
\end{corollary}
\begin{proof}
By the previous corollary and the chain rule we have
\begin{align*}
F_* \circ (F^{-1})_* = (F \circ F^{-1})_*  &= \mathrm{Id}_{T_{F(p)} M}, \\
(F^{-1})_* \circ F_* = (F^{-1} \circ F)_* &= \mathrm{Id}_{T_p N},
\end{align*}
so $F_* : T_p N \to T_{F(p)}M$ is a bijective linear map.
\end{proof}
\begin{corollary}
(Invariance of dimension) Let $U \subseteq \R^n$ and $V \subseteq \R^m$ be diffeomorphic open sets. Then $n =m$.
\end{corollary}
\begin{proof}
If $F : U \to V$ is a diffeomorphism then by the previous corollary it induces an isomorphism $F_* : T_p U \to T_{F(p)} V$ of vector spaces. Therefore
\[
n = \dim(T_p \R^n) = \dim(T_p U) = \dim(T_{F(p)} V) = \dim(T_{F(p)} \R^m) = m
\]
\end{proof}
The above theorem holds in the case where the sets are merely homeomorphic, but that requires algebraic topology to prove and is decidedly non-trivial.

\begin{proposition}
If $M$ is a smooth manifold of dimension $m$, then for each $p \in M$, the tangent space $T_p M$ has dimension $m$.
\end{proposition}
\begin{proof}
Choose a coordinate chart $(U, \phi)$ around $p$. Then we have a diffeomorphism $\phi : U \to \phi(U)$, so $\phi_* : T_p U \to T_{\phi(p)} \phi(U)$ is an isomorphism. Then
\[
\dim(T_p M) = \dim(T_{\phi(p)} \phi(U)) = m.
\]
\end{proof}

\subsection{A Basis for the Tangent Space}

Knowing the dimension of the tangent space brings us to the following question: what is a basis of the tangent space? We have a main result.
\begin{theorem}
Let $M$ be a smooth manifold and let $p \in M$. Choose a coordinate chart $(U, \phi) = (U, x^1, \dots, x^m)$ around $p$. Then
\[
\left\{ \left. \frac{\pd }{\pd x^1} \right|_p, \dots, \left. \frac{\pd }{\pd x^m} \right|_p \right\}
\]
is a basis of $T_p M$.
\end{theorem}
\begin{proof}
We have, for $f \in C_{\phi(p)}^\infty (\R^n)$,
\[
\phi_* \left( \left. \frac{\pd }{\pd x^i} \right|_p \right)(f) = \left. \frac{\pd }{\pd x^i} \right|_p f \circ \phi = \left. \frac{\pd (f \circ \phi \circ \phi^{-1})}{\pd r^i} \right|_{\phi(p)} = \left( \left. \frac{\pd }{\pd r^i} \right|_{\phi(p)} \right)(f).
\]
Since $\phi_*$ is an isomorphism and isomorphisms send bases to bases, the fact that 
\[
\left\{ \left. \frac{\pd }{\pd r^1} \right|_{\phi(p)}, \dots, \left. \frac{\pd }{\pd r^m} \right|_{\phi(p)} \right\}
\]
is a basis of $T_{\phi(p)} \phi(U)$ implies that the proposed basis of $T_p M$ is indeed a basis.
\end{proof}
We will sometimes write $\frac{\pd}{\pd x^i}$ instead of $\left. \frac{\pd }{\pd x^i} \right|_{p}$ if the base point of the tangent vector is understood.

Of course, the basis of the tangent space depends on the choice of coordinate chart. What are the changes of coordinates? 
\begin{proposition}
Suppose $(U, x^1, \dots, x^m)$ and $(V, y^1, \dots, y^m)$ are two coordinate charts on a manifold $M$. Then on $U \cap V$,
\[
\frac{\pd}{\pd x^j} = \sum_i \frac{\pd y^i}{\pd x^j} \frac{\pd}{\pd y^i}.
\]
(One can remember this by thinking of the $\pd y^i$'s as cancelling.)
\end{proposition}
\begin{proof}
Since $\{ \pd / \pd x^i|_p \}$ and $\{ \pd / \pd y^i|_p \}$ are both bases of the tangent space $T_p M$, for each $p \in U \cap V$, there is an $m \times m$ matrix $[a^i_j]$ (depending on $p$) such that
\[
\frac{\pd}{\pd x^j} = \sum_k a^k_j \frac{\pd}{\pd y^k}
\]
on $U \cap V$. Evaluating both sides at $y^i$ gives
\begin{align*}
\frac{\pd y^i}{\pd x^j} =  \sum_k a^k_j \frac{\pd y^i}{\pd y^k} = \sum_k a^k_j \delta^i_k = a^i_j.
\end{align*}
\end{proof}
 
\end{document}
  