\documentclass[11pt]{article}
\usepackage[utf8]{inputenc}
\usepackage{amsmath, amsthm, amssymb, amsfonts, mathtools, tikz-cd, float}
\usepackage[left=2.5cm,right=2.5cm]{geometry}
\usepackage[shortlabels]{enumitem}

\newcommand{\Int}{\mathrm{Int}}
\newcommand{\R}{\mathbb{R}}
\newcommand{\Z}{\mathbb{Z}}
\newcommand{\pd}{\partial}
\renewcommand{\epsilon}{\varepsilon}
\renewcommand{\hat}{\widehat}
\renewcommand{\tilde}{\widetilde}

\newtheorem{theorem}{Theorem}[section]
\newtheorem{corollary}{Corollary}[theorem]
\newtheorem{lemma}[theorem]{Lemma}
\newtheorem{proposition}{Proposition}

\newtheorem{definition}{Definition}

\pagestyle{myheadings}
\begin{document}

\section{Smooth Structures, Examples (May 12)}

\subsection{More on Maximal Atlases}

Consider the two atlases $\mathcal{A}_1 = \{(\R^n, Id)\}$ and $\mathcal{A}_2 = \{(B_1(x), Id) : x \in \R^n\}$ on $\R^n$. These two atlases determine the same maximal atlas, or the same smooth structure. Why? We have three equivalent reasons
\begin{itemize}
\item for any $(U, \phi) \in \mathcal{A}_1$ and $(V, \psi) \in \mathcal{A}_2$, the charts $(U, \phi)$ and $(V, \psi)$ are $C^\infty$ compatible.
\item $\mathcal{A}_1 \cup \mathcal{A}_2$ is a $C^\infty$ atlas.
\item $\mathcal{A}_1$ and $\mathcal{A}_2$ belong to the same maximal atlas.
\end{itemize}
Define a relation $\sim$ on the atlases by $\mathcal{A}_1 \sim \mathcal{A}_2$ if and only if $\mathcal{A}_1 \cup \mathcal{A}_2$ is another $C^\infty$ atlas. Symmetry and reflexivity are immediate. For transitivity, suppose $\mathcal{A}_1 \cup \mathcal{A}_2$ and $\mathcal{A}_2 \cup \mathcal{A}_3$ are $C^\infty$ atlases. Choose $(U_1, \phi_1) \in \mathcal{A}_1$ and $(U_3, \phi_3) \in \mathcal{A}_3$. We obtain a diffeomorphism
\[
\phi_1 \circ \phi_3^{-1} = \phi_1 \circ \phi_2^{-1} \circ \phi_2 \circ \phi_3^{1} 
\]
defined on $\phi_3(U_{13} \cap U_2)$. Since $\{U_2 : (U_2, \phi_2) \in \mathcal{A}_2$ covers $M$, the map $\phi_1 \circ \phi_3^{-1}$ is smooth at every point of $\phi_3(U_{13})$. Therefore $\sim$ is an equivalence relation.

Now given an atlas $\mathcal{A}$ on $M$, we can talk about the equivalence class $[\mathcal{A}]$. Define 
\[
\mathcal{M} = \bigcup_{\mathcal{A}^\prime \in [\mathcal{A}]} \mathcal{A}^\prime.
\]
Then $\mathcal{M}$ is a new atlas on $M$; it is the unique maximal atlas containing $\mathcal{A}$. (Exercise.)

So we can make the
\begin{definition}
A smooth $n$-manifold $M$ is a topological $n$-manifold with a maximal atlas. The choice of maximal atlas is called a smooth structure on $M$.
\end{definition}

Considering the previous remarks, we arrive at a sufficient condition for a space to be a smooth manifold: If $M$ is a topological space for which
\begin{enumerate}
\item $M$ is Hausdorff, second-countable, and
\item $M$ admits a $C^\infty$ atlas $\mathcal{A}$
\end{enumerate}
then $M$ is a smooth manifold with smooth structure $\mathcal{M} = \bigcup_{\mathcal{A}^\prime \in [\mathcal{A}]} \mathcal{A}^\prime$.

\subsection{Examples}

\begin{enumerate}
\item (Open subsets) Let $M$ be a smooth $n$-manifold with a smooth atlas $\mathcal{A} = \{ (U_\alpha, \phi_\alpha) \}$. Let $A \subseteq M$ be an open set. Then $\mathcal{A}_A = \{ (U_\alpha \cap A, \phi_\alpha |_{U_\alpha \cap A}) \}$ is a smooth atlas on $A$, so $A$ is a smooth $n$-manifold.

\item (Finite dimensional vector spaces) Let $V$ be a finite dimensional real vector space. Choose a basis $\beta = \{v_1, \dots, v_n\}$ of $V$, and consider the isomorphism $\Phi : V \to \R^n$ given by $\Phi(v_i) = e_i$. 

Define a norm on $V$ by $\|\sum a_i v_i\| := \|\sum a_i e_i \|$, where the norm on the left is the standard Euclidean norm. With this norm we may define an open ball in $V$ as $B_r(v_0) = \{ v \in V : \| v - v_0 \| < r\}$. This gives a topology on $V$. Since all norms on finite dimensional vector spaces are equivalent, this topology does not depend on our choice of basis.

Then $\Phi$ is an isometry (it does not change distances), so it takes balls to balls and so does its inverse. That is, $\Phi$ is a homeomorphism, so we have a $C^\infty$ atlas $\{(V, \Phi)\}$ on $V$, making $V$ a smooth $n$-manifold.

This atlas determines a maximal atlas on $V$. Does this maximal atlas depend on the choice of basis? No. Choose another basis $\beta^\prime$ of $V$ and define $\Phi^\prime : V \to \R^n$ similarly. Then we'll get another $C^\infty$ atlas $\{(V, \Phi^\prime)\}$ on $V$. The charts $(U, \Phi)$ and $(V, \Phi^\prime)$ are $C^\infty$-compatible, for the transition map $\Phi^\prime \circ \Phi^{-1}$ is a linear isomorphism of $\R^n$ with itself (certainly $C^\infty$). 

\textbf{Remark:} We also could have talked about complex vector spaces, since $\mathbb{C} \cong \R^2$.

\item (Matrices, general linear group) $\mathrm{Mat}_{m \times n}(\R) \cong \R^{mn}$, so $\mathrm{Mat}_{m \times n}(\R)$ is a smooth manifold of dimension $mn$. 

The general linear group is $GL(n, \R) = \{A \in \mathrm{Mat}_{m \times n}(\R) : \det(A) \neq 0\}$. By continuity of $\det$ it is an open subset of $\mathrm{Mat}_{m \times n}(\R)$, so by the first example we know it's a smooth $n^2$-dimensional manifold.
\end{enumerate}

\end{document}
