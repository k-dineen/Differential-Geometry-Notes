\documentclass[11pt]{article}
\usepackage[utf8]{inputenc}
\usepackage{amsmath, amsthm, amssymb, amsfonts, mathtools, tikz-cd, float,cancel}
\usepackage[hidelinks]{hyperref}
\usepackage[left=2.5cm,right=2.5cm]{geometry}
\usepackage[shortlabels]{enumitem}

\hypersetup{linktoc=all}

\newcommand{\Int}{\mathrm{Int}}
\newcommand{\R}{\mathbb{R}}
\newcommand{\C}{\mathbb{C}}
\newcommand{\Z}{\mathbb{Z}}
\newcommand{\pd}{\partial}
\newcommand{\supp}{\mathrm{supp}}
\newcommand{\sgn}{\mathrm{sgn}}
\renewcommand{\epsilon}{\varepsilon}
\renewcommand{\tilde}{\widetilde}
\renewcommand{\hat}{\widehat}

% HOW TO READ THESE
% (definiton/theory/corollary/lemma) a.b.c is the cth respective object of section a, subsection b

\newtheorem{definition}{Definition}[subsection]
\newtheorem{theorem}{Theorem}[subsection]
\newtheorem{corollary}{Corollary}[subsection]
\newtheorem{lemma}{Lemma}[subsection]
\newtheorem{proposition}{Proposition}[subsection]

\pagestyle{myheadings}
\title{MAT367 Course Notes}
\author{Kain Dineen}

\begin{document}
\maketitle

The following are course notes for the course MAT367 (Differential Geometry) offered in the Summer of 2020, taught by Ahmed Ellithy. The course notes are based off of handwritten notes created during lectures. They may contain errors or other false statements. The dates for each lecture are included, and any additional sections are supplementary material. (Exercises to very important/relevant problems, etc.)

\tableofcontents

\newpage
\section{Introduction (May 5)}

\subsection{Trying to Define Things}

The straightforward approach is
\begin{definition}
A set $S \subseteq \R^3$ is a surface if there is an open set $U \subseteq \R^2$ and a smooth function $f : U \to \R$ for which $S = \Gamma_f$ is the graph of $f$.
\end{definition}
This isn't a great definition though. Its problem is that it's way too specific. The sphere $S^2 = \{x^2 + y^2 + z^2 = 1\}$ fails to be a surface under this definition, as it's not the graph of a function. We can remedy this by thinking about the following question:
\[
\textit{If we were standing on a surface, what should our surroundings look like?}
\]
Here's another attempt at defining a surface, albeit in an imprecise way.
\begin{definition}
A set $S \subseteq \R^3$ is a surface if for every $p \in S$ there is a neighbourhood of $p$ in $S$ that "looks like a piece of the plane".
\end{definition}
In more precise (but still not formal) wording, we are "locally diffeomorphic to pieces of $\R^2$". It turns out that this condition is equivalent to $S$ being locally a graph; that follows from the implicit function theorem.

We'd like to generalize the above notions to define a $k$-dimensional "surface" in $\R^n$. Following in the footsteps of the previous definition, we obtain a new
\begin{definition}
A set $S \subseteq \R^n$ is a $k$-dimensional manifold if it "locally looks like $\R^n$".
\end{definition}
Equivalently, if for each $p \in S$ there is an open $U \subseteq \R^n$ containing $p$ such that $S \cap U$ is the graph of a smooth function from an open subset of $\R^k$ to $\R^{n-k}$.

The key idea with the last two definitions is that they are \textit{local} - they are concerned with describing "pieces" of the surface or manifold, as opposed to the first definition being "global" by describing the entire surface.

\subsection{Leaving $\R^n$ for the Intrinsic View}
Almost all of the geometry that is done on manifolds depends only on the manifold itself, and not on the space in which the manifold lies. (An example of Riemannian geometry is curvature.) Moreover, there are many sets we'd like to call manifolds whose points do not lie in Euclidean space. An example is \textit{real projective space} $\R P^n$, which is defined as the quotient $(\R^{n+1} \setminus \{0\})/(x \sim \lambda x)$, where $\lambda \neq 0$. The real projective space contains equivalence classes of points of Euclidean space, so it is not a subset of Euclidean space. Therefore we'd like to define manifolds so that $\R P^n$ is an $n$-dimensional manifold. 

Concisely, we would like to study manifolds \textit{intrinsically}: we would like to drop all of the unnecessary data around our manifold and consider only the key properties of what a manifold should be.

\newpage
\section{Defining Manifolds (May 7)}

\subsection{Submanifolds of $\R^n$}

We'll formally write out the definition of a $k$-manifold $M$ in $\R^n$ now.
\begin{definition}
A subset $M \subseteq \R^n$ is a $k$-dimensional manifold if for every $p \in M$ there is an open neighbourhood $U$ of $p$ in $\R^n$, an open $V \subseteq \R^k$, and a function $f : V \to U \cap M$ such that
\begin{enumerate}
\item $f$ is a homeomorphism,
\item $f$ is smooth,
\item $Df(x)$ has rank $k$ at every $x \in V$.
\end{enumerate}
\end{definition}
The first two conditions are natural. Why the third? We'd like the \textit{tangent space to $M$ at $p$} to be a $k$-dimensional subspace of $\R^n$. If $Df(x)$ has rank $k$, then $Df(x)(\R^k)$ is a $k$-dimensional subspace of $\R^n$, which is what we would like $T_p M$ to be (roughly).

We have an equivalent definition, stated here as a theorem:
\begin{theorem}
$M \subseteq \R^n$ is a $k$-manifold if and only if for each $p \in M$ there is an open neighbourhood $U$ of $p$ in $\R^n$, an open $V \subseteq \R^k$, and a smooth $f : V \to \R^{n-k}$ such that $U \cap M = \Gamma_f$ (up to a permutation of the coordinates in $U$).
\end{theorem}
That last condition is a little odd, but what it means is that we can consider graphs of the form $(x, f(x))$ and $(f(y), y)$. This is essential in ensuring that, say, $S^1 = \{x^2 + y^2 = 1\}$ is a manifold. The definition may be shown to be equivalent to the old one using the implicit function theorem.

\subsection{Topological Manifolds}

By way of the subspace topology, every manifold in $\R^n$ is Hausdorff and second countable. It turns out that these are the conditions we would like our abstract manifolds to have in order to exclude some pathological cases.

\begin{definition}
A topological space $M$ is locally Euclidean of dimension $m$ if for each $p \in U$ there is an open neighbourhood $U$ of $p$ in $M$ and a map $\phi : U \to \R^m$ which is a homeomorphism onto its image. The pair $(U, \phi)$ is called a coordinate chart, $U$ is called a coordinate neighbourhood, and $\phi$ is called a coordinate system. 
\end{definition}

\begin{definition}
$M$ is a topological manifold of dimension $m$ if it is Hausdorff, second countable, and locally Euclidean of dimension $m$.
\end{definition}
Is the dimension of a topological manifold well-defined? That is, if $(U, \phi)$ and $(V, \psi)$ are two coordinate charts with $U \cap V \neq \emptyset$ and $\phi(U) \subseteq \R^n$, $\psi(V) \subseteq \R^m$, is $n = m$? Consider the \textit{transition mapping}
\[
\psi \circ \phi^{-1} : \phi(U \cap V) \to \psi(U \cap V).
\]
This is a homeomorphism from an open subset of $\R^n$ to an open subset of $\R^m$. If $n \neq m$, this contradicts a non-trivial theorem called \textit{Invariance of Domain}. We will not prove it here.

If we drop the Hausdorff condition, then the "line with two origins" becomes a topological manifold. If we drop the countable basis condition, then the "long line" becomes a topological manifold. These are both topological spaces that intuitively should not be manifolds - the extra conditions excludes them from being so.

\subsection{Defining a Smooth Manifold}

How should we define a smooth function on a manifold, say, $f : M \to \R$? The reasonable thing to do is to say that $f$ is smooth if $f \circ \phi^{-1}$ is smooth, for some coordinate system $\phi$. Then we run into a problem - this isn't independent of the choice of coordinate system, so long as $M$ is only a topological manifold. We will define a \textit{smooth structure} on $M$ which allows us to make this natural definition.

\begin{definition}
Two coordinate charts $(U, \phi)$ and $(V, \psi)$ are said to be smoothly compatible (or $C^\infty$-compatible) if the transition mappings are diffeomorphisms, i.e.
\begin{align*}
&\psi \circ \phi^{-1} : \phi(U \cap V) \to \psi(U \cap V) \\
&\phi \circ \psi^{-1} : \psi(U \cap V) \to \phi(U \cap V)
\end{align*}
are $C^\infty$ maps of open subsets of Euclidean space.
\end{definition}

Smooth compatability is clearly a reflexive and symmetric relation. Is it transitive? Unfortunately, the answer is no. Suppose $(U_1, \phi_1)$ is smoothly compatible with $(U_2, \phi_2)$ and similarly for $(U_2, \phi_2)$ with $(U_3, \phi_3)$. The natural thing to do is write
\[
\phi_3 \circ \phi_1^{-1} = (\phi_3 \circ \phi_2^{-1}) \circ (\phi_2 \circ \phi_1^{-1}).
\]
But this only makes sense on $\phi_1(U_1 \cap U_2 \cap U_3)$, which may be empty!

\begin{definition}
A smooth atlas (or $C^\infty$ atlas) on $M$ is a collection of pairwise smoothly compatible coordinate charts covering $M$.
\end{definition}

We can now properly define a smooth function on a manifold. For unsaid technical reasons, however, it's beneficial to consider a little more structure. (Unfortunately the rest of the lecture went a little fast, as we ran out of time.)

\begin{definition}
A smooth maximal atlas is a smooth atlas not contained in any other smooth atlas.
\end{definition}

\begin{definition}
A smooth manifold of dimension $n$ is a Hausdorff, second countable topological manifold of dimension $n$ equipped with a smooth maximal atlas $\mathcal{A}$. The smooth maximal atlas $\mathcal{A}$ is called a smooth structure on $M$.
\end{definition}

\begin{lemma}
Any smooth atlas for $M$ is contained in a unique maximal smooth atlas.
\end{lemma}
The proof for this lemma proceeds roughly as follows: first one proves that if a two coordinate charts are smoothly compatible with a given atlas (meaning they are compatible with every chart in the atlas), then they are themselves compatible. Then one picks a smooth atlas and adjoins (by union) all of the charts with which the smooth atlas is compatible. It is then shown that this larger atlas is the desired unique maximal atlas.

Because of this lemma, we have a simple "test" for a smooth manifold.
\begin{corollary}
A topological space $M$ is a smooth manifold if and only if
\begin{enumerate}
\item It is Hausdorff and second countable,
\item It admits a smooth atlas.
\end{enumerate}
\end{corollary}

\newpage
\section{Smooth Structures, Examples (May 12)}

\subsection{More on Maximal Atlases}

Consider the two atlases $\mathcal{A}_1 = \{(\R^n, Id)\}$ and $\mathcal{A}_2 = \{(B_1(x), Id) : x \in \R^n\}$ on $\R^n$. These two atlases determine the same maximal atlas, or the same smooth structure. Why? We have three equivalent reasons
\begin{itemize}
\item for any $(U, \phi) \in \mathcal{A}_1$ and $(V, \psi) \in \mathcal{A}_2$, the charts $(U, \phi)$ and $(V, \psi)$ are $C^\infty$ compatible.
\item $\mathcal{A}_1 \cup \mathcal{A}_2$ is a $C^\infty$ atlas.
\item $\mathcal{A}_1$ and $\mathcal{A}_2$ belong to the same maximal atlas.
\end{itemize}
Define a relation $\sim$ on the atlases by $\mathcal{A}_1 \sim \mathcal{A}_2$ if and only if $\mathcal{A}_1 \cup \mathcal{A}_2$ is another $C^\infty$ atlas. Symmetry and reflexivity are immediate. For transitivity, suppose $\mathcal{A}_1 \cup \mathcal{A}_2$ and $\mathcal{A}_2 \cup \mathcal{A}_3$ are $C^\infty$ atlases. Choose $(U_1, \phi_1) \in \mathcal{A}_1$ and $(U_3, \phi_3) \in \mathcal{A}_3$. We obtain a diffeomorphism
\[
\phi_1 \circ \phi_3^{-1} = \phi_1 \circ \phi_2^{-1} \circ \phi_2 \circ \phi_3^{1} 
\]
defined on $\phi_3(U_{13} \cap U_2)$. Since $\{U_2 : (U_2, \phi_2) \in \mathcal{A}_2$ covers $M$, the map $\phi_1 \circ \phi_3^{-1}$ is smooth at every point of $\phi_3(U_{13})$. Therefore $\sim$ is an equivalence relation.

Now given an atlas $\mathcal{A}$ on $M$, we can talk about the equivalence class $[\mathcal{A}]$. Define 
\[
\mathcal{M} = \bigcup_{\mathcal{A}^\prime \in [\mathcal{A}]} \mathcal{A}^\prime.
\]
Then $\mathcal{M}$ is a new atlas on $M$; it is the unique maximal atlas containing $\mathcal{A}$. (Exercise.)

So we can make the
\begin{definition}
A smooth $n$-manifold $M$ is a topological $n$-manifold with a maximal atlas. The choice of maximal atlas is called a smooth structure on $M$.
\end{definition}

Considering the previous remarks, we arrive at a sufficient condition for a space to be a smooth manifold: If $M$ is a topological space for which
\begin{enumerate}
\item $M$ is Hausdorff, second-countable, and
\item $M$ admits a $C^\infty$ atlas $\mathcal{A}$
\end{enumerate}
then $M$ is a smooth manifold with smooth structure $\mathcal{M} = \bigcup_{\mathcal{A}^\prime \in [\mathcal{A}]} \mathcal{A}^\prime$.

\subsection{Examples}

\begin{enumerate}
\item (Open subsets) Let $M$ be a smooth $n$-manifold with a smooth atlas $\mathcal{A} = \{ (U_\alpha, \phi_\alpha) \}$. Let $A \subseteq M$ be an open set. Then $\mathcal{A}_A = \{ (U_\alpha \cap A, \phi_\alpha |_{U_\alpha \cap A}) \}$ is a smooth atlas on $A$, so $A$ is a smooth $n$-manifold.

\item (Finite dimensional vector spaces) Let $V$ be a finite dimensional real vector space. Choose a basis $\beta = \{v_1, \dots, v_n\}$ of $V$, and consider the isomorphism $\Phi : V \to \R^n$ given by $\Phi(v_i) = e_i$. 

Define a norm on $V$ by $\|\sum a_i v_i\| := \|\sum a_i e_i \|$, where the norm on the left is the standard Euclidean norm. With this norm we may define an open ball in $V$ as $B_r(v_0) = \{ v \in V : \| v - v_0 \| < r\}$. This gives a topology on $V$. Since all norms on finite dimensional vector spaces are equivalent, this topology does not depend on our choice of basis.

Then $\Phi$ is an isometry (it does not change distances), so it takes balls to balls and so does its inverse. That is, $\Phi$ is a homeomorphism, so we have a $C^\infty$ atlas $\{(V, \Phi)\}$ on $V$, making $V$ a smooth $n$-manifold.

This atlas determines a maximal atlas on $V$. Does this maximal atlas depend on the choice of basis? No. Choose another basis $\beta^\prime$ of $V$ and define $\Phi^\prime : V \to \R^n$ similarly. Then we'll get another $C^\infty$ atlas $\{(V, \Phi^\prime)\}$ on $V$. The charts $(U, \Phi)$ and $(V, \Phi^\prime)$ are $C^\infty$-compatible, for the transition map $\Phi^\prime \circ \Phi^{-1}$ is a linear isomorphism of $\R^n$ with itself (certainly $C^\infty$). 

\textbf{Remark:} We also could have talked about complex vector spaces, since $\mathbb{C} \cong \R^2$.

\item (Matrices, general linear group) $\mathrm{Mat}_{m \times n}(\R) \cong \R^{mn}$, so $\mathrm{Mat}_{m \times n}(\R)$ is a smooth manifold of dimension $mn$. 

The general linear group is $GL(n, \R) = \{A \in \mathrm{Mat}_{m \times n}(\R) : \det(A) \neq 0\}$. By continuity of $\det$ it is an open subset of $\mathrm{Mat}_{m \times n}(\R)$, so by the first example we know it's a smooth $n^2$-dimensional manifold.
\end{enumerate}

\newpage

\section{More Examples, Quotients (May 14)}

\subsection{More Examples}
\begin{enumerate}
\item (The circle) Define $S^1 = \{x^2 + y^2 = 1\} \subseteq \R^2$. We can define four functions on open sets of $\R$, the collection of which form a set of functions of which $S^1$ is locally the graph. Define an open cover $\{V_1, V_2, V_3, V_4\}$ of $S^1$ by
\begin{alignat*}{2}
V_1 &= S^1 \cap ((0, \infty) \times (-1,1)) \qquad &&\text{"open right half"}\\
V_2 &= S^1 \cap ((\infty, 0) \times (-1,1)) \qquad &&\text{"open left half"} \\
V_3 &= S^1 \cap ((-1, 1) \times (0, \infty)) \qquad &&\text{"open top half"}\\
V_4 &= S^1 \cap ((-1, 1) \times (-\infty, 0))\qquad &&\text{"open bottom half"}
\end{alignat*}
Define $f_1, f_2, f_3, f_4 : (-1, 1) \to \R$ by
\begin{alignat*}{2}
f_1 (y) &= \sqrt{1-y^2} \qquad &&\text{so that } \Gamma_{f_1} = V_1 \\
f_2 (y) &= -\sqrt{1-y^2} \qquad &&\text{so that } \Gamma_{f_2} = V_2  \\
f_3 (x) &= \sqrt{1-x^2} \qquad &&\text{so that } \Gamma_{f_3} = V_3  \\
f_4 (x) &= -\sqrt{1-x^2} \qquad &&\text{so that } \Gamma_{f_4} = V_4 
\end{alignat*}
What are the charts? Define $\phi_1 : V_1 \to (-1,1)$ by $\phi_1(x,y) = y$. This is continuous with continuous inverse $\phi_1^{-1}(y) = (\sqrt{1-y^2}, y)$. The other coordinate systems $\phi_2, \phi_3, \phi_4$ are defined similarly. Consider 
\[
\mathcal{A} = \{ (V_1, \phi_1),(V_2, \phi_2),(V_3, \phi_3),(V_4, \phi_4) \}.
\]
We claim that $\mathcal{A}$ is a smooth atlas on $S^1$. For example, one transition map is $\phi_1 \circ \phi_3^{-1} : \phi_3(V_{13}) \to \phi_1(V_{13})$, which is a map from $(0, 1)$ to itself. It is given by
\[
(\phi_1 \circ \phi_3^{-1})(t) = \phi_1(t, \sqrt{1-t^2}) = \sqrt{1-t^2},
\]
which is a diffeomorphism of $(0, 1)$ with itself. As a similar proposition holds for the other transition maps, we conclude that $(S^1, \mathcal{A})$ is a smooth manifold of dimension $1$.

Let $f : \R^2 \to \R$ be $f(x,y) = x^2+y^2$. Then $S^1 = f^{-1}(1)$ (preimage). We get a collection of $1$-dimensional manifolds covering $\R^2 \setminus \{0\}$; we say that $\{f^{-1}(r) : r > 0\}$ is a \textit{one-dimensional foliation} of $\R^2 \setminus \{0\}$. (More on that in a later lecture.)

\item (Level sets) Consider a smooth map $F : \R^{n+1} \to \R$. Let $c \in \R$ be such that $F^{-1}(c) \neq \emptyset$ and $\nabla F(a) \neq 0$ for each $a \in F^{-1}(c)$. 

For example, if $F(x) = \|x\|^2$, then $S^n = F^{-1}(1)$ and $\left. \nabla F \right|_{F^{-1}(c)} \neq 0$. (We say $\{F^{-1}(r) : r > 0$ is an \textit{$n$-dimensional foliation} of $\R^{n+1} \setminus \{0\}$.)

Choose $a \in F^{-1}(c)$. Then $DF(a) \neq 0$, so there is an $i$ such that $\frac{\pd F}{\pd x_i}(a) \neq 0$. Then the equation $F(x_1, \dots, x_i, \dots, x_{n+1}) = c$ can be solved locally for $x_i$ in terms of the other coordinates, i.e. $F^{-1}(c)$ is the graph of a smooth function near $a$.

Making this precise, the implicit function theorem provides us with a neighbourhood $U$ of $(a_1, \dots, \hat{a_i}, \dots, a_{n+1})$ in $\R^n$ and a smooth function $g : U \to \R$ satisfying
\begin{itemize}
\item $g(a_1, \dots, \hat{a_i}, \dots, a_{n+1}) = a_i$,
\item $F(x_1, \dots, g(x_1, \dots, \hat{x_i}, \dots, x_{n+1}), \dots, x_{n+1}) = c$ for all $(x_1, \dots, \hat{x_i}, \dots, x_{n+1}) \in U$,
\end{itemize}
i.e.
\[
\Gamma_g = \{(x_1, \dots, g(x_1, \dots, \hat{x_i}, \dots, x_{n+1}), \dots, x_{n+1}) \in \R^{n+1} : (x_1, \dots, \hat{x_i}, \dots, x_{n+1}) \in U\} = V \cap F^{-1}(c)
\]
for some neighbourhood $V$ of $a$ in $\R^{n+1}$.

So we conclude that if $\nabla F(a) \neq 0$ for all $a \in F^{-1}(c) \neq \emptyset$, then $F^{-1}(c)$ is locally the graph of a function. What are the charts? $(V \cap F^{-1}(c), \phi)$, where $\phi : V \cap F^{-1}(c) \to U$ is given by $\phi(x_1, \dots, x_{n+1}) = (x_1, \dots, \hat{x_i}, \dots, x_{n+1})$ with the inverse $\phi^{-1}(x_1, \dots, \hat{x_i}, \dots, x_{n+1}) = (x_1, \dots, g(x_1, \dots, \hat{x_i}, \dots, x_{n+1}), \dots, x_{n+1})$. This is clearly a chart.

Now consider the collection of such charts $\mathcal{A} = \{ (V_a \cap F^{-1}(c), \phi_a) \}$. Consider a transition mapping $\phi_a \circ \phi_b^{-1} : \phi_b(V_{ab}) \to \phi_a(V_{ab})$. This is
\begin{align*}
(\phi_a \circ \phi_b^{-1})(x_1, \dots, x_i, \dots, \hat{x_j}, \dots, x_{n+1}) &= \phi_a(x_1, \dots, x_i, \dots, g_b(x_1, \dots, \hat{x_j}, \dots, x_{n+1}), \dots, x_{n+1}) \\
&= (x_1, \dots, \hat{x_i}, \dots, g_b(x_1, \dots, \hat{x_j}, \dots, x_{n+1}), \dots, x_{n+1})
\end{align*}
which is $C^\infty$, and similarly for its inverse. So $\mathcal{A}$ is a $C^\infty$ atlas on $F^{-1}(c)$, making $F^{-1}(c)$ a smooth manifold of dimension $n$.

\item (Products) Consider two smooth manifolds $M$ and $N$ of dimensions $m$ and $n$, respectively. Equip them with smooth atlases $\mathcal{A}_M$ and $\mathcal{A}_N$, respectively. Define
\[
\mathcal{A}_{M \times N} = \{ (U \times V, \phi \times \psi) : (U, \phi) \in \mathcal{A}_M \text{ and } (V, \psi) \in \mathcal{A}_N \}.
\]
$\mathcal{A}_{M \times N}$ is a smooth atlas on $M \times N$, making $M \times N$ a smooth manifold of dimension $m + n$. To see this, note that the sets $U \times V$ certainly cover $M \times N$, and that the products of homeomorphisms are homeomorphisms. If $(U_1 \times V_1, \phi_1 \times \psi_1), (U_2 \times V_2, \phi_2 \times \psi_2) \in \mathcal{A}_{M \times N}$, then the transition map
\[ 
(\phi_1 \times \psi_1) \circ (\phi_2 \times \psi_2)^{-1} : (\phi_2 \times \psi_2)((U_1 \times V_1) \cap (U_2 \times V_2)) \to (\phi_1 \times \psi_1)((U_1 \times V_1) \cap (U_2 \times V_2))
\]
is, by set theory, equal to
\[
(\phi_1 \circ \phi_2^{-1}) \times (\psi_1 \times \psi_2^{-1}) : \phi_2(U_{12}) \times \psi_2({V_{12}}) \to \phi_1(U_{12}) \times \psi_1(V_{12}),
\]
which is clearly a diffeomorphism.

For example, the cylinder $S^1 \times \R$ is a smooth manifold of dimension $2$, and the torus $S^1 \times S^1$ is a smooth manifold of dimension $2$. We also have the higher tori $T^n = S^1 \times \dots \times S^n$, a smooth manifold of dimension $n$. 

(Algebraic topology remark: $T^n \not\cong S^n$, as the former has first fundamental group $\Z^n$, whereas the latter is simply connected for $n \geq 2$.)
\end{enumerate}

\subsection{Gluing Manifolds}
Due to the informal visual nature of this part of the lecture, the examples can only be described in words.

\begin{enumerate}
\item Glue the endpoints of $[0, 1]$ to get the circle. They aren't homeomorphic however, since removing an interior point from $[0, 1]$ disconnects it, whereas the circle will remain connected if a point is removed.

\item
Glue the two vertical sides of $[0,1]^2$ to get a cylinder. (Note: in order to visualize this, we need to go up one dimension.)

\item Glue the two vertical sides of $[0, 1]^2$, but with points identified "by reflecting through the centre $(1/2,1/2)$". This produces a Mobius strip.

\item Glue the opposite sides of $[0,1]^2$ together as in example 2, but with each opposite side glued. This produces a torus.

\item Glue the opposite vertical sides of $[0,1]^2$ together as in example 2, and the opposite horizontal sides together as in example 3. This produces a "Klein bottle", an example of a manifold which cannot be embedded in $\R^3$.
\end{enumerate}

\subsection{The Quotient Topology}

Let $S$ be a topological space and $\sim$ an equivalence relation on $S$. Let $\pi : S \to S /_\sim$ be the projection map $\pi(x) = [x]$. Topologize $S /_\sim$ by declaring $U \subseteq S /_\sim$ to be open if and only if $\pi^{-1}(U)$ is open in $S$. This topology on $S /_\sim$ is called the \emph{quotient topology} - it is the finest topology on $S /_\sim$ with respect to which $\pi$ is continuous, as is easily seen.

Now consider a function $f : S \to Y$, where $Y$ is a set. Suppose $f$ is constant on the fibres of $\pi$ (i.e. $f$ is constant on every equivalence class of $\sim$). Then $f$ induces a map $\tilde{f} : S /_\sim \to Y$ for which the following diagram is commutative:
\[
\begin{tikzcd}[column sep = large, row sep = large]
S \arrow[rd, "f"] \arrow[d, "\pi"] \\
S /_\sim \arrow[r, dashed, "\tilde{f}"] & Y
\end{tikzcd} 
\]
The function $\tilde{f}$ is defined in the obvious way: $\tilde{f}([x]) = f(x)$. The new function $\tilde{f}$ is well-defined since we assumed $f$ was constant on equivalence classes. We say that $f$ \emph{descends to the quotient}. If $Y$ is a topological space, we have a very useful lemma.
\begin{lemma}
Suppose $f : S \to Y$ is a function of topological spaces, and that $\sim$ is an equivalence relation on $S$ on whose equivalence classes $f$ is constant. Then the induced map $\tilde{f} : S /_\sim \to Y$ is continuous if and only if $f$ is continuous.
\end{lemma}
\begin{proof}
If $\tilde{f}$ is continuous, then $f = \tilde{f} \circ \pi$ is continuous as a composition of continuous maps. If $f$ is continuous, then given $U$ open in $Y$, $f^{-1}(U)$ is open in $S$. But $f^{-1}(U) = \pi^{-1}(\tilde{f}^{-1}(U))$, so by the definition of the quotient topology, $\tilde{f}^{-1}(U)$ is open in $S /_\sim$, proving continuity of $\tilde{f}$.
\end{proof}

Let's discuss the example of gluing the endpoints of the interval. Define $\sim$ on $I = [0,1]$ by $x \sim x$ for $x \in (0, 1)$ and $x \sim y$ for $x,y \in \{0,1\}$. We claim that $I /_\sim \cong S^1$. An explicit homeomorphism can be found by descending to the quotient.

Define $f : I \to S^1$ by $f(t) = (\cos 2\pi t, \sin 2\pi t)$. Then $f(0) = f(1) = (1,0)$, so $f$ is constant on the equivalence classes of $\sim$. Then $f$ descends to a continuous map $\tilde{f} : I /_\sim \to S^1$, given by
\[
\tilde{f}([t]) = \begin{cases} 
(\cos 2\pi t, \sin 2 \pi t), & [t] \neq [0] \\
(1,0), & t = [0] = [1]
\end{cases}
\]
which is bijective. Since $I /_\sim = \pi(I)$ is compact and $S^1$ is Hausdorff, the map $\tilde{f}$ is a homeomorphism of topological spaces. So indeed, $I /_\sim \cong S^1$.

In order to tackle the question of "when is a quotient a manifold", we need to derive some conditions for when the quotient of a space is Hausdorff or second countable. Here's a simple necessary condition.
\begin{lemma}
If $S /_\sim$ is Hausdorff, then equivalence classes are closed in $S$.
\end{lemma}
\begin{proof}
Each $\{[x]\} = \{\pi(x)\}$ is closed in $S /_\sim$ by Hausdorffness, so by continuity $\pi^{-1}(\{\pi(x)\}) = [x]$ is closed in $S$.
\end{proof}
For a simple application of this necessary condition, consider $\R / (0, \infty)$ - the quotient space obtained by identifying all points of $(0, \infty)$. The lemma dictates that $\R / (0, \infty)$ is not Hausdorff because the equivalence class $(0, \infty)$ is not closed in $\R$.

\subsection{Open Equivalence Relations}

\begin{definition}
An equivalence relation $\sim$ on a space $S$ is said to be open if the projection $\pi : S \to S /_\sim$ is an open mapping. Equivalently, $\sim$ is open if and only if 
\[
\pi^{-1}(\pi(U)) = \bigcup_{x \in U}[x]
\]
is open in $S$, for each $U$ open in $S$.
\end{definition}

This definition is worth making, as the projections need not be open in general. Consider $\R / \{-1, 1\}$. The interval $(-2, 0)$ is open, but
\[
\pi^{-1}(\pi((-2, 0))) = \bigcup_{-2 < x < 0}[x] = (-2, 0) \cup \{1\}
\]
is not open in $\R$. Therefore $\sim$ identifying $-1$ and $1$ on $\R$ is not an open equivalence relation. (Note that $\R / \{-1,1\}$ is not a topological manifold, as it is homeomorphic to the symbol $\propto$ with the ends extending infinitely.)

\begin{definition}
The graph of an equivalence relation $\sim$ on $S$ is the set $R = \{(x, y) \in S \times S : x \sim y\}$.
\end{definition}
\begin{theorem}
Suppose $\sim$ is an open equivalence relation on $S$. Then $S /_\sim$ is Hausdorff if and only if the graph $R$ of $\sim$ is closed in $S \times S$.
\end{theorem}
\begin{proof}
Was left as an exercise in class, so here's a solution. We have a sequence of equivalent statements
\begin{align*}
R \text{ is closed} &\iff S \times S \setminus R \text{ is open} \\
&\iff \text{for all } (x, y) \in S \times S \setminus R \text{ there are open sets } U, V \text{ such that } (x,y) \in U \times V \subseteq S \times S \setminus R \\
&\iff \text{for all } x \not\sim y \text{ in } S \text{ there are open sets } U \ni x, V \ni Y \text{ such that } (U \times V) \cap R = \emptyset \\
&\iff \text{for all } [x] \neq [y] \text{ in } S /_\sim \text{there are open sets } U \ni x, V \ni y \text{ such that } \pi(U) \cap \pi(V) = \emptyset
\end{align*}
This last statement is equivalent to $S /_\sim$ being Hausdorff, which we now prove. If this statement is true, then $\pi(U)$ and $\pi(V)$ are disjoint open (because $\sim$ is open) sets of $S /_\sim$ separating $[x]$ and $[y]$, which shows that $S /_\sim$ is Hausdorff. Conversely, suppose $S /_\sim$ is Hausdorff. Given $[x] \neq [y]$ in $S /_\sim$, we can find disjoint open sets $U \ni [x]$, $V \ni [y]$ of $S /_\sim$. By surjectivity, $U = \pi(\pi^{-1}(U))$ and $V = \pi(\pi^{-1}(V))$, so $\pi^{-1}(U)$ and $\pi^{-1}(V)$ are open sets of $S$ containing $x$ and $y$, respectively, satisfying the condition of the last statement. So the last statement is equivalent to $S /_\sim$ being Hausdorff.
\end{proof}
With it is a corollary - a classic exercise in point-set topology.
\begin{corollary}
$S$ is Hausdorff if and only if $\Delta = \{(x, x) \in S \times S : x \in S\}$ is closed.
\end{corollary}
\begin{proof}
Let $\sim$ be the equivalence relation identifying every point only with itself. Then $\sim$ is an open equivalence relation and $R = \Delta$. The spaces $S$ and $S /_\sim$ are homeomorphic, so the statement follows from the theorem immediately.
\end{proof}
It turns out that the above theorem and its corollary are equivalent. It's not too hard to see that the corollary implies the theorem by using the fact that $\pi$ is continuous and open.

What about second countability?
\begin{theorem}
If $\sim$ is an open equivalence relation on $S$ and $\{B_n\}$ is a countable basis of $S$, then $\{\pi(B_n)\}$ is a countable basis of $S /_\sim$.
\end{theorem}
\begin{proof}
Was left as an exercise in class, so here's a solution. Note that the collection $\{\pi(B_n))\}$ is a collection of open sets because $\pi$ is an open mapping. Let $U \subseteq S /_\sim$ be open and consider $[x] \in U$. Then $x \in \pi^{-1}(U)$, so we can find a $B_n$ with $x \in B_n \subseteq \pi^{-1}(U)$. Then $[x] = \pi(x) \subseteq \pi(B_n) \subseteq \pi(\pi^{-1}(U)) = U$, proving that $\{B_n\}$ is a basis of $S /_\sim$.
\end{proof}
To summarize,
\begin{itemize}
\item quotient spaces of Hausdorff spaces under open equivalence relations are Hausdorff if and only if the graph of the relation is closed
\item quotient spaces of second-countable spaces under open equivalence relations are second-countable, and bases for the quotient are obtained in the obvious way.
\end{itemize}

\subsection{Real Projective Space}

Define $\sim$ on $\R^{n+1} \setminus \{0\}$ by $x \sim \lambda x$ for $\lambda \neq 0$. The quotient space $(\R^{n+1} \setminus \{0\}) /_\sim$ is denoted $\R P^n$ and is called \emph{real projective space}. It may be thought of as the set of lines passing through the origin.

Each element of $\R P^n$ can be thought of as a pair of antipodal points on $S^n$, which motivates the following
\begin{theorem}
Define $\sim$ on $S^n$ by identifying antipodal points, i.e. $x \sim \pm x$. Define $f : \R^{n+1} \setminus \{0\} \to S^n$ by $f(x) = \frac{x}{\|x\|}$. Then $f$ induces a homeomorphism $\R P^n \xrightarrow{\sim} S^n /_\sim$.
\end{theorem}
The proof will be essentially the proof given in class, but much more complete and explicit about how maps induce other maps.
\begin{proof}
Consider the following diagram:
\[
\begin{tikzcd}[column sep = large, row sep = large]
\R^{n+1} \setminus \{0\} \arrow{r}{f} \arrow{d}{\pi_1} & S^n \arrow{d}{\pi_2} \\
\R P^n & S^n /_\sim
\end{tikzcd}
\]
where $\pi_1$ and $\pi_2$ are the projections to each quotient space as shown in the diagram. The map $\pi_2 \circ f : \R^{n+1} \setminus \{0\} \to S^n /_\sim$ is given by
\[
(\pi_2 \circ f)(x) = \pi_2\left( \frac{x}{\|x\|} \right) = \left\{ -\frac{x}{\|x\|}, \frac{x}{\|x\|} \right\} ,
\]
which is continuous and constant on the fibres of $\pi_1$; the lines through the origin. It thus induces a continuous map $\tilde{f} : \R P^n \to S^n /_\sim$ for which the following diagram is commutative:
\[
\begin{tikzcd}[column sep = large, row sep = large]
\R^{n+1} \setminus \{0\} \arrow{r}{f} \arrow{d}{\pi_1} \arrow{rd}{\pi_2 \circ f} & S^n \arrow{d}{\pi_2} \\
\R P^n \arrow[dashed]{r}{\tilde{f}} & S^n /_\sim
\end{tikzcd}
\]
We define a continuous inverse of $\tilde{f}$. Consider $g : S^n \to \R^{n+1} \setminus \{0\}$ given by $g(x) = x$. As before, consider the diagram
\[
\begin{tikzcd}[column sep = large, row sep = large]
\R^{n+1} \setminus \{0\} \arrow{d}{\pi_1} & S^n \arrow{d}{\pi_2} \arrow{l}{g} \\
\R P^n & S^n /_\sim
\end{tikzcd}
\]
The map $\pi_1 \circ g : S^n \to \R P^n$ is given by
\[
(\pi_1 \circ g)(x) = \pi_1(x) = \{ \lambda x : \lambda \neq 0 \} = [x]_1,
\]
which is continuous and constant on the fibres of $\pi_2$; antipodal points on the $n$-sphere. It thus induces a continuous map $\tilde{g} : S^n /_\sim \to \R P^n$ for which the following diagram is commutative:
\[
\begin{tikzcd}[column sep = large, row sep = large]
\R^{n+1} \setminus \{0\} \arrow{d}{\pi_1} & S^n \arrow{d}{\pi_2} \arrow{l}{g} \arrow{ld}{\pi_1 \circ g}\\
\R P^n & S^n /_\sim \arrow[dashed]{l}{\tilde{g}}
\end{tikzcd}
\]
We claim that $\tilde{f}$ and $\tilde{g}$ are inverses to each other, which will show that $\tilde{f}$ is a homeomorphism $\R P^n \xrightarrow{\sim} S^n /_\sim$. We have
\begin{align*}
(\tilde{g} \circ \tilde{f})([x]_1) &= (\tilde{g} \circ \tilde{f} \circ \pi_1)(x) = (\tilde{g} \circ \pi_2 \circ f)(x) = (\pi_1 \circ g \circ f)(x) = \pi_1\left(g\left( \frac{x}{\|x\|} \right)\right) = \pi_1\left(\frac{x}{\|x\|}\right) = [x]_1 \\
(\tilde{f} \circ \tilde{g})([x]_2) &= (\tilde{f} \circ \tilde{g} \circ \pi_2)(x) = (\tilde{f} \circ \pi_1 \circ g)(x) = (\pi_2 \circ f \circ g)(x) = \pi_2(f(x)) = \pi_2 \left( \frac{x}{\|x\|} \right) = [x]_2 
\end{align*}
So $\tilde{f}$ is a homeomorphism $\R P^n \xrightarrow{\sim} S^n /_\sim$.
\end{proof}
In particular, $\R P^n$ is compact! Note that we could have just explicitly defined
\begin{alignat*}{2}
&\tilde{f} : \R P^n \to S^n /_\sim \qquad &&\tilde{f}([x]_1) := \pi_2(f(x)) = \left[ \frac{x}{\|x\|} \right]_2 \\
&\tilde{g} : S^n /_\sim \to \R P^n \qquad &&\tilde{g}([x]_2) := \pi_1(g(x)) = [x]_1
\end{alignat*}
checked for well-definedness and continuity, and then we'd have been done. That's how the proof on page 362 of Tu goes. However, the abuse of tikz diagrams makes it very clear where the homeomorphism and its inverse come from, and that they're continuous (which is basically what Tu is doing anyway).

% may need work
\subsection{Visualizing $\R P^2$}
In order to visualize $\R P^2$ we will consider some homeomorphisms. Define
\begin{align*}
H^2 &= \{(x,y,z) \in \R^3 : x^2 + y^2 + z^2 = 1, z \geq 0\} \\
D^2 &= \{(x,y) \in \R^2 : x^2 + y^2 \leq 1\}.
\end{align*}
Consider the maps
\begin{alignat*}{2}
&\phi : H^2 \to D^2 \qquad &&\phi(x,y,z) = (x,y) \\
&\psi : D^2 \to H^2 \qquad &&\psi(x,y) = (x,y,\sqrt{1-x^2-y^2})
\end{alignat*}
which are continuous inverses of each other. Define equivalence relations on $H^2$ and $D^2$ as follows:
\begin{itemize}
\item On $H^2$: identify antipodal points on the equator, call the projection $\pi_3$
\item On $D^2$: identify antipodal points on the boundary, call the projection $\pi_4$
\end{itemize}
Considering diagrams similar to those in the previous proof, the map $\pi_4 \circ \phi$ induces a continuous map $\tilde{\phi} : H^2 /_\sim \to D^2 /_\sim$ with $\tilde{\phi} \circ \pi_3 = \pi_4 \circ \phi$, and the map $\pi_3 \circ \psi$ induces a continuous map $\tilde{\psi} : D^2 /_\sim \to H^2 /_\sim$ with $\tilde{\psi} \circ \pi_4 = \pi_3 \circ \psi$. The maps $\tilde{\phi}$ and $\tilde{\psi}$ are continuous inverses of each other (which can be seen using just these given compositions), which shows that we have a homeomorphism $H^2 /_\sim \xrightarrow{\sim} D^2 /_\sim$. 

If we accept on faith that there is a homeomorphism $S^2 /_\sim \xrightarrow{\sim} H^2 /_\sim$, then we have a sequence of homeomorphisms
\[
\R P^2 \xrightarrow{\sim} S^2 /_\sim \xrightarrow{\sim} H^2 /_\sim \xrightarrow{\sim} D^2 /_\sim.
\]
Therefore we can visualize the real projective plane $\R P^2$ as a disk with the antipodal boundary points identified. Such a homeomorphism $S^2 /_\sim \xrightarrow{\sim} H^2 /_\sim$ can be shown by a proof similar to the previous quotient space homeomorphisms that we did, by considering the inclusion map $i : H^2 \to S^2$ and its obvious inverse, and working through steps similar to the proofs of the previous homeomorphisms.

\newpage

\section{Smooth Maps and Differentiable Structures (May 19)}

\subsection{Smooth Maps on a Manifold}
The notion of the pullback of a function on a manifold (which by MAT257 we know is a $0$-form on a manifold - not that that's important right now) is the following:
\begin{definition}
Let $F : M \to N$ and $f : N \to \R$ be functions. The pullback of $f$ by $F$ is the function $F^*f:M \to \R$ defined by $F^*f = f \circ F$. That is, the pullback of $f$ by $F$ is the unique function for which the following diagram commutes:
\[
\begin{tikzcd}[column sep = large, row sep = large]
M \arrow[rd, "F^*f"] \arrow[d, "F"] \\
N \arrow[r, "f"] & \R
\end{tikzcd} 
\]
\end{definition}
Now for the main definitions.
\begin{definition}
Fix a smooth manifold $M$. A function $f : M \to \R$ is $C^\infty$ at $p \in M$ if there is a chart $(U, \phi)$ about $p$ such that $f \circ \phi^{-1} : \phi(U) \to \R$ is $C^\infty$ at $\phi(p)$, in the usual sense. Alternatively, $f$ is $C^\infty$ at $p$ if the pullback $(\phi^{-1})^*f$ of $f$ by the inverse of some coordinate system $\phi$ about $p$ is $C^\infty$ at $\phi(p)$.
\end{definition}
We'd like to show that this does not depend on the choice of chart about $p$. If $(V, \psi)$ is another chart about $p$, then
\[
f \circ \psi^{-1} = (f \circ \phi^{-1}) \circ (\phi \circ \psi^{-1}),
\]
is $C^\infty$ at $\psi(p)$ on the open set $\psi(U \cap V)$, since $\phi \circ \psi^{-1}$ is $C^\infty$ at $\psi(p)$ and $f \circ \phi^{-1}$ is $C^\infty$ at $\phi(p)$. Therefore smoothness of a function on a manifold at a point doesn't depend on the choice of chart about that point. We will say that $f$ is $C^\infty$ on $M$ if it is $C^\infty$ at every point of $M$. Note that if $f : M \to \R$ is $C^\infty$ at $p$, then $f = (f \circ \phi^{-1}) \circ \phi$ is continuous at $p$.

These considerations give us a
\begin{proposition}
Let $f : M \to \R$ be a continuous function on a smooth manifold $M$. The following are equivalent:
\begin{enumerate}[(i)]
\item $f : M \to \R$ is $C^\infty$.
\item There is an atlas $\mathcal{A}$ of $M$ such that for any $(U, \phi) \in \mathcal{A}$, the function $f \circ \phi^{-1} : \phi(U) \to \R$ is $C^\infty$.
\end{enumerate}
\end{proposition}
Note that we implicitly assume $\mathcal{A}$ in the above is a subset of our choice of maximal atlas for $M$. When we say $M$ is a smooth manifold, we also assume a choice of maximal atlas has been made.

What about maps between manifolds? The definition is a natural extension of the one we just made.
\begin{definition}
Let $N$ and $M$ be smooth manifolds and let $F : N \to M$ be continuous. We say $F$ is $C^\infty$ at $p \in N$ if there is a chart $(V, \psi)$ about $F(p)$ and a chart $(U, \phi)$ about $p$ such that 
\[
\psi \circ F \circ \phi^{-1} : \phi(U \cap F^{-1}(V)) \to \R^m
\]
is $C^\infty$ at $\phi(p)$.
\end{definition}
Note that continuity of $F$ was essential, for if that were not the case, the set $\phi(U \cap F^{-1}(V))$ may not be open, in which case we may not be able to talk about smoothness at $p$. 

As before, we check that this is independent of the charts. Choose charts $(\tilde{U}, \tilde{\phi})$ about $p$ and $(\tilde{V}, \tilde{\psi})$ about $F(p)$. Then
\[
\tilde{\psi} \circ F \circ \tilde{\phi}^{-1} = (\tilde{\psi} \circ \psi^{-1}) \circ (\psi \circ F \circ \phi^{-1}) \circ (\phi \circ \tilde{\phi}^{-1})
\]
is $C^\infty$ at $\tilde{\phi}(p)$ by similar reasoning as before. We say that $F : N \to M$ is $C^\infty$ if it is so at every point of $N$.

We have a similar proposition coming from the independence of charts:
\begin{proposition}
Let $F : N \to M$ be a continuous function of smooth manifolds $N$ and $M$. The following are equivalent:
\begin{enumerate}[(i)]
\item $F$ is $C^\infty$ on $N$.
\item There are atlases $\mathcal{A}$ of $N$ and $\mathcal{B}$ of $M$ such that for every $(U, \phi) \in \mathcal{A}$ and $(V, \psi) \in \mathcal{B}$, the map $\psi \circ F \circ \phi^{-1} : \phi(U \cap F^{-1}(V)) \to \R^m$ is $C^\infty$.
\end{enumerate}
\end{proposition}

% everything after this needs checking

We need to make sure this is actually a generalization of the notion of smoothness we know from calculus. We will make sure that our definition is the usual notion of smoothness when the manifolds are Euclidean spaces, and we will make sure that smoothness is preserved by compositions.
\begin{proposition}
If $N = \R^n$ and $M = \R^m$ are given their usual smooth structures, then $F : N \to M$ is smooth as defined above if and only if it is smooth as a function of Euclidean spaces.
\end{proposition}
\begin{proof}
Choose the atlases $\{(\R^n, \mathrm{Id}_{\R^n})\}$ on $\R^n$ and $\{(\R^m, \mathrm{Id}_{\R^m})\}$ on $\R^m$. Then $F : N \to M$ is smooth as defined above if and only if
\[
\mathrm{Id}_{\R^m} \circ F \circ \mathrm{Id}_{\R^n}^{-1} : N \to M
\]
is smooth. But this function is just $F : \R^n \to \R^m$.
\end{proof}
Note that this holds if $N$ and $M$ had merely been open sets of Euclidean spaces, for the usual smooth structure on them (i.e. the one we do ordinary calculus with) is the maximal atlas corresponding to the restrictions of the charts given above to those open sets.

\begin{proposition}
If $F : N \to M$ and $G : M \to P$ are $C^\infty$ maps of manifolds, then $G \circ F : N \to P$ is $C^\infty$.
\end{proposition}
\begin{proof}
Suppose $p \in N$. Choose charts $(U, \phi)$ about $p$, $(V, \psi)$ about $F(p)$, and $(W, \sigma)$ about $G(F(p))$. Then 
\[
\sigma \circ (G \circ F) \circ \phi^{-1} = (\sigma \circ G \circ \psi^{-1}) \circ (\psi \circ F \circ \phi^{-1})
\]
is $C^\infty$ at $\phi(p)$, since $\sigma \circ G \circ \psi^{-1}$ is $C^\infty$ at $\psi(F(p))$ and $\psi \circ F \circ \phi^{-1}$ is $C^\infty$ at $\phi(p)$.
\end{proof}

We have one last property: vector-valued functions behave how we want them to.
\begin{proposition}
Let $N$ be a smooth manifold and $F : N \to \R^m$ be a continuous function. The following are equivalent:
\begin{enumerate}[(i)]
\item $F$ is $C^\infty$.
\item Each component function $F^i : N \to \R$ is smooth.
\end{enumerate}
\end{proposition}
\begin{proof}
The proof was left as an exercise, so here's a solution. We have
\begin{align*}
F \text{ is } C^\infty &\iff \text{ for every chart } (U, \phi) \text{ on } N \text{, the map } F \circ \phi^{-1} : \phi(U) \to \R^m \text{ is } C^\infty \\
&\iff \text{ for each } i \text{ and for every chart } (U, \phi) \text{ on } N \text{, the map } F^i \circ \phi^{-1} : \phi(U) \to \R \text{ is } C^\infty \\
&\iff \text{ for each } i \text{, the map } F^i : N \to \R \text{ is } C^\infty.
\end{align*}
\end{proof}

Just as two vector spaces or groups are equivalent if they are isomorphic, or two topological spaces are equivalent if they are homeomorphic, or two sets are equivalent if they are in bijection with eachother, we have a notion of "isomorphism" or equivalence of smooth manifolds.
\begin{definition}
A function $F : N \to M$ of smooth manifolds is said to be a diffeomorphism if it is smooth and has a smooth inverse.
\end{definition}
Then we can state: \emph{differential topology is the study of properties of smooth manifolds invariant under diffeomorphism}.

\subsection{Differentiable Structures}

We can exhibit two diffeomorphic but unequal smooth structures on $\R$. Define two atlases
\begin{alignat*}{2}
\mathcal{A}_1 &= \{(\R, \mathrm{Id})\} \qquad &&\text{(call this one $\R$)}\\
\mathcal{A}_2 &= \{ (\R, \psi(x) := x^3) \} \qquad &&\text{(call this one $\R'$)}
\end{alignat*}
These charts are not $C^\infty$ compatible, since $\mathrm{Id} \circ \psi^{-1}$ sends $x$ to $\sqrt[3]{x}$; not a diffeomorphism. Therefore the smooth structures corresponding to $\mathcal{A}_1$ and $\mathcal{A}_2$ are different.

Nevertheless, define $f : \R \to \R'$ by $f(x) = \sqrt[3]{x}$. Then 
\[
\psi \circ f \circ \mathrm{Id}^{-1} : \R \to \R' \qquad x \mapsto x
\]
is a diffeomorphism!

We can exhibit non-diffeomorphic smooth structures on manifolds; see the exotic sphere $S^7$. Even better, $\R^4$ has uncountably many smooth structures \emph{up to diffeomorphism}. It is known that every topological manifold of dimension $<4$ admits a unique smooth structure, up to diffeomorphism.

\newpage

\section{Inverse Function Theorem, Tangent Spaces (May 21)}

\subsection{Diffeomorphisms and Coordinate Systems}
By convention, any manifold labelled $N$ will have dimension $n$ and any labelled $M$ will have dimension $m$.
\begin{definition}
A diffeomorphism $F : N \to M$ of smooth manifolds is a bijective smooth map with smooth inverse.
\end{definition}
\begin{proposition}
Coordinate systems are diffeomorphisms.
\end{proposition}
\begin{proof}
Let $M$ be a smooth manifold and $(U, \phi)$ a coordinate chart on $M$. Choose the atlases
\begin{align*}
&\{(U, \phi)\} \text{ on $U$} \\
&\{\phi(U), \mathrm{Id}\} \text{ on $\phi(U)$}.
\end{align*}
Then 
\begin{align*}
&\mathrm{Id} \circ \phi \circ \phi^{-1} : \phi(U) \to \phi(U) \\
&\phi \circ \phi^{-1} \circ \mathrm{Id}^{-1} : \phi(U) \to \phi(U)
\end{align*}
are both smooth, implying that $\phi$ and $\phi^{-1}$ are smooth, respectively.
\end{proof}
The converse is true; it uses maximality of the smooth structure.
\begin{proposition}
Diffeomorphisms from open subsets of manifolds to open subsets of Euclidean space are coordinate systems belonging to the manifold's smooth structure.
\end{proposition}
\begin{proof}
Was left as an exercise in class, so here's a solution. Let $F : U \to F(U)$ be a diffeomorphism of the open subset $U$ of the smooth manifold $M$ with an open subset $F(U) \subseteq \R^m$. Then $(U, F)$ is a coordinate chart on $M$. Choose a coordinate chart $(V, \psi)$ for $M$. If $U \cap V = \emptyset$ we are done, and otherwise, the transition maps are $F \circ \phi^{-1}$ and $\phi \circ F^{-1}$, both of which are clearly smooth. So the transition map is a diffeomorphism, and so $(U, F)$ is a coordinate chart belonging to the smooth structure by maximality.
\end{proof}

\subsection{Coordinate Derivatives, Inverse Function Theorem}

In calculus, we take derivatives. How can we take derivatives of functions on manifolds? The first thing we can try is differentiating with respect to local coordinates. If $(U, \phi)$ is a coordinate system on a manifold, we will write $(U, \phi) = (U, x^1, \dots, x^m)$ to mean that $x^i$ is the $i$th component of $\phi$. More specifically, if $r^1, \dots, r^m$ are the coordinates on $\R^m$, then $x^i = r^i \circ \phi$.

\begin{definition}
Let $f : M \to \R$ be a smooth function on the smooth $M$. Let $p \in M$ and let $(U, \phi) = (U, x^1, \dots, x^m)$ be a coordinate chart around $p$. Define
\[
\left. \frac{\pd f}{\pd x^i} \right|_p := \left. \frac{\pd (f \circ \phi^{-1})}{\pd r^i} \right|_{\phi(p)}
\]
as the $i$th partial derivative of $f$ at $p$ with respect to the coordinates $(U, x^1, \dots, x^m)$.
\end{definition}
What about maps between manifolds? We can do something similar. Let $F : N \to M$ be a smooth map between smooth manifolds. Let $(U, \phi) = (U, x^1, \dots, x^n)$ and $(V, \psi) = (V, y^1, \dots, y^m)$ be coordinate charts on $N$ and $M$, respectively. Define \emph{the $i$th component of $F$ with respect to the coordinates $(V, y^1, \dots, y^m)$} by $F^i := y^i \circ F = r^i \circ \psi \circ F$. Then $F^i : N \to \R$, so by our previous definition we can look at
\[
\left. \frac{\pd F^i}{\pd x^j} \right|_{p} = \left. \frac{\pd (F^i \circ \phi^{-1})}{\pd r^j} \right|_{\phi(p)} = \left. \frac{\pd (r^i \circ \psi \circ F \circ \phi^{-1})}{\pd r^j} \right|_{\phi(p)} = \left. \frac{\pd (\psi \circ F \circ \phi^{-1})^i}{\pd r^j} \right|_{\phi(p)}.
\]
We will call the $m \times n$ matrix $\begin{bmatrix}
\left. \frac{\pd F^i}{\pd x^j} \right|_p
\end{bmatrix}$ the \emph{Jacobian of $F$ at $p$ (relative to the coordinates $(U, x^1, \dots, x^n)$ and $(V, y^1, \dots, y^m)$)}. 

The Jacobian itself is not independent of the coordinate systems, but since transition maps are diffeomorphisms, its rank is independent of the coordinate systems chosen. Precisely, if $(\tilde{U}, \tilde{\phi})$ and $(\tilde{V}, \tilde{\psi})$ are alternate coordinate charts around $p$ and $F(p)$, respectively, then we have
\[
\tilde{\psi} \circ F \circ \tilde{\phi}^{-1} = (\tilde{\psi} \circ \psi^{-1} ) \circ (\psi \circ F \circ \phi^{-1}) \circ (\phi \circ \tilde{\phi}^{-1}),
\]
implying
\[
D(\tilde{\psi} \circ F \circ \tilde{\phi}^{-1})(\tilde{\phi}(p)) = \underbrace{D(\tilde{\psi} \circ \psi^{-1})(\psi(F(p)))}_{\in GL(m, \R)} \cdot D(\psi \circ F \circ \phi^{-1})(\phi(p)) \cdot \underbrace{D(\phi \circ \tilde{\phi}^{-1})(\tilde{\phi}(p))}_{\in GL(n, \R)},
\]
and so linear algebra tells us that
\[
\mathrm{rank}(D(\psi \circ F \circ \phi^{-1})(\phi(p))) = \mathrm{rank}(D(\tilde{\psi} \circ F \circ \tilde{\phi}^{-1})(\tilde{\phi}(p))).
\]
Therefore "the rank of the Jacobian of $F$ at $p$" is a well-defined quantity, independent of the coordinate charts. We state this as a proposition.
\begin{proposition}
If $F : N \to M$ is $C^\infty$ at $p$, then the rank of the Jacobian of $F$ at $p$ is the same no matter what coordinate charts around $p$ and $F(p)$ are used to calculate it.
\end{proposition}
In particular, if $m = n$, then we are led to a generalization of the inverse function theorem, as we can then speak of invertibility of the Jacobian.

\begin{theorem}
(Inverse function theorem for manifolds) Let $F : N \to M$ be a smooth map of smooth manifolds of the same dimension. If the Jacobian of $F$ at $p \in N$ is invertible, then there is an open neighbourhood $U$ of $p$ in $N$ and an open neighbourhood $V$ of $F(p)$ in $M$ such that $\left. F \right|_U : U \to V$ is a diffeomorphism.
\end{theorem}
\begin{proof}
Was left as an exercise in class, so here's a solution. Choose coordinate charts $(U, \phi) = (U, x^1, \dots, x^n)$ at $p$ and $(V, \psi) = (V, y^1, \dots, y^n)$ at $F(p)$. Then $\begin{bmatrix}
\left. \frac{\pd F^i}{\pd x^j} \right|_p
\end{bmatrix}$ is invertible, but as we saw above, this is equivalent to saying $\begin{bmatrix}
\left. \frac{\pd (\psi \circ F \circ \phi^{-1})^i}{\pd r^j} \right|_{\phi(p)}
\end{bmatrix}$ is invertible. By the inverse function theorem in $\R^n$, the map $\psi \circ F \circ \phi^{-1}$ is a diffeomorphism on a small neighbourhood of $\phi(p)$ in $\phi(U \cap F^{-1}(V))$. Since coordinate systems are diffeomorphisms, $F$ is a diffeomorphism on a small neighbourhood of $p$.
\end{proof}
(The following was not part of the lecture.) Note that the converse of the above theorem is true; if $F$ restricts to a diffeomorphism in a neighbourhood of $p$, then the Jacobian with respect to any choices of coordinates is invertible. This can be seen by taking two coordinate charts $(U, \phi)$ around $p$ and $(V, \psi)$ around $F(p)$ and noting that since $\psi \circ F \circ \phi^{-1}$ is then a diffeomorphism of open sets of $\R^n$, the Jacobian of $F$ with respect to these coordinate systems is invertible (and hence with respect to any coordinate systems). Therefore we have the following slightly stronger theorem:
\begin{theorem}
(Stronger inverse function theorem for manifolds) Let $F : N \to M$ be a smooth map of smooth manifolds of the same dimension. Then $F$ is a local diffeomorphism at $p \in N$ if and only if the Jacobian  of $F$ at $p$ is invertible.
\end{theorem}
Of course, \emph{local diffeomorphism at $p$} means that $F$ restricts to a diffeomorphism on an open neighbourhood of $p$.

We would like a "coordinate-free" derivative. In MAT257, the derivative of a map $F : \R^n \to \R^m$ at $p$ was thought of as the map $F_{*} : T_p \R^n \to T_{F(p)}\R^m$ defined by $F_*(v_p) = (DF(p)v)_{F(p)}$, where the subscript indicates the tangent space in which the vector lies. The difficulty in generalizing this to manifolds lies in defining the tangent space of an abstract manifold.

\subsection{Abstracting the Tangent Space}

If $M$ is a submanifold in $\R^n$ in the MAT257 sense, then we can define the tangent space as follows. Suppose $p \in M$. Then there is an open neighbourhood $V$ of $p$ in $\R^n$, an open set $U \subseteq \R^k$, and a $C^\infty$ homeomorphism $\phi : U \to V \cap M$ with $\mathrm{rank}( D\phi(q) ) = k$ for each $q \in U$. If $q = \phi^{-1}(p)$, then let $T_p U$ be the "set of all vectors in $\R^k$ thought of as pointing from $q$". Then we define $T_p M := D\phi(q)(T_p U)$. Since the derivative has rank $k$ at $q$, the space $T_p M$ will be a vector subspace of $T_p \R^n$ of dimension $k$. It is not hard to see that this is independent of the "parametrization" chosen near $p$.

The problem with this is that it doesn't generalize to abstract manifolds. We'd like to modify the definition so that it abstracts. 

We will first attempt to do so by using curves. Let $p \in \R^n$ and $v \in T_p \R^n$. If $F : \R^n \to \R^m$ is smooth, then we can speak of its Jacobian at a point $p \in \R^n$. If $\gamma : (-\epsilon, \epsilon) \to \R^n$ is a smooth curve with $\gamma(0) = p$ and $\gamma'(0) = v$, then
\[
\left. \frac{d}{dt} \right|_{t=0} F \circ \gamma = DF(p) \cdot v,
\]
so we can think of $DF(p)$ as a map sending tangent vectors to tangent vectors. (A picture would really help here.)

Define $A = \{\text{smooth curves with $\gamma(0) = p$}\}$. Define $\sim$ on $A$ by $\gamma \sim \tilde{\gamma}$ if and only if $\gamma'(0) = \tilde{\gamma}'(0)$. Then we can think of a vector $v \in T_p \R^n$ as the equivalence class $[\gamma]$ of a curve $\gamma$ with $\gamma(0) = p$, and we can think of $T_{F(p)} \R^m$ as $A /_\sim$.

This generalizes to manifolds, since we know what a smooth curve is on a manifold. But who wants to work with equivalence classes? We don't.

\subsection{Germs and Derivations}

Introduce a smooth function $f : \R^n \to \R$. We can speak of the directional derivative of $f$. We have
\[
D_vf = \left. \frac{d}{dt} \right|_{t=0} f \circ \gamma = \nabla f(p) \cdot v,
\]
where the quantity $\nabla f(p) \cdot v$ is independent of $\gamma$. We can therefore choose to identify $v \in T_p \R^n$ with the map $D_v : C^\infty (\R^n) \to \R$. But the value of $D_v f$ only depends on "the local behaviour of $f$ at $p$", and so we would like to consider two inputs of $D_v$ to be equivalent if they are equal on a smaller neighbourhood of $p$. For this, we develop germs.

\begin{definition}
Let $f : U \to \R$ and $g : V \to \R$ be smooth functions defined on open neighbourhoods of $p$. We will say that $f \sim g$ if and only if $\left. f \right|_W = \left. g \right|_W$ for some open neighbourhood $W \subseteq U \cap V$ of $p$. Denote by $C^\infty_p(\R^n)$ the set of all such equivalence classes. The equivalence class $[f]$ is called the germ of $f$ at $p$. 
\end{definition}

The map $D_v$ is constant on germs, so it induces a map $D_v : C^\infty_p(\R^n) \to \R$ (note the notational abuse). The set of germs at $p$ has some nice algebraic properties. That it is an "algebra" was not covered in lecture.
\begin{proposition}
$C_p^\infty(\R^n)$ is a vector space over $\R$. It can be made into a ring with multiplication of germs, and it can be made into an "algebra over $\R$"; a ring which is also a vector space over $\R$ with the vector space scalar multiplication satisfying the homogeneity condition $a(vw) = (av) \cdot w = v \cdot (aw)$.
\end{proposition}
\begin{proof}
Was left as an exercise in class, so here's a solution. We define three operations on $C_p^\infty(\R^n)$:
\begin{itemize}
\item Vector addition: $[f] + [g] := [f + g]$.
\item Vector scaling: $a[f] := [af]$.
\item Ring multiplication: $[f] \cdot [g] := [fg]$.
\end{itemize}
We must first check that these operations are well defined. If $[f] = [\tilde{f}]$ and $[g] = [\tilde{g}]$, then $f = \tilde{f}$ on a neighbourhood of $p$ and $g = \tilde{g}$ on another neighbourhood of $p$. It then follows that $f + g = \tilde{f} + \tilde{g}$, $af = a\tilde{f}$, and $fg = \tilde{f}\tilde{g}$ on the intersections of these neighbourhoods. By definition we have $[f + g] = [\tilde{f} + \tilde{g}]$, $[af] = [a\tilde{f}]$, and $[fg] = [\tilde{f}\tilde{g}]$, implying that our three operations are well-defined.

$C_p^\infty(\R^n)$ is clearly a vector space over $\R$ under the first two operations, and is also a ring over the first and last operation. Homogeneity of the ring multiplication with respect to vector scaling follows from the corresponding assertion for $C^\infty (\R^n)$. (At this point it is just definition pushing.)
\end{proof}
Note that the ring $C_p^\infty (\R^n)$ is commutative and has unity - the identity element of the multiplication is the germ $[x \mapsto 1]$. From now on we will abuse notation (even more) and let $f$ denote its germ $[f] \in C_p^\infty(\R^n)$.

We note two properties of our map $D_v : C^\infty_p(\R^n) \to \R$:
\begin{enumerate}
\item $D_v$ is linear.
\item $D_v$ satisfies the "Leibnitz rule"
\[
D_v (fg) = f(p) D_v(g) + D_v(f) g(p).
\]
\end{enumerate}
Any map $D : C^\infty_p(\R^n) \to \R$ satisfying the above properties is called a \emph{derivation at $p$}. The set of all derivations at $p$ is denoted $\mathcal{D}_p$. It turns out that this view of the tangent space is what generalizes to manifolds. Before we prove the "identification theorem", we need two lemmas.

\begin{lemma}
If $f$ is $C^\infty$ on an open ball $U$ centred at $p$, then there are smooth $g_i \in C^\infty(U)$ such that $g_i(p) = \frac{\pd f}{\pd x_i}(p)$ and
\[
f(x) = f(p) + \sum_{i=1}^n (x^i - p^i) g_i(x).
\]
\end{lemma}
\begin{proof}
Define $\gamma (t) = p + t(x-p)$. Then
\begin{align*}
f(x) - f(p) &= \int_0^1 \frac{d}{dt} f(\gamma(t)) \, dt = \int_0^1 \sum_{i=1}^n \left. \frac{\pd f}{\pd x^i} \right|_{\gamma(t)} (x^i - p^i) \, dt = \sum_{i=1}^n (x^i - p^i)\underbrace{\int_0^1 \frac{\pd f}{\pd x^i}(p + t(x-p)) \, dt}_{g_i(x)}.
\end{align*}
\end{proof}

\begin{lemma}
Derivations of constants are zero.
\end{lemma}
\begin{proof}
Let $c \in \R$. Then
\begin{align*}
D(c) &= c \cdot D(1) = c \cdot D(1\cdot 1) \\
&= c \cdot 1 \cdot D(1) + c \cdot D(1) \cdot 1 \\
&= 2c \cdot D(1) \\
&= 2 D(c),
\end{align*}
implying $D(c) = 0$.
\end{proof}

\begin{theorem}
We can identify $T_p \R^n$ with $\mathcal{D}$. More specifically,
\begin{enumerate}
\item $\mathcal{D}_p$ is a vector space over $\R$.
\item The map $\Phi : T_p \R^n \to \mathcal{D}_p$ sending $v$ to $D_v$ is a vector space isomorphism.
\end{enumerate}
\end{theorem}
\begin{proof}
\begin{enumerate}
\item 
Was left as an exercise in class, so here's a proof. We must check that if $a \in \R$ and $D_1, D_2 \in \mathcal{D}_p$, then the function $a D_1 + D_2$ is a derivation. It is linear as a sum of linear functions. If $f, g \in C_p^\infty(\R^n)$, then
\begin{align*}
(aD_1 + D_2)(fg) &= a D_1(fg) + D_2(fg) \\
&= a \left[ f(p) D_1(g) + D_1(f)g(p) \right] + f(p)D_2(g) + D_2(f) g(p) \\
&= f(p) \left[ aD_1(g) + D_2(g) \right] + \left[ aD_1(f) + D_2(f) \right] g(p) \\
&= f(p)(aD_1 + D_2)(g) + (aD_1 + D_2)(f) g(p),
\end{align*}
so $aD_1 + D_2$ satisfies the Leibnitz rule and is thus a derivation a $p$. Therefore $\mathcal{D}_p$ is a vector space over $\R$.

\item 
We check linearity, injectivity, and surjectivity.
\begin{itemize}
\item
Linearity: if $a \in \R$ and $v_1, v_2 \in T_p\R^n$, then for $f \in C_p^\infty(\R^n)$ we have
\[
\Phi(av_1 + v_2)(f) = D_{av_1 + v_2}(f) = Df(p)(av_1 + v_2) = a Df(p)v_1 + Df(p)v_2 = (a \Phi(v_1) + \Phi(v_2))(f).
\]

\item 
Injectivity: suppose $D_v(f) = 0$ for all $f \in C_p^\infty(\R^n)$. In particular, $D_v x^i = 0$ for the $i$th coordinate map $x^i \in C_p^\infty(\R^n)$. Expanded, this says
\[
0 = D_v x^i = Dx^i(p)v = e_i^Tv = v^i,
\]
where $e_i$ is the $i$th standard basis vector of $\R^n$. So $v = 0$ if $\Phi(v) = 0$.

\item 
Surjectivity: suppose $D \in \mathcal{D}_p$. For any $f \in C_p^\infty(\R^n)$, we have, by the two lemmas,
\begin{align*}
Df &= D \left(f(p) + \sum_{i=1}^n (x^i - p^i) g_i \right) \\
&= \sum_{i=1}^n \left[(p^i - p^i) Dg_i + D(x^i - p^i) g_i(p)\right] \\
&= \sum_{i=1}^n Dx^i \frac{\pd f}{\pd x^i}(p),
\end{align*}
so if we take $v = (Dx^1, \dots, Dx^n)$ then $Df = D_vf$ for all $f \in C_p^\infty(\R^n)$. Therefore $\Phi(v) = D$.
\end{itemize}
So $\Phi$ is a vector space isomorphism $T_p \R^n \xrightarrow{\sim} \mathcal{D}_p$.
\end{enumerate}
\end{proof}

We can finally define the tangent space to a point on an abstract manifold. The space of germs $C_p^\infty(M)$ for $p \in M$ is defined in the exact same way as in $\R^n$. 
\begin{definition}
Let $M$ be a smooth manifold. We say $v : C_p^\infty(M) \to \R$ is a derivation at $p \in M$ if $v$ is linear and satisfies the Leibnitz rule. We define the tangent space $T_p M$ to $M$ at $p$ to be the set of all derivations at $p$.
\end{definition}

(The following was not part of the lecture and is included for completeness.) Now we can finally define the derivative of a map between manifolds.

\begin{definition}
Let $F : N \to M$ be a smooth map of smooth manifolds and let $p \in N$. The map $F$ induces a linear map $F_* : T_p N \to T_{F(p)} M$ defined by
\[
(F_* X_p)(f) = X_p(f \circ F),
\]
where $X_p \in T_p N$ is a derivation at $p$ and $f \in C^\infty_{F(p)}(M)$.
\end{definition}

\newpage

\section{Tangent Spaces and the Differential (May 26)}

\subsection{Derivatives and the Chain Rule on Manifolds}

Having defined the smooth functions on a manifold, in order to proceed with generalizing calculus to manifolds, we must now differentiate them. The notion of the derivative comes from the \emph{differential} or \emph{pushforward} of a smooth map $F : \R^n \to \R^m$ of standard calculus; it is the derivative $DF(p) : T_p \R^n \to T_{F(p)} \R^m$. As we are now working with derivations as our tangent vectors, the definition must be adjusted accordingly.

\begin{definition}
Let $F : N \to M$ be a $C^\infty$ map of smooth manifolds. We define the differential of $F$ at $p$ by the linear map
\[
F_* : T_p N \to T_{F(p)}M, \qquad F_*(X_p)(f) = X_p(f \circ F),
\]
where $X_p \in T_p N$ and $f \in C_{F(p)}^\infty(M)$. The map $F_*$ is sometimes denoted $F_{*, p}$, and is not to be confused with the pullback operation $F^* : C(M) \to C(N)$ on continuous functions. 
\end{definition}
Let's make sure this makes sense; i.e. that $F_*(X_p)$ is actually a derivation at $F(p)$. Linearity follows immediately from linearity of $X_p$ on $C_p^\infty(N)$. If $f, g \in C_{F(p)}^\infty(M)$, then
\begin{alignat*}{2}
F_*(X_p)(fg) &= X_p((fg) \circ F)  &&\text{by definition}\\
&= X_p((f \circ F)(g \circ F)) \\
&= (f \circ F)(p) X_p(g \circ F) + X_p(f \circ F) (g \circ F)(p) \qquad &&\text{$X_p$ a derivation at $p$} \\
&= f(F(p)) F_*(X_p)(g) + F_*(X_p)(f) g(F(p)).
\end{alignat*}
Therefore $F_*$ is indeed a map $T_p N \to T_{F(p)}M$. That it is also linear is obvious.

We need to make sure this properly generalizes the derivative of a $C^\infty$ map $F : \R^n \to \R^m$. Thinking of a tangent vector $v \in T_p \R^n$ as its directional derivative $D_v : C_p^\infty(\R^n) \to \R$, we have, for $f \in C_{F(p)}^\infty(\R^m)$,
\[
F_*(D_v)(f) = D_v(f \circ F) = \nabla f(F(p))( DF(p) \cdot v ) = D_{DF(p)v}(f),
\]
where the latter directional derivative is at $F(p)$. Therefore $F_*(D_v) = D_{DF(p)v}$. If we again identify derivations of germs at $F(p)$ with tangent vectors in $T_{F(p)} \R^m$, we conclude that the differential between manifolds generalizes the derivative from calculus. 

The differential is the same as the derivative, definition wise. Does it hold the same properties. The answer is "yes".
\begin{theorem}
(Chain Rule) Let $F : N \to M$ and $G : M \to P$ be $C^\infty$ maps of smooth manifolds. Then $(G \circ F)_{*, p} = G_{*, F(p)} \circ F_{*, p}$.
\end{theorem}
\begin{proof}
If $X_p \in T_p N$ and $f \in C_{G(F(p))}^\infty P$, then
\[
(G_{*, F(p)} \circ F_{*, p})(X_p)(f) = G_{*, F(p)}(F_{*,p}(X_p))(f) = F_{*,p}(X_p)(f \circ G) = X_p(f \circ G \circ F) = (G \circ F)_{*, p}(X_p)(f)
\]
\end{proof}
If the "base point" is understood, then we will often omit it and simply write $F_*$ and $G_*$, in which case the chain rule reads as $(G \circ F)_* =G_* \circ F_*$.

\subsection{Dimension of Tangent Spaces}

We present some very useful corollaries of the chain rule.
\begin{corollary}
The differential of the identity map $\mathrm{Id} : M \to M$ is the identity map $\mathrm{Id}_* : T_p M \to T_p M$.
\end{corollary}
\begin{proof}
If $X_p \in T_p M$ and $f \in C_p^\infty(M)$ then
\[
\mathrm{Id}_*(X_p)(f) = X_p(f \circ \mathrm{Id}) = X_p(f),
\] 
so $\mathrm{Id}_*(X_p) = X_p$.
\end{proof}
\begin{corollary}
If $F : N \to M$ is a diffeomorphism of smooth manifolds and $p \in N$, then $F_* : T_p N \to T_{F(p)} M$ is an isomorphism of vector spaces.
\end{corollary}
\begin{proof}
By the previous corollary and the chain rule we have
\begin{align*}
F_* \circ (F^{-1})_* = (F \circ F^{-1})_*  &= \mathrm{Id}_{T_{F(p)} M}, \\
(F^{-1})_* \circ F_* = (F^{-1} \circ F)_* &= \mathrm{Id}_{T_p N},
\end{align*}
so $F_* : T_p N \to T_{F(p)}M$ is a bijective linear map.
\end{proof}
\begin{corollary}
(Invariance of dimension) Let $U \subseteq \R^n$ and $V \subseteq \R^m$ be diffeomorphic open sets. Then $n =m$.
\end{corollary}
\begin{proof}
If $F : U \to V$ is a diffeomorphism then by the previous corollary it induces an isomorphism $F_* : T_p U \to T_{F(p)} V$ of vector spaces. Therefore
\[
n = \dim(T_p \R^n) = \dim(T_p U) = \dim(T_{F(p)} V) = \dim(T_{F(p)} \R^m) = m
\]
\end{proof}
The above theorem holds in the case where the sets are merely homeomorphic, but that requires algebraic topology to prove and is decidedly non-trivial.

\begin{proposition}
If $M$ is a smooth manifold of dimension $m$, then for each $p \in M$, the tangent space $T_p M$ has dimension $m$.
\end{proposition}
\begin{proof}
Choose a coordinate chart $(U, \phi)$ around $p$. Then we have a diffeomorphism $\phi : U \to \phi(U)$, so $\phi_* : T_p U \to T_{\phi(p)} \phi(U)$ is an isomorphism. Then
\[
\dim(T_p M) = \dim(T_{\phi(p)} \phi(U)) = m.
\]
\end{proof}

\subsection{A Basis for the Tangent Space}

Knowing the dimension of the tangent space brings us to the following question: what is a basis of the tangent space? We have a main result.
\begin{theorem}
Let $M$ be a smooth manifold and let $p \in M$. Choose a coordinate chart $(U, \phi) = (U, x^1, \dots, x^m)$ around $p$. Then
\[
\left\{ \left. \frac{\pd }{\pd x^1} \right|_p, \dots, \left. \frac{\pd }{\pd x^m} \right|_p \right\}
\]
is a basis of $T_p M$.
\end{theorem}
\begin{proof}
We have, for $f \in C_{\phi(p)}^\infty (\R^n)$,
\[
\phi_* \left( \left. \frac{\pd }{\pd x^i} \right|_p \right)(f) = \left. \frac{\pd }{\pd x^i} \right|_p f \circ \phi = \left. \frac{\pd (f \circ \phi \circ \phi^{-1})}{\pd r^i} \right|_{\phi(p)} = \left( \left. \frac{\pd }{\pd r^i} \right|_{\phi(p)} \right)(f).
\]
Since $\phi_*$ is an isomorphism and isomorphisms send bases to bases, the fact that 
\[
\left\{ \left. \frac{\pd }{\pd r^1} \right|_{\phi(p)}, \dots, \left. \frac{\pd }{\pd r^m} \right|_{\phi(p)} \right\}
\]
is a basis of $T_{\phi(p)} \phi(U)$ implies that the proposed basis of $T_p M$ is indeed a basis.
\end{proof}
We will sometimes write $\frac{\pd}{\pd x^i}$ instead of $\left. \frac{\pd }{\pd x^i} \right|_{p}$ if the base point of the tangent vector is understood.

Of course, the basis of the tangent space depends on the choice of coordinate chart. What are the changes of coordinates? 
\begin{proposition}
Suppose $(U, x^1, \dots, x^m)$ and $(V, y^1, \dots, y^m)$ are two coordinate charts on a manifold $M$. Then on $U \cap V$,
\[
\frac{\pd}{\pd x^j} = \sum_i \frac{\pd y^i}{\pd x^j} \frac{\pd}{\pd y^i}.
\]
(One can remember this by thinking of the $\pd y^i$'s as cancelling.)
\end{proposition}
\begin{proof}
Since $\{ \pd / \pd x^i|_p \}$ and $\{ \pd / \pd y^i|_p \}$ are both bases of the tangent space $T_p M$, for each $p \in U \cap V$, there is an $m \times m$ matrix $[a^i_j]$ (depending on $p$) such that
\[
\frac{\pd}{\pd x^j} = \sum_k a^k_j \frac{\pd}{\pd y^k}
\]
on $U \cap V$. Evaluating both sides at $y^i$ gives
\begin{align*}
\frac{\pd y^i}{\pd x^j} =  \sum_k a^k_j \frac{\pd y^i}{\pd y^k} = \sum_k a^k_j \delta^i_k = a^i_j.
\end{align*}
\end{proof}

\newpage
\section{Curves, Submanifolds (May 28)}

\subsection{A Local Expression for the Differential}

Let $N, M$ be smooth manifolds and $F : N \to M$ a $C^\infty$ map. For $p \in N$, the differential $F_{*,p} : T_p N \to T_{F(p)}M$ is linear, so if we fix coordinate charts $(U, x^1, \dots, x^n)$ near $p$ and $(V, y^1, \dots, y^m)$ near $F(p)$, then we can speak about the matrix of $F_{*,p}$ relative to the bases $\{ \left. \frac{\pd}{\pd x^i} \right|_p \}$ and  $\{ \left. \frac{\pd}{\pd y^i} \right|_{F(p)} \}$. It turns out that this matrix is precisely the Jacobian of $F$ relative to these two coordinate systems.

Let $A = [a^i_j]$ be the matrix of $F_{*,p}$ relative to the above bases. That is, for $j = 1, \dots, n$,
\[
F_{*,p} \left( \left. \frac{\pd}{\pd x^j} \right|_p \right) = \sum_{i=1}^m a^i_j \left. \frac{\pd}{\pd y^i} \right|_{f(p)}.
\]
Applying $y^i$ to both sides of the above equation gives
\[
a^i_j = \sum_{k=1}^m a^k_j \left. \frac{\pd y^i}{\pd y^k} \right|_{F(p)} = F_{*,p} \left( \left. \frac{\pd}{\pd x^j} \right|_p \right)(y^i) = \left. \frac{\pd F^i}{\pd x^j} \right|_p.
\]
Therefore $A = [\left. \frac{\pd F^i}{\pd x^j} \right|_p]$. We state this fact formally as a proposition.
\begin{proposition}
Let $F : N \to M$ be a $C^\infty$ map and let $p \in N$. Choose coordinate charts $(U, x^1, \dots, x^n)$ near $p$ and $(V, y^1, \dots, y^m)$ near $F(p)$. Then the matrix representation of the linear transformation $F_{*,p} : T_pN \to T_{F(p)}M$ in the bases given by these coordinate charts is the Jacobian $[\left. \frac{\pd F^i}{\pd x^j} \right|_p]$ relative to these coordinate systems.
\end{proposition}
Recall that the inverse function theorem stated that, assuming the hypotheses of the above proposition, $F$ is a local diffeomorphism at $p$ if and only if its Jacobian was nonsingular. The above proposition therefore gives us a "coordinate-free" inverse function theorem.
\begin{theorem}
(Inverse function theorem) Let $F : N \to M$ be a $C^\infty$ map of manifolds of the same dimension and suppose $p \in N$. Then $F$ is a local diffeomorphism at $p$ if and only if its differential $F_{*,p}$ is an isomorphism.
\end{theorem}

\subsection{Curves on Manifolds}

We'd like to be able to relate the abstract tangent space $T_p M$, a set of derivations, to "tangent vectors" of curves in $M$ at $p$.
\begin{definition}
A $C^\infty$ curve in a manifold $M$ is a smooth map $\gamma : (a, b) \to M$. We will usually assume that $0 \in (a, b)$ and that $\gamma(0) = p$.
\end{definition}
How can we discuss that tangent vector? First, consider the case $M = \R^n$. Let $\beta : (a, b) \to \R^n$ be a $C^\infty$ curve with $\beta(0) = p$. Then
\[
\beta'(0) = \left. \frac{d}{dt} \right|_0 \beta,
\]
and so we can think of $\beta'$ as a map $c \mapsto \beta' \cdot c$. 
\begin{definition}
The velocity vector of $\gamma$ at $t_0$ is the differential 
\[
\gamma'(t_0) := \gamma_*\left( \left. \frac{d}{dt} \right|_{t_0} \right).
\]
\end{definition}
Suppose $X_p = \gamma'(0)$ and that $f \in C_p^\infty(M)$. Then
\[
X_p(f) = \gamma'(0)(f) = \gamma_*\left( \left. \frac{d}{dt} \right|_{t_0} \right)(f) = \left. \frac{d}{dt} \right|_0 (f \circ \gamma).
\]
If $M = \R^n$, this is the directional derivative of $f$ at $p$ in the direction $\gamma'(0)$ (this means the standard derivative). Note that the right side of the above equation is independent of the curve $\gamma$.

The following proposition says that every tangent vector is the velocity vector of some curve. Morally, manifolds are locally like $\R^n$, and since velocity vectors of curves are "local things", we can transfer them over to manifolds easily.
\begin{proposition}
For any $X_p \in T_p M$, there is a smooth curve $\gamma : (a,b) \to M$ with $\gamma(0) = p$ and $\gamma'(0) = X_p$.
\end{proposition}
\begin{proof}
Choose a coordinate chart $(U, \phi) = (U, x^1, \dots, x^n)$ near $p$. There are scalars $a^1, \dots, a^n$ such that $X_p = \sum a^i \left. \frac{\pd}{\pd x^i} \right|_p$. Then $\phi_*(X_p) = \sum a^i \left. \frac{\pd}{\pd r^i} \right|_{\phi(p)}$. Define $\beta : (-\epsilon, \epsilon) \to \R^n$ by
\[
\beta(t) = \phi(p) + t(a^1, \dots, a^n),
\]
where $\epsilon > 0$ is small enough so that the entire curve lies in $\phi(U)$. Then $\beta$ is a smooth curve satisfying $\beta(0) = \phi(p)$. There are scalars $b^1, \dots, b^n$ such that $\beta'(0) = \sum b^k \left. \frac{\pd}{\pd r^k} \right|_{\phi(p)}$. Applying $r^i$ to both sides gives
\[
b^i = \sum b^k \left. \frac{\pd r^i}{\pd r^k} \right|_{\phi(p)} = \beta_* \left( \left. \frac{d}{dt} \right|_0 \right) = \left. \frac{d}{dt} \right|_0 (r^i \circ \beta) = a^i,
\]
which implies that $\beta'(0) = \phi_*(X_p) = \sum a^i \left. \frac{\pd}{\pd r^i} \right|_{\phi(p)}$. If $\gamma = \phi^{-1} \circ \beta$, then $\gamma$ is a $C^\infty$ curve in $M$ with $\gamma(0) = p$ and
\[
\gamma'(0) = (\phi^{-1} \circ \beta_*) \left( \left. \frac{d}{dt} \right|_0 \right) = \phi_*^{-1} \left( \beta_* \left( \left. \frac{d}{dt} \right|_0 \right) \right) = X_p.
\]
\end{proof}
Of course, we could have chosen any curve in $\phi(U)$ whose tangent vector is $\phi_*(X_p)$. However, there's no loss in taking the simplest possible one: the line through $\phi(p)$ in the direction $(a^1, \dots, a^n)$ (which is, after some identifications, just the tangent vector $\phi_*(X_p)$).

Consider a smooth map $F : N \to M$. If $X_p \in T_pN$ is equal to $\gamma'(0)$ for some smooth curve $\gamma$ (the previous proposition ensures this is always the case), then
\begin{align*}
F_{*, p}(X_p) &= F_{*, p}(\gamma'(0)) = F_{*, p}\left( \gamma_{*, 0}\left( \left. \frac{d}{dt} \right|_0 \right) \right) = (F \circ \gamma)_{*, 0} \left( \left. \frac{d}{dt} \right|_0 \right) = (F \circ \gamma)'(0).
\end{align*} 
This gives us a way to compute the differential of a smooth map using curves. If we go back to $\R^n$ and $\R^m$, then this is just the directional derivative of $F$ at $\gamma(0) = p$ in the direction $\gamma'(0)$ (identifying tangent vectors with arrows).

\subsection{Immersions and Submersions}

We want a submanifold to be a subset of a smooth manifold that is also a smooth manifold which, in some sense, inherits the smooth structure from the larger manifold. We will make a few definitions.

\begin{definition}
Let $F : N \to M$ be a $C^\infty$ map. We define the rank of $F$ at $p$ to be the rank of the linear map $F_{*, p}$. Equivalently, it is the dimension of $F_{*,p}(T_pN)$, or the rank of the Jacobian of $F$ at $p$ relative to any two charts. (Recall that we showed that this quantity is independent of the charts used.)
\end{definition}

We will mostly be concerned with smooth maps of constant rank. Studying the rank of a smooth map allows us to study smooth manifolds using linear algebra, something we already know very well and that is often very easy to work with.

\begin{definition}
$F$ is an immersion at $p$ if $F_{*,p}$ is injective, and is an immersion if this is the case for all $p \in N$. This definition implies that $n \leq m$, and that if $F$ is an immersion, it has constant rank $n$.
\end{definition}
\begin{definition}
$F$ is a submersion at $p$ if $F_{*, p}$ is surjective, and is a submersion if this is the case for all $p \in N$. This definition implies that $n \geq m$, and that if $F$ is a submersion, it has constant rank $m$.
\end{definition}
The following two examples are the "canonical" immersions and submersions, in the sense that every immersion is locally the canonical immersion, and every submersion is locally the canonical submersion. 

The "canonical immersion" is the map $i : \R^n \to \R^m$, $n < m$, defined by $i(x^1, \dots, x^n) = (x^1, \dots, x^n, 0, \dots, 0)$. This map is clearly $C^\infty$, and its Jacobian relative to the standard coordinates is the matrix
\[
\begin{pmatrix}
I_{n \times n} \\ 0
\end{pmatrix},
\]
which clearly shows $i_{*, p}$ is injective for all $p \in \R^n$. Therefore $i$ is an immersion.

The "canonical submersion" is the map $\pi : \R^n \to \R^m$, $n > m$, defined by $\pi(x^1, \dots, x^n) = (x^1, \dots, x^m)$. This map is clearly $C^\infty$, and its Jacobian relative to the standard coordinates is the matrix
\[
\begin{pmatrix}
I_{m \times m} & 0
\end{pmatrix},
\]
which clearly shows $\pi_{*, p}$ is surjective for all $p \in \R^n$. Therefore $\pi$ is a submersion.

We now state the theorems which say that these are indeed the "canonical" examples of their kind.

\begin{theorem}
(Immersion theorem) Let $F : N \to M$ be an immersion and $p \in N$. There are coordinate charts $(U, \phi)$ near $p$ and $(V, \psi)$ near $F(p)$ such that
\[
\psi \circ F \circ \phi^{-1} : (x^1, \dots, x^n) \mapsto (x^1, \dots, x^n, 0, \dots, 0).
\]
\end{theorem}
\begin{theorem}
(Submersion theorem) Let $F : N \to M$ be an immersion and $p \in N$. There are coordinate charts $(U, \phi)$ near $p$ and $(V, \psi)$ near $F(p)$ such that
\[
\psi \circ F \circ \phi^{-1} : (x^1, \dots, x^n) \mapsto (x^1, \dots, x^m).
\]
\end{theorem}
We will prove both of these later as a corollary of the following theorem, whose proof will be given later.
\begin{theorem}
(Constant rank theorem) Let $F : N \to M$ have constant rank $r$ and $p \in N$. Then there are charts $(U, \phi)$ near $p$ and $(V, \psi)$ near $F(p)$ such that
\[
\psi \circ F \circ \phi^{-1} : (x^1, \dots, x^n) \mapsto (x^1, \dots, x^r, 0, \dots, 0).
\]
\end{theorem}

\begin{definition}
A smooth map $F : N \to M$ is said to be an embedding if it is an immersion and a topological embedding (i.e. a homeomorphism onto its image in the subspace topology).
\end{definition}
We have a few properties of immersions and submersions.

\begin{proposition}
\begin{enumerate}
\item $F : N \to M$ is a local diffeomorphism if and only if it is an immersion and a submersion

\item An immersion is locally injective, and $F$ is an immersion if and only if it is locally an embedding.

\item Submersions are open.
\end{enumerate}
\end{proposition}

%check these
\begin{proof}
Was left as an exercise in class, so here's a solution.
\begin{enumerate}
\item
$F$ is a local diffeomorphism if and only if $F_{*,p}$ is an isomorphism, if and only if $F_{*,p}$ is injective and surjective, if and only if $F$ is an immersion and a submersion at $p$.

\item
Suppose that $F$ is an immersion and that $p \in N$. By the immersion theorem there are charts $(U, \phi)$ near $p$ and $(V, \psi)$ near $F(p)$ such that $\psi \circ F \circ \phi^{-1} : \phi(U \cap F^{-1}(V)) \to \R^m$ takes on the form 
\[
(\psi \circ F \circ \phi^{-1})(x^1, \dots, x^n) = (x^1, \dots, x^n, 0, \dots, 0).
\]
We claim that $F|_{U \cap F^{-1}(V)}$ is injective. Suppose $F(a) = F(b)$ for some $a,b \in U$. Then $(\psi \circ F \circ \phi^{-1})(\phi(a)) = (\psi \circ F \circ \phi^{-1})(\phi(b))$, and so the above equation implies that $\phi(a) = \phi(b)$. Since $\phi$ is bijective, $a = b$. Therefore $F$ is locally injective. Moreover, the above equation implies that on $U \cap F^{-1}(V)$, $F$ is composition of embeddings. Therefore $F$ is a local embedding.

If $F$ is a local embedding, then for each $p \in N$ there is an open neighbourhood $U$ of $p$ such that $F|_U$ is an embedding. Therefore $F|_U$ is an immersion, which implies that $(F|_U)_{*,p}$, and therefore $F_{*,p}$, is injective, so $F$ is an immersion at $p$. This holds for all $p \in N$, so $F$ is an immersion. 
 
\item
Suppose $F$ is a submersion. Let $O \subseteq N$ be open and suppose $F(p) \in F(O)$. Since $p \in O \subseteq N$, there are, by the submersion theorem, coordinate charts $(U, \phi)$ at $p$ and $(V, \psi)$ at $F(p)$ such that $\psi \circ F \circ \phi^{-1} : \phi(U \cap F^{-1}(V)) \to \R^m$ takes on the form 
\[
(\psi \circ F \circ \phi^{-1})(x^1, \dots, x^n) = (x^1, \dots, x^m),
\]
where $n \geq m$. This implies that on $U \cap F^{-1}(V)$, $F$ is a composition of open mappings (the "canonical submersion" $\pi$ is certainly open). Moreover, $U \cap F^{-1}(V) \cap O$ is an open neighbourhood of $p$, and so we can find an open neighbourhood $W$ of $p$ contained in $U \cap F^{-1}(V) \cap O$. Then $F(W)$ is an open neighbourhood of $F(p)$ contained in $F(O)$, which implies that $F(O)$ is open. 
\end{enumerate}
\end{proof}

\subsection{Submanifolds}

\begin{proposition}
Let $M$ be a $k$-dimensional manifold in $\R^n$, in the MAT257 sense. Then the inclusion map $i : M \to \R^n$ is an embedding.
\end{proposition}
\begin{proof}
That the inclusion map is a topological embedding is obvious. Given a coordinate map $\phi : U \to V \cap M$, the pair $(V \cap M, \phi^{-1})$ is a coordinate chart on $M$. Then $i \circ \phi$ is simply $\phi$, which is injective since $D\phi(q)$ is injective. Therefore $i$ is an immersion.
\end{proof}
Therefore the smooth structure on $M$ is, in some sense, "inherited" from the ambient space $\R^n$. That is, if $F : \R^n \to \R$ is $C^\infty$, then $F|_M$ is a $C^\infty$ map of smooth manifolds. (The local converse is true.)

\begin{definition}
$S \subseteq M$ is a(n embedded) submanifold of the smooth manifold $M$ if it is a smooth manifold such that the inclusion map $i : S \to M$ is an embedding.
\end{definition}
The following proposition is immediate.
\begin{proposition}
If $F : N \to M$ is a $C^\infty$ map between manifolds and $S \subseteq N$ is a submanifold as we just defined it, then $F|_S : S \to M$ is $C^\infty$.
\end{proposition}
We list some examples of submanifolds.
\begin{enumerate}
\item
Every "MAT257 manifold" in $\R^n$ is a submanifold of $\R^n$.

\item
Every open subset $U \subseteq M$ of a smooth manifold is a submanifold, since the inclusion $i : U \to M$ is an embedding. 

In fact, the only submanifolds of $M$ with the same dimension as $M$ are the open sets. This was left as an exercise in class, so here's a solution. Suppose $S \subseteq M$ is a submanifold of $M$ with $\dim S = \dim M$. Then the inclusion map $i : S \to M$ is an embedding. For $p \in S$, the differential $i_{*,p} : T_pS \to T_pM$ is an injective linear map between two vector spaces of the same dimension, so it is invertible. By the inverse function theorem, $i$ is a local diffeomorphism at $p$, which implies that $p$ is contained in some open subset of $M$ contained in $S$. Therefore $S$ is open.

\item
The graph $\Gamma_f$ of $f(x) = |x|$ defined on $\R$ is a smooth manifold, but it is not a submanifold of $\R^2$. Consider the atlas $\{(\Gamma_f, \pi)\}$, where $\pi : (x,f(x)) \mapsto x$. The inclusion map $i : \Gamma_f \to \R^2$ is not $C^\infty$, since $(i \circ \pi^{-1})(x) = (x, |x|)$.

\item Embeddings give rise to submanifolds in the following sense.
\begin{proposition}
Let $F : N \to M$ be an embedding. Then there is a unique smooth structure on $F(N)$ such that $F(N)$ is a submanifold of $M$ and that $F : N \to F(N)$ a diffeomorphism.
\end{proposition}
\begin{proof}
Let $\mathcal{A}$ be an atlas on $N$. Define $\mathcal{A}' = \{ (F(U), \phi \circ F^{-1}) : (U, \phi) \in \mathcal{A} \}$. Each set $F(U)$ is open in $F(N)$, and each $\phi \circ F^{-1} : F(U) \to \phi(U)$ is a homeomorphism, because $F$ is a homeomorphism onto its image. If $(U_1, \phi_1 \circ F^{-1}), (U_2, \phi_2 \circ F^{-1}) \in \mathcal{A}'$, then
\[
(\phi_1 \circ F^{-1}) \circ (\phi_2 \circ F^{-1})^{-1} = \phi_1 \circ \phi_2^{-1}
\]
is $C^\infty$. Therefore $\mathcal{A}'$ is a $C^\infty$ atlas on $F(N)$, making it a smooth manifold of the same dimension as $N$. 

Consider a chart $(W, \sigma)$ on $N$ and a chart $(F(U), \phi \circ F^{-1})$ on $F(N)$. Then
\[
(\phi \circ F^{-1}) \circ F \circ \sigma^{-1} = \phi \circ \sigma^{-1}
\]
is $C^\infty$ because $\phi$ and $\sigma$ are both coordinate systems on $N$, and
\[
\sigma \circ F^{-1} \circ (\phi \circ F^{-1})^{-1} = \sigma \circ F^{-1} \circ F \circ \phi^{-1} = \sigma \circ \phi^{-1}
\]
is $C^\infty$ for the same reason. Therefore $F : N \to F(N)$ is a diffeomorphism. That the smooth structure corresponding to $\mathcal{A}'$ is the unique one on $F(N)$ with respect to which $F : N \to F(N)$ is a diffeomorphism is clear.

The inclusion map $i : S \hookrightarrow M$ is the composition of a diffeomorphism and an embedding: $S \xrightarrow{F^{-1}} N \xrightarrow{F} M$, and so it is an embedding itself. Therefore $F(N)$ is an embedded submanifold.
\end{proof}

\item
Let $U \subseteq N$ be an open subset of a smooth manifold and $F : U \to M$ be a $C^\infty$ map into a smooth manifold $M$. Then $\Gamma_f = \{ (x, f(x)) : x \in U \}$ is a submanifold of $N \times M$. This can be proved by defining $F : U \to N \times M$ by $F(x) = (x, f(x))$ and showing that this is an embedding.
\end{enumerate}

\subsection{Regular Submanifolds}

\begin{definition}
Suppose $M$ is an $n$-dimensional manifold. $S \subseteq M$ is a regular submanifold of dimension $k$ if for each $p \in S$, there exists a chart $(U, x^1, \dots, x^n)$ of $M$ near $p$ such that $U \cap S$ is defined by the vanishing of $n - k$ of the coordinate functions; we may as well assume it is defined by the vanishing of the last $n-k$ coordinates. That is,
\[
U \cap S = \{ q \in S : x^{k+1}(q) = \cdots = x^n(q) = 0 \}.
\]
Such a coordinate chart is called an adapted chart relative to $S$.
\end{definition}
If $(U, \phi) = (U, x^1, \dots, x^n)$ is an adapted chart relative to $S$, define $\phi_S : U \cap S \to \R^k$ by $\phi_S(q) = (x^1(q), \dots, x^k(q))$. The pair $(U \cap S, \phi_S)$ is a coordinate chart on $S$ in the subspace topology. (That is, $\phi_S = \pi \circ \phi|_S$.)

If $\{ (U \cap S, \phi_S) \}$ is a collection of adapted charts relative to $S$ covering $S$, it is not hard to see that they form a $C^\infty$ atlas on $S$, making $S$ a manifold of dimension $k$. $S$ is said to have \emph{codimension $n-k$ in $M$}.

Note that the definition of a submanifold we gave in which the inclusion was required to be a smooth embedding (an "embedded submanifold") is equivalent to the definition of a regular manifold. We state this as a theorem - without proof for now.
\begin{theorem}
$S \subseteq M$ is a regular submanifold if and only if it is an embedded submanifold.
\end{theorem}
\begin{theorem}
(Whitney embedding theorem) Any smooth manifold of dimension $n$ can be embedded in $\R^{2n}$.
\end{theorem}
The Klein bottle is an example of a manifold of dimension $2$ which cannot be embedded in $\R^3$, but can be embedded in $\R^4$.

\newpage
\section{Equivalence of Regular and Embedded Submanifolds (June 2)}

\subsection{Regular Submanifolds}

Recall the definition of a regular submanifold.
\begin{definition}
Let $M$ be a smooth manifold. $S \subseteq M$ is a regular submanifold of dimension $k$ if for each $p \in S$ there is a chart $(U, \phi) = (U, x^1, \dots, x^n)$ for $M$ at $p$ such that $U \cap S$ is defined by the vanishing of exactly $n-k$ of the coordinates (we will usually take these to be the last such coordinates). Such a chart is called an adapted chart relative to $S$.
\end{definition}
If $\{(U, \phi)\}$ is an atlas for $M$ of adapted charts relative to $S$, then it is not hard to see that $\{(U \cap S, \phi_S)\}$ is an atlas for $S$ in the subspace topology, where $\phi_S := \pi \circ \phi|_S$. Therefore $S$ is a smooth manifold of dimension $k$.

A regular submanifold "inherits" the smooth structure from $M$ in the following sense:
\begin{proposition}
If $f : M \to \R$ is $C^\infty$ and $S \subseteq M$ is a regular submanifold, then $f|_S : S \to \R$ is $C^\infty$.
\end{proposition}
\begin{proof}
For any adapted chart $(U, \phi)$ relative to $S$, $f \circ \phi^{-1}$ is $C^\infty$. Then $f \circ \phi_S^{-1}$ is $C^\infty$, since it is the composition $f \circ \phi^{-1} \circ g$, where $g : (x^1, \dots, x^k) \mapsto (x^1, \dots, x^k, 0, \dots, 0)$ is the "canonical immersion".
\end{proof}

For example, consider a $C^\infty$ function $f : \R \to \R$. Then $\Gamma_f$ becomes a smooth manifold with the atlas $\{(\Gamma_f, \pi)\}$, where $\pi : (x,f(x)) \mapsto x$. For an open set $U \subseteq \R^2$ intersecting $\Gamma_f$, define $\psi : U \to \R^2$ by $\psi(x,y) = (x, y-f(x))$. Then $\psi$ is a local diffeomorphism, which implies that, after shrinking $U$, the pair $(U, \psi)$ is a coordinate chart belonging to the standard smooth structure on $\R^2$. Moreover, $\Gamma_f \cap U$ is defined by the vanishing of the last coordinate of $\psi$, so $(U, \psi)$ is an adapted chart relative to $\Gamma_f$. We can do this at any point of $\Gamma_f$, so we can conclude that $\Gamma_f$ is a regular submanifold of $\R^2$ of dimension $1$.

What is the tangent space to a regular submanifold $S \subseteq M$? Note that we cannot write $T_pS \subseteq T_pM$, since the elements are not even the same. However, if $v \in T_pS$, there is a unique $\tilde{v} \in T_pM$ such that for any $f \in C_p^\infty(M)$, $\tilde{v}(f) = v(f|_S)$. (Uniqueness is immediate, and existence follows by defining $\tilde{v}$ by that formula.) Let $\Phi$ be the map $v \mapsto \tilde{v}$. Linearity is obvious, and for injectivity, suppose $\Phi(v) = \tilde{v} = 0$. Fix an adapted chart $(U, x^1, \dots, x^n)$ at $p$, so that if $y^i = x^i|_S$, then $(U \cap S, y^1, \dots, y^k)$ is a chart on $S$ at $p$. Then $\{\left. \frac{\pd}{\pd y^i}\right|_p\}$ is a basis of $T_pS$, so
\[
v = \sum v(y^i)\left. \frac{\pd}{\pd y^i}\right|_p = \sum v(x^i|_S)\left. \frac{\pd}{\pd y^i}\right|_p = \sum \tilde{v}(x^i)\left. \frac{\pd}{\pd y^i}\right|_p = 0,
\]
so $\Phi$ is injective. Therefore we may think of the $k$-dimensional subspace $\Phi(T_pS) \subseteq T_pM$ as "$T_pS$ living inside $T_pM$".

\subsection{Embedded Submanifolds}

Recall the definition of an embedded submanifold.
\begin{definition}
Let $M$ be a smooth manifold. $S \subseteq M$ is an embedded submanifold of dimension $k$ if it is a smooth manifold of dimension $k$ such that the inclusion map $i : S \hookrightarrow M$ is an embedding (topological embedding and an immersion).
\end{definition}

Let $M$ be a smooth manifold and $S \subseteq M$ a subset which is also a smooth manifold. Is it true that the inclusion $i : S \hookrightarrow M$ is $C^\infty$? Not always. Consider the case $\Gamma_f$ for $f(x) = |x|$. Then $\Gamma_f$ is a smooth manifold and a subset of the smooth manifold $\R^2$, but the inclusion $\Gamma_f \hookrightarrow \R^2$ is not smooth.

Give $S$ the subspace topology, so that $i : S \hookrightarrow M$ is a topological embedding. Suppose $S$ is equipped with a smooth structure such that $i$ is $C^\infty$. We claim that $i$ is then an embedding, in the sense that, in addition to being a topological embedding, it is an immersion. (The proof will be a homework exercise.)

An embedded submanifold "inherits" the smooth structure from $M$ in the following sense:
\begin{proposition}
If $f : M \to \R$ is $C^\infty$ and $S \subseteq M$ is an embedded submanifold, then $f|_S : S \to \R$ is $C^\infty$.
\end{proposition}
\begin{proof}
$f|_S = f \circ i$.
\end{proof}

What is the tangent space to an embedded submanifold $S \subseteq M$? The inclusion $i : S \hookrightarrow M$ has injective differential $i_{*,p} : T_pS \to T_pM$, and so we can think of the $k$-dimensional subspace $i_{*,p}(T_pS) \subseteq T_pM$ as "$T_pS$ living inside $T_pM$". Moreover, in reference to the tangent space of a regular submanifold, we have $i_{*,p} = \Phi$, since
\[
i_{*,p}(v)(f) = v(f \circ i) = v(f|_S) = \tilde{v}(f)
\]
for every $f \in C_p^\infty(M)$ and $v \in T_pS$.

\subsection{Equivalence of the Two}

After noticing the similarities between regular and embedded submanifolds, one might ask whether or not they are the same. The answer is yes.
\begin{theorem}
Let $M$ be a smooth manifold and $S \subseteq M$. $S$ is a regular submanifold of dimension $k$ if and only if $S$ is an embedded submanifold of dimension $k$.
\end{theorem}
\begin{proof}
Suppose $S$ is a regular submanifold of dimension $k$. It is given the subspace topology, so $i : S \hookrightarrow M$ is a topological embedding. Let $(U, \phi)$ be an adapted chart relative to $S$. Then $(U \cap S, \phi_S)$ is a coordinate chart on $S$. The coordinate representation of $i$ in these two charts is
\[
\phi \circ i \circ \phi_S^{-1} : (x^1, \dots, x^k) \mapsto (x^1, \dots, x^k, 0, \dots, 0),
\]
since $U \cap S$ is defined by the vanishing of the last $n-k$ coordinates. In this form it is clear that $i : S \hookrightarrow M$ is an immersion, so $S$ is an embedded submanifold.

The converse follows from the following slightly more general proposition.
\end{proof}
\begin{proposition}
If $f : N \to M$ is an embedding, then $f(N)$ is a regular submanifold of $M$.
\end{proposition}
\begin{proof}
Let $p \in N$. By the immersion theorem, we can find coordinate charts $(U, \phi) = (U, x^1, \dots, x^n)$ at $p$ and $(V, \psi) = (V, y^1, \dots, y^m)$ at $f(p)$ with respect to which $f$, in coordinates, takes on the form
\[
\psi \circ f \circ \phi^{-1} : \phi(U \cap f^{-1}(V)) \to \R^m, \qquad(x^1, \dots, x^n) \mapsto (x^1, \dots, x^n, 0, \dots, 0).
\]
By possibly shrinking $U$, assume that $f(U) \subseteq V$. We may do this by replacing $U$ with $U \cap f^{-1}(V)$, which is open in $N$; we will still have a coordinate chart at $p$ and the above identity will still hold.

We show that $f(U)$ is defined by the vanishing of $y^{n+1}, \dots, y^m$. More precisely, that
\[
f(U) = \{ z \in V : y^{n+1}(z) = \cdots = y^m(z) = 0 \}.
\]
Suppose $q \in U$. Then $f(q)$ satisfies $\psi(f(q)) = (\psi \circ f \circ \phi^{-1})(\phi(q))$, of which the last $m-n$ coordinates vanish. This proves the $\subseteq$ inclusion. Conversely, suppose $z \in V$ satisfies $y^{n+1}(z) = \cdots = y^m(z) = 0$. Then $\psi(z)$ is in the image of $\psi \circ f \circ \phi^{-1}$ because of the vanishing of the last $m-n$ coordinates, so there is a $q \in \phi(U)$ such that $(\psi \circ f \circ \phi^{-1})(q) = \psi(z)$, implying $z = f(\phi^{-1}(q)) \in f(U)$. This proves the $\supseteq$ inclusion, and completes the proof that $f(U)$ is defined by the vanishing of $y^{n+1}, \dots, y^m$.

Since $f$ is a homeomorphism onto its image, $f(U)$ is open in the subspace topology on $f(N)$, so we can find an open set $W$ of $M$ such that $f(U) = W \cap f(N)$. Then
\begin{align*}
(V \cap W) \cap f(N) &= V \cap f(U) \\
&= f(U) \qquad \text{(because we made } f(U) \subseteq V)
\end{align*}
is defined by the vanishing of $y^{n+1}, \dots, y^m$, which implies that $(V \cap W, y^1, \dots,  y^m)$ is an adapted chart at $f(p)$ relative to $f(N)$. Therefore $f(N)$ is a regular submanifold of $M$, of the same dimension as $N$.
\end{proof}

Therefore \emph{embedded submanifolds and regular submanifolds are one and the same thing.}

\newpage

\section{Level Sets, Tangent Bundles (June 4)}

\subsection{Regular and Critical Values}

Recall that we showed that if $g : \R^{n+1} \to \R$ was a $C^\infty$ function such that $\nabla g(x) \neq 0$ for each $x \in g^{-1}(0)$, then $g^{-1}(0)$ was a manifold in $\R^{n+1}$ of dimension $n$. In our new terminology, $g^{-1}(0)$ is a regular submanifold of $\R^{n+1}$ of co-dimension $1$. We are going to generalize this example.

\begin{definition}
Let $F : N \to M$ be a $C^\infty$ map between smooth manifolds and suppose $c \in M$. We call the set $F^{-1}(c)$ the level set of $F$ with level $c$. We say that $c$ is a critical value of $F$ if there is a $p \in F^{-1}(c)$ such that $F_{*,p}$ is not surjective, and we call $p$ a critical point of $F$. Otherwise, $c$ is a regular value of $F$ and the level set is said to be regular. 
\end{definition}

Suppose that $F : M \to \R^k$ has $0$ as a regular value, and that $M$ is a smooth manifold of dimension $n$. In generalizing the first example, we'd like to show that $F^{-1}(0)$ is a regular submanifold of $M$ of co-dimension $k$. Let's first informally discuss why this should be true.

If we choose $p \in F^{-1}(0)$, then $F_{*,p}$ is surjective. By linear algebra, this implies that there are $n-k$ linearly independent directions $v \in T_pM$ such that $F_{*,p}(v) = 0$, corresponding to "how many directions we can move around at $p$ and stay in $F^{-1}(0)$". Similarly, there are $k$ linearly independent directions $v \in T_pM$ such that $F_{*,p}(v) \neq 0$, corresponding to "how many directions we can move around at $p$ to exit $F^{-1}(0)$".

\begin{theorem}
If $F : M \to \R^k$ has $0$ as a regular value, then $F^{-1}(0)$ is a regular submanifold of $M$ of co-dimension $k$.
\end{theorem}
\begin{proof}
Given $p \in F^{-1}(0)$, let $(U, \phi) = (U, x^1, \dots, x^n)$ be a coordinate chart for $M$ at $p$. Since $F_{*,p}$ is surjective, assume without loss of generality that the first $k\times k$ submatrix of the Jacobian of $F$ relative to this coordinate chart is non-singular. The Jacobian of the $C^\infty$ map $(F^1, \dots, F^k, x^{k+1}, \dots, x^n)$ is, relative to this coordinate chart,
\[
\begin{pmatrix}
\frac{\pd (F^1, \dots, F^k)}{\pd (x^1, \dots, x^k)} & * \\
0 & I_{n-k},
\end{pmatrix}
\]
so by the inverse function theorem there is an open set $\tilde{U} \subseteq U$ such that $(\tilde{U}, F^1, \dots, F^k, x^{k+1}, \dots, x^n)$ is a coordinate chart on $M$ at $p$. The set $\tilde{U} \cap F^{-1}(0)$ is defined by setting the first $k$ coordinates of this chart to be $0$, which proves that $F^{-1}(0)$ is a regular submanifold of co-dimension $k$.
\end{proof}
\begin{corollary}
(Regular level set theorem) Let $F : N \to M$ be a $C^\infty$ map and suppose $c \in M$ is a regular value of $F$. Then $F^{-1}(c)$ is a regular submanifold of $N$ of co-dimension $m$.
\end{corollary}
This is a special case of a more general theorem.
\begin{theorem}
(Constant rank level set theorem) Let $F : N \to M$ be a $C^\infty$ map and suppose $c \in M$ is such that $F$ has constant rank $k$ in some neighbourhood of $F^{-1}(c)$ in $N$. Then $F^{-1}(c)$ is a regular submanifold of $N$ of co-dimension $k$.
\end{theorem}
We also have the following equivalent characterization of submanifolds.
\begin{theorem}
$S \subseteq M$ is a submanifold of co-dimension $k$ if and only if for each $p \in S$ there exists a $C^\infty$ map $F : U \to \R^k$ defined on a neighbourhood $U$ of $p$ such that $0$ is a regular value of $F$ and $U \cap S = F^{-1}(0)$.
\end{theorem}

\subsection{Motivating the Tangent Bundle}

We want to define vector fields, differential forms, tensor fields, Riemannian metrics, etc. In order to talk about them and to say that they are smooth, we need the concept of the tangent bundle.

For example, what is a "smooth choice" of tangent vectors $X_p \in T_pM$, for $p \in M$? We also want to make the dual choice; what is a "smooth choice" of \emph{covectors} $\omega_p$, for $p \in M$. (This will be a differential $1$-form).

We also want to make a "smooth choice" of $k$-dimensional subspaces $E_p$ of $T_pM$, for $T_pM$. This will bring up the question "Does there exist a submanifold $S \subseteq M$ such that for each $p \in S$, $i_{*,p}(T_pS) = E_p$?". This will be answered by the \emph{Frobenius theorem}.

We'd like to talk about $(k,l)$-tensors: multilinear maps
\[
T_{p(k,l)} : \underbrace{T_p^*M \times \cdots \times T_p^*M}_{\text{$k$ times}} \times \underbrace{T_pM \times \cdots \times T_pM}_{\text{$l$ times}} \to \R,
\]
where $T_p^*M$ is the dual space to $T_pM$. We want to make a "smooth choice" of $T_{p(k,l)}$ for $p \in M$. Such a $T$ will be called a \emph{smooth $(k,l)$-tensor field}. In particular, a differential $k$-form will be an alternating $(0,k)$-tensor.

Everything is easy in $\R^n$ because the tangent spaces are just copies of $\R^n$. On abstract manifolds, things are more difficult, and so we must develop the notion of a \emph{tangent bundle}.

\subsection{The Tangent Bundle is a Smooth Manifold}

\begin{definition}
Let $M$ be a smooth manifold. The tangent bundle $TM$ is defined to be the set
\[
TM = \bigcup_{p \in M} T_pM = \bigsqcup_{p \in M} T_pM,
\]
where the disjoint union is the set
\[
\bigsqcup_{p \in M} T_pM = \bigcup_{p \in M} (\{p\} \times T_pM).
\]
\end{definition}
Up to notation these are the same thing, since any two $T_pM$, $T_qM$ for $p \neq q$ are already disjoint. The tangent bundle $TM$ comes with a natural projection map $\pi : TM \to M$ defined by $\pi(p, v) = v$, or alternatively, $\pi(v) = p$ if $v \in T_pM$. 

How should we topologize $TM$? We do \emph{not} want to choose the coarsest topology with respect to which $\pi$ is continuous, since the open sets would then be $\pi^{-1}(U) = TU$, which is too large.

Let $(U, \psi) = (U, x^1, \dots, x^n)$ be a coordinate chart on $M$. Then $TU = \bigcup_{p\in U}T_pM$. For $p \in U$, we have a basis $\{ \left. \frac{\pd}{\pd x^i}\right|_p \}$ of $T_pM$. Define $\tilde{\phi} : TU \to \phi(U) \times \R^n$ by
\[
\tilde{\phi}\left( \sum c^i \left. \frac{\pd}{\pd x^i}\right|_p \right) := (x^1(p), \dots, x^n(p), c^1, \dots, c^n) = (\phi(p), c^1, \dots, c^n).
\]
Note that the $c^i$'s are functions of $v \in T_pU$. The map $\tilde{\phi}$ is bijective with inverse
\[
\tilde{\phi}^{-1}(\phi(p), c^1, \dots, c^n).
\]
Equip $TU$ with the unique topology with respect to which $\tilde{\phi}$ is a homeomorphism. That is, declare $V \subseteq TU$ to be open if and only of $\tilde{\phi}(V)$ is open in $\phi(U) \times \R^n \subseteq \R^{2n}$.

Having topologized the tangent bundle of every coordinate neighbourhood in $M$, how do we topologize $TM$? Define
\[
\mathcal{T} = \{ A \subseteq TM : \text{$A$ is open in $TU_\alpha$ for every coordinate open set $U_\alpha$}\}.
\]
It is not hard to see that $\mathcal{T}$ is a topology on $TM$. We have the following proposition:
\begin{proposition}
For a smooth manifold $M$, the projection $\pi : TM \to M$ is a continuous open mapping.
\end{proposition}
Note that by construction $TM$ is locally Euclidean of dimension $2n$. In addition, we have the following proposition, making $TM$ into a topological $2n$-manifold.

\begin{proposition}
$TM$ is second-countable and Hausdorff.
\end{proposition}
\begin{proof}
Since $M$ is second-countable we may choose a countable set of coordinate neighbourhoods $U_\alpha$. Each $TU_\alpha$ is homeomorphic to $\phi_\alpha(U_\alpha) \times \R^n \subseteq \R^{2n}$, so we may choose a countable basis $\mathcal{B}_\alpha$ of $TU_\alpha$. Let $\mathcal{B} = \bigcup_\alpha \mathcal{B}_\alpha$. The set $\mathcal{B}$ is a countable collection of open sets of $TM$; we show it's a basis.

Given $A \subseteq TM$ open and $(p, v) \in A$, choose one of the coordinate neighbourhoods $U_\alpha$ at $p$. Since $A$ is open in $TU_\alpha$ and $(p, v) \in TU_\alpha$, we can find an element $V \in \mathcal{B}_\alpha$ such that $(p,v) \in V \subseteq A \subseteq TU_\alpha$. This proves that $\mathcal{B}$ is a countable basis of $TM$.

Now suppose $(p, v), (q, w) \in TM$. If $p = q$, then choosing a coordinate chart $(U, \phi)$ at $p = q$ shows that $v, w$ are distinct points of $TU \cong \phi(U) \times \R^n$. This set is Hausdorff as it is a subspace of $\R^{2n}$, so we're done. Otherwise, $p \neq q$, so by Hausdorffness of $M$ we can find disjoint open neighbourhoods $U$ of $p$ and $V$ of $q$ in $M$. Then $\pi^{-1}(U)$ and $\pi^{-1}(V)$ are disjoint open neighourboods of $(p, v)$ and $(q, w)$ in $M$, respectively, which proves that $TM$ is Hausdorff.
\end{proof}
In short, if $M$ is a topological $n$-manifold, then $TM$ is a topological $2n$-manifold.

Having given $TM$ a topology and a topological manifold structure, how do we give it a smooth structure? This is easy.
\begin{proposition}
Suppose $M$ is a smooth manifold of dimension $n$. If $\{(U_\alpha, \phi_\alpha)\}$ is a smooth atlas on $M$, then $\{(TU_\alpha, \tilde{\phi_\alpha})\}$ is a smooth atlas on $TM$. Therefore $TM$ is a smooth manifold of dimension $2n$.
\end{proposition}
\begin{proof}
Clearly $TM = \bigcup_\alpha TU_\alpha$. All we have to show is that on $(TU_\alpha) \cap (TU_\beta)$ the maps $\tilde{\phi_\alpha}$ and $\tilde{\phi_\beta}$ are $C^\infty$ compatible.

To make the notation nicer, suppose $(TU, \tilde{\phi})$ and $(TV, \tilde{\psi})$ are charts on $TM$, where $(U, \phi) = (U, x^1, \dots, x^n)$ and $(V, \psi) = (V, y^1, \dots, y^n)$ are coordinate charts on $M$. If $p \in U \cap V$, then
\begin{align*}
(\tilde{\psi} \circ \tilde{\phi}^{-1})(\phi(p), c^1, \dots, c^n) &= \tilde{\psi} \left( \sum_j c^j \left. \frac{\pd}{\pd x^j}\right|_p \right) \\
&= \tilde{\psi} \left( \sum_j c^j \sum_i \frac{\pd y^i}{\pd x^j}(p) \left. \frac{\pd}{\pd y^i}\right|_p \right)  \qquad \text{coord. change } \frac{\pd}{\pd x^j} = \sum_i \frac{\pd y^i}{\pd x^j}\frac{\pd}{\pd y^i} \\
&= \tilde{\psi} \left( \sum_i \left( \sum_j c^j \frac{\pd y^i}{\pd x^j}(p) \right) \left. \frac{\pd}{\pd y^i}\right|_p \right) \\
&= \left( (\psi \circ \phi^{-1})(\phi(p)), b^1, \dots, b^n \right),
\end{align*}
where 
\[
b^i = \sum_j c^j \frac{\pd y^i}{\pd x^j}(p) = \sum_j c^j \frac{\pd (\psi \circ \phi^{-1})^i}{\pd r^j}(\phi(p)).
\]
So $\tilde{\psi} \circ \tilde{\phi}^{-1}$ is $C^\infty$ because $\psi \circ \phi^{-1}$ is. It follows that the atlas we gave $TM$ is smooth, making $TM$ a smooth manifold of dimension $2n$.
\end{proof}
\begin{corollary}
The projection $\pi : TM \to M$ is $C^\infty$.
\end{corollary}
\begin{proof}
For any coordinate map $\phi$ on $M$,
\[
\phi \circ \pi \circ \tilde{\phi}^{-1} : (x^1, \dots, x^n, c^1, \dots, c^n) \mapsto (x^1, \dots, x^n),
\]
which is clearly $C^\infty$.
\end{proof}
\begin{corollary}
If $M$ is a smooth $n$-manifold covered by a single coordinate chart $(M, \phi)$, then $TM$ is diffeomorphic to $M \times \R^n$.
\end{corollary}
We want to use the tangent bundle to work with "global" objects on a manifold. Here's one example.
\begin{definition}
Let $F : N \to M$ be a $C^\infty$ map. Define the global differential $F_* : TN \to TM$ by $F_*((p, v)) = F_{*,p}(v)$. (Note the abuse of notation.)
\end{definition}
\begin{proposition}
The global differential $F_* : TN \to TM$ is $C^\infty$.
\end{proposition}
\begin{proof}
Given $(p, v) \in TN$, let $(TU, \tilde{\phi})$ be a chart at $(p,v)$ and $(TV, \tilde{\psi})$ be a chart at $F_{*,p}(v)$, where $\phi = (x^1, \dots, x^n)$ and $\psi = (y^1, \dots, y^m)$ are local coordinates on $N$ and $M$, respectively. Then
\begin{align*}
(\tilde{\psi} \circ F_* \circ \tilde{\phi}^{-1})(\phi(p), c^1, \dots, c^n) &= (\tilde{\psi} \circ F_*)\left( \sum_j c^j \left. \frac{\pd}{\pd x^j} \right|_p\right) \\
&= \tilde{\psi} \left( \sum_j c^j F_{*,p}\left( \left. \frac{\pd}{\pd x^j} \right|_p \right) \right) \\
&= \tilde{\psi}\left( \sum_j c^j \sum_i \frac{\pd F^i}{\pd x^j}(p) \left. \frac{\pd}{\pd y^i}\right|_{F(p)} \right) \\
&= \tilde{\psi}\left( \sum_i \left( \sum_j c^j \frac{\pd F^i}{\pd x^j}(p) \right) \left. \frac{\pd }{\pd y^i}\right|_{F(p)} \right) \\
&= \left((\psi \circ F \circ \phi^{-1})(\phi(p)), b^1, \dots, b^n \right),
\end{align*}
where
\[
b^i = \sum_j c^i \frac{\pd F^i}{\pd x^j}(p) = \sum_j c^i \frac{\pd (\psi \circ F \circ \phi^{-1})^i}{\pd r^j}(\phi(p)).
\]
Since $\psi \circ F \circ \phi^{-1}$ is $C^\infty$ at $p$, we conclude that the global differential is $C^\infty$ at $(p,v)$.
\end{proof}

\subsection{Sections, Algebraic Structures}

We'd like to describe vector fields on manifolds as functions that take each point $p$ to a tangent vector $X(p) = X_p \in T_pM$. We can easily describe such functions with the following definition.
\begin{definition}
A section of the tangent bundle $TM$ is a right inverse of the projection $\pi : TM \to M$. We say that a section is smooth if it is smooth relative to the smooth structures on $M$ and $TM$.
\end{definition}
So if $X : M \to TM$ is a section of $TM$, then $\pi(X(p)) = p$, implying that $X(p) \in T_pM$ for each $p \in M$. This is the property we want. With this language, a \emph{smooth vector field} on $M$ is a smooth section of the tangent bundle.

\begin{proposition}
Let $X, Y$ be smooth sections of $TM$. Then
\begin{enumerate}[(i)]
\item
Define $X + Y : M \to TM$ by $(X+Y)(p) := X(p) + Y(p)$. Then $X+Y$ is another smooth section on $TM$.
\item
For $f \in C^\infty(M)$, define $fX : M \to TM$ by $(fX)(p) = f(p) \cdot X(p)$. Then $fX$ is another smooth section on $TM$.
\end{enumerate}
\end{proposition}
The above proposition states that if $\Gamma(TM)$ denotes the set of all smooth sections on $TM$, then $\Gamma(TM)$ is both a real vector space and a module over the ring $C^\infty(M)$ of smooth functions on $M$. (A module can be thought of as taking a vector space and replacing the base field with a commutative ring with unity, but here's a precise definition anyway.)
\begin{definition}
Let $R$ be a ring with unity. A left-R module consists of an abelian group $(M,+)$ and an operation $\cdot : R \times M \to M$ such that
\begin{enumerate}
\item $r \cdot (x+y) = r\cdot x + r\cdot y$ 
\item $(r+s) \cdot x = r \cdot x + r \cdot y$
\item $(rs) \cdot x = r \cdot(s \cdot x)$ 
\item $1 \cdot x = x$
\end{enumerate}
(Long story short, we swap the scalars in a vector space with the elements of a ring, which obey the same laws as those scalars. We lose out on being able to invert those scalars.)
\end{definition}
\begin{definition}
A derivation $D$ on an algebra $A$ over $\R$ is a linear map $D : A \to A$ satisfying the Leibnitz rule.
\end{definition}
\begin{proposition}
Let $M$ be a smooth manifold. The set
\[
\mathrm{Der} = \{ X : C^\infty(M) \to C^\infty(M) : \text{X is a derivation }\}
\]
is a module over $C^\infty(M)$. Moreover, the map $\Phi : \Gamma(TM) \to \mathrm{Der}$ defined by $\Phi(X)(f) = X(f)$ is a module isomorphism.
\end{proposition}
The map $\Phi$ in the above proposition is similar to the old map $T_p\R^n \to \mathcal{D}_p$, $v \mapsto D_v$, which was also an isomorphism. 

Suppose $X$ is a smooth section of $TM$. Then we can define, by abuse of notation, a map $X : C^\infty(M) \to C^\infty(M)$ by $X(f)(p) = X(p)(f)$. One of the problems on Homework 3 gives the following proposition (whose proof, for obvious reasons, shall not be given).
\begin{proposition}
Let $X$ be a section of $TM$. Then $X$ is smooth if and only if for each $f \in C^\infty(M)$, $X(f) \in C^\infty(M)$ as defined above.
\end{proposition}

\newpage

\section{Bump Functions, Partitions of Unity (June 9)}

\subsection{Bump Functions}

Hereafter, $M$ denotes a smooth manifold. We will present two of the fundamental tools in manifold theory: the bump function, and the partition of unity. They allow "local phenomena" to be translated to global phenomena. For example, the integration of a differential form on a manifold is first defined locally, and then extended to the entire manifold using a partition of unity.

Bump functions allow us to, in particular, extend functions to an entire manifold. We will not (in lecture) cover the details of the construction of bump functions or of partitions of unity.

\begin{theorem}
(Existence of bump functions) Let $q \in M$ and let $U$ be an open neighbourhood of $q$. There exists a $\rho \in C^\infty(M)$ such that $\supp(\rho) \subseteq U$ and $\rho|_{\tilde{U}} \equiv 1$ on a neighbourhood $\tilde{U} \subseteq U$ of $q$. 
\end{theorem}

\begin{corollary}
($C^\infty$ extension lemma for a point) Let $U$ be an open neighbourhood of a point $p \in M$ and suppose $f \in C^\infty(U)$. Then there exists an $\tilde{f} \in C^\infty(M)$ and an open neighbourhood $\tilde{U} \subseteq U$ of $p$ such that $\tilde{f}|_{\tilde{U}} = f |_{\tilde{U}}$. 
\end{corollary}
\begin{proof}
Choose a $\rho \in C^\infty(M)$ such that $\supp(\rho) \subseteq U$ and $\rho|_{\tilde{U}} \equiv 1$ on a neighbourhood $\tilde{U} \subseteq U$ of $q$. Define
\[
\tilde{f}(x) = \begin{cases} 
\rho(x)f(x), & x \in U \\
0, & x \not\in U
\end{cases}.
\]
The function $\tilde{f}$ is $C^\infty$ on $U$ because it is a product of $C^\infty$ functions on $U$. If $x \not\in U$, then in particular $x \not\in \supp(\rho)$, so we can find a neighbourhood $V$ of $p$ such that $\tilde{f}|_V \equiv 0$. That is, $\tilde{f}$ is also $C^\infty$ on $M \setminus U$, and clearly it is an extension since $\rho|_{\tilde{U}} \equiv 1$. Therefore the function $\tilde{f}$ is the desired extension.
\end{proof}

\begin{corollary}
Let $F : N \to M$ be a continuous map of manifolds. Then $F$ is $C^\infty$ if and only if $F^*(C^\infty(M)) \subseteq C^\infty(N)$. (That is, if and only if $F$ pulls back $C^\infty$ functions to $C^\infty$ functions.)
\end{corollary}
\begin{proof}
Suppose that $F^*(C^\infty(M)) \subseteq C^\infty(N)$. Let $(V, \psi) = (V, y^1, \dots, y^m)$ be a coordinate chart for $M$ intersecting the image of $F$. We wish to show that $y^i \circ F = F^*(y^i) \in C^\infty(V)$. While the coordinate function $y^i$ is $C^\infty$, it is merely a member of $C^\infty(V)$ and not $C^\infty(M)$. This is where extensions come in. There is, given $F(p) \in V$, a $\tilde{y^i} \in C^\infty(M)$ agreeing with $y^i$ on some open neighbourhood $\tilde{V} \subseteq V$ of $F(p)$. Since $\tilde{y^i} \in C^\infty(M)$, we have $\tilde{y^i} \circ F \in C^\infty(M)$. Since this function agrees with $y^i \circ F$ on $F^{-1}(\tilde{V})$, we have that $F$ is $C^\infty$ at $p$. Since $p \in N$ was arbitrary, $F$ is $C^\infty$. The other direction is obvious.
\end{proof}

\subsection{Partitions of Unity}

\begin{definition}
A $C^\infty$ partition of unity is a collection of nonnegative $C^\infty$ functions $\{\rho_\alpha\}_{\alpha \in A}$ such that
\begin{enumerate}[(i)]
\item The collection $\{\supp(\rho_\alpha)\}_{\alpha \in A}$ is locally finite.
\item $\sum \rho_\alpha \equiv 1$. (Hence the name.)
\end{enumerate}
Note that the second condition makes sense because at each point, the sum is finite. If $\{U_\alpha\}_{\alpha \in A}$ is an open cover of $M$, we say that $\{\rho_\alpha\}_{\alpha \in A}$ is subordinate to $\{U_\alpha\}_{\alpha \in A}$ if $\supp(\rho_\alpha) \subseteq U_\alpha$ for each $\alpha \in A$.
\end{definition}

\begin{theorem}
(Existence of partitions of unity) Let $\{U_\alpha\}_{\alpha \in A}$ be an open cover of $M$. 
\begin{enumerate}[(i)]
\item
There is a $C^\infty$ partition of unity $\{\phi_k\}_{k=1}^\infty$ which is compactly supported such that for each $k$, there is some $\alpha \in A$ such that $\supp(\phi_k) \subseteq U_\alpha$.
\item 
There is a $C^\infty$ partition of unity $\{\rho_\alpha\}_{\alpha \in A}$ subordinate to $\{U_\alpha\}_{\alpha \in A}$.
\end{enumerate}
\end{theorem}

\subsection{Applications}

\begin{corollary}
Let $A \subseteq M$ be closed and let $U$ be an open neighbourhood of $A$. Then there exists an $f \in C^\infty(M)$ such that $f|_A \equiv 1$ and $\supp(f) \subseteq U$.
\end{corollary}
\begin{proof}
Consider the open cover $\{ U, M \setminus A \}$ of $M$. By (ii) of the existence theorem of partitions of unity, we can find a partition of unity $\{ \rho_1, \rho_2 \}$ which is subordinate to $\{U,M\setminus A\}$; say, $\supp(\rho_1) \subseteq U$ and $\supp(\rho_2) \subseteq M \setminus A$. Since $\rho_2 \equiv 0$ on $A$. $\rho_1 \equiv 1$ on $A$, so we may take $f = \rho_1$.
\end{proof}

In the preceding corollary, we call such a function $f$ a \emph{bump function for $A$ supported in $U$}. 

We can use partitions of unity to discuss smooth functions on arbitrary subsets of manifolds. First, we define what it means for a function to be smooth on an arbitrary subset of a manifold. Then we see that, at least on closed sets, such smooth functions can be extended to the entire manifold.

\begin{definition}
Let $A \subseteq M$ be a subset of a smooth manifold and let $f : A \to \R$ be a function. If $p \in A$, we say that $f$ is $C^\infty$ at $p$ if there is an open neighbourhood $W_p$ of $p$ in $M$ and an $\tilde{f} \in C^\infty(W_p)$ such that $\tilde{f}|_{W_p\cap A} = f|_{W_p\cap A}$. We say that $f$ is $C^\infty$ on $A$ if this condition holds for all $p \in A$.
\end{definition}

\begin{theorem}
($C^\infty$ extension lemma for a closed set) Let $A \subseteq M$ be closed and $f : A \to \R$ be a $C^\infty$ function on $A$ as just defined. If $U$ is any open neighbourhood of $A$ in $M$, then $f$ extends to a $C^\infty$ function on $M$ which is supported in $U$.
\end{theorem}

\begin{proof}
For each $q \in A$ there is an open neighbourhood $W_q \subseteq U$ such that $f$ admits an extension $f_q \in C^\infty(W_q)$ agreeing with $f$ on $W_q \cap A$. Then the set $\{W_q : q \in A\} \cup \{ M \setminus A \}$ is an open covering of $M$. Choose a partition of unity $\{\rho_q : q \in A\} \cup \{\rho_0\}$ such that $\supp(\rho_q) \subseteq W_q$ and $\supp(\rho_0) \subseteq M \setminus A$. 

The product function $\rho_q \cdot f_q$ is only defined on $W_q$, but we can extend it to all of $M$ smoothly by declaring it to be zero outside of $W_q$. This extension is well-defined because $\rho_q \cdot f_q$ is identically zero on the overlap $W_q \setminus \supp(\rho_q)$. It is $C^\infty$ because on $W_q$ it is a product of smooth functions, and outside $W_q$ it is identically zero in a neighbourhood of each point (by the definition of support). We will abuse notation and hereafter let $\rho_q \cdot f_q$ denote this smooth extension.

Define $\tilde{f} : M \to \R$ by
\[
\tilde{f} = \sum_{q \in A} \rho_q \cdot f_q.
\]
This sum is well-defined by the local finiteness condition; at each point of $M$ all but finitely many of the $\rho_q$'s are zero, and so the sum is finite in a neighbourhood of every point. It is $C^\infty$ for the same reason. Note that since $\supp(\rho_0) \subseteq M \setminus A$, the function $\rho_0$ is identically $0$ on $A$, and so all of the other functions sum to $1$ on $A$. Therefore, if $x \in A$, then each $f_q(x) = f(x)$ whenever it is defined, and if it is not we have $\rho_p(x) = 0$, so
\[
\tilde{f}(x) = \sum_{q \in A} \rho_q(x) \cdot f(x) = \left(\sum_{q \in A} \rho_q(x)\right)f(x) = 1 \cdot f(x) = f(x).
\]
So $\tilde{f}$ is actually an extension of $f$. Finally, 
\begin{alignat*}{2}
\supp(\tilde{f}) &\subseteq \overline{\bigcup_{q \in A} \supp(\rho_q \cdot f_q)}\\
&= \bigcup_{q \in A} \overline{\supp(\rho_q \cdot f_q)}  \qquad &&\text{local finiteness} \\
&= \bigcup_{q \in A}\supp(\rho_q \cdot f_q) \qquad &&\text{supports are closed} \\
&\subseteq \bigcup_{q \in A}W_q \qquad &&\text{subordinate assumption} \\
&\subseteq U, \qquad &&\text{each $W_q \subseteq U$ assumption} 
\end{alignat*}
where the last inclusion follows from the assumption that each $W_q$ was contained in $U$. Therefore $\tilde{f}$ is a smooth extension of $f$ to the entire manifold supported in $U$.
\end{proof}

Of course, we can ask the same questions for submanifolds.

% mistake in the statement?
\begin{theorem}
\begin{enumerate}
\item
Let $S \subseteq M$ be a submanifold. Then $f \in C^\infty(S)$ if and only if $f$ is $C^\infty$ as a function on the subset $S$ of $M$, as defined earlier.

\item
Let $S \subseteq M$ be a smooth manifold and $f : S \to \R$ a function. If it is true that $f \in C^\infty(S)$ if and only if if $f$ is $C^\infty$ as a function on the subset $S$ of $M$, as defined earlier, then $S$ is a submanifold of $M$.
\end{enumerate}

\end{theorem}
\begin{corollary}
If $S \subseteq M$ is a closed submanifold, then any smooth function $f : S \to \R$ can be extended smoothly to all of $M$.
\end{corollary}
\begin{proof}
$S$ is a closed subset of $M$ on which, by (1) of the preceding theorem, $f$ is $C^\infty$ according to the definition given earlier. By the $C^\infty$ extension lemma on a closed set, we may extend $f$ smoothly to all of $M$.
\end{proof}

\subsection{Whitney Embedding Theorem, Easy Case (Incomplete)}

We now have the machinery necessary for stating and proving a weak case of the Whitney Embedding Theorem in the case that the manifold is compact.

\begin{theorem}
Every smooth compact $n$-manifold may be embedded in $\R^{N}$, for some $N$.
\end{theorem}
\begin{proof}
See Lee's smooth manifolds book, page 134, theorem 6.15.
\end{proof}

\newpage

\section{Bump Functions and Partitions of Unity (Additional Reading, Incomplete)}

\subsection{Bump Functions}

Define $f : \R \to \R$ by
\[
f(t) = \begin{cases} 
e^{-1/t} & t > 0 \\
0 & t \leq 0
\end{cases}.
\]
This will be the basic function off of which our bump functions will be modelled. The construction of bump functions on manifolds proceeds in four steps. 
\begin{lemma}
The function $f$ defined above is $C^\infty$.
\end{lemma}
\begin{proof}
We claim that for $t > 0$ and $k \geq 0$, there is a polynomial $p_{2k}$ of degree $2k$ such that $f^{(k)}(t) = p_{2k}(1/t)e^{-1/t}$. For $k = 0$ this is obvious, so suppose that this holds true for some $k \geq 0$. Then we have
\begin{align*}
f^{(k+1)}(t) &= \frac{d}{dt} p_{2k} \left( \frac{1}{t} \right) e^{-1/t} \\
&= -\frac{1}{t^2} p_{2k}' \left( \frac{1}{t} \right) e^{-1/t} + \frac{1}{t^2}p_{2k} \left( \frac{1}{t} \right)e^{-1/t} \\
&= \underbrace{\left[ -\frac{1}{t^2} p_{2k}' \left( \frac{1}{t} \right) + \frac{1}{t^2}p_{2k} \left( \frac{1}{t} \right) \right]}_{p_{2(k+1)}(1/t)} e^{-1/t},
\end{align*}
so by induction the claim holds true.

(Finish this. Since $f$ is $C^\infty$ on $\R \setminus 0$, all we need to do is show that each $f^{(k)}(0)$ makes sense and is equal to $0$, by induction.)
\end{proof}

\begin{lemma}
Given real numbers $r_1 < r_2$, there is a $C^\infty$ function $h : \R \to \R$ such that $h^{-1}(1) = (-\infty, r_1]$, $h^{-1}(0) = [r_2, \infty)$, and $0 < h(t) < 1$ for $t \in (r_1, r_2)$.
\end{lemma}
\begin{proof}
Taking $f$ as defined above, define
\[
h(t) = \frac{f(r_2 - t)}{f(r_2 - t) + f(t - r_1)}.
\]
This function is well-defined because $f(r_2 - t) + f(t - r_1) = 0$ if and only if each is zero, which is true if and only if $t \geq r_2$ and $t \leq r_1$, which is clearly impossible. Then $h$ is $C^\infty$. It is clear that $0 < h(t) < 1$ for $t \in (r_1, r_2)$, so we are left with checking the other two conditions.

$h(t) = 0$ if and only if $f(r_2 - t) = 0$, which holds if and only if $r_2 - t \leq 0$, which holds if and only if $t \geq r_2$. So $h^{-1}(0) = [r_2, \infty)$.

$h(t) = 1$ if and only if $f(t - r_1) = 0$, which holds if and only if $t - r_1 \leq 0$, which holds if and only if $t \leq r_1$. So $h^{-1}(1) = (\infty, r_1]$.
\end{proof}

\begin{lemma}
Given real numbers $0 < r_1 < r_2$, there is a $C^\infty$ function $H : \R^n \to \R$ such that $H^{-1}(1) = \overline{B_{r_1}(0)}$, $H^{-1}(0) = \R^n \setminus B_{r_2}(0)$, and $0 < H(x) < 1$ for $x \in B_{r_2}(0) \setminus \overline{B_{r_1}(0)}$.
\end{lemma}
\begin{proof}
With $h$ as in the previous lemma, define $H(x) = h(\|x\|)$. The function $H$ is $C^\infty$ because when $\|x\| < r_1$, it is identically $1$, and it is a composition of $C^\infty$ maps away from the origin. The rest of the lemma is clear.
\end{proof}

Using this, we now work in a coordinate chart to get a bump function on a manifold.

\begin{theorem}
(Existence of bump functions)
Given a smooth manifold $M$, a point $q \in M$, and a neighbourhood $U$ of $q$, there exists a $\rho \in C^\infty(M)$ such that $\rho |_V \equiv 1$ on some neighbourhood $V \subseteq U$ of $q$, and $\supp(\rho) \subseteq U$.
\end{theorem}
\begin{proof}
Choose a coordinate chart $(W, \phi)$ at $q$ such that $\phi(q) = 0$. The set $\phi(W \cap U)$ is an open neighbourhood of the origin, so we can find $0 < r_1 < r_2$ such that 
\[
0 = \phi(q) \in B_{r_1}(0)  \subset B_{r_2}(0) \subset \phi(W \cap U).
\]
This implies that
\[
q \in \phi^{-1}(B_{r_1}(0)) \subset \phi^{-1}(B_{r_2}(0)) \subset W \cap U \subseteq U
.\]
Let $H$ be as in the previous lemma. Define 
\[
\rho(x) = \begin{cases} 
(H \circ \phi)(x) & x \in U \cap W \\
0 & x \not\in U \cap W
\end{cases}.
\]
The function $\rho$ is $C^\infty$ on $U \cap W$ because it is a composition of $C^\infty$ functions on an open set. If $x \not\in U \cap W$, then $x \not\in \phi^{-1}(\overline{B_{r_2}(0)})$, so we can find a neighbourhood of $x$ on which $\rho$ is identically zero. (Finish this.)
\end{proof}

\newpage

%this section needs work, kind of rushed
\section{Vector Fields, Integral Flows, and the Lie Derivative (June 11)}

\subsection{Smoothness Criteria for Vector Fields}

We will discuss vector fields in a little more detail and provide a few criteria for a vector field to be $C^\infty$. Note that given a coordinate chart $(U, x^1, \dots, x^n)$ on a manifold $M$, there are functions $a^1, \dots, a^n : U \to \R$ such that for every $p \in U$,
\[
X_p = \sum_i a^i \left. \frac{\pd}{\pd x^i} \right|_p.
\]
We will call the functions $a^1, \dots, a^n$ the components of $X$ in the chart. The first criterion is the one you would expect.

\begin{theorem}
(Smoothness in terms of components) Let $X$ be a section of $TM$. Then $X$ is $C^\infty$ if and only if for every coordinate chart $(U, x^1, \dots, x^n)$ on $M$, the component functions of $X$ in the chart are $C^\infty$ on $U$.
\end{theorem}
\begin{proof}
A coordinate chart $(U, \phi) = (U, x^1, \dots, x^n)$ induces a chart $(TU, \tilde{\phi})$ on $TM$. In these coordinates, if $p \in U$ and if $a^1, \dots, a^n$ are the components of $X$ in $U$, then
\[
\tilde{\phi} \circ X : p \mapsto (x^1(p), \dots, x^n(p), a^1(p), \dots, a^n(p)),
\]
in which it is clear that $X$ is $C^\infty$ if and only if each $a^i$ is $C^\infty$ on $U$.
\end{proof}

We can also think of vector fields as derivations. Let $s : M \to TM$ be a section. If $f \in C^\infty(M)$, define $D_s(f) : M \to \R$ by $D_s(f)(p) := s_p([f])$. It is easy to see that $D_s$ is a derivation on $C^\infty(M)$. With this we may present our second smoothness criterion.

\begin{theorem}
(Smoothness in terms of action on functions as a derivation) Let $s : M \to TM$ be a section. Then $s$ is $C^\infty$ if and only if for every $f \in C^\infty(M)$, the function $D_s(f)$ is $C^\infty$.
\end{theorem}
\begin{proof}
Suppose $s$ is $C^\infty$. Let $(U, x^1, \dots, x^n)$ be a coordinate chart on $M$. If $a^1, \dots, a^n$ are the components of $s$ in this chart, then on $U$ we have
\[
D_s(f) = \sum_i a^i \frac{\pd f}{\pd x^i},
\]
which is certainly $C^\infty$ on $U$. Since this is true for all charts, we have that $D_s(f) \in C^\infty(M)$.

The converse is a simple homework exercise which is done by extending the coordinate functions in any given chart.
\end{proof}

We would like to explore the link between vector fields and derivations some more. Recall our notation: $\Gamma(TM)$ for the smooth sections on $TM$, and $\mathrm{Der}(C^\infty(M))$ for the derivations on $C^\infty(M)$. These sets are both real vector spaces and $C^\infty(M)$-modules. It turns out that our association of a smooth section with a derivation is an isomorphism with respect to both of these structures.

\begin{theorem}
(Smooth sections $\cong$ Derivations) The map $\Phi : \Gamma(TM) \to \mathrm{Der}(C^\infty(M))$ defined by $\Phi(s) = D_s$ is an isomorphism of vector spaces and of modules.
\end{theorem}
\begin{proof}
Checking that $\Phi$ is a homomorphism (with respect to both structures) and injective is a homework exercise. As is the case with vector spaces (and in extreme similarity modules), to be a bijective homomorphism is sufficient for being an isomorphism. We shall only check surjectivity.

Suppose $D \in \mathrm{Der}(C^\infty(M))$. For $p \in M$ let us define $D_p : C_p^\infty(M) \to \R$ by $D_p([f]) = D(\tilde{f})(p)$, where $\tilde{f}$ is a smooth extension of $f$ to all of $M$. (We know we can do this with bump functions.) It is clear that this doesn't depend on the representative of $f$, so we only need to check that it also doesn't depend on the extension of $f$. 

Choose a representative $f : U \to \R$ of the germ. Let $\tilde{f}_1, \tilde{f}_2 \in C^\infty(M)$ be any two extensions of $f$ which both agree with $f$ on some open neighbourhood $V \subseteq U$ of $p$. Let $\rho$ be a bump function at $p$ supported in $V$. Then $\rho \cdot (\tilde{f}_1 - \tilde{f}_2) \equiv 0$ on $M$, so by linearity we have $D(\rho \cdot (\tilde{f}_1 - \tilde{f}_2)) = 0$. By the Leibnitz rule,
\[
0 = D(\rho )(\tilde{f}_1 - \tilde{f}_2) + \rho D(\tilde{f}_1 - \tilde{f}_2).
\]
Evaluating at $p$ gives $D(\tilde{f}_1 - \tilde{f}_2) = 0$, implying that $D(\tilde{f}_1)(p) = D(\tilde{f}_2)(p)$ by linearity. So the function $D_p$ is well defined. 

It is not too hard to see that $D_p \in T_pM$. Define $s : M \to TM$ by $p \mapsto D_p$. Since $D_p \in T_pM$ for each $p \in M$, the map $s$ is a section. We have
\[
D_s(f)(p) = s_p([f]) = D_p([f]) = D(f)(p),
\]
so $D_s = D$. It remains to check that $s$ is a smooth section of $TM$.

Let $(U, \phi) = (U, x^1, \dots, x^n)$ be a chart on $M$ at $p$. Then we have component functions $a^1, \dots, a^n$ of $s$ on $U$. We must first extend the coordinate functions to all of $M$ to use the fact $D_s(x^j) = a^j$. Extend $x^j$ to
$\tilde{x^j} \in C^\infty(M)$ agreeing with $x^j$ on a neighbourhood $\tilde{U}$ of $p$. Then, on $\tilde{U}$,
\[
D_s(\tilde{x^j}) = \left( \sum_i a^i \frac{\pd}{\pd x^i}\right)(\tilde{x^j}) = \sum_i a^i \frac{\pd x^j}{\pd x^i} = a^j,
\] 
so each $a^j$ is $C^\infty$ on $\tilde{U}$. Therefore $s$ is a smooth section, so we can conclude that $\Phi(s)$ makes sense and equals $D$. So $\Phi$ is surjective.
\end{proof}
\begin{corollary}
A section $X$ is $C^\infty$ if and only if $X(f) \in C^\infty(M)$ for every $f \in C^\infty(M)$.
\end{corollary}
We shall hereafter denote by $\mathfrak{X}(M)$ the set of smooth vector fields on $M$. 

\subsection{Integral Flows}

We shall begin the study of ordinary differential equations on manifolds. Everything that happens in $\R^n$ locally should also happen on manifolds, since manifolds are locally modelled by patches of $\R^n$. Therefore it is reasonable to try and generalize differential equations to manifolds. We begin with some definitions.
\begin{definition}
Let $X \in \mathfrak{X}(M)$. An integral curve of $X$ is a $C^\infty$ curve $c : (a,b) \to M$ such that for each $t$, $c'(t) = X_{c(t)}$. We say that the curve starts at $p$ if $c(0) = p$, and we say that it is maximal if its domain may not be extended.
\end{definition}
Let $(U, x^1, \dots, x^n)$ be a chart at $p$. Suppose $c : (a,b) \to M$ is an integral curve for $X$ starting at $p$. Then, in $U$, if $a^1, \dots, a^n$ are the components of $X$, we have
\[
c'(t) = X_{c(t)} = \sum_i (a^i \circ c)(t) \left. \frac{\pd }{\pd x^i} \right|_{c(t)}.
\]
If $\dot{c}^i(t)$ denotes the ordinary calculus derivative of the function $x^i \circ c$ at $t$, then we can write
\[
c'(t) = \sum_i \dot{c}^i(t) \left. \frac{\pd}{\pd x^i} \right|_{c(t)}.
\]
Therefore we have a system of ODEs
\begin{align*}
(x^1 \circ c)'(t) &= (a^1 \circ c)(t) \\
&\vdots \\
(x^n \circ c)'(t) &= (a^n \circ c)(t) \\
c(0) &= p
\end{align*}
Let us now recall some theorems about ODEs.
\begin{theorem}
(Existence and Uniqueness) Let $V \subseteq \R^n$ be open and $f : V \to \R^n$ a $C^\infty$ function. Then the differential equation
\[
\begin{cases} 
\frac{dy}{dt} = f(y) \\
y(0) = p_0
\end{cases}
\]
has a unique $C^\infty$ solution $y : (a(p_0), b(p_0)) \to V$ defined on a maximal open interval.
\end{theorem}
Note that the function $f$ here is basically a vector field. The corresponding section is $V \to TV$, $x \mapsto (x,f(x))$, where we identify $TV$ with $V \times \R^n$. We do not have the luxury of doing this on manifolds. The uniqueness condition simply means that if $z : (-\epsilon_1, \epsilon_2) \to V$ is another solution, then $z$ and $y$ agree on the interval of existence of $z$; the maximality condition ensures that this interval of existence is no larger than that of $y$.

A direct corollary of the existence and uniqueness theorem for ODEs in $\R^n$ is the following:
\begin{corollary}
If $U \subseteq M$ is a coordinate neighbourhood and $p \in U$ and $X \in \mathfrak{X}(U)$, then there is a unique maximal integral curve of $X$ in $U$ starting at $p$.
\end{corollary}

It is natural to ask what happens when we let the initial point vary. We would expect that if the time is fixed and we vary the initial point, the result varies smoothly. This result is, in fact, true.

\begin{theorem}
(Smooth dependence on initial conditions) Let $V \subseteq \R^n$ be open and $f : V \to \R^n$ be $C^\infty$. For each $p_0 \in V$ there is an open neighbourhood $W \subseteq V$ of $p$, an $\epsilon > 0$, and a $C^\infty$ function $y : (-\epsilon, \epsilon) \times W \to V$ such that
\[
\frac{\pd y}{\pd t}(t, q) = f(y(t,q))
\]
and $y(0,q) = q$ for all $(t,q) \in (-\epsilon,\epsilon) \times W$.
\end{theorem}

It follows that if $U$ is a coordinate neighbourhood in $M$ and $X \in \mathfrak{U}$
, then for any $p \in U$ there is an open neighbourhood $W$ of $p$ in $U$, an $\epsilon > 0$, and a $C^\infty$ map $F : (-\epsilon, \epsilon) \times W \to U$ such that for each $q \in W$, the curve $F(t, q)$ is an integral curve of $X$ in $U$ starting at $q$. We will sometimes write $F_t(q)$ to mean $F(t,q)$.

We would like it to be true that $F_t(F_s(q)) = F_{t+s}(q)$, whenever these make sense. If we visualize a curve and a point $q$ on the curve, then this means that moving for $t+s$ time units is the same as moving for $s$ times units and then moving for $t$ time units. Fortunately, this is true; it is a very simple consequence of uniqueness. If $s$ is fixed, then both $F_t(F_s(q))$ and $F_{t+s}(q)$ are integral curves of $X$ starting at $F_s(q)$.

We now make many more definitions.
\begin{definition}
The map $F$ above is called the local flow generated by $X$. The curve $t \mapsto F_t(q)$ is called the flow line. If $F$ is defined on $\R \times M$, then $F$ is called a global flow. A vector field that admits a global flow is said to be complete.
\end{definition}
Not every vector field admits a global flow, as we know from our ODEs class. For example, the ODE $\frac{dx}{dt} = x^2$ with initial condition $x(0) = x_0 \in \R$ with $x_0 \neq 0$ has solution $x(t) = \frac{x_0}{1-tx_0}$, which does not exist everywhere.

If $F$ is a global flow, then for every $t \in \R$ we have $F_t^{-1} = F_{-t}$, which is easy to check. We therefore have a diffeomorphism $F_t : M \to M$ for each $t \in \R$, which can be thought of as sending every point to its position after "flowing for $t$ time units". We generalize this slightly.
\begin{definition}
Let $\mathrm{Diff(M)}$ denote the group of all homomorphisms of a smooth manifold $M$ under composition. A group homomorphism $G : \R \to \mathrm{Diff}(M)$ is called a one-parameter group of diffeomorphisms of $M$.
\end{definition}
So, of course, a global flow is an example of a one-parameter group of diffeomorphisms of a manifold. 

We shall now define what it means to be a local flow independently of any vector field.

\begin{definition}
A local flow about $p$ in an open set $U$ of a manifold is a $C^\infty$ map $F : (-\epsilon, \epsilon) \times W \to V$, where $\epsilon > 0$ and $W \subseteq U$ is an open neighbourhood of $p$, such that
\begin{enumerate}[(i)]
\item
$F_0(q) = q$ for all $q \in W$.
\item
$F_t(F_s(q)) = F_{t+s}(q)$ whenever both sides are defined.
\end{enumerate}
\end{definition}

If we have a local flow $F(t,q)$ as defined above then we may recover the vector field $X$ of which $F$ is a local flow by observing that
\[
F(0,q) = q \qquad \text{ and } \qquad \frac{\pd F}{\pd t}(0, q) = X_{F(0,q)} = X_q.
\]

Let us consider an example. The function $F : \R \times \R^2 \to \R^2$ defined by
\[
F \left( t, \begin{bmatrix}
x \\ y
\end{bmatrix} \right) := \begin{bmatrix}
\cos t & -\sin t \\
\sin t & \cos t
\end{bmatrix}\begin{bmatrix}
x \\ y
\end{bmatrix}
\]
is the global flow on $\R^2$ generated by the vector field 
\[
X = -y \frac{\pd}{\pd x} + x \frac{\pd}{\pd y},
\]
since
\[
\frac{\pd F}{\pd t}(t,(x,y)) =\begin{bmatrix}
0 & -1 \\ 1 & 0
\end{bmatrix}\begin{bmatrix}
x \\ y
\end{bmatrix}= \begin{bmatrix}
-y \\ x
\end{bmatrix} = -y \frac{\pd}{\pd x} + x \frac{\pd}{\pd y}.
\]
\subsection{The Lie Derivative}
We defined smooth functions on manifolds and gave a notion of the directional derivative using tangent vectors. Since we have defined smooth vector fields, can we develop a notion of the derivative of a vector field? Or can we develop a "directional derivative" of a vector field? This is what we attempt to do.

Let $X,Y$ be vector fields. In calculus we define the derivative of a real valued function as
\[
f'(p) = \lim_{t \to 0} \frac{f(p+t) - f(p)}{t},
\]
assuming it exists. We cannot readily generalize this to manifolds because for distinct nearby points $p,q$ in a manifold, the elements of $T_pM$ and $T_qM$ cannot be compared. We can get around this by using the local flow of another vector field $X$ to "transport" $Y_q \in T_qM$ to $T_pM$. 

Let $F$ be the local flow of $X \in \mathfrak{X}(M)$ starting at $p$. Because of the identity $F_t \circ F_s = F_{t+s}$ when they make sense, every $F_t$ is a diffeomorphism onto its image with inverse $F_{-t}$. The following definition therefore makes sense.

\begin{definition}
For $X,Y \in \mathfrak{X}(M)$ and $p \in M$, let $F$ be a local flow of $X$ on a neighbourhood of $p$. Define the Lie derivative of $Y$ with respect to $X$ at $p$ to be the vector
\[
(\mathcal{L}_XY)_p := \lim_{t \to 0} \frac{F_{-t *}(Y_{F_t(p)}) - Y_p}{t},
\]
if the limit exists.
\end{definition}
\begin{figure}[H]
\centering
\includegraphics[width = 0.8\textwidth]{lie_derivative_1.jpg}
\caption{Transporting $Y_{F_t(p)}$ along the flow to $T_pM$ to be compared with $Y_p$.}
\end{figure}

%pushforward of vector fields part may need elaboration
Actually, since each $F_{-t}$ is a diffeomorphism onto its image, we can push vector fields forward and rewrite this as
\[
(\mathcal{L}_XY)_p = \lim_{t \to 0} \frac{(F_{-t *} Y)_p - Y_p}{t}  = \left. \frac{d}{dt} \right|_{t=0} (F_{-t*}Y)_p,
\]
which shows that a sufficient condition for the Lie derivative $(\mathcal{L}_XY)_p$ to exist is that $\{ F_{-t*}Y \}$ be a smooth family of vector fields on $M$.

\begin{theorem}
If $X,Y \in \mathfrak{X}(M)$, then $\mathcal{L}_XY \in \mathfrak{X}(M)$.
\end{theorem}

\begin{figure}[H]
\centering
\includegraphics[width = 0.9\textwidth]{lie_derivative_2.jpg}
\caption{}
\end{figure}
Consider the figure above. Define 
\[
\gamma(t) = (F_t \circ G_t \circ F_{-t} \circ G_{-t})(p),
\]
which we may think of as taking the point $p$ and travelling clockwise around the "square". Do we end up back at the point $p$? The answer is, in general, "no"; however, we can talk about this using the Lie derivative. 

We can also think of this scenario in terms of pushing forward vectors. Considering the diagram, think of a tangent vector $v \in T_pM$. We can push it forward by $G_{-t}$ to get a tangent vector at the "bottom right corner" of the square. Then we can push that forward by $F_{-t}$ to get one at the "bottom left". And that by $G_{t}$. And that by $F_t$. Do we arrive at the same tangent vector $v$? Not always; this difference is something that the Lie derivative measures. (One affirmative case is the origin and the coordinate vector fields on $\R^2$, as one can check). We have a theorem.

\begin{theorem}
\begin{enumerate}
\item $\gamma'(0) = 0$ and $\frac{1}{2} \gamma''(0) = \mathcal{L}_XY|_p$.
\item If $\mathcal{L}_XY \equiv 0$, then $\gamma(t) = p$ for all $t$.
\end{enumerate}
\end{theorem}

Later on we will see that we have
\[
\mathcal{L}_XY = [X,Y] = XY - YX,
\]
so the Lie derivative isn't actually anything new.

\newpage

\section{Lie Groups and Algebras (Additional Reading, Incomplete)}

We will write the group operation for an arbitrary Lie group as multiplication, but in some cases (e.g. $\R^n$) we will use addition when necessary.

\subsection{Motivation and Definitions}

A Lie group is a manifold with a group structure such that the group operations of multiplication and inversion are smooth. We can think of Lie groups as the smooth analogue of topological groups; topological spaces with continuous group multiplication and inversion. A Lie group is a "homogeneous" space, in the sense that left-multiplication by a fixed element is a diffeomorphism of the group with itself and so, in some sense, the group is locally everywhere the same. More precisely, if $G$ is a Lie group, $g \in G$, and $\ell_g : G \to G$ is left-multiplication by $g$, then $\ell_g$ is a diffeomorphism of $G$ with itself taking the identity element $e$ to $g$. Therefore, in the study of Lie groups, it suffices to study the properties of the group at the identity. We now make these notions precise.

\begin{definition}
A Lie group is a smooth manifold $G$ equipped with a group structure such that multiplication
\[
\mu : G \times G \to G \qquad (g, h) \mapsto gh
\]
and inversion
\[
\i : G \to G \qquad g \mapsto g^{-1}
\]
are smooth.
\end{definition}
Since $\ell_g^{-1} = \ell_{g^{-1}}$, left-multiplication is a diffeomorphism of $G$ with itself, and similarly for right-multiplication. We have the following equivalent condition for being a Lie group.

\begin{proposition}
$G$ is a Lie group if and only if the map $k : G \times G \to G$ defined by $k(g,h) = gh^{-1}$ is smooth.
\end{proposition}
\begin{proof}
If $G$ is a Lie group, then $k = \mu \circ (\mathrm{id}_G, \i)$ is smooth. Conversely, if $k$ is smooth, then $\i = k \circ (e, \mathrm{id}_G)$ is smooth and $\mu = k \circ (\mathrm{id}_G, \i)$ is smooth, where $e$ denotes the identity element of $G$.
\end{proof}

\begin{definition}
A Lie subgroup of the Lie group $G$ is an immersed submanifold $H$ that is also a subgroup such that the group operations on $H$ are smooth.
\end{definition}

We must impose the condition that the group operations on $H$ are smooth; $H$ is merely an immersed submanifold, so it does not follow that the group operations on $H$ are smooth. If $H$ is embedded then this is the case, which is the content of the following theorem. First, recall the following technical lemma.

\begin{lemma}
Let $F : N \to M$ be a smooth map and let $S$ be a subset of $M$ containing $F(N)$. If $S$ is a regular submanifold of $M$, then the restriction of the codomain of $F$ to $S$ is smooth.
\end{lemma}
\begin{proof}
For convenience, denote by $\tilde{F} : N \to S$ the map obtained by restricting the codomain of $F$. Let $p \in N$, and suppose the dimensions of $N,M,S$ are $n,m,s$, respectively. Choose a slice chart $(V, \psi) = (V, y^1, \dots, y^m)$ on $M$ at $F(p)$ such that $V \cap S$ is defined by the vanishing of $y^{s+1},\dots,y^m$. Let $\psi_S = (y^1, \dots, y^s)$ be the induced coordinate chart on $S$ at $F(p)$. Choose an open neighbourhood $U$ of $p$ with $F(U) \subseteq V$. Then $F(U) \subseteq V \cap S$, so if $q \in U$ we have
\[
(\psi \circ F)(q) = (y^1(F(q)), \dots, y^s(F(q)), 0, \dots, 0),
\]
so on $U$ we have
\[
\psi_S \circ \tilde{F} = (y^1 \circ F, \dots, y^s \circ F).
\]
Thus $\tilde{F}$ is smooth on $U$. Since $p$ was arbitrary, $\tilde{F}$ is smooth.
\end{proof}

For a counterexample of the preceding lemma in the case of an immersed submanifold, consider a parametrization of the figure eight in $\R^2$.

\begin{theorem}
If $H$ is an abstract subgroup of the Lie group $G$ and is also an embedded submanifold of $G$, then $H$ is a Lie subgroup of $G$.
\end{theorem}
\begin{proof}
Embedded submanifolds are immersed submanifolds, so all we have to check is the smoothness of the operations on $H$. Let $i : H \hookrightarrow G$ be the inclusion map. Then multiplication on $H$ is given by $\mu \circ (i, i)$, where $\mu$ is the multiplication on $G$. The image of this map lies in the regular submanifold $H$, so restricting the codomain leaves us with the multiplication on $H$. This is smooth by the preceding lemma, since $H$ is an embedded submanifold. Similarly for the inversion map on $H$.
\end{proof}

We have the following important theorem, due to Cartan, which provides us with many examples of Lie groups.

\begin{theorem}
(Closed subgroup theorem) A closed subgroup of a Lie group is an embedded Lie subgroup.
\end{theorem}

\subsection{Examples of Lie Groups}

\begin{enumerate}
\item
$\R^n$ with addition.
\item
$\R^*$, the non-zero real numbers, with multiplication.
\item
The (direct) product of Lie groups is another Lie group.
\item
$\C \setminus \{0\}$ with complex multiplication, and its embedded Lie subgroup $S^1$. $\C \setminus \{0\}$ is a Lie group for obvious reasons, and $S^1$ is an embedded Lie subgroup by either the closed subgroup theorem, or the theorem above.
\item
$\mathrm{Mat}_{n \times n}(\R)$ with matrix multiplication, and similarly $\mathrm{Mat}_{n \times n}(\C)$. This can be proven by identifying with $\R^{n^2}$ and $\C^{n^2}$, respectively.
\item
$GL(n, \R)$ is an abstract subgroup of $\mathrm{Mat}_{n \times n}(\R)$ and an embedded submanifold, since it is an open set, so $GL(n,\R)$ is an embedded Lie subgroup of $G$. Similarly with $\R$ replaced by $\C$.
\item
The orthogonal group $O(n)$ and the special linear group $SL(n, \R)$ are embedded Lie subgroups of $GL(n, \R)$, since they are embedded submanifolds and abstract subgroups. The unitary group $U(n)$ and the complex special linear group $SL(n, \C)$ are embedded Lie subgroups of $GL(n, \C)$ for similar reasons.
\end{enumerate}

\subsection{The Differential of det at I}

To compute the differential of a map on a subgroup of $GL(n, \R)$, we need a curve of invertible matrices. Since $\det(e^X) = e^{\mathrm{Tr}(X)}$, the matrix exponential is useful for this purpose.

Consider the determinant $\det : GL(n, \R) \to \R$. It is a smooth map because it is given by a polynomial. After the usual identifications, its differential at the identity is a map $\det_{*, I} : \R^{n \times n} \to \R$.

\begin{theorem}
$\det_{*,I}(X) = \mathrm{Tr}(X)$.
\end{theorem}
\begin{proof}
Let $c(t) = e^{tX}$. Then $c$ is a smooth curve starting at $I$ with $c'(0) = X$. Therefore
\begin{align*}
\det_{*,I}(X) = \left. \frac{d}{dt} \right|_{t = 0} \det(e^{tX}) = \left. \frac{d}{dt} \right|_{t = 0} e^{t \mathrm{Tr}(X)} = \mathrm{Tr}(X). 
\end{align*}
\end{proof}

\subsection{Some Properties of Lie Groups}

It is possible to do the next exercise without invoking the closed subgroup theorem, but I would like to give an example of its application (due to the relative lack of content thus far).

\begin{theorem}
(Exercise 15.3, slight variation) The connected component of a Lie group $G$ containing the identity is an embedded Lie subgroup.
\end{theorem}
\begin{proof}
Let $G_0$ be the connected component of $G$ which contains $e$. Let $\mu : G \times G \to G$ be multiplication and $\i : G \to G$ be inversion. These are both continuous maps. Fix $x \in G_0$. The image $\mu(\{x\} \times G_0)$ is connected, and it intersects $G_0$ because $\mu(x,e) = x \in G_0$. (This is where $e \in G_0$ is used.) Therefore $\mu(\{x\} \times G_0) \subseteq G_0$ and similarly $\i(G_0) \subseteq G_0$. It follows that $G_0$ is a subgroup of $G$. 

Since $G$ is a manifold, it is locally connected and thus the connected components of $G$ are open. Then $G \setminus G_0$ is either empty or the union of the other connected components of $G$, so $G_0$ is closed. Then $G_0$ is a closed subgroup of $G$, so by the closed subgroup theorem $G_0$ is an embedded Lie subgroup of $G$. (To prove this without invoking the closed subgroup theorem, note that $G_0$ is open and so it is an embedded submanifold of $G$, so by the theorem directly following the statement of the closed subgroup theorem, it is an embedded Lie subgroup.)
\end{proof}
It is not hard to see that every connected component of a Lie group $G$ is of the form $gG_0$ for some $g \in G$.

The following property holds for the more general topological group, whose space is merely a topological space and whose group operations are continuous with respect to the group's topology.

\begin{theorem}
(Exercise 15.4) An open subgroup $H$ of a connected Lie group $G$ is equal to $G$.
\end{theorem}
\begin{proof}
$H$ is a subgroup, so it contains the identity $e$ of $G$ and is thus non-empty, so to show $H = G$ it suffices to show that $H$ is closed. If $g \in G \setminus H$, then $gH$ is an open neighbourhood of $g$ disjoint from $H$ (because the cosets partition the group). Therefore $G$ is nonempty, open, and closed, so we must have $H = G$.
\end{proof}

\subsection{The Tangent Space at I}
(Finish this and the following sections.)
\subsection{Left-Invariant Vector Fields}
\subsection{Lie Algebras}
\subsection{The Differential as a Lie Algebra Homomorphism}

\newpage

\section{More on Flows and Lie Derivatives (July 7)}

\subsection{The Fundamental Theorem of Flows}

Consider $X \in \mathfrak{X}(M)$. Let $(U, \phi) = (U, x^1, \dots, x^n)$ be a coordinate chart on $M$. The theory of ODEs in $\R^n$ applies almost exactly to the integral curves of $X|_U \in \mathfrak{X}(U)$, which follows from $U$ being diffeomorphic to an open subset of $\R^n$. In particular,
\begin{enumerate}
\item
Given $p \in U$, there is, by the existence and uniqueness theorems, a unique maximal integral curve of $X|_U$ starting at $p$.

\item
By "collecting" all of the maximal integral curves from the previous step, we get flows. More precisely, we define
\[
\mathcal{D} = \left\{ (t, p) \in \R \times M : t \text{ lies in the domain of the maximal integral curve of $X|_U$ starting at } p \right\}.
\]
Then, for $p \in U$, we let
\[
\mathcal{D}^{(p)} = \{ t \in \R : (t, p) \in D \};
\]
this is just the domain of the maximal integral curve of $X|_U$ starting at $p$. In particular, $0 \in \mathcal{D}^{(p)}$.

Now define $F : \mathcal{D} \to U$ by $F(t,p) = \gamma_p(t)$, where $\gamma_p : \mathcal{D}^{(p)} \to U$ is the maximal integral curve of $X|_U$ starting at $p$.

We are led to the following question: does there exist an interval $(-\epsilon, \epsilon)$ such that $(-\epsilon, \epsilon) \subseteq \mathcal{D}^{(p)}$ for all $p \in U$? If the answer is yes, then the vector field $X|_U$ is \emph{complete}; each of its maximal integral curves exists for all $t \in \R$. This is known as the "Uniform Time Lemma".

Moreover, the function $F$ is smooth, which follows immediately from the theorem of smooth dependence on initial conditions of ODEs in $\R^n$.

\item
Given $p \in U$, there is, by smooth dependence on initial conditions, a neighbourhood $W \subseteq U$ of $p$ such that the hypotheses of the uniform time lemma are satisfied on $W$. (?)
\end{enumerate}
All of the above was in a coordinate chart, and so we have seen nothing new. Everything followed directly from the theory of ODEs in $\R^n$. We must then ask if the above claims hold in general on $M$, not restricted to a single coordinate open set. The answer is yes, all of these claims hold when $U$ is replaced with $M$.

\begin{theorem}
(Fundamental Theorem of Flows) Suppose $X \in \mathfrak{X}(M)$. Then there exists a unique smooth maximal flow $F : \mathcal{D} \to M$, where $\mathcal{D} \subseteq \R \times M$, generated by $X$, satisfying
\begin{enumerate}[(a)]
\item
For each $p \in M$, the curve $\gamma_p(t) = F(t, p)$ is the unique maximal integral curve of $X$ starting at $p$.
\item
If $s \in \mathcal{D}^{(p)}$, then $\mathcal{D}^{(F(s,p))} = \mathcal{D}^{(p)} - s$.
\item
For each $t \in \R$, the set $M_t = \{p \in M : (t, p) \in \mathcal{D}\}$ is open in $M$, and $F_t : M_t \to M_{-t}$ is a diffeomorphism.
\end{enumerate}
\end{theorem}
\begin{proof}
The proof is left as an exercise (likely as one of the homework problems). It is also in Lee.
\end{proof}

\subsection{More on Lie Derivatives}

Suppose $X,Y \in \mathfrak{X}(M)$, and let $F$ be the flow of $X$. If $M = \R^n$, then we can compare $Y_{F_t(p)}$ and $Y_p$, since $T_p\R^n = \R^n$ for every $p \in \R^n$. We cannot, however, do this for abstract manifolds, since the elements of the distinct tangent spaces of $M$ cannot even be compared. We remedy this using the pushforward of $F_{-t}$.
\begin{definition}
The Lie derivative of $Y$ with respect to $X$ is the vector field $\mathcal{L}_XY$ defined by
\[
(\mathcal{L}_XY)_p = \lim_{t \to 0} \frac{(F_{-t})_{*,F_t(p)}(Y_{F_t(p)}) - Y_p}{t},
\]
when the limit exists.
\end{definition}
\begin{proposition}
The above limit exists for all $p \in M$, and $\mathcal{L}_XY \in \mathfrak{X}(M)$.
\end{proposition}
\begin{proof}
Exercise.
\end{proof}

Consider the question of whether or not the flows of $X$ and $Y$ commute. Let $G$ be the flow of $Y$. We then may ask ourselves: given $p \in M$, do we have $F_s(G_t(p)) = G_t(F_s(p))$?  Define a function $A$ mapping $(s,t)$ to $F_s(G_t(p)) - G_t(F_s(p))$. We have the following proposition:
\begin{proposition}
\[
\frac{\pd^2 A}{\pd s \pd t} = \frac{\pd^2 A}{\pd t \pd s} = \mathcal{L}_XY,
\]
and if $\mathcal{L}_XY = 0$, then $A \equiv 0$.
\end{proposition}
\begin{proof}
Exercise.
\end{proof}
So $\mathcal{L}_XY$ measures "how much the flows commute".

For another perspective, consider a smooth function $f \in C^\infty(M)$ and a vector field $X \in \mathfrak{X}(M)$. We can ask ourselves the question "how does $f$ change along the integral curve of $X$ starting at $p$?". If $F$ is the flow of $X$, then we would like to discuss
\[
\left. \frac{d}{dt} \right|_{t = 0} f \circ F_t(p)
\]
for $p \in M$. A straightforward application of the chain rule gives
\[
\left. \frac{d}{dt} \right|_{t = 0} f \circ F_t(p) = f_{*,p}(\gamma_p'(0)) = f_{*,p}(X_p) = (Xf)(p).
\]
We can also define $\left. \frac{d}{dt} \right|_{t = 0} f \circ F_t(p)$ as the first order term in the Taylor expansion of $f \circ F_t(p)$ at $0$:
\[
f \circ F_t(p) = f(p) + t(Xf)(p) + o(t).
\]
Let us apply the same idea in order to get the Lie derivative. Define $G$ to be the map
\[
G : t \mapsto (F_{-t})_{*,F_t(p)}(Y_{F_t(p)}) \in T_pM.
\]
The Taylor series expansion of $G$ at $0$ is
\[
G(t) = G(0) + tG'(0) + o(t) = Y_p + t(\mathcal{L}_XY)_p + o(t),
\]
as we will see later. Therefore we may actually define $(\mathcal{L}_XY)_p$ to be the first order term of the Taylor expansion of the map $G$ defined above.

We end up having three equivalent definitions of the Lie derivative $\mathcal{L}_XY$: given $p \in M$, define $(\mathcal{L}_XY)_p$ as
\begin{enumerate}
\item
the limit
\[
(\mathcal{L}_XY)_p := \lim_{t \to 0} \frac{(F_{-t})_{*,F_t(p)}(Y_{F_t(p)}) - Y_p}{t}.
\]
\item
the first-order term of the Taylor expansion of the map $G$ defined by
\[
G : t \mapsto (F_{-t})_{*,F_t(p)}(Y_{F_t(p)}).
\]
\item
the Lie bracket 
\[
[X,Y] = X_pY - Y_pX.
\]
\end{enumerate}

\newpage

\section{Lie Derivatives, Lie Algebras, and Frobenius' Theorem (July 10)}

(This lecture was pushed a day forward as the instructor could not make the usual lecture time on Thursday.)

\subsection{Equivalent Conditions for Zero Lie Derivative }

Suppose $X, Y \in \mathfrak{X}(M)$, and let $F$ denote the flow of $X$. Suppose $\mathcal{L}_XY$ is identically zero on $M$. Define a $C^\infty$ curve $H : \mathcal{D}^{(p)} \to T_pM$ by
\[
H : t \mapsto (F_{-t})_{*, F_t(p)}(Y_{F_t(p)}).
\]
Since $H$ is a smooth curve in a vector space, we can identify its derivative with an element of the vector space $T_pM$. We have
\begin{align*}
H'(t_0) &= \left. \frac{d}{dt} \right|_{t = t_0} (F_{-t})_{*, F_t(p)}(Y_{F_t(p)}) \\
&= \left. \frac{d}{dt} \right|_{t = t_0} (F_{-(t - t_0) - t_0})_{*, F_{t - t_0 + t_0}(p)}(Y_{F_{t - t_0 + t_0}(p)}) \\
&= \left. \frac{d}{dt} \right|_{t = 0} (F_{-t - t_0})_{*, F_{t + t_0}(p)}(Y_{F_{t + t_0}(p)}) \\
&= \left. \frac{d}{dt} \right|_{t = 0} (F_{t_0} \circ F_{-t})_{*,F_t(F_{t_0}(p))} (Y_{F_t(F_{t_0}(p))}) \\
&= \left. \frac{d}{dt} \right|_{t = 0}  (F_{t_0})_{*,F_{t_0}(p)} \left( (F_{-t})_{*,F_t(F_{t_0}(p))}(Y_{F_t(F_{t_0}(p))}) \right) \\
&= (F_{t_0})_{*,F_{t_0}(p)} \left. \frac{d}{dt} \right|_{t = 0}    (F_{-t})_{*,F_t(F_{t_0}(p))}(Y_{F_t(F_{t_0}(p))})  \\
&= (F_{t_0})_{*,F_{t_0}(p)} (\mathcal{L}_XY)_{F_{t_0}(p)} \\
&= 0.
\end{align*}
So $H$ is constant. Therefore $H(t) = Y_p$ for all $t \in \mathcal{D}^{(p)}$. Applying $(F_t)_{*, p}$ to both sides gives
\[
Y_{F_t(p)} = (F_t)_{*,p} (Y_p) \text{ for all } t \in \mathcal{D}^{(p)}.
\]
When this condition is satisfied, we say that $Y$ is \emph{invariant under the flow of $X$}. We have just shown the following proposition.

\begin{proposition}
If $\mathcal{L}_XY$ is identically zero on $M$, then $Y$ is invariant under the flow of $X$.
\end{proposition}
We will eventually show that the hypothesis $\mathcal{L}_XY \equiv 0$ also implies that $X$ is invariant under the flow of $Y$, which will be a corollary of the identity $\mathcal{L}_XY = - \mathcal{L}_YX$.

One way to imagine this situation is through curves. Suppose $\gamma(s)$ is a $C^\infty$ curve in $M$ starting at $p$ with $\gamma'(0) = Y_p$. Suppose the conclusion of the above proposition is satisfied. We have
\begin{align*}
(F_t \circ \gamma)'(0) = (F_t)_{*,p}(\gamma'(0)) = (F_t)_{*,p}(Y_p) = Y_{F_t(p)},
\end{align*}
the last step following from the previous proposition. In particular, choose $\gamma$ to be the unique integral curve of $Y$ starting at $p$. Then, for all $s$ at which $\gamma$ is defined,
\[
(F_t \circ \gamma)'(s) = (F_t)_{*,\gamma(s)}(\gamma'(s)) = (F_t)_{*,p}(Y_{\gamma(s)}) = Y_{(F_t \circ \gamma)(s)},
\]
the last step again following from the previous proposition. We have proven the following proposition.

\begin{proposition}
If $Y$ is invariant under the flow $F$ of $X$, then for each $t$, $F_t$ takes integral curves of $Y$ to integral curves of $Y$.
\end{proposition}

Now suppose the conclusion of this proposition holds. Let $F$ be the flow of $X \in \mathfrak{X}(M)$ and $G$ the flow of $Y \in \mathfrak{X}(M)$. Assume for simplicity that the vector fields are complete. Fix $p \in M$. For a fixed $t_0$, this conclusion implies that $F_{t_0}(G_s(p))$ is the integral curve of $Y$ starting at $F_{t_0}(p)$. But then $F_{t_0}(G_s(p)) = G_s(F_{t_0}(p))$ for all $s$ for which it makes sense. Since $t_0$ and $p$ were arbitrary, we have $F_t \circ G_s = G_s \circ F_t$ for all $s,t$ for which it makes sense. We say, in this case, that the flows of $X$ and $Y$ \emph{commute}. We have proven the following proposition.

\begin{proposition}
If $X$ and $Y$ are complete and if the flow of $X$ takes integral curves of $Y$ to integral curves of $Y$, then the flows of $X$ and $Y$ commute.
\end{proposition}

Note that each proposition implies the next one. It turns out that they are actually equivalent, which we now state and prove. The theorem holds without the assumption that $X$ and $Y$ are complete.

\begin{theorem}
(Equivalent conditions for zero Lie derivative, complete case) Let $X, Y \in \mathfrak{X}(M)$ be complete vector fields. The following are equivalent:
\begin{enumerate}
\item $\mathcal{L}_XY \equiv 0$.
\item $Y$ is invariant under the flow of $X$.
\item The flow of $X$ takes integral curves of $Y$ to integral curves of $Y$.
\item The flows of $X$ and $Y$ commute.
\end{enumerate}
\end{theorem}
\begin{proof}
We have shown that (1) $\implies$ (2) $\implies$ (3) $\implies$ (4). All we need to prove is that (4) implies (1), which is very easy. Let $F$ denote the flow of $X$ and $G$ the flow of $Y$. Given $p \in M$, we have
\[
F_t(G_s(p)) = G_s(F_t(p))
\]
for all $t,s$. Differentiating with respect to $s$ at $0$ gives
\[
(F_t)_{*,p}(Y_p) = Y_{F_t(p)},
\]
implying
\[
Y_p = (F_{-t})_{*,F_t(p)}(Y_{F_t(p)}).
\]
Since this holds for all $t$, $(\mathcal{L}_XY)_p = 0$. Since $p$ was arbitrary, $\mathcal{L}_XY \equiv 0$.
\end{proof}
Note that the fourth condition in the theorem is symmetric in $X$ and $Y$. Therefore all of the conditions of the theorem hold when $X$ and $Y$ are switched: in particular, $\mathcal{L}_XY \equiv 0$ if and only if $\mathcal{L}_YX \equiv 0$.

\subsection{A Formula for the Lie Derivative}

We will now derive a very simple formula for the Lie derivative. Given $X, Y \in \mathfrak{X}(M)$, let $F$ be the flow of $X$. Fix $p \in M$ and consider $H : \mathcal{D}^{(p)} \to T_pM$ defined by 
\[
H : t \mapsto (F_{-t})_{*, F_t(p)}(Y_{F_t(p)}).
\]
Then $H$ is a smooth function into the vector space $T_pM$. We may therefore identify $H'(0)$ with an element of $T_pM$; that element is precisely $H'(0) = (\mathcal{L}_XY)_p$. Since $H$ takes elements in a vector space, we can look at its Taylor series expansion near $t = 0$:
\[
(F_{-t})_{*,F_t(p)}(Y_{F_t(p)}) = H(t) = H(0) + tH'(0) + o(t) = Y_p + t(\mathcal{L}_XY)_p + o(t).
\]
Applying $(F_t)_{*,p}$ to both sides and rearranging
\[
Y_{F_t(p)} - (F_t)_{*,p}(Y_p) = t (F_t)_{*,p}((\mathcal{L}_XY)_p) + o(t).
\]
Suppose $f \in C^\infty(M)$. Evaluating both sides at $f$ gives
\[
(Y(f) \circ F_t)(p) - (Y(f \circ F_t))(p) = t (\mathcal{L}_XY)((f \circ F_t))(p) + o(t).
\]
Note that $(Y(f) \circ F_t)(p)$ and $f \circ F_t$ are functions of $t$ into vector spaces, so we can look at their Taylor series expansions near $t = 0$:
\begin{alignat*}{2}
(Y(f) \circ F_t)(p) &= Y_p(f) + t \left. \frac{d}{dt} \right|_{t = 0} (Y(f) \circ F_t)(p) + o(t) &&= Y_p(f) + t X_p(Y(f)) + o(t) \\
f \circ F_t & &&= f + t X(f) + o(t).
\end{alignat*}
Substituting these back in gives us
\[
\bcancel{Y_p(f)} + tX_p(Y(f)) - (Y(\bcancel{f} + t X(f) + o(t)))(p) = t (\mathcal{L}_XY)((f + t X(f) + o(t)))(p) + o(t),
\]
(the backslashes indicate that those terms cancel out) simplifying to
\[
t \Big[ X_p(Y(f)) - Y_p(X(f)) \Big] = t (\mathcal{L}_XY)((f + t X(f) + o(t)))(p) + o(t).
\]
By linearity on the right side, we can take out the term $tX(f) + o(t)$ from the argument of $\mathcal{L}_XY$. This term will be absorbed into the rightmost $o(t)$. Thus
\[
t \Big[ X_p(Y(f)) - Y_p(X(f)) \Big] = t (\mathcal{L}_XY)_p(f) + o(t).
\]
Using the Lie bracket $[X,Y] := XY - YX$, this may be written as 
\[
t [X,Y]_p(f) = t(\mathcal{L}_XY)_p(f) + o(t).
\]
Dividing by $t$ and taking the limit $t \to 0$ gives us the following
\begin{theorem}
$\mathcal{L}_XY = [X,Y]$.
\end{theorem}
With this formula in hand, we may very easily compute Lie derivatives. Note that this implies $\mathcal{L}_XY = -\mathcal{L}_YX$. Also, with this, we can say that $X$ and $Y$ commute if $[X, Y] = 0$.

Note the following properties of the Lie bracket:
\begin{enumerate}
\item Bilinearity.
\item Anticommutativity.
\item The \emph{Jacobi identity} 
\[
\sum_{\text{cyclic}} [X, [Y,Z]] = 0,
\]
or equivalently
\[
[X, [Y,Z]] + [Y, [Z, X]] + [Z, [X,Y]] = 0.
\]
\end{enumerate}
All of these are relatively obvious from the definition. Note that every property of the Lie bracket implies a property of the Lie derivative (after all, they are the same). Note that the Lie bracket is \emph{not} in general associative; the Jacobi identity ruins any hope of associativity in general.

\subsection{Lie Algebras}

Let us abstract away from vector fields on manifolds and consider only the structure a vector space is given when we define on it a "product" holding properties similar to the Lie bracket.

\begin{definition}
A Lie algebra is a vector space $V$ over a field $F$ together with a bilinear, anticommutative operation $[\cdot, \cdot] : V \times V \to V$ which satisfies the Jacobi identity.
\end{definition}
Note that a Lie algebra is, in general, not an algebra. Recall that an \emph{algebra} is a vector space $V$ over a field equipped with a "product" $\cdot : V \times V \to V$ making $V$ into a ring (with or without identity) which satisfies a homogeneity condition with respect to the vector space's scalar multiplication. 

However, if $(V, \cdot)$ is an algebra, we can consider the ring commutator on $V$ with respect to $\cdot$:
\[
[v, w] := v \cdot w - w \cdot v.
\]
One easily checks that the operation defined above gives the algebra $V$ a Lie algebra structure.

The most familiar example of a Lie algebra to us is $\mathfrak{X}(M)$ with the Lie bracket $[X,Y] = XY - YX$. (Note that the "multiplication" $XY$ here is a vector field defined by $(XY)(f) = X(Y(f))$, viewing vector fields as derivations of $C^\infty(M)$.) There are three main algebraic structures at play here:
\begin{itemize}
\item A real vector space structure.
\item A $C^\infty(M)$-module structure
\item A real Lie algebra structure.
\end{itemize}

\begin{definition}
A derivation on a Lie algebra $V$ is a linear map $D : V \to V$ with respect to the vector space structure satisfying the Leibnitz rule
\[
D([X,Y]) = [D(X),Y] + [X, D(Y)].
\]
\end{definition}

The next proposition provides an important class of derivations, one special member of which we are very familiar with.

\begin{proposition}
For $X$ in a Lie algebra $V$, define $\mathrm{ad}_X : V \to V$ by $\mathrm{ad}_X(Y) = [X,Y]$. Then $\mathrm{ad}_X$ is a derivation on $V$.
\end{proposition}
\begin{proof}
Linearity of $\mathrm{ad}_X$ follows from bilinearity of $[\cdot, \cdot]$. The Jacobi identity may be written as
\[
[X,[Y,Z]] = [[X,Y],Z] + [Y,[X,Z]],
\]
or alternatively,
\[
\mathrm{ad}_X([Y,Z]) = [\mathrm{ad}_X(Y), Z] + [Y, \mathrm{ad}_X(Z)].
\]
\end{proof}
\begin{corollary}
For a fixed $X \in \mathfrak{X}(M)$, the map $\mathcal{L}_X : \mathfrak{X}(M) \to \mathfrak{X}(M)$ defined by $\mathcal{L}_X(Y) = \mathcal{L}_XY$ is a derivation of the Lie algebra $\mathfrak{X}(M)$.
\end{corollary}
Let us note some important properties of the Lie derivative, whose proofs are relatively straightforward computations.
\begin{proposition}
The Lie derivative satisfies
\begin{enumerate}[(i)]
\item $\mathcal{L}_XY = -\mathcal{L}_YX$.
\item $\mathcal{L}_X([Y,Z]) = [\mathcal{L}_XY, Z] + [Y, \mathcal{L}_XZ]$.
\item $\mathcal{L}_{[X,Y]}Z = \mathcal{L}_X(\mathcal{L}_YZ) - \mathcal{L}_Y(\mathcal{L}_XZ)$.
\item If $g \in C^\infty(M)$, $\mathcal{L}_X(gY) = g\mathcal{L}_XY + X(g) \cdot Y$.
\item If $F : M \to N$ is a diffeomorphism, $F_*(\mathcal{L}_XY) = \mathcal{L}_{F_*X}(F_*Y)$.
\end{enumerate}
\end{proposition}

%careful with those calculations in (iv)
\begin{proof}
(i) and (ii) are immediate. For (iii), the Jacobi identity may be written as
\[
[[X,Y], Z] = [X, [Y,Z]] - [Y, [X,Z]],
\]
which is the desired identity. For (iv), if $f \in C^\infty(M)$ then
\begin{align*}
(\mathcal{L}_X(gY))(f) &= X((gY)(f)) - (gY)(X(f)) \\
&= X(g \cdot Y(f)) - g \cdot Y(X(f)) \\
&= g \cdot X(Y(f)) + Y(f) \cdot X(g) - g \cdot Y(X(f)) \\
&= g (\mathcal{L}_XY)(f) - (X(g)Y)(f),
\end{align*}
so the identity holds. The last is left as an exercise (was not included in lecture).
\end{proof}

\subsection{Mini Frobenius' Theorem}

Suppose $X \in \mathfrak{X}(M)$. Given $p \in M$, we can think of finding the integral curve of $X$ starting at $p$ as finding a $1$-dimensional (immersed) submanifold of $M$ to which $X$ is everywhere tangent. If we increase to $k$ vector fields, when can we find $k$-dimensional submanifolds of $M$ to which those vector fields are everywhere tangent? Such questions lead us to Frobenius' theorem.

Let $(U, \phi) = (U, x^1, \dots, x^n)$ be a chart on $M$. Suppose $f \in C^\infty(M)$. Then the function $\frac{\pd f}{\pd x^i}$ is again a $C^\infty$ function on $M$, for each $i$. Equality of second-order mixed partials gives
\[
\frac{\pd^2 f}{\pd x^i \pd x^j} = \frac{\pd^2 f}{\pd x^j \pd x^i},
\]
implying that the Lie bracket
\[
\left[ \frac{\pd}{\pd x^i}, \frac{\pd}{\pd x^j} \right] = 0.
\]
That is, the coordinate vector fields commute. Is this condition also sufficient? That is, is the following statement true? 

"Suppose $X_1, \dots, X_k$ is a \emph{(smooth) $k$-frame}; a collection of $k$ smooth vector fields such that $(X_1)_p, \dots, (X_k)_p$ are linearly independent for all $p \in M$. Suppose $[X_i, X_j] = 0$ for each $1 \leq i, j \leq k$. Are $X_1, \dots, X_k$ coordinate vector fields? That is, for $p \in M$, does there exist a chart $(U, x^1, \dots, x^n)$ near $p$ such that
\[
\frac{\pd}{\pd x^i} = X_i \text{ for } 1 \leq i \leq k?"
\] 
(We do not require $k = n$.)

% why can we choose such an epsilon??
Suppose $X,Y$ is a smooth commuting $2$-frame on $M$. Let $p \in M$ and let $U$ be a neighbourhood of $p$ such that the flows $F$ of $X$ and $G$ of $Y$ are defined on some $(-\epsilon, \epsilon) \times U$. Define $A : (-\epsilon, \epsilon)^2 \to U$ by 
\[
A(s,t) = G_s(F_t(p)) = F_t(G_s(p)).
\]
(Equality holds since $[X,Y] = 0$.) $A$ is $C^\infty$, and
\begin{align*}
A_{*, (s_0, t_0} \left( \frac{\pd}{\pd s} \right) &= \left. \frac{\pd A}{\pd s} \right|_{(s_0, t_0)} = Y_{A(s_0, t_0)} \\
A_{*, (s_0, t_0} \left( \frac{\pd}{\pd s} \right) &= \left. \frac{\pd A}{\pd t} \right|_{(s_0, t_0)} = X_{A(s_0, t_0)}.
\end{align*}
The fact that $X,Y$ is a $2$-frame implies that $A$ is an immersion. By the immersion theorem there is a chart $(U, \phi) = (U, x^1, \dots, x^n)$ near $p$ such that
\begin{align*}
\phi \circ A : (-\epsilon', \epsilon')^2 &\to \R^n \\
(s,t) &\mapsto (s,t,0,\dots,0)
\end{align*}
for some $\epsilon' \in (0, \epsilon]$. Then
\begin{align*}
\frac{\pd}{\pd x^1} &= Y \\
\frac{\pd}{\pd x^2} &= X
\end{align*}
on $A((-\epsilon', \epsilon')^2)$. Since $A$ is an embedding on $(-\epsilon', \epsilon')^2$, $S = A((-\epsilon', \epsilon')^2)$ is a $2$-dimensional embedded submanifold of $M$ such that for every $q \in S$, $T_qS = \mathrm{span}(\{X_q,Y_q\})$, and for which $(U, \phi)$ is an adapted chart. Moreover, $X$ and $Y$ are coordinate vector fields with respect to the chart $\phi_S = \pi \circ S$ on $S$. In this case, we call $S$ an \emph{integral submanifold of the $2$-frame $X,Y$.} We have proven the following theorem:

\begin{theorem}
(Mini-Frobenius) Let $X_1, \dots, X_k$ be a smooth commuting $k$-frame on $M$. Let $p \in M$. There exists a $k$-dimensional integral submanifold $S$ of this frame which contains $p$. Also, there exists an adapted chart $(U, \phi)$ near $p$ relative to $S$ such that the coordinate vector fields relative to $\phi_S$ are
\[
\frac{\pd }{\pd x^i} = X_i \text{ for } 1 \leq i \leq k.
\]
\end{theorem}
(We may state the result for arbitrary $k$ since the only thing that changes in the proof is the notation.)

\subsection{Leading up to Frobenius' Theorem}

We will be able to weaken the hypotheses of the preceding theorem. We will replace "commuting" with "such that $[X_i, X_j] \in \mathrm{span}(\{X_1, \dots, X_k\})$".

Let $S$ be a $k$-dimensional submanifold of $M$ which is an integral submanifold for the $k$-frame $X_1, \dots, X_k$ (i.e. $T_qS = \mathrm{span}(\{ (X_1)_q, \dots, (X_k)_q \})$ for all $q \in S$). Then
\[
[X_i, Z_j] = \sum a_\ell X_\ell
\]
for some functions $a_\ell$. The proof is an exercise. (Hint: write in coordinates the Lie bracket.) This shows that if $X_p, Y_p \in T_pS$, then $[X,Y]_p \in S$. In particular, the condition stated in the previous paragraph is necessary; Frobenius' theorem tells us that it is sufficient.

Now let $X, Y$ be a $2$-frame. Let $(U, \phi)$ be a chart as before. Then 
\[
\frac{\pd}{\pd x^1} = Y \qquad \frac{\pd}{\pd x^2} = X
\]
on $S$. We chose $\phi$ so that the integral curves of $Y$ on $S$ are precisely the integral curves of $\pd / \pd x^1$, and similarly for $X$ and $\pd / \pd x^2$, since $\phi \circ A : (s,t) \mapsto (s,t,0,\dots,0)$. But we do not necessarily have this on all of $U$; we have still not shown that $X$ and $Y$ are coordinate vector fields on $U$. We shall remedy this.

Consider the map
\[
H : (s,t, x^3, \dots, x^n) \mapsto G_s(F_t(\phi^{-1}(0,0,x^3,\dots,x^n))).
\]

\begin{proposition}
Let $\psi$ be the inverse of $H$, defined above. Then $\psi$ is a coordinate chart whose first two coordinate vector fields are $Y$ and $X$, respectively.
\end{proposition}

We have a

\begin{theorem}
Let $X_1, \dots, X_k$ be a smooth commuting $k$-frame on $M$. Let $p \in M$. Then there exists a chart $(U, \phi)$ near $p$ such that 
\[
\frac{\pd}{\pd x^i} = X_i
\]
for $1 \leq i \leq k$. (What does this mean when $k = 1$?)
\end{theorem}

\begin{lemma}
(Lemma for Frobenius' theorem) If $X_1, \dots, X_k$ is a smooth $k$-frame such that $[X_i, X_j] \in \mathrm{span}(\{ X_1, \dots, X_k \})$, then there is a smooth commuting $k$-frame $Y_1, \dots, Y_k$ such that $\mathrm{span}(\{ X_1, \dots, X_k \}) = \mathrm{span}(\{ Y_1, \dots, Y_k \})$.
\end{lemma}

"Commuting" is a necessary and sufficient condition for the vector fields to be coordinate vector fields.

\newpage

\section{Frobenius' Theorem in the Language of Bundles (July 14)}

\subsection{What We've Done}

Let us recall what we worked on last time, along with some corollaries. (Note that the "frame" terminology we used last lecture was not used in the standard way, so from now on we will avoid it.)

\begin{theorem}
Let $X_1, \dots, X_k$ be smooth linearly independent vector fields satisfying $[X_i, X_j] = 0$ for $1 \leq i, j \leq k$. Then for each $p \in M$, there is a coordinate chart $(U, x^1, \dots, x^n)$ at $p$ such that $X_i = \frac{\pd}{\pd x^i}$ for $i = 1, \dots, k$. The converse holds.

Moreover, $S = \{x^{k+1} = \cdots = x^n = 0\}$ is a $k$-dimensional integral submanifold of $X_1, \dots, X_k$ containing $p$.

Even better, $S^* = \{x^{k+1} = a^{k+1}, \dots, x^n = a^n\}$ for appropriate constants $a^{k+1}, \dots, x^n \in \R$ is also an integral submanifold of $X_1, \dots, X_k$. 
\end{theorem}
The $S^*$'s are an example of a $k$-dimensional foliation of $U$. A foliation of a manifold can be thought of as a partition of the manifold by lower dimensional submanifolds that "fit together nicely". This notion will be made precise later.

If $k = 1$, then all of this follows from the fundamental theorem of flows.

\subsection{Frobenius' Theorem in the Language of Vector Fields}

The property that $X_1, \dots, X_k$ is necessary and sufficient for them to be coordinate vector fields, but it is not necessary for the existence of integral submanifolds.

\begin{proposition}
Let $S$ be an integral submanifold of the linearly independent $X_1, \dots, X_k \in \mathfrak{X}(M)$. Then
\[
[X_i, X_j] = \sum_{\ell = 1}^k a^\ell_{ij} X_\ell
\]
for some $a^\ell_{ij} \in C^\infty(U)$ and for any $1 \leq i,j \leq n$.
\end{proposition}
\begin{proof}
This is an assignment problem.
\end{proof}

This property above is also sufficient for the existence of integral submanifolds.

\begin{lemma}
Let $X_1, \dots, X_k$ be smooth linearly independent vector fields which satisfy the conclusion of the preceding proposition. Then for each $p \in M$, there is a neighbourhood $U$ of $p$ and $Y_1, \dots, Y_k \in \mathfrak{X}(U)$ such that $[Y_i, Y_j] = 0$ for $1 \leq i,j \leq k$ and $\mathrm{span}(\{ (Y_1)_q, \dots, (Y_k)_q \}) = \mathrm{span}(\{ (X_1)_q, \dots, (X_k)_q \})$ for all $q \in U$.
\end{lemma}

With the preceding proposition and lemma in mind, we have Frobenius' theorem. However, the language used to state it is rather clunky, so we will develop some language that will allow us to state Frobenius' theorem very nicely in the language of subbundles of the tangent bundle.

\subsection{Subbundles}

Choose a $k$-dimensional subspace $\Delta_p \subseteq T_pM$ at each point $p \in M$. Does there exist a $k$-dimensional submanifold $S$ of $M$ such that $\Delta_q = T_qS$ for all $q \in S$? (Here, we abuse notation and write $T_qS$ to mean $i_{*,q}(T_qS)$ rather than $T_qS$, as the former is actually a subspace of $T_qM$.) The answer is not always.

Let $\Delta = \bigcup_{p \in M} \Delta_p$. Then $\Delta$ is a subset of $TM$. We call $\Delta$ a \emph{rank-$k$ distribution of $M$}, or alternatively, a \emph{rank-$k$ subbundle of $TM$}.

\begin{definition}
$\Delta \subseteq TM$ is a \emph{rank-$k$ subbundle of $TM$} if $\Delta_q = \pi|_\Delta^{-1}(\{q\})$ is a $k$-dimensional subspace of $T_qM$, for all $q \in M$. We say that $\Delta$ is a \emph{$C^\infty$ rank-$k$ subbundle} if $\pi|_\Delta : \Delta \to M$ is smooth.
\end{definition}

This definition of smoothness of a subbundle of $TM$ is rather clunky and hard to use. We will develop an equivalent notion which is much easier to work with.

\begin{definition}
A \emph{section of $\Delta$} is a map $X : M \to \Delta$ such that $\pi \circ X = \mathrm{id}$. We denote by $\Gamma(\Delta)$ the space of all smooth sections of $\Delta$. 
\end{definition}

We will see later on an assignment that $\Delta$ will be a submanifold of $TM$. (Of what dimension?)

\begin{proposition}
(Local frame criterion for subbundles)
A rank-$k$ subbundle $\Delta$ of $TM$ is $C^\infty$ if and only if for all $p \in M$, there is a neighbourhood $U$ of $p$ and $X_1, \dots, X_K \in \mathfrak{X}(U)$ such that $\{ (X_1)_q, \dots, (X_k)_q \}$ forms a basis for $\Delta_q$, for every $q \in U$. (We call such a collection $X_1, \dots, X_k$ a local basis of $\Delta$.)
\end{proposition}

We will also call a $C^\infty$ subbundle of $TM$ a \emph{(smooth) distribution}.

\begin{definition}
A $C^\infty$ rank-$k$ subbundle $\Delta$ is said to be involutive if for every $p \in M$, there is a neighbourhood $U$ of $p$ in $M$ and a local basis $X_1, \dots, X_k$ defined on $U$ such that
\[
[X_i, X_k] = \sum_{\ell = 1}^k a^\ell_{ij} X_\ell
\]
for some $a^\ell_{ij} \in C^\infty(U)$ and for all $1 \leq i, j \leq k$. 
\end{definition}
Alternatively, $\Delta$ is involutive if the Lie bracket of any pair of smooth local sections of $\Delta$ is again a smooth local section of $\Delta$. A \emph{smooth local section of $\Delta$} is a smooth map $\sigma : U \to \Delta$ defined on an open set $U \subseteq M$ with $\pi \circ \sigma = \mathrm{id}$.

The following is an algebraic characterization of involutivity. The proof is very easy.

\begin{proposition}
$\Delta$ is involutive if and only if $\Gamma(\Delta)$ is a Lie subalgebra of $\Gamma(TM)$.
\end{proposition}
\begin{proof}
If $\Delta$ is involutive, then it is closed under the Lie bracket. Since $\Gamma(\Delta)$ is also a vector subspace of $\Gamma(TM)$, $\Gamma(\Delta)$ is a Lie subalgebra of $\Gamma(TM)$.

Suppose conversely that $\Gamma(\Delta)$ is closed under the Lie bracket. Let $X, Y$ be smooth local sections of $\Delta$ defined on an open set $U$ in $M$. Given $p \in M$, choose a bump function $\rho \in C^\infty(M)$ which is identically $1$ on a neighbourhood $V \subseteq U$ of $p$ and is supported in $U$. Then $\rho X, \rho Y$ are smooth global sections of $\Delta$, so $[\rho X, \rho Y]$ is also a smooth global section of $\Delta$ by hypothesis. We have
\[
[\rho X, \rho Y] = \rho^2 [X, Y] + \rho (X\rho)Y - \rho (Y\rho)X,
\]
which equals $[X,Y]$ on $V$. Thus $[X,Y]_p \in \Delta_p$. Therefore $[X,Y]$ is a smooth local section of $\Delta$, implying that $\Delta$ is involutive.
\end{proof}

\subsection{Frobenius' Theorem in the Language of Bundles}

\begin{definition}
A $C^\infty$ distribution $\Delta$ of rank $k$ is said to be integrable if for all $p \in M$, there is an integral submanifold $S$ of $\Delta$ containing $p$. (That is, $T_qS = \Delta_q$ for all $q \in S$.)
\end{definition}

\begin{definition}
A $C^\infty$ rank-$k$ distribution $\Delta$ is said to be completely integrable if for all $p \in M$, there is a "flat chart" $(U, \phi)$ of $M$ at $p$: $\frac{\pd}{\pd x^i}$ is a local basis of $\Delta$.

If this is true, then each $S = \{ x^{k+1} = a^{k+1}, \dots, x^n = a^n \}$ is an integral submanifold for $\Delta$, for appropriately chosen constants $a^{k+1}, \dots, a^n \in \R$.
\end{definition}

Complete integrability implies integrability by our previous work. We have the following proposition.

\begin{proposition}
Every integrable distribution is involutive.
\end{proposition}
\begin{proof}
Let $\Delta$ be a smooth rank-$k$ distribution that is integrable. Let $X,Y$ be smooth local sections of $\Delta$ defined on an open set $U$. Given $p \in M$, let $S$ be an integral submanifold of $\Delta$ containing $p$. Then $X_p, Y_p \in T_pS$, which implies that $[X,Y]_p \in T_pS = \Delta_p$. Since this is true for all $p \in U$, $[X,Y]$ is also a local section of $\Delta$. Therefore $\Delta$ is involutive.
\end{proof}

This gives rise to the following implications:
\[
\text{completely integrable} \implies \text{integrable} \implies \text{involutive}.
\]
Frobenius' theorem states that these are equivalences.

\begin{theorem}
(Frobenius' Theorem) Every involutive distribution is completely integrable.
\end{theorem}

The proof is essentially what we have already done. There is no more hard work to be done.

Remark: Given $p \in M$, there is a maximal integral submanifold of $\Delta$ containing $p$. This is a non-trivial fact. These maximal integral submanifolds form a "$k$-dimensional foliation of $M$, whose "leaves" are the maximal integral submanifolds.

Later we will give a PDE-theoretic version of Frobenius' theorem. This is only natural, since for $k = 1$, Frobenius' theorem is essentially the fundamental theorem of flows.

\newpage

\section{$1$-forms (July 20)}

\subsection{Motivation, Einstein Notation}

\begin{definition}
Given a point $p$ on the smooth manifold $M$, the dual space to $T_pM$ is called the cotangent space at $p$ and is denoted by $T_p^*M$. Its elements are called covectors (at $p$).
\end{definition}

Recall that $T_p^*M = \mathrm{Hom}(T_pM, \R)$ is an $n$-dimensional vector space over $\R$.

We define a \emph{covector field} on $M$ to be a choice of a covector at each point, similarly to how a vector field is a choice of a (tangent) vector at each point. Since the cotangent spaces are the duals to the tangent spaces, many familiar constructions regarding the tangent space can be done for the cotangent space. In particular, we will develop a vector bundle called the \emph{cotangent bundle}, and give it a topology and smooth structure so that we can talk about a \emph{smooth covector field}, which we may also call a \emph{(differential) $1$-form}.

We now detour into the world of Einstein notation, whose benefits should become obvious after its definition is given.
\begin{enumerate}
\item
Indices should be placed as follows:
\begin{alignat*}{2}
&\text{vectors: } \, v_i \qquad &&\text{coefficients of vectors: } \, a^i \\
&\text{covectors: } \theta^i \qquad &&\text{coefficients of covectors: } b_i
\end{alignat*}
The main idea is that if the index $i$ appears twice, once up and once down, and if you are summing over that index, then the object in question is basis-independent.

For example, let $v_1, \dots, v_n$ be a basis of $T_pM$ and $\theta^1, \dots, \theta^n$ the dual basis. Then we may write fixed elements $v \in T_pM$, $\theta \in T_p^*M$ as linear combinations $v = \sum a^i v_i$ and $\theta = \sum b_i \theta^i$. Consider the following three objects:
\[
\sum a^i b_i \qquad \sum a^i \theta^i \qquad \sum b_i v_i.
\]
The first object is basis-independent because it is simply $\theta(v)$, and in this one there is a summation with the same index appearing once above and once below: a simple calculation gives
\[
\theta(v) = \sum_i b_i \theta^i \left( \sum_j a^j v_j \right) = \sum_{i,j} a^j b_i \theta^i(v_j) = \sum_{i,j} a^jb_i\delta^j_i = \sum_i a^ib_i.
\]
The other two quantities are not basis-independent, which can be seen by, for example, considering their representations in the bases $-v_1, \dots, -v_n$ and $-\theta^1, \dots, -\theta^n$.

For one more example of a basis-dependent object, consider the inner product $\langle  v,w \rangle = \sum a^i c^i$, where $w = \sum c^i v_i$. This very much depends on the choice of basis, and, as we can see, the index $i$ appears twice above.

\item
If an index appears twice, once up and once down, then the summation symbol is omitted and it is understood that a summation is occurring over all possible values of that index.

For example, in our old notation we would write $v = \sum a^i v_i$, but now we just write $v = a^i v_i$ and it is understood that $i$ is being summed from $1$ to $n$.
\end{enumerate}
For another example, the coordinates $x^1, \dots, x^n$ on $\R^n$ are written with upper indices because they give the coefficients of vectors.

\subsection{The Cotangent Bundle}

The construction of the cotangent bundle is almost identical to that of the tangent bundle, so we will be very brief.

\begin{definition}
The cotangent bundle is the disjoint union $T^*M := \cup T_p^*M$. It comes with a projection map $\pi : T^*M \to M$, sending a covector at $p$ to the point $p$.
\end{definition}

We topologize $T^*M$ as follows: let $(U, \phi) = (U, x^1, \dots, x^n)$ be a coordinate chart on $M$. Let $\{ \lambda_p^1, \dots, \lambda_p^n \} $ be the dual basis to $\{ \left. \frac{\pd}{\pd x^1} \right|_p, \dots, \left. \frac{\pd}{\pd x^i} \right|_p \}$, for each $p \in U$. We define $\tilde{\phi} : T^*U \to \phi(U) \times \R^n$ by $\tilde{\phi}(c_i \lambda_p^i) := (x^1(p), \dots, x^n(p), c_1, \dots, c_n)$. (We are using Einstein notation here.) Then $\tilde{\phi}$ is bijective.

Give to $T^*U$ the unique topology which makes $\tilde{\phi}$ a homeomorphism. Define $\tau \subseteq \mathcal{P}(T^*M)$ by
\[
\tau = \{ A : A \cap T^*U \text{ is open in } U \text{ for all coordinate open sets } U \}.
\]
Then $\tau$ is a topology on $T^*M$. One checks that this makes $T^*M$ into a second-countable Hausdorff topological space.

Consider the collection of all coordinate charts $\{ (T^*U, \tilde{\phi}) \}$ on $T^*M$. This collection makes $T^*M$ into a $2n$-dimensional topological manifold, and can easily be seen to form a $C^\infty$ atlas on $T^*M$, making $T^*M$ into a $2n$-dimensional smooth manifold.

\textbf{Remark: } $TM$ and $T^*M$ are obviously locally diffeomorphic by construction, but are they diffeomorphic? The answer is yes, and one can see this easily after developing the notion of a Riemannian metric.

\subsection{One-Forms}

Having produced a smooth structure on the cotangent bundle, we now proceed with the promised definition of a covector field. 

\begin{definition}
A smooth covector field is a smooth section of the cotangent bundle (i.e. a smooth right inverse of $\pi$). A smooth section of $T^*M$ is also called a (differential) $1$-form.
\end{definition}

Let $f \in C^\infty(M)$. We define a $1$-form $df$ as follows: $(df)_p(v) = v(f)$, where $v \in T_pM$. For each $p$, $(df)_p$ is linear, so $df$ is a $1$-form. Alternatively, $(df)_p(X_p) = X_p([f])$.

\begin{proposition}
Let $f \in C^\infty(M)$. For $p \in M$ and $X_p \in T_pM$, 
\[
f_{*,p}(X_p) = (df)_p(X_p) \left. \frac{d}{dt} \right|_{f(p)}.
\]
\end{proposition}
\begin{proof}
If $f_{*,p}(X_p) = a \left. \frac{d}{dt} \right|_{f(p)}$, then
\[
a = f_{*,p}(X_p)(\mathrm{Id}) = X_p(\mathrm{Id} \circ f) = X_p(f) = (df)_p(X_p).
\]
\end{proof}
Under the usual identification of $T_{f(p)}\R$ with $\R$, this agrees with the previous notion of the differential. We therefore also call $df$ the \emph{differential} of $f$. 

If $(U, x^1, \dots, x^n)$ is a coordinate chart on $M$, then what can we say about $dx^1, \dots, dx^n$? If $p \in U$, then
\[
(dx^i)_p \left( \left. \frac{\pd}{\pd x^j} \right|_p \right) = \left. \frac{\pd x^i}{\pd x^j} \right|_p = \delta^i_j,
\]
implying that $(dx^1)_p, \dots, (dx^n)_p$ is the dual basis to $\left. \frac{\pd}{\pd x^i}\right|_p, \dots, \left. \frac{\pd}{\pd x^n} \right|_p$. Suppose $df = c_i dx^i$ on $U$. We have
\[
c_k = (c_i dx^i) \left( \frac{\pd}{\pd x^k} \right) = df \left( \frac{\pd}{\pd x^k} \right) = \frac{\pd f}{\pd x^k}.
\]
Therefore $df = \frac{\pd f}{\pd x^i} dx^i$ on $U$. In particular, $df$ is a smooth $1$-form as defined earlier, since in local coordinates it is the map
\[
(x^1, \dots, x^n) \mapsto \left(x^1, \dots, x^n, \frac{\pd f}{\pd x^1}, \dots, \frac{\pd f}{\pd x^n} \right).
\]
We state these facts formally as a proposition.
\begin{proposition}
If $f \in C^\infty(M)$, then the $1$-form $df$ defined by $(df)_p(X_p) = X_p(f)$ is a smooth $1$-form on $M$. Moreover, if $x^1, \dots, x^n$ are local coordinates on $M$, then, in these coordinates, $df$ is of the form 
\[
df = \frac{\pd f}{\pd x^i} dx^i.
\]
\end{proposition}

\subsection{Equivalent Notions of Smoothness for 1-forms}

Denote by $\Omega^1(M)$ the space of all smooth $1$-forms on $M$. This becomes a vector space over $\R$ by pointwise addition and scalar multiplication of covectors. It also becomes a $C^\infty(M)$-module as follows: if $\omega \in \Omega^1(M)$ and $f \in C^\infty(M)$, let $f\omega$ be the $1$-form defined by $(f\omega)_p(X_p) = f(p)\omega_p(X_p)$.

Recall the following characterization of smooth vector fields: 
\begin{proposition}
Let $X$ be a section of $TM$. The following are equivalent:
\begin{enumerate}
\item $X : M \to TM$ is $C^\infty$.
\item For any coordinate chart $(U, x^1, \dots, x^n)$, $X = a^i \frac{\pd}{\pd x^i}$ on $U$ for some $a^1, \dots, a^n \in C^\infty(U)$.
\item $X$ takes $C^\infty$ functions to $C^\infty$ functions.
\end{enumerate}
\end{proposition}

We will develop something similar for $1$-forms.

\begin{proposition}
Let $\omega$ be a section of $T^*M$. Then $\omega$ is $C^\infty$ if and only if for every chart $(U, x^1, \dots, x^n)$ on $M$, $\omega = c_i dx^i$ on $U$ for some $c_1, \dots, c_n \in C^\infty(U)$.
\end{proposition}
\begin{proof}
In local coordinates, $\omega$ sends $(x^1, \dots, x^n)$ to $(x^1, \dots, x^n, c_1, \dots, c_n)$, so $\omega$ is smooth on $U$ if and only if the $c_1, \dots, c_n$ are smooth.
\end{proof}
A $1$-form will not take functions to functions, so we cannot directly translate the third smoothness criterion for vector fields into one for covector fields. Instead, a covector field takes vector fields to functions, as we now define.

\begin{definition}
Let $\omega$ be a $1$-form on $M$. If $X$ is a vector field on $M$, define $\omega(X) : M \to \R$ by $\omega(X)(p) = \omega_p(X_p)$.
\end{definition}
Immediately we have the following property, which is a consequence of linearity of $\omega_p$ for each $p$.
\begin{proposition}
Let $\omega$ be a $1$-form on $M$. Then $\omega$ as a map from $\mathfrak{X}(M)$ to functions on $M$ is $C^\infty(M)$-linear (i.e. linear with respect to the module structure on $\mathfrak{X}(M)$).
\end{proposition}
It is surprising that this property is also sufficient for $\omega$ to be a $1$-form, assuming that $\omega$ takes smooth vector fields to smooth functions, which we state and prove in (2) of the following theorem.

\begin{theorem}
\begin{enumerate}
\item
Let $\omega$ be a $1$-form. Then $\omega \in \Omega^1(M)$ if and only if $\omega$ takes smooth vector fields to smooth functions.

\item
Every $C^\infty(M)$-linear map $\omega : \mathfrak{X}(M) \to C^\infty(M)$ is, under the obvious identification, a $1$-form.
\end{enumerate}
\end{theorem}
\begin{proof}
\begin{enumerate}
\item
Suppose $\omega \in \Omega^1(M)$. Given $X \in \mathfrak{X}(M)$, let $(U, x^1, \dots, x^n)$ be a coordinate chart on $M$. Write $X = X^i \frac{\pd}{\pd x^i}$ on $U$ and $\omega = a_i dx^i$ on $U$. Then
\[
\omega(X) = (a_i dx^i)X = a_i dx^i(X) = a_i X^i
\]
on $U$. Since the functions $a_i, X^i$ are all smooth on $U$, $\omega(X)$ is smooth on $U$. Hence $\omega(X) \in C^\infty(M)$.

Conversely, suppose $\omega$ takes smooth vector fields to smooth functions. Let $(U, x^1, \dots, x^n)$ be a coordinate chart on $M$. Write $\omega = a_i dx^i$ on $U$. Extend $\frac{\pd}{\pd x^k}$ to $\tilde{X_k} \in \mathfrak{X}(M)$ using a bump function supported in $U$ which is identically $1$ on some neighbourhood $W_j$ of $p \in U$. Then, on $W_j$,
\[
\omega(\tilde{X}_k) = (a_i dx^i)\left( \frac{\pd}{\pd x^k} \right) = a_k,
\]
implying that $a_k \in C^\infty(W_j)$. If $W = W_1 \cap \cdots \cap W_n$, then $\omega$ is $C^\infty$ on $W$. Therefore $\omega \in \Omega^1(M)$.

\item
Suppose $\omega : \mathfrak{X}(M) \to C^\infty(M)$ is $C^\infty(M)$-linear. We claim that $\omega(X)(p)$ depends only on $X_p$. By linearity we may, without loss of generality, assume that $X_p = 0$. Let $(U, x^1, \dots, x^n)$ be a coordinate chart on $M$ at $p$, and write $X = a^i \frac{\pd}{\pd x^i}$ on $U$. By considering a bump function $\psi$ supported in $U$ with $\psi(p) = 1$, extend each $a^i$ and $\frac{\pd}{\pd x^i}$ to all of $M$ by considering $\psi a^i$ and $\psi \frac{\pd}{\pd x^i}$. Then these extensions are smooth, and
\[
\psi^2 \omega(X) = \omega(\psi^2 X) = \omega \left( \psi a^i \psi \frac{\pd}{\pd x^i} \right) = \psi a^i \omega \left( \psi \frac{\pd}{\pd x^i} \right).
\]
Evaluation at $p$ gives $\omega(X)(p) = 0$.

Now, define, for each $p \in M$, $\omega_p : T_pM \to \R$ by $\omega_p(v) = \omega(X)(p)$, where $X$ is any smooth vector field on $M$ with $X_p = v$. The previous paragraph shows that this is well-defined. Then the $1$-form $\omega : M \to T^*M$ just defined has the same action on smooth vector fields as the $\omega$ we started with.
\end{enumerate}
\end{proof}

By (2) of the preceding theorem, we may identify $\Omega^1(M)$ with the set of all $C^\infty(M)$-linear maps $\mathfrak{X}(M) \to C^\infty(M)$. Continuing with the usual trend of abusing notation the highest degree, we will often consider these sets to be equal.

\subsection{Pullbacks of 1-forms}

Let $F : N \to M$ be $C^\infty$. Denote by $F^{*,p} : T_{F(p)}^*M \to T_p^*N$ the dual map to $F_{*,p}$. (This is the notation that was used in lecture.) If $\theta \in T_{F(p)}^*M$, then $F^{*,p}(\theta) = \theta \circ F_{*,p} \in T_p^*N$. If we have a $1$-form, then we can do this at every point.

Note that while vector fields cannot be pushed forward, in general, covector fields may always be pulled back. Let $\omega$ be a $1$-form on $M$. Define $F^*\omega$ by $(F^*\omega)_p := F^{*,p}(\omega_{F(p)}) = \omega_{F(p)} \circ F_{*,p}$.

We have the following properties of the pullback operation on $1$-forms, whose proofs are straightforward computations.
\begin{proposition}
\begin{enumerate}
\item
$F^*(g\omega) = (F^*g)F^*\omega$.
\item
If $g \in C^\infty(M)$, $F^*(dg) = d(F^*g)$.
\item
$F^* : \Omega^1(M) \to \Omega^1(N)$ is $\R$-linear. (Note that (1) ensures that $F^*$ will not, in general, be $C^\infty(M)$-linear.)
\end{enumerate}
\end{proposition}

\newpage

\section{The Exterior Derivative of a 1-form (July 21)}

\subsection{A Remark on Coordinates}

On $\R^n$, there is a natural isomorphism $\mathfrak{X}(\R^n) \to \Omega^1(\R^n)$ given by $a^i\frac{\pd}{\pd x^i} \mapsto \sum a^i dx^i$. As the notation suggests, this isomorphism is coordinate-dependent. However, $\R^n$ comes with a canonical set of global coordinates, so we are justified in calling this isomorphism natural.

If $M$ is a smooth manifold with a global coordinate system, then we can definitely find such an isomorphism, but it will no longer be "natural". This is similar to how, given a finite dimensional vector space $V$, one constructs an isomorphism with the dual space of $V$ by first choosing a basis of $V$.

If $M$ does not have a global coordinate system, then there is no guarantee that such an isomorphism will make sense on the overlaps of coordinate charts. We are, however, able to create a "local isomorphism" of sorts: if $(U, x^1, \dots, x^n)$ is a coordinate chart on $M$, then consider the map $\mathfrak{X}(U) \to \Omega^1(U)$ defined by $a^i \frac{\pd}{\pd x^i} \mapsto \sum a^i dx^i$. This makes sense because $\{ \frac{\pd}{\pd x^1}, \dots, \frac{\pd}{\pd x^n} \}$ and $\{dx^1, \dots, dx^n\}$ are frames of $TM$ and $T^*M$, respectively, over $U$. Moreover, these sets actually form bases of the $C^\infty(M)$-modules $\mathfrak{X}(U)$ and $\Omega^1(U)$, respectively. 

\subsection{The Exterior Derivative}

If $f \in C^\infty(M)$, then we have a smooth $1$-form $df = \frac{\pd f}{\pd x^i} dx^i$. We can imagine this as a map $d : C^\infty(M) \to \Omega^1(M)$. We call this map the \emph{exterior derivative}. Eventually, we will define $d$ on the space of all smooth forms of all degrees; here, $C^\infty(M)$ will be $\Omega^0(M)$, the space of smooth "$0$-forms".

\begin{proposition}
The exterior derivative $d$ has the following properties:
\begin{enumerate}
\item $d$ is $\R$-linear.
\item $d(fg) = f dg + g df$.
\item $d(f/g) = (g df - f dg)/g^2$.
\item $d$ commutes with pullback.
\item If $f$ is constant, $df = 0$.
\item If $df = 0$, then $f$ is constant on the components of $M$.
\end{enumerate}
\end{proposition}
\begin{proof}
We will only prove (6), which was left as an exercise in class. Suppose $df = 0$ for some $f \in C^\infty(M)$. Let $(U, x^1, \dots, x^n)$ be a coordinate chart on $M$, and suppose without loss of generality that $U$ is connected. Then $U$ is contained in some connected component of $M$. Here, $df = \frac{\pd f}{\pd x^i} dx^i = 0$, so $\frac{\pd f}{\pd x^i} = 0$ on $U$ for each $i$. Therefore $f$ is constant on $U$, implying that $f$ is locally constant on $M$. 

Let $M'$ be a connected component of $M$. Choose $x_0 \in M'$, and consider the set $W =  \{ x \in M' : f(x) = f(x_0) \}$. Then $W$ is non-empty and closed by definition. If $x \in W$, then we can find a neighbourhood $U \subseteq M'$ of $x$ on which $f$ is constant. But then $U \subseteq W$, implying that $W$ is open. Since $M'$ is connected, we must have $W = M'$.
\end{proof}
(It is a more general and easy fact of point-set topology that a locally constant function on a connected space is constant. The proof follows the second paragraph of the previous one nearly verbatim.)

Note that $d$ is \emph{not} a derivation on $C^\infty(M)$, despite (1) and (2) of the previous proposition. It cannot be a derivation since its codomain is $\Omega^1(M)$ and not $C^\infty(M)$.

\subsection{Closed and Exact 1-forms}

About the exterior derivative $d$ we ask two main questions. Is $d$ injective? Absolutely not, since $d(\text{const}) = 0$. Is $d$ surjective? The answer is "it depends on the topology of $M$".

Let us attempt to derive a necessary and sufficient condition for $\omega \in \Omega^1(M)$ to equal $df$ for some $f \in C^\infty(M)$. Suppose $\omega = a_idx^i$ for some local coordinates $x^1, \dots, x^n$ on $U \subseteq M$. To ask if there is a function $f \in C^\infty(M)$ with $\omega = df$ is to ask if there is a solution to the following overdetermined system of PDEs:
\[
\begin{cases} 
a_i = \frac{\pd f}{\pd a^i} \\
i = 1, \dots, n
\end{cases},
\]
which we will actually be able to answer using Frobenius' theorem.


If $\omega = df$, then
\[
\frac{\pd a_i}{\pd x^j} = \frac{\pd}{\pd x^j} \frac{\pd f}{\pd x^i} = \frac{\pd}{\pd x^i} \frac{\pd f}{\pd x^j} = \frac{\pd a_j}{\pd x^i}.
\]
Therefore a necessary condition for $\omega = df$ is
\[
\tag{*}
\frac{\pd a_i}{\pd x^j} = \frac{\pd a_j}{\pd x^i} \text{ for all charts } (U, x^1, \dots, x^n), \text{ where } \omega = a_idx^i \text{ on } U.
\]
We would like to formulate this in a coordinate-independent way. We can write (*) as
\[
0 = \frac{\pd}{\pd x^i}\left(\omega \left( \frac{\pd}{\pd x^j} \right)\right) - \frac{\pd}{\pd x^j} \left(\omega \left( \frac{\pd}{\pd x^i} \right)\right),
\]
which tempts us to write that this is equivalent to
\[
X(\omega(Y)) - Y(\omega(X)) = 0 \text{ for all } X, Y \in \mathfrak{X}(M).
\]
But not all vector fields are coordinate vector fields! (As we saw with the "mini Frobenius' theorem", the coordinate vector fields are precisely those that commute. Not all vector fields commute.)  There is, however, a slightly weaker condition we can replace this with that turns out to be the coordinate-independent formulation of (*) which we want.

\begin{proposition}
$\omega \in \Omega^1(M)$ satisfies (*) if and only if $X(\omega(Y)) - Y(\omega(X)) = \omega([X,Y])$ for all $X, Y \in \mathfrak{X}(M)$.
\end{proposition}
\begin{proof}
Exercise.
\end{proof}

Sadly, our condition is not actually sufficient. Consider, on $M = \R^2 \setminus \{0\}$, the $1$-form
\[
\omega = \frac{-y}{x^2+y^2} dx + \frac{x}{x^2+y^2} dy.
\]
One checks that $\omega$ satisfies (*). We show, however, that $\omega \neq df$ for $f \in C^\infty(\R^2 \setminus \{0\})$. Consider
\[
\theta(x,y) = \begin{cases} 
\arctan(y/x) & x > 0, y > 0, \\
\pi + \arctan(y/x) & x < 0, \\
2\pi + \arctan(y/x) & x > 0, y < 0, \\
\pi/2 & x = 0, y > 0, \\
3\pi/2 & x = 0, y < 0
\end{cases}
\]
defined on the open set $U = \{(x,y) : x < 0 \text{ or } x \geq 0, y \neq 0\}$. Then $\omega = d\theta$ on $U$. If $\omega = df$ for some $f \in C^\infty(\R^2 \setminus \{0\})$, then $df = d\theta$, implying that $\theta = f + \text{const}$. This contradicts the fact that $\theta$ cannot be continuously extended to all of $\R^2 \setminus \{0\}$. See Spivak's \emph{Calculus on Manifolds} for more elaboration on this example.

We now introduce some terminology that will help us describe our problem.
\begin{definition}
$\omega \in \Omega^1(M)$ is closed if it satisfies (*), and exact if there is an $f \in C^\infty(M)$ with $\omega = df$.
\end{definition}
We showed already that every exact form is closed in our derivation of the condition (*). We are therefore seeking an answer to the converse question: is every closed form exact? A sufficient condition in $\R^n$ is that the domain of the form in question be open and \emph{star-convex} - see the aforementioned book for a proof of this fact.

\begin{theorem}
Every closed $1$-form is locally exact: for all closed $\omega \in \Omega^1(M)$ and every $p \in M$, there is an open neighbourhood $U$ of $p$ such that $\omega = df$ for some $f \in C^\infty(U)$.
\end{theorem}
\begin{proof}
Here is a vague outline of the proof:
\begin{enumerate}
\item 
Find $X^1, \dots, X^n$ spanning the $C^\infty$ distribution $\Delta$.
\item
Show that $\Delta$ is involutive if and only if (*) is satisfied.
\item
Then $\Delta$ is completely integrable by Frobenius' theorem. Show that this is true if and only if $\omega = df$ for some $f \in C^\infty(U)$.
\end{enumerate}
\end{proof}
A full proof of this theorem may be provided later by the author, if he feels up to the task of writing one.

\subsection{A Remark on Geometry}

Given $f \in C^\infty(M)$, can we define $\nabla f \in \mathfrak{X}(M)$? If $M = \R^n$, then $\nabla f = \sum \frac{\pd f}{\pd x^i} \frac{\pd}{\pd x^i}$, which is characterized by the property that $\langle \nabla f(p), v \rangle = (df)_p(v)$ for $v \in T_p\R^n$. This definition does not directly carry over to manifolds since it is very much coordinate-dependent.

There is no straightforward generalization of $\nabla f$ to smooth manifolds available to us right now. The characterization of the gradient is very geometrical in nature, but the smooth manifolds we've been working on do not come with any "geometry" of sorts. 

To this end, one defines a \emph{Riemannian metric}, which is, roughly speaking, a smoothly varying choice of inner product on every tangent space of $M$. With a Riemannian metric, one may define the gradient of a function; even better, one can define the divergence of a vector field!

\textbf{Morally,} differential forms are the tools with which vector calculus is generalized to manifolds.

\newpage

\section{$k$-forms (July 24)}

\subsection{Multilinear Algebra}

\begin{definition}
An $(\ell, k)$-tensor on a real vector space $V$ is a multilinear map
\[
T : \underbrace{V^* \times \cdots \times V^*}_{\ell \text{ times}} \times V\underbrace{ \times \cdots \times V}_{k \text{ times}} \to \R.
\]
\end{definition}
In this course we will mainly be concerned with $(0,k)$-tensors, and we'll mainly refer to them as $k$-tensors. Why are these important? We have some reasons:
\begin{enumerate}
\item
The set of tensors have a rich algebraic structure. (They will form a "graded algebra.")
\item
They give us the objects we can integrate over. (It turns that the multilinear algebraic properties of forms allow us to define their integrals in a coordinate independent way.)
\item
They provide the framework needed to generalize vector calculus to manifolds.
\end{enumerate}

\begin{definition}
A $k$-tensor (hereafter this refers to a $(0,k)$-tensor, as defined above) is alternating if 
\[
f(v_1, \dots, v_i, \dots, v_j, \dots, v_k) = -f(v_1, \dots, v_j, \dots, v_i, \dots, v_k)
\]
for all $v_1, \dots, v_k \in V$.
\end{definition}
We have multiple characterizations of algebraic tensors that will make working with them easier.
\begin{proposition}
Let $f$ be a $k$-tensor on $V$. The following are equivalent:
\begin{enumerate}
\item $f$ is alternating.
\item $f(v_1, \dots, v_k) = 0$ whenever $v_i = v_k$ for some $i \neq j$.
\item $f(v_1, \dots, v_k) = 0$ whenever $\{v_1, \dots, v_k\}$ is linearly independent.
\item For all $\sigma \in S_k$, $\sigma f = \sgn(\sigma)f$, where $\sigma f$ is defined as the $k$-tensor $(\sigma f)(v_1, \dots, v_k) = f(v_{\sigma(1)}, \dots, v_{\sigma(k)})$.
\end{enumerate}
\end{proposition}

Let us introduce some notation for the spaces of different kinds of tensors.
\begin{itemize}
\item $T_k(V)$ for the vector space of $k$-tensors.
\item $A_k(V)$ for the vector space of alternating $k$-tensors.
\item $S_k(V)$ for the vector space of symmetric $k$-tensors; those $k$-tensors $f$ satisfying $\sigma f = f$ for any $\sigma \in S_k.$
\end{itemize}

Now we define projection operators:
\begin{align*}
\mathrm{Sym} : &T_k(V) \to S_k(V) \\
&f \mapsto \sum_{\sigma \in S_k} \sigma f
\end{align*}
and
\begin{align*}
\mathrm{Alt} : &T_k(V) \to A_k(V) \\
&f \mapsto \frac{1}{k!} \sum_{\sigma} (\sgn(\sigma))\sigma f.
\end{align*}
The reason for the mysterious $1/k!$ in the definition of the operator $\mathrm{Alt}$ is a technical one: it makes a lot of results come out nicer. In particular,
\begin{itemize}
\item $f$ is symmetric if and only if $f = \mathrm{Sym}(f)$,
\item $f$ is alternating if and only if $f = \mathrm{Alt}(f)$.
\end{itemize}

\begin{definition}
For $f \in T_k(V)$ and $g \in T_\ell(V)$, define the tensor product $f \otimes g \in T_{k + \ell}(V)$ by $(f \otimes g)(v_1, \dots, v_k, v_{k+1}, \dots, v_{k+\ell}) \coloneqq f(v_1, \dots, v_k)g(v_{k+1}, \dots, v_{k+\ell})$.
\end{definition}
We want a product operation of the form $A_k(V) \times A_\ell(V) \to A_{k+\ell(V)}$. The tensor product does not satisfy this property, unfortunately. Our projection operators will help us define it, however.

\begin{definition}
For $f \in A_k(V)$ and $g \in A_\ell(V)$, define the wedge product $f \wedge g \in A_{k+\ell}(V)$ by
\[
f \wedge g = \frac{(k+\ell)!}{k!\ell!}\mathrm{Alt}(f \otimes g).
\]
\end{definition}
The mysterious scalar multiple is, again, there for technical reasons. We also have
\[
f \wedge g = \frac{1}{k!\ell!} \sum_{\sigma \in S_{k+\ell}} \sgn(\sigma)\sigma(f \otimes g).
\]
Here are some properties of the wedge product $\wedge : A_k(V) \times A_\ell(V) \to A_{k+\ell}(V)$:
\begin{enumerate}
\item Bilinearity.
\item Associativity.
\item Anticommutatitivty: $f \wedge g = (-1)^{k\ell}g \wedge f$. (This is the reason we always sum over increasing indices!)
\item Fix a basis $e_1, \dots, e_n$ of $V$ and let $\alpha^1, \dots, \alpha^n$ be the dual basis for $V^* = A_1(V)$. For any increasing multi-index $I \subseteq \{1, \dots, n\}$ of length $k$, define $\alpha^I$ as the unique element of $A_k(V)$ sending $e_J = (e_{j_1}, \dots, e_{j_k})$ to $\delta^I_J$, where $J$ is another increasing multi-index of length $k$ from $\{1, \dots, n\}$. 

Then 
\[
\{\alpha^I : I \text{ an increasing multi-index of length } k \text{ from } \{1, \dots, n\}\}
\]
forms a basis of $A_k(V)$. In particular, $\dim(A_k(V)) = {n \choose k}$. Also, $a^I = a^{i_1} \wedge \cdots \wedge a^{i_k}$.

\item
For any $\omega^1, \dots, \omega^k \in V^*$ and $v_1, \dots, v_k \in V$, $\omega^1 \wedge \cdots \wedge \omega^k(v_1, \dots, v_k) = \det(\omega^i(v_j))$.

\item
The wedge product is actually characterized by the above properties.
\end{enumerate}

Because of these properties, we will hereafter denote by $\bigwedge^k(V)$ the space of alternating $k$-tensors on a vector space $V$.

\begin{definition}
An $\R$-algebra $A$ is said to be graded if $A = \bigoplus_{k=0}^\infty A^k$, where each $A^k$ is an $\R$-vector space, such that the multiplication $A^k \times A^\ell$ maps into $A^{k+\ell}$. A graded algebra $A$ is said to be anticommutative if $ab = (-1)^{k\ell}ba$ for $a \in A^k$ and $b \in A^\ell$.
\end{definition}

Define $\bigwedge(V^*) \coloneqq \bigoplus_{k=0}^\infty \bigwedge^k(V^*) = \bigoplus_{k=0}^n \bigwedge^k(V^*)$. The properties of the wedge product make $\bigwedge(V^*)$ an associative anticommutatitve graded algebra over $\R$ of dimension $\sum_{k=0}^n {n \choose k} = 2^n$.

\subsection{$k$-forms On Manifolds}

We developed a notion of smoothness for $1$-forms on manifolds. We defined a $1$-form $\omega$ on $M$ to be smooth if it was smooth as a section of the cotangent bundle. We will follow a similar approach by giving the union of all of the spaces $A_k(T_pM)$, over $p \in M$, a smooth structure, which will allow us to talk about a smooth $k$-form. (Along the way, our notation will change a little.)

Let $(U, x^1, \dots, x^n)$ be a coordinate chart on $M$ containing $p$. Then we have a basis $\{\left. \frac{\pd}{\pd x^1} \right|_p, \dots, \left. \frac{\pd}{\pd x^n} \right|_p\}$ of $T_pM$ with the dual basis $\{dx^1_p ,\dots, dx^n_p\}$. Therefore
\[
\{ dx_p^{i_1} \wedge \cdots \wedge dx_p^{i_k} : 1 \leq i_1 < \cdots < i_k \leq n \}
\]
is a basis of $\bigwedge^k(T_p^*M)$.

Define $\bigwedge^k(T^*M) \coloneqq \bigcup_{p \in M} \bigwedge^k(T_p^*M)$. We call $\bigwedge^k(T^*M)$ the \emph{bundle of alternating $k$-tensors.} This comes with a projection map
\begin{align*}
\pi : &\bigwedge^k(T^*M) \to M \\
&\omega \mapsto p \qquad \text{ whenever } \omega \in \bigwedge^k(T_p^*M)
\end{align*}
We can equip $\bigwedge^k(T^*M)$ with a topology and smooth structure making it into a rank ${n \choose k}$ vector bundle. In fact, there is a unique topology and smooth structure for which this is the case. The construction is very similar to that for $TM$ and for $T^*M$. The idea is that for a chart $(U, \phi)$, define $\tilde{\phi} : \bigwedge^k(T^*U) \to \phi(U) \times \R^n$ by
\[
\tilde{\phi} : \omega \mapsto (\phi(p), \{c_I\}_I) \qquad \text{ whenever } \omega = \sum_I c_Idx^I \in \bigwedge^k(T_p^*M).
\]
(The sum is over increasing multi-indices $I$.) A detailed proof that $\bigwedge^k(T^*M)$ is a smooth rank-${n \choose k}$ vector bundle is left as an exercise. We will sometimes call $\bigwedge^k(T^*M)$ the \emph{$k$th exterior power of the cotangent bundle.}

With a smooth structure on $\bigwedge^k(T^*M)$, we can talk begin to talk about smooth forms of higher degree.

\begin{definition}
A (differential) $k$-form on $M$ is a section of the $k$th exterior power of the cotangent bundle $\pi : \bigwedge^k(T^*M) \to M$. 
\end{definition}
For example, if $(U, x^1, \dots, x^n)$ is a chart on $M$, we can define $dx^I : U \to \bigwedge^k(T^*M)$ by $d^I = dx^{i_1} \wedge \cdots \wedge dx^{i_k}$, where the wedge product of two forms is defined pointwise. Thus $dx^I$ is a $k$-form on $U$.

Just as $1$-forms act on vector fields, $k$-forms act on $k$-tuples of vector fields. Let $\omega$ be a $k$-form on $M$. For $X_1, \dots, X_k \in \mathfrak{X}(M)$, define $\omega(X_1, \dots, X_k) : M \to \R$ pointwise: $\omega(X_1, \dots, X_k)(p) \coloneqq \omega_p(X_{1p}, \dots, X_{kp})$. We note the following important property: if $h : M \to \R$ is a function, then
\[
\omega(X_1, \dots, hX_i, \dots, X_k) = h\omega(X_1,\dots,X_k).
\]

We now begin to discuss smooth $k$-forms. The definition is exactly what one would expect.

\begin{definition}
A $k$-form $\omega$ on $M$ is smooth if it is smooth as a section of $\bigwedge^k(T^*M)$. The set of all smooth $k$-forms on $M$ is denoted $\Omega^k(M)$. We have $\Omega^k(M) = \Gamma(\bigwedge^k(T^*M))$, using vector bundle notation. We also define $\Omega^0(M) = C^\infty(M)$.
\end{definition}

The space $\Omega^k(M)$ is an $\R$-vector space and a $C^\infty(M)$-module, as we should expect by now. We have some equivalent conditions for smoothness of a $k$-form. The proofs are left as easy exercises.
\begin{proposition}
Let $\omega$ be a $k$-form on $M$. The following are equivalent:
\begin{enumerate}
\item $\omega$ is smooth as a section of $\bigwedge^k(T^*M)$.
\item For any chart $(U, x^1, \dots, x^n)$, $\omega = \sum_I c_I dx^I$ for some $c_I \in C^\infty(U)$, where the sum is over all increasing multi-indices $I$.
\item
By its action on vector fields, $\omega : \mathfrak{X}(M) \times \cdots \times \mathfrak{X}(M) \to C^\infty(M)$, and is $C^\infty(M)$-multilinear.
\end{enumerate}
\end{proposition}

The next proposition is a higher degree form of a surprising result that we saw for $1$-forms. The proof is identical.
\begin{proposition}
Every $C^\infty(M)$-multilinear map $\omega : \mathfrak{X}(M) \times \cdots \times \mathfrak{X}(M) \to C^\infty(M)$ is a $k$-form.
\end{proposition}

Let's see some examples. Let $f^1, \dots, f^k \in C^\infty(M)$. Then we have $df^1, \dots, df^k \in \Omega^1(M)$. If $(U, x^1, \dots, x^n)$ is a chart, then $df^1 \wedge \cdots \wedge df^k = \sum c_I dx^I$, where the sum is over increasing multi-indices $I$. If $p \in M$, then evaluating at $p$ gives
\[
df_p^1 \wedge \cdots \wedge df_p^k = \sum c_I(p)dx_p^{i_1} \wedge \cdots \wedge dx_p^{i_k}.
\]
Evaluation at $\left. \frac{\pd}{\pd x^I} \right|_p$ (which means exactly what you think it means) gives
\[
c_I(p) = df_p^1 \wedge \cdots \wedge dx_p^k \left( \left. \frac{\pd}{\pd x^I} \right|_p \right) = \det \left(df_p^i \left( \left. \frac{\pd}{\pd x^{i_j}} \right|_p \right)\right) = \frac{\pd (f^1, \dots, f^k)}{\pd (x^{i_1}, \dots, x^{i_k})}(p),
\]
which is the determinant of the Jacobian evaluated at $p$. Therefore
\[
df^1 \wedge \cdots \wedge df^k = \sum_{1 \leq i_1 < \dots < i_k \leq n} \frac{\pd (f^1, \dots, f^k)}{\pd (x^{i_1}, \dots, x^{i_k})} dx^{i_1} \wedge \cdots \wedge dx^{i_k}.
\]
So $df^1 \wedge \cdots \wedge df^k \in \Omega^k(M)$. This leads us to ask the question: is it true in general that wedges of smooth forms on $M$ are also smooth forms on $M$? The answer is yes.

Suppose $\omega \in \Omega^k(M)$ and $\eta \in \Omega^\ell(M)$. In local coordinates, we have
\[
\omega \wedge \eta = \left( \sum_I c_Idx^I \right) \wedge \left( \sum_J b_Jdx^J \right) = \sum_{I,J} c_Ib_Jdx^{IJ} \in \Omega^{k+\ell}(M),
\]
where $IJ$ is the multi-index $\{ i_1, \dots, i_k, j_1, \dots, j_\ell \}$, and all sums are over increasing multi-indices. Therefore the wedge product gives us a map $\wedge : \Omega^k(M) \times \Omega^\ell(M) \to \Omega^{k+\ell}(M)$. (We are not being too careful here, but it doesn't really matter in the end.)

We extend the wedge product to $0$-forms in the obvious way: since $\Omega^0(M) = C^0(M)$, $f \wedge \omega = f\omega$ for $f \in \Omega^0(M)$ and $\omega \in \Omega^k(M)$.

\subsection{Pullbacks of $k$-forms}

Let $F : N \to M$ be a smooth map. We define 
\begin{align*}
F^{*,p} : &\bigwedge^k(T_{F(p)}^*M) \to \bigwedge^k(T_p^*N) \\
&\theta \mapsto F^{*,p}(\theta) \coloneqq \theta \circ (F_{*,p}, \dots, F_{*,p}).
\end{align*}
That is, if $\theta \in \bigwedge^k(T_{F(p)}^*M)$ and $v_1, \dots, v_k \in T_{F(p)}M$, then $F^{*,p}(\theta)(v_1, \dots, v_k) = \theta(F_{*,p}(v_1), \dots, F_{*,p}(v_k))$.

With this, we define the pullback of a $k$-form as follows: if $\omega$ is a $k$-form on $M$, define $F^*\omega$ on $N$ by
\[
(F^*\omega)_p \coloneqq F^{*,p}\omega_{F(p)} = \omega_{F(p)} \circ (F_{*,p}, \dots, F_{*,p}).
\]

The pullback has the following properties:
\begin{proposition}
\begin{enumerate}
\item
$F^*(a\omega + \eta) = aF^*\omega + F^*\eta$.
\item
For any $k$-form $\omega$ and $\ell$-form $\eta$, $F^*(\omega \wedge \eta) = F^*\omega \wedge F^*\eta$. (The pullback distributes over the wedge product.)
\item
$F^* : \Omega^k(M) \to \Omega^k(N)$.
\end{enumerate}
\end{proposition}
\begin{proof}
We will prove only (3). In local coordinates,
\begin{align*}
F^*\omega &= F^*(\sum c_Idx^I) \\
&= \sum (c_I \circ F) F^*(dx^{i_1} \wedge \cdots \wedge dx^{i_k}) \\
&= \sum (c_I \circ F) d(x^{i_1} \circ F) \wedge \cdots \wedge d(x^{i_k} \circ F)
\end{align*}
the sum, as always, ranging over increasing multi-indices. Since each $d(x^{i_1} \circ F) \wedge \cdots \wedge d(x^{i_k} \circ F)$ is a smooth $k$-form, $F^*\omega$ must be a smooth $k$-form.
\end{proof}

\subsection{A Remark About Top Degree Forms}

Let $M,N$ be smooth manifolds of common dimension $n$ with charts $(V, y^1, \dots, y^n)$ and $(U, x^1, \dots, x^n)$, respectively, and let $F : N \to M$ be a smooth map with $F(U) \subseteq V$, for simplicity. Then
\[
\Omega^n(V) = \{ f dy^1 \wedge \cdots \wedge dy^n : f \in C^\infty(V) \}
\]
is a $1$-dimensional $C^\infty(V)$-module. On $U$, 
\[
F^*(dy^1 \wedge \cdots \wedge dy^n) = dF^1 \wedge \cdots \wedge dF^n = \det \left( \frac{\pd F^i}{\pd x^j} \right) dx^1 \wedge \cdots \wedge dx^n,
\]
giving us the very important identity
\[
\boxed{F^*(f dy^1 \wedge \cdots \wedge dy^n) = (f \circ F)\det\left(\frac{\pd F^i}{\pd x^j} \right) dx^1 \wedge \cdots \wedge dx^n}
\]
that we will (likely) use extensively.

Define
\[
\Omega^*(M) \coloneqq \bigoplus_{k=0}^\infty \Omega^k(M) = \bigoplus{k=0}^n \Omega^k(M).
\]
Equipped with the wedge product, $\Omega^*(M)$ is an associative anticommutative graded algebra over $\R$. This is what was meant in the first section of this lesson by "tensors have a very rich algebraic structure." As we can see, the algebraic structure of differential forms on a manifold is \emph{extremely} rich. In particular, $\Omega^*(M)$ is studied extensively in algebraic topology. (See, for example, de Rham cohomology.)

Next time, we will develop the exterior derivative
\begin{align*}
&d : \Omega^k(M) \to \Omega^{k+1}(M) \\
&d : \Omega^*(M) \to \Omega^*(M).
\end{align*}

\newpage

\section{The Exterior Derivative of a $k$-form (July 29)}

\subsection{Motivating the Local Definition}

\begin{definition}
An antiderivation on a graded algebra $A = \bigoplus_k A^k$ is an $\R$-linear map $D : A \to A$ such that
\[
D(\omega \cdot \tau) = D(\omega) \cdot \tau + (-1)^k \omega \cdot D(\tau) \qquad \text{ whenever } \omega \in A^k.
\]
An element $\omega \in A^k$ is said to be (homogeneous) of degree $k$. The antiderivation $D$ is said to be of degree $m$ if $\deg(D(\omega)) = \deg(\omega) + m$ for all $\omega \in A$.
\end{definition}

Recall that $\Omega^*(M) = \bigoplus_k \Omega^k(M)$. The exterior derivative $d$ on $\Omega^*(M)$ that we wish to define will be an antiderivation of degree $1$.

On $0$-forms, we defined $d$ in a coordinate-independent way as $d : f \mapsto (X \mapsto X(f))$. Alternatively, we could have defined it in each coordinate chart and showed that the definition is independent of the chart. Specifically, if $(U, x^i)$ is a coordinate chart on $M$, we can define $df$ on $U$ as the $1$-form $df = \frac{\pd f}{\pd x^i} dx^i$. To show that this is independent of the coordinate chart, let $(V, y^i)$ be another chart with $U \cap V \neq \emptyset$. Then, on $U \cap V$,
\[
\frac{\pd f}{\pd x^i} dx^i = \frac{\pd f}{\pd x^i}  \frac{\pd x^i}{\pd y^j} dy^j  = \frac{\pd f}{\pd y^j} dy^j,
\]
which shows that the local definition of $df$ is coordinate-independent. We can actually define the exterior derivative $d$ locally in the same manner for forms of higher degree, and show that our definition is independent of the chart we defined it in.

\subsection{Defining $d$ Locally For $1$-forms}

Some time ago, we asked when a $1$-form $\omega$ was expressible as $df$ for some $0$-form $f$. A sufficient condition for \emph{local exactness} is that $\frac{\pd \omega_i}{\pd x^j} - \frac{\pd \omega_j}{\pd x^i} = 0$ on every chart. This condition is also expressible in a coordinate independent manner: 
\[
\frac{\pd \omega_i}{\pd x^j} - \frac{\pd \omega_j}{\pd x^i} = 0 \iff X(\omega(Y)) - Y(\omega(X)) = \omega([X, Y]) \qquad \text{ for all } X, Y \in \mathfrak{X}(U).
\]
The left side in the above equivalence is antisymmetric in $i$ and $j$, so we might hope that if we define
\[
d\omega \coloneqq \sum_{i < j} \left( \frac{\pd \omega_i}{\pd x^j} - \frac{\pd \omega_j}{\pd x^i} \right) dx^i \wedge dx^j,
\]
then this $2$-form would be independent. Note that this is equivalent to saying that $d\omega = d\omega_i \wedge dx^i$. The following lemma shows that this is indeed a coordinate-independent definition.

\begin{lemma}
Let $(U, x^i)$ and $(V, y^i)$ be coordinate charts on $M$ such that $\omega = a_i dx^i = b_i dy^i$ on $U \cap V$. Then $da_i \wedge dx^i = db_i \wedge dy^i$ on $U \cap V$.
\end{lemma}
\begin{proof}
On $U \cap V$,
\begin{align*}
da_i \wedge dx^i &= \frac{\pd a_i}{\pd x^j} dx^j \wedge dx^i \\
&= \frac{\pd}{\pd x^j} \left( \omega \left( \frac{\pd}{\pd x^i} \right) \right) dx^j \wedge dx^i \\
&= \frac{\pd}{\pd x^j} \left( b_k dy^k \left( \frac{\pd}{\pd x^i} \right) \right) dx^j \wedge dx^i \\
&= \frac{\pd}{\pd x^j} \left( b_k  \frac{\pd y^k}{\pd x^i}  \right) dx^j \wedge dx^i \\
&=  \left( \frac{\pd b_k}{\pd x^j} \frac{\pd y^k}{\pd x^i} + b_k \frac{\pd^2 y^k}{\pd x^j \pd x^i}  \right) dx^j \wedge dx^i \\
&=  \frac{\pd b_k}{\pd x^j} \frac{\pd y^k}{\pd x^i} dx^j \wedge dx^i \\
&=  \frac{\pd b_k}{\pd x^j} \frac{\pd y^k}{\pd x^i} \left( \frac{\pd x^j}{\pd y^\ell} dy^\ell \right) \wedge \left( \frac{\pd x^i}{\pd y^m} dy^m \right) \\
&= \left( \frac{\pd b_k}{\pd x^j} \frac{\pd x^j}{\pd y^\ell} \right) \left( \frac{\pd y^k}{\pd x^i} \frac{\pd x^i}{\pd y^m} \right) dy^\ell \wedge dy^m \\
&= \frac{\pd b_k}{\pd y^\ell} \delta^k_m dy^\ell \wedge dy^m \\
&= \frac{\pd b_k}{\pd y^\ell}  dy^\ell \wedge dy^k \\
&= db_k \wedge dy^k
\end{align*}
\end{proof}
Therefore the exterior derivative of a $1$-form is a well-defined $2$-form.

\subsection{Higher-Degree Exterior Derivatives}

\begin{definition}
Let $\omega \in \Omega^k(M)$. Define $d\omega$ locally as follows: if $(U, x^i)$ is a coordinate chart, then define $d\omega$ on $U$ by $d\omega \coloneqq d\omega_I \wedge dx_I$. One shows that this definition is coordinate-independent by a calculation similar to (but more tedious than) the previous one, giving rise to a $(k+1)$-form $d\omega \in \Omega^k(M)$.
\end{definition}

\begin{proposition}
$d : \Omega^k(M) \to \Omega^{k+1}(M)$ is an $\R$-linear map satisfying
\begin{enumerate}[(i)]
\item
$d$ is an antiderivation of degree $1$ on $\Omega^*(M)$.
\item
$d^2 = 0$.
\item
As just defined, $df$ agrees with the differential of a $0$-form $f$ as defined way before this lecture. 
\end{enumerate}
\end{proposition}
\begin{proof}
Exercise!
\end{proof}

\subsection{Outlining Uniqueness}

It turns out that the properties above actually characterize $d$.

\begin{theorem}
There is a unique $\R$-linear map $d : \Omega^k(M) \to \Omega^{k+1}(M)$ satisfying properties (i)-(iii) of the previous proposition.
\end{theorem}
\begin{proof}
We provide an outline of the proof here. A full proof is given in e.g. Tu or Lee. We have shown existence already. For uniqueness, suppose that $D : \Omega^k(M) \to \Omega^{k+1}(M)$ also satisfies those three properties and is $\R$-linear. We proceed in three main steps.
\begin{enumerate}
\item
Show that $D$ is a \emph{local operator}: for any $\omega \in \Omega^k(M)$, $(D\omega)_p$ depends only on $\omega$ in a neighbourhood of $p$.
\item
Given $\omega \in \Omega^k(M)$ and a chart $(U, x^i)$, write $\omega = a_I dx^I$ on $U$. Since $D$ is a local operator, $D|_U : \Omega^k(U) \to \Omega^{k+1}(U)$ defined by $D|_U \eta \coloneqq (D\tilde{\eta})|_U$, where $\tilde{\eta}$ is an extension of $\eta$ to $M$, is well-defined.
\item
One then shows that $D\omega = da_I \wedge dx^I = d\omega$ on $U$, proving uniqueness.
\end{enumerate}
\end{proof}

\begin{theorem}
If $F : N \to M$ is smooth, then $F^*(d\omega) = d(F^*\omega)$ for all $\omega \in \Omega^k(M)$.
\end{theorem}

\newpage

\section{Introducing Cartan's Calculus (August 4)}

\subsection{Plan}

We will develop \emph{Cartan's calculus}, which could be described as the calculus of differential forms. We will, in particular, do four things:
\begin{enumerate}
\item
Develop a global intrinsic formula for the exterior derivative.
\item
Develop interior multiplication of forms, a certain antiderivation $\i_X$ of degree $-1$.
\item
Develop the notion of the Lie derivative of a $k$-form.
\item
Discuss how the previous three concepts interact with each other. In particular, we will prove \emph{Cartan's homotopy formula:} $\mathcal{L}_X = d\i_X + \i_X d$.
\end{enumerate}

The focus of today's lecture is (1).

\subsection{A Global Intrinsic Formula For The Exterior Derivative of a $1$-form}

Suppose $\omega \in \Omega^1(M)$. Recall that we said that $\omega$ is closed if
\[
\frac{\pd \omega_j}{\pd x^i} - \frac{\pd \omega_i}{\pd x^j} = 0 \qquad \text{ on every chart } (U, x^1, \dots, x^n).
\]
Notice two things:
\begin{enumerate}[(i)]
\item
$\frac{\pd \omega_j}{\pd x^i} - \frac{\pd \omega_i}{\pd x^j}$ is antisymmetric in $i,j$, and it is the $i,j$-th component of $d\omega$:
\[
d\omega = \sum_{1 \leq i < j \leq n} \left( \frac{\pd \omega_j}{\pd x^i} - \frac{\pd \omega_i}{\pd x^j} \right) dx^i \wedge dx^j.
\]
This component is not coordinate-independent, but the property that it is zero is. Thus $\omega$ is closed if and only if $d\omega = 0$. We use this to generalize the notions of closedness and exactness to higher degree forms.
\end{enumerate}

\begin{definition}
$\omega \in \Omega^k(M)$ is closed if $d\omega = 0$, and exact if $\omega = d\eta$ for some $\eta \in \Omega^{k-1}(M)$.
\end{definition}

\begin{enumerate}[(i)]
\setcounter{enumi}{1} % shitty solution to get the definition there, but it works

\item
We showed that $\frac{\pd \omega_j}{\pd x^i} - \frac{\pd \omega_i}{\pd x^j} = 0$ on a coordinate open set $U$ if and only if for all $X,Y \in \mathfrak{X}(U)$, $X(\omega(Y)) - Y(\omega(X)) - \omega([X, Y]) = 0$. One shows with an easy bump function argument that if this holds on every chart, then this holds for vector fields on $M$ as well.
\end{enumerate}

We therefore have that $\omega \in \Omega^1(M)$ is closed if and only if $d\omega = 0$ , if and only if $X(\omega(Y)) - Y(\omega(X)) - \omega([X,Y]) = 0$ for all $X, Y \in \mathfrak{X}(M)$. This might lead one to thing that $d\omega$ and $X(\omega(Y)) - Y(\omega(X)) - \omega([X,Y])$ are related. In fact, they're the same thing.

\begin{theorem}
For every $\omega \in \Omega^1(M)$, $d\omega$ is, by its action on vector fields, given by
\[
d\omega (X, Y) = X(\omega(Y)) - Y(\omega(X)) - \omega([X,Y]).
\]
\end{theorem}
\begin{proof}
I will give a different proof than the one given in class. By linearity of both sides in $\omega$, we may assume that $\omega = f dg$, where $f, g \in C^\infty(U)$ for some (coordinate) open set $U$. If $X$ and $Y$ are smooth vector fields, then the left side is
\[
d(f dg)(X, Y) = df \wedge dg (X, Y) = df(X) dg(Y) - dg(X) df(Y) = (Xf)(Yg) - (Xg)(Yf),
\]
and the right side is
\[
X(f Yg) - Y(f Xg) - f dg([X, Y]) = ((Xf) (Yg) + f XYg) - ((Yf) (Xg) + f YX g) - f(XYg - YXg),
\]
which simplifies to equal the left side.
\end{proof}

\subsection{A Global Intrinsic Formula For The Exterior Derivative of a $k$-form}

The general case is a not-so-straightforward of the case for $1$-forms.

\begin{theorem}
For every $\omega \in \Omega^k(M)$, $d\omega$ is, by its action on vector fields, given by
\[
d\omega(X_0, \dots, X_k) = \sum_{i=0}^k (-1)^i X_i \omega \left(X_0, \dots, \hat{X_i}, \dots, X_k \right) + \sum_{0 \leq i < j \leq k} (-1)^{i+j} \omega \left( [X_i, X_j], \dots, \hat{X_i}, \dots, \hat{X_j}, \dots, X_k \right).
\]
\end{theorem}
\begin{proof}
We will be able to give a very short proof of this once we develop Cartan's homotopy formula. For now, we outline a proof that does not use this. Define 
\[
D\omega : \mathfrak{X}(M) \times \cdots \times \mathfrak{X}(M) \to C^\infty(M)
\]
by the right-hand side of the formula. We can use uniqueness of the exterior derivative to prove that $d\omega = D\omega$. This proceeds in two main steps.
\begin{enumerate}
\item
Show that $D\omega$ is a $(k+1)$-form by showing that $D\omega$ is $C^\infty(M)$-multilinear and alternating.
\item
Show that $D$ satisfies the characterizing properties of the exterior derivative to show that $d\omega = D\omega$, and conclude the result.
\end{enumerate}
\end{proof}

\end{document}
