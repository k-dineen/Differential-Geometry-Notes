\documentclass[11pt]{article}
\usepackage[utf8]{inputenc}
\usepackage{amsmath, amsthm, amssymb, amsfonts, mathtools, tikz-cd, float}
\usepackage[left=2.5cm,right=2.5cm]{geometry}
\usepackage[shortlabels]{enumitem}

\newcommand{\Int}{\mathrm{Int}}
\newcommand{\R}{\mathbb{R}}
\newcommand{\Z}{\mathbb{Z}}
\newcommand{\pd}{\partial}
\renewcommand{\epsilon}{\varepsilon}
\renewcommand{\hat}{\widehat}
\renewcommand{\tilde}{\widetilde}
\newcommand{\supp}{\mathrm{supp}}

\newtheorem{theorem}{Theorem}[section]
\newtheorem{corollary}{Corollary}[theorem]
\newtheorem{lemma}[theorem]{Lemma}
\newtheorem{proposition}{Proposition}[section]

\newtheorem{definition}{Definition}

\pagestyle{myheadings}

\begin{document}

\section{More on Flows and Lie Derivatives (July 7)}

\subsection{The Fundamental Theorem of Flows}

Consider $X \in \mathfrak{X}(M)$. Let $(U, \phi) = (U, x^1, \dots, x^n)$ be a coordinate chart on $M$. The theory of ODEs in $\R^n$ applies almost exactly to the integral curves of $X|_U \in \mathfrak{X}(U)$, which follows from $U$ being diffeomorphic to an open subset of $\R^n$. In particular,
\begin{enumerate}
\item
Given $p \in U$, there is, by the existence and uniqueness theorems, a unique maximal integral curve of $X|_U$ starting at $p$.

\item
By "collecting" all of the maximal integral curves from the previous step, we get flows. More precisely, we define
\[
\mathcal{D} = \left\{ (t, p) \in \R \times M : t \text{ lies in the domain of the maximal integral curve of $X|_U$ starting at } p \right\}.
\]
Then, for $p \in U$, we let
\[
\mathcal{D}^{(p)} = \{ t \in \R : (t, p) \in D \};
\]
this is just the domain of the maximal integral curve of $X|_U$ starting at $p$. In particular, $0 \in \mathcal{D}^{(p)}$.

Now define $F : \mathcal{D} \to U$ by $F(t,p) = \gamma_p(t)$, where $\gamma_p : \mathcal{D}^{(p)} \to U$ is the maximal integral curve of $X|_U$ starting at $p$.

We are led to the following question: does there exist an interval $(-\epsilon, \epsilon)$ such that $(-\epsilon, \epsilon) \subseteq \mathcal{D}^{(p)}$ for all $p \in U$? If the answer is yes, then the vector field $X|_U$ is \emph{complete}; each of its maximal integral curves exists for all $t \in \R$. This is known as the "Uniform Time Lemma".

Moreover, the function $F$ is smooth, which follows immediately from the theorem of smooth dependence on initial conditions of ODEs in $\R^n$.

\item
Given $p \in U$, there is, by smooth dependence on initial conditions, a neighbourhood $W \subseteq U$ of $p$ such that the hypotheses of the uniform time lemma are satisfied on $W$. (?)
\end{enumerate}
All of the above was in a coordinate chart, and so we have seen nothing new. Everything followed directly from the theory of ODEs in $\R^n$. We must then ask if the above claims hold in general on $M$, not restricted to a single coordinate open set. The answer is yes, all of these claims hold when $U$ is replaced with $M$.

\begin{theorem}
(Fundamental Theorem of Flows) Suppose $X \in \mathfrak{X}(M)$. Then there exists a unique smooth maximal flow $F : \mathcal{D} \to M$, where $\mathcal{D} \subseteq \R \times M$, generated by $X$, satisfying
\begin{enumerate}[(a)]
\item
For each $p \in M$, the curve $\gamma_p(t) = F(t, p)$ is the unique maximal integral curve of $X$ starting at $p$.
\item
If $s \in \mathcal{D}^{(p)}$, then $\mathcal{D}^{(F(s,p))} = \mathcal{D}^{(p)} - s$.
\item
For each $t \in \R$, the set $M_t = \{p \in M : (t, p) \in \mathcal{D}\}$ is open in $M$, and $F_t : M_t \to M_{-t}$ is a diffeomorphism.
\end{enumerate}
\end{theorem}
\begin{proof}
The proof is left as an exercise (likely as one of the homework problems). It is also in Lee.
\end{proof}

\subsection{More on Lie Derivatives}

Suppose $X,Y \in \mathfrak{X}(M)$, and let $F$ be the flow of $X$. If $M = \R^n$, then we can compare $Y_{F_t(p)}$ and $Y_p$, since $T_p\R^n = \R^n$ for every $p \in \R^n$. We cannot, however, do this for abstract manifolds, since the elements of the distinct tangent spaces of $M$ cannot even be compared. We remedy this using the pushforward of $F_{-t}$.
\begin{definition}
The Lie derivative of $Y$ with respect to $X$ is the vector field $\mathcal{L}_XY$ defined by
\[
(\mathcal{L}_XY)_p = \lim_{t \to 0} \frac{(F_{-t})_{*,F_t(p)}(Y_{F_t(p)}) - Y_p}{t},
\]
when the limit exists.
\end{definition}
\begin{proposition}
The above limit exists for all $p \in M$, and $\mathcal{L}_XY \in \mathfrak{X}(M)$.
\end{proposition}
\begin{proof}
Exercise.
\end{proof}

Consider the question of whether or not the flows of $X$ and $Y$ commute. Let $G$ be the flow of $Y$. We then may ask ourselves: given $p \in M$, do we have $F_s(G_t(p)) = G_t(F_s(p))$?  Define a function $A$ mapping $(s,t)$ to $F_s(G_t(p)) - G_t(F_s(p))$. We have the following proposition:
\begin{proposition}
\[
\frac{\pd^2 A}{\pd s \pd t} = \frac{\pd^2 A}{\pd t \pd s} = \mathcal{L}_XY,
\]
and if $\mathcal{L}_XY = 0$, then $A \equiv 0$.
\end{proposition}
\begin{proof}
Exercise.
\end{proof}
So $\mathcal{L}_XY$ measures "how much the flows commute".

For another perspective, consider a smooth function $f \in C^\infty(M)$ and a vector field $X \in \mathfrak{X}(M)$. We can ask ourselves the question "how does $f$ change along the integral curve of $X$ starting at $p$?". If $F$ is the flow of $X$, then we would like to discuss
\[
\left. \frac{d}{dt} \right|_{t = 0} f \circ F_t(p)
\]
for $p \in M$. A straightforward application of the chain rule gives
\[
\left. \frac{d}{dt} \right|_{t = 0} f \circ F_t(p) = f_{*,p}(\gamma_p'(0)) = f_{*,p}(X_p) = (Xf)(p).
\]
We can also define $\left. \frac{d}{dt} \right|_{t = 0} f \circ F_t(p)$ as the first order term in the Taylor expansion of $f \circ F_t(p)$ at $0$:
\[
f \circ F_t(p) = f(p) + t(Xf)(p) + o(t).
\]
Let us apply the same idea in order to get the Lie derivative. Define $G$ to be the map
\[
G : t \mapsto (F_{-t})_{*,F_t(p)}(Y_{F_t(p)}) \in T_pM.
\]
The Taylor series expansion of $G$ at $0$ is
\[
G(t) = G(0) + tG'(0) + o(t) = Y_p + t(\mathcal{L}_XY)_p + o(t),
\]
as we will see later. Therefore we may actually define $(\mathcal{L}_XY)_p$ to be the first order term of the Taylor expansion of the map $G$ defined above.

We end up having three equivalent definitions of the Lie derivative $\mathcal{L}_XY$: given $p \in M$, define $(\mathcal{L}_XY)_p$ as
\begin{enumerate}
\item
the limit
\[
(\mathcal{L}_XY)_p := \lim_{t \to 0} \frac{(F_{-t})_{*,F_t(p)}(Y_{F_t(p)}) - Y_p}{t}.
\]
\item
the first-order term of the Taylor expansion of the map $G$ defined by
\[
G : t \mapsto (F_{-t})_{*,F_t(p)}(Y_{F_t(p)}).
\]
\item
the Lie bracket 
\[
[X,Y] = X_pY - Y_pX.
\]
\end{enumerate}
\end{document}
  

  