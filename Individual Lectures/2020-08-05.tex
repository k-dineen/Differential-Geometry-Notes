\documentclass[11pt]{article}
\usepackage[utf8]{inputenc}
\usepackage{amsmath, amsthm, amssymb, amsfonts, mathtools, tikz-cd, float}
\usepackage[left=2.5cm,right=2.5cm]{geometry}
\usepackage[shortlabels]{enumitem}
\usepackage{cancel}

\newcommand{\Int}{\mathrm{Int}}
\newcommand{\R}{\mathbb{R}}
\newcommand{\Z}{\mathbb{Z}}
\newcommand{\pd}{\partial}
\renewcommand{\epsilon}{\varepsilon}
\renewcommand{\hat}{\widehat}
\renewcommand{\tilde}{\widetilde}
\newcommand{\supp}{\mathrm{supp}}
\newcommand{\sgn}{\mathrm{sgn}}

\newtheorem{theorem}{Theorem}[section]
\newtheorem{corollary}{Corollary}[theorem]
\newtheorem{lemma}[theorem]{Lemma}
\newtheorem{proposition}{Proposition}[section]

\newtheorem{definition}{Definition}

\pagestyle{myheadings}

\begin{document}

\section{Cartan Calculus Continued (August 5)}

Our plan is to continue developing the Cartan calculus.

\subsection{Interior Multiplication}

We will first define \emph{interior multiplication} on a vector space, and then define it on a manifold pointwise.

\begin{definition}
Given a vector space $V$ and $\beta \in \bigwedge^k(V^*)$ with $k \geq 2$, define, for $v \in V$, $i_v\beta \in \bigwedge^{k-1}(V^*)$ by
\[
i_v \beta(v_1, \dots, v_{k-1}) = \beta(v, v_1, \dots, v_{k-1}).
\]
The map $i_v\beta$ is called the interior multiplication (or contraction) of $\beta$ with $v$. If $k = 1$, we define $i_v\beta$ as the scalar $\beta(v)$, and if $k = 0$, we define $i_v\beta$ to be $0$.
\end{definition}

It is obvious that $i : V \times \bigwedge^k(V^*) \to \bigwedge^{k-1}(V^*)$ is linear with respect to the vector space structures in both arguments. We list some properties of interior multiplication.

\begin{proposition}
\begin{enumerate}
\item
If $\alpha^1, \dots, \alpha^k \in V^* = \bigwedge^1(V^*)$ and $v \in V$, then
\[
i_v(\alpha^1 \wedge \cdots \wedge \alpha^k) = \sum_{i=1}^{k-1} (-1)^{i-1} \alpha_1 \wedge \cdots \wedge \widehat{\alpha^i} \wedge \cdots \wedge \alpha^k.
\]

\item
$i_v^2 = 0$. (Compare with $d^2 = 0$.)

\item
For any $\beta \in \bigwedge^k(V^*)$ and $\gamma \in \bigwedge^\ell(V^*)$, one has $i_v(\beta \wedge \gamma) = i_v\beta \wedge \gamma + (-1)^k \beta \wedge i_v\gamma$.
\end{enumerate}
\end{proposition}
\begin{proof}
\begin{enumerate}
\item
Expand along the first column in
\[
i_v(\alpha^1 \wedge \cdots \wedge \alpha^k)(v_1, \dots, v_{k-1}) = \det \begin{pmatrix}
\alpha^1(v) & \alpha^1(v_1) & \cdots & \alpha^k(v_{k-1}) \\
\vdots      & \vdots        & \ddots & \vdots \\
\alpha^k(v) & \alpha^k(v_1) & \cdots & \alpha^k(v_{k-1})
\end{pmatrix}
\]
\item
If $k \geq 2$, then
\[
i_v^2\beta(v_1, \dots, v_{k-2}) = \beta(v,v,v_1,\dots,v_{k-2}) = 0,
\]
since $\beta$ is alternating. If $k = 1$ or $k = 0$, then this is obvious.
\item
Reduce to the case where $\beta$ and $\gamma$ are of the form $\alpha^1 \wedge \cdots \wedge \alpha^k$ and $\alpha^{k + 1} \wedge \cdots \wedge \alpha^{k+\ell}$ by linearity. 
\end{enumerate}
\end{proof}

So $i_v : \bigwedge^*(V^*) \to \bigwedge^*(V^*)$ is an antiderivation of degree $-1$ whose square is zero. Compare with $d$!

Having defined interior multiplication on a vector space, we make the obvious generalization to manifolds by defining interior multiplication pointwise.

\begin{definition}
Given $X \in \mathfrak{X}(M)$ and $\omega \in \Omega^k(M)$, define $i_X \omega$ as the $(k - 1)$-form given by $(i_X \omega)_p \coloneqq i_{X_p}\omega_p$.
\end{definition}

For $X_1, \dots, X_{k-1} \in \mathfrak{X}(M)$, $i_X\omega(X_1, \dots, X_{k-1}) = \omega(X,X_1,\dots,X_{k-1})$ is a smooth function on $M$, so we conclude that $i_X\omega \in \Omega^{k-1}(M)$, since any form that takes smooth vector fields to smooth functions must be a smooth form.

By the way we defined the edge cases $k = 0, 1$ for the interior multiplication of tensors on vector spaces, we have the following matching edge cases for the interior multiplication of forms on manifolds: if $k = 1$, then $i_X\omega = \omega(X)$, and if $k = 0$, then $i_X\omega = 0$. 

The interior multiplication $i_X : \Omega^k(M) \to \Omega^{k-1}(M)$ has the following properties:
\begin{enumerate}
\item $\R$-linearity. (We now have to specify the type of linearity, because simply "linearity" could refer to the $C^\infty(M)$-module structure or to the $\R$-vector space structure.)
\item For $\omega \in \Omega^k(M)$ and $\eta \in \Omega^\ell(M)$, 
\[
i_X(\omega \wedge \eta) = i_X\omega \wedge \eta + (-1)^k \omega \wedge i_X\eta.
\]
\item
$i_X^2 = 0$.
\end{enumerate}
As before, $i_X$ is an antiderivation on the graded algebra $\Omega^*(M)$ of degree $-1$ whose square is zero. Again, compare with $d$! Since $i_X\omega$ is defined pointwise, the map $i : \mathfrak{X}(M) \times \Omega^k(M) \to \Omega^{k-1}(M)$ is $C^\infty(M)$-linear in both arguments.

We note an important property of $i_X$ whose proof is obvious.

\begin{proposition}
$i_X \circ i_Y + i_Y \circ i_X = 0$.
\end{proposition}

\subsection{Lie Derivative of Forms}

Fix a smooth vector field $X \in \mathfrak{X}(M)$ with flow $F$. We generally want to study how things change along the flow of $X$ at a point. We defined the Lie derivative on $\Omega^0(M)$ as
\[
(\mathcal{L}_X f)_p \coloneqq \lim_{t \to 0} \frac{f(F_t(p)) - f(p)}{t} = \left. \frac{d}{dt} \right|_{t=0}f(F_t(p)) = X_p(f).
\]
We also defined the Lie derivative on $\mathfrak{X}(M)$ using pushforwards to compare the tangent vectors:
\[
(\mathcal{L}_XY)_p \coloneqq \lim_{t \to 0} \frac{(F_{-t})_{*,F_t(p)}(Y_{F_t(p)}) - Y_p}{t} = \left. \frac{d}{dt} \right|_{t = 0} (F_{-t})_{*,F_t(p)}(Y_{F_t(p)}).
\]
Just as vector fields push forwards and differential forms pull back, we now define the Lie derivative of a $k$-form by using pullbacks to compare the forms:
\[
(\mathcal{L}_X \omega)_p \coloneqq \lim_{t \to 0} \frac{F_t^*(\omega_{F_t(p)}) - \omega_p}{t} = \left. \frac{d}{dt} \right|_{t = 0} F_t^*(\omega_{F_t(p)}).
\]
\begin{proposition}
The limit always exists, and $\mathcal{L}_X\omega \in \Omega^k(M)$ whenever $\omega \in \Omega^k(M)$.
\end{proposition}
\begin{proof}
Write $\mathcal{L}_X\omega$ in local coordinates.
\end{proof}

We could also have defined $(\mathcal{L}_X \omega)_p$ as the first order term in the Taylor expansion of $F_t^*(\omega_{F_t(p)})$:
\[
F_t^*(\omega_{F_t(p)}) = \omega_p + t(\mathcal{L}_X \omega)_p + o(t).
\]

\subsection{Properties of These Operations}

We state and prove many properties of this Lie derivative, including how it interacts with the exterior derivative and with interior multiplication.

\begin{theorem}
\begin{enumerate}
\item
$\mathcal{L}_X : \Omega^*(M) \to \Omega^*(M)$ is a derivation: it is $\R$-linear and satisfies
\[
\mathcal{L}_X (\omega \wedge \eta) = (\mathcal{L}_X \omega) \wedge \eta + \omega \wedge \mathcal{L}_X \eta.
\]

\item
The Lie derivative commutes with the exterior derivative: $\mathcal{L}_X \circ d = d \circ \mathcal{L}_X$.

\item
We have a "global intrinsic formula" for the Lie derivative:
\[
\mathcal{L}_X(\omega(X_1, \dots, X_k)) = (\mathcal{L}_X\omega)(X_1, \dots, X_k) + \sum_{i=1}^k \omega(X_1, \dots, \mathcal{L}_XX_i, \dots, X_k).
\]

\item
(Cartan's magic formula) $\mathcal{L}_X = d \circ i_X + i_X \circ d$.

\item
$\mathcal{L}_X \circ i_Y - i_Y \circ \mathcal{L}_X = i_{[X, Y]}$. 
\end{enumerate}
\end{theorem}
\begin{proof}
For most of these properties, we shall merely outline a proof.
\begin{enumerate}
\item
One has
\[
\mathcal{L}_X(\omega \wedge \eta) = \left. \frac{d}{dt} \right|_{t = 0} (F_t^*\omega \wedge F_t^* \eta).
\]
Assume without loss of generality that $\omega = f dg$ and $\eta = u dv$ for some smooth functions $f, g, u, v$. Then work out the computation.

\item
One has
\[
d(F_t^*\omega) = d\omega + td(\mathcal{L}_X\omega) + o(t).
\]
Since pullback and exterior differentiation commute, this is also equal to
\[
F_t^*(d\omega) = d\omega + t \mathcal{L}_X(d\omega) + o(t).
\]
As they are equal, we may cancel the $d\omega$ terms, divide by $t$, and take the limit $t \to 0$. This gives the desired $\mathcal{L}_X \circ d = d \circ \mathcal{L}_X$.

\item
By the definition of $\mathcal{L}_X \omega$, we obtain
\[
F_t^* \omega (X_1, \dots, X_k) = \omega(X_1, \dots, X_k) + t(\mathcal{L}_X\omega)(X_1, \dots, X_k) + o(t).
\]
Note that
\begin{align*}
\omega( (F_t)_*(X_1), \dots, (F_t)_*(X_k) )  &= \omega( X_1 - t\mathcal{L}_XX_1 + o(t), \dots, X_k - t\mathcal{L}_XX_k + o(t) )  \\
&= \omega(X_1, \dots, X_k) - t\sum_{i=1}^k \omega(X_1, \dots, \mathcal{L}_XX_i, \dots, X_k) + o(t).
\end{align*}
Substituting this into the first equation and moving the sum, as well as the $o(t)$ terms, to the right gives
\[
\omega(X_1, \dots, X_k) \circ F_t = \omega(X_1, \dots, X_k) + t\left( (\mathcal{L}_X \omega)(X_1, \dots, X_k) + \sum_{i=1}^k \omega(X_1, \dots, \mathcal{L}_XX_i, \dots, X_k) \right) + o(t).
\]
Taylor expansion of the left hand side at $t = 0$ gives
\[
\omega(X_1, \dots, X_k) \circ F_t = \omega(X_1, \dots, X_k) + t \mathcal{L}_X(\omega(X_1, \dots, X_k)) + o(t),
\]
so the constant terms on both sides cancel, leaving us with
\[
t \mathcal{L}_X(\omega(X_1, \dots, X_k)) + o(t) = t\left( (\mathcal{L}_X \omega)(X_1, \dots, X_k) + \sum_{i=1}^k \omega(X_1, \dots, \mathcal{L}_XX_i, \dots, X_k) \right) + o(t).
\]
Dividing by $t$ and taking the limit $t \to 0$ gives the desired formula.

\item
This is another proof in which we will abuse uniqueness properties. First, a lemma;
\begin{lemma} (Uniqueness of the Lie derivative)
$\mathcal{L}_X : \Omega^k(M) \to \Omega^k(M)$ is the unique $\R$-linear map satisfying
\begin{enumerate}[(i)]
\item $\mathcal{L}_Xf = X(f)$ for $f \in \Omega^0(M)$,
\item $\mathcal{L}_X \circ d = d \circ \mathcal{L}_X$,
\item $\mathcal{L}_X (\omega \wedge \eta) = (\mathcal{L}_X \omega) \wedge \eta + \omega \wedge \mathcal{L}_X \eta$.
\end{enumerate}
\begin{proof}
Let $D : \Omega^k(M) \to \Omega^k(M)$ be an $\R$-linear map satisfying properites (i)-(iii). Let $\omega \in \Omega^k(M)$ and let $(U, x^1, \dots, x^n)$ be a chart on $M$. Since property (iii) is satisfied, $D$ is a local operator; $D\omega$ on $U$ depends only on $\omega$ on $U$, roughly. (The notion of a local operator was precisely defined a few lectures ago, and is in the textbook.)

On $U$ we thus have
\begin{alignat*}{2}
D\omega &= D\left(\sum a_I dx^I\right) \\
&= \sum \left( Da_I dx^I + a_I D(dx^I) \right) \qquad &&\text{$\R$-linearity + property (iii)}\\
&= \sum \left( Da_I dx^I + a_I \left[ d(Dx^{i_1}) \wedge \cdots \wedge d(Dx^{i_k}) \right] \right) \qquad &&\text{property (ii)} \\
&= \sum \left( X(a_I) dx^I + a_I \left[ dX^{i_1} \wedge \cdots \wedge dX^{i_k} \right] \right) \qquad &&\text{property (i)}.
\end{alignat*}
Since this does not depend on $D$, we have shown that $D$ is unique. Therefore $\mathcal{L}_X$, satisfying these properties, must be unique.
\end{proof}
\end{lemma}

With the lemma in mind, to prove Cartan's magic formula we may prove that $i_X \circ d + d \circ i_X$ is an $\R$-linear map satisfying properties (i)-(iii), and conclude by the uniqueness lemma that $\mathcal{L}_X = i_X \circ d + d \circ i_X$. These properties are all relatively straightforward to check and will be omitted.

\item
Show this for a $1$-form, and then use that to prove it for a $k$-form.

\end{enumerate}
\end{proof}

\subsection{An Easy Proof of The Global Formula For $d\omega$}

Suppose $\omega \in \Omega^1(M)$ and $X, Y \in \mathfrak{X}(M)$. We compute:
\begin{alignat*}{2}
d\omega (X, Y) &= i_Y(i_X(d\omega)) \\
&= i_Y \mathcal{L}_X\omega - i_Ydi_X\omega \qquad &&\text{Cartan's magic formula}\\
&= \mathcal{L}_X i_Y \omega - i_{[X, Y]} \omega - i_Ydi_X\omega \qquad &&\mathcal{L}_Xi_Y - i_Y\mathcal{L}_X = i_{[X, Y]}\\
&= X(\omega(Y)) - \omega([X, Y]) - Y(\omega(X)) \qquad &&i_Ydi_X\omega = Y(\omega(X)).
\end{alignat*}
This gives a straightforward proof for the $k = 1$ case. To prove the more general formula
\[
d\omega (X_0, \dots, X_k) = \sum_{i=0}^k (-1)^{i-1} X_i \omega(X_0, \dots, \hat{X_i}, \dots, X_k) + \sum_{0 \leq i < j \leq k} (-1)^{i + j} \omega([X_i, X_j], X_0, \dots, \hat{X_i}, \dots, \hat{X_j}, \dots, X_k),
\]
one uses induction as well as the formulas we have developed here. We shall not give the proof.

\subsection{Bringing it All Together (Cartan Calculus)}

Let us summarize what happened over the last two lectures.

For $X \in \mathfrak{X}(M)$, we introduced three "operators":
\begin{itemize}
\item
$d : \Omega^k(M) \to \Omega^{k+1}(M)$, an antiderivation of degree $1$ - exterior differentiation.
\item 
$i_X : \Omega^k(M) \to \Omega^{k-1}(M)$, an antiderivation of degree $-1$ - interior multiplication.
\item
$\mathcal{L}_X : \Omega^k(M) \to \Omega^k(M)$, a derivation - the Lie derivative.
\end{itemize}
They each interact with the wedge product $\wedge$:
\begin{itemize}
\item
$d(\omega \wedge \eta) = d\omega \wedge \eta + (-1)^k \omega \wedge d\eta$.
\item
$i_X(\omega \wedge \eta) = i_X\omega \wedge \eta + (-1)^k \omega \wedge i_X\eta$.
\item
$\mathcal{L}_X(\omega \wedge \eta) = \mathcal{L}_X\omega \wedge \eta + \omega \wedge \mathcal{L}_X\eta$.
\end{itemize}
And finally, they interact with each other:
\begin{align*}
d^2, i_X^2 &= 0 \\
\mathcal{L}_X\mathcal{L}_Y - \mathcal{L}_Y  \mathcal{L}_X &= \mathcal{L}_{[X, Y]} \\
i_Xi_Y + i_Yi_X &= 0 \\
d\mathcal{L}_X - \mathcal{L}_Xd &= 0 \\
\mathcal{L}_Xi_Y - i_Y\mathcal{L}_X &= i_{[X, Y]} \\
di_X + i_Xd &= \mathcal{L}_X.
\end{align*}
This completes our study of the Cartan calculus.

\end{document}
  

  