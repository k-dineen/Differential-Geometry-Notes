\documentclass[11pt]{article}
\usepackage[utf8]{inputenc}
\usepackage{amsmath, amsthm, amssymb, amsfonts, mathtools, tikz-cd, float}
\usepackage[left=2.5cm,right=2.5cm]{geometry}
\usepackage[shortlabels]{enumitem}
\usepackage{cancel}

\newcommand{\Int}{\mathrm{Int}}
\newcommand{\Bd}{\mathrm{Bd}}
\newcommand{\R}{\mathbb{R}}
\renewcommand{\H}{\mathbb{H}}
\newcommand{\Z}{\mathbb{Z}}
\newcommand{\pd}{\partial}
\renewcommand{\epsilon}{\varepsilon}
\renewcommand{\hat}{\widehat}
\renewcommand{\tilde}{\widetilde}
\newcommand{\supp}{\mathrm{supp}}
\newcommand{\sgn}{\mathrm{sgn}}

\newtheorem{theorem}{Theorem}[section]
\newtheorem{corollary}{Corollary}[theorem]
\newtheorem{lemma}[theorem]{Lemma}
\newtheorem{proposition}{Proposition}[section]

\newtheorem{definition}{Definition}

\pagestyle{myheadings}

\begin{document}

\section{Manifolds With Boundary Continued (August 11)}

We will study the concepts that carry over from smooth manifolds to those with boundary. These will be essential in developing integration on manifolds, as well as stating and proving Stokes' theorem.

\subsection{Things That Carry Over}

\begin{enumerate}
\item
\textbf{Tangent vectors:} The definition of the $\R$-algebra $C^\infty_p(M)$ carries over word-for-word to manifolds with boundary, the elements being equivalence classes under the equivalence relation of agreeing on open sets containing $p$. We then define the tangent space $T_pM$ to be the set of point-derivations of $C_p^\infty(M)$, as before.

Alternatively, we could, by the fourth homework assignment, define $T_pM$ as the set of all point-derivations at $p$ of $C^\infty(M)$. Recall that a \emph{point derivation at $p$ of $C^\infty(M)$} is an $\R$-linear map $D : C^\infty(M) \to \R$ satisfying
\[
D(fg) = D(f)g(p) + f(p)D(g).
\]
The two notions coincide by Problem 1(b) on Homework 4.

For example, consider the half-plane $\H^2$ and a point $p \in \pd \H^2$. The tangent space $t_p\H^2$ is a $2$-dimensional vector space with origin $p$. The vector $\pd / \pd y |_p$ is a tangent vector to $\H^2$ at $p$, so $-\pd / \pd y |_p$  is also a tangent vector to $\H^2$ at $p$. This is the prototype "outward-pointing" tangent vector. (For $\H^n$, it is $-\pd / \pd x^n |_p$.)


\item
\textbf{Cotangent spaces and forms:} The definitions are identical. $T_p^M$ is the dual space of $T_pM$ as defined before, and a $k$-form is a section of $\bigwedge^k(T^*M) \coloneqq \cup_{p \in M} \bigwedge^k(T_p^*M)$. A smooth $k$-form is a smooth section of $\bigwedge^k(T^*M)$, whose topology and smooth structure are defined in the exact same way as before.


\item
\textbf{Orientations:} The definition is the exact same. The fact that $(U, -x^1, \dots, x^n)$ is a chart whenever $(U, x^1, \dots, x^n)$ is a chart allows for the following fact: \emph{a pointwise orientation on $M$ is smooth if and only if for every $p \in M$ there is a chart $(U, x^1, \dots, x^n)$ at $p$ such that for every $q \in U$, the orientation on $T_qM$ induces by$\{ \pd / \pd x^1 |_q, \dots, \pd / \pd x^n |_q \}$ is consistent with the orientation chosen on $T_qM$.} See Lemma 21.4 of Tu.

Note that the fact $(U, -x^1, \dots, x^n)$ is a chart whenever $(U, x^1, \dots, x^n)$ is a chart would not have worked for $n = 1$ unless we also allowed the left half line to be a local model for manifolds with boundary.
\end{enumerate}

However, there is a bit to be said about vector fields and orientations regarding the boundary of a manifold, since the boundary of a manifold with boundary is itself a smooth manifold.

\subsection{Boundary Vector Fields and Orientation}

The boundary $\pd M$ is an $(n - 1)$-dimensional embedded submanifold of $M$. That is, the inclusion map $i : \pd M \hookrightarrow M$ is a smooth embedding (topological embedding/homeomorphism onto its image given the subspace topology and an immersion).

We make the following abuse of notation: $i_{*, p}(T_p \pd M) = T_p \pd M$. Then, if $p \in \pd M$ and if $(U, x^1, \dots, x^n)$, we have
\begin{align*}
T_p \pd M &= \left\{ \left. \frac{\pd}{\pd x^1} \right|_p, \dots, \left. \frac{\pd}{\pd x^{n-1}} \right|_p \right\}, \\
T_p M &= \left\{ \left. \frac{\pd}{\pd x^1} \right|_p, \dots, \left. \frac{\pd}{\pd x^n} \right|_p \right\}.
\end{align*}

We say that $X_p$ for $p \in \pd M$ is \emph{inward-pointing} if $X_p \not\in T_p \pd M$ and if there is a smooth curve $c : [0, \epsilon] \to M$ such that $c(0) = p$ and $c'(0) = X_p$, and we say $X_p$ is \emph{outward-pointing} if $-X_p$ is inward-pointing.

A \emph{vector field along $\pd M$} is a function $X$ assigning to each point $p \in \pd M$ a vector $X_p$ in $T_pM$ (as opposed to $T_p\pd M$). In a coordinate chart $(U, x^1, \dots, x^n)$, we may write such a vector field as $X = a^i \pd / \pd x^i$ for functions $a^1, \dots, a^n : U \cap \pd M \to \R$. We will say that such a vector field is smooth if these functions are smooth functions on $U \cap \pd M$, in any coordinate chart. It was left as an exercise to show that $X_p$ is outward pointing if and only if $a^n(p) < 0$; we state and prove this formally here.

\begin{proposition}
(Exercise 22.3 of Tu) Let $M$ be a manifold with boundary and let $p \in \pd M$. The tangent vector $X_p \in T_pM$ is inward-pointing if and only if in any coordinate chart $(U, x^1, \dots, x^n)$ centred at $p$, the coefficient of $\pd / \pd x^n |_p$ is positive.
\end{proposition}
\begin{proof}
Suppose that $X_p$ is inward-pointing. Then $X_p \not\in T_p \pd M$, and there is a smooth curve $c : [0, \epsilon] \to M$ such that $c(0) = p$ and $c'(0) = X_p$. Let $(U, \phi) = (U, x^1, \dots, x^n)$ be a chart centred at $p$. Then $\phi \circ c$ is a smooth curve in $\H^n$ starting at $0$ with $(\phi \circ c)'(0) = \phi_{*,p}(X_p)$. Then $c^n(0) = 0$ and $c^n(t) > 0$ for $t > 0$, so $\dot{c}^n(0) > 0$, for $X_p \not\in T_p \pd M$. But this is precisely the coefficient of $\pd / \pd x^n |_p$.

The converse is easy and is left as an exercise to the reader ;)
\end{proof}

The following will be useful for giving $\pd M$ an orientation.

\begin{proposition}
If $M$ is a manifold with boundary, then there exists a smooth outward-pointing vector field on $\pd M$.
\end{proposition}
\begin{proof}
Cover $\pd M$ in coordinate charts and take, in each chart, the smooth outward-pointing vector field in that chart along $\pd M$ given by the negative of the last coordinate vector field. Then use a partition of unity.
\end{proof}

The existence of a smooth outward-pointing vector field along $\pd M$ allows us to define an orientation on $\pd M$ whenever $M$ is oriented.

\begin{proposition}
Let $M$ be an oriented $n$-manifold with boundary. If $\omega$ is an orientation form for $M$ and $X$ is a smooth outward-pointing vector field along $\pd M$, then $i_X \omega$ is a smooth non-vanishing top-degree form on $\pd M$. Hence $\pd M$ is orientable.
\end{proposition}

The orientation induced by $i_X\omega$ is called the \emph{boundary orientation} of $\pd M$. Checking that this orientation is well-defined is left as an exercise.

\begin{proof}
Suppose for the sake of contradiction that $i_X\omega$ vanishes at some point $p \in \pd M$. Given a basis $v_1, \dots, v_{n-1}$ of $T_p \pd M$, the fact that $X_p$ is outward-pointing gives us that $X_p, v_1, \dots, v_{n-1}$ is a basis of $T_pM$. Then
\[
0 = (i_X\omega)_p (v_1, \dots, v_{n-1}) = \omega_p(X_p, v_1, \dots, v_{n-1}),
\]
implying that $\omega_p \equiv 0$, a contradiction.
\end{proof}

Let $(U, \phi) = (U, x^1, \dots, x^n)$ be a chart near a boundary point $p$ in the oriented atlas for $M$. Then $dx^1 \wedge \cdots \wedge dx^n$ is an orientation form for $U$. Moreover, $-\pd / \pd x^n$ is a smooth outward-pointing vector field along $U \cap \pd M$, so
\[
i_{-\pd / \pd x^n}(dx^1 \wedge \cdots \wedge dx^n)
\]
is a smooth non-vanishing top-degree form on $U \cap \pd M$. Then
\begin{align*}
i_{-\pd / \pd x^n} (dx^1 \wedge \cdots \wedge dx^n)\left( \frac{\pd}{\pd x^1}, \dots, \frac{\pd}{\pd x^{n-1}} \right) &= dx^1 \wedge \cdots \wedge dx^n \left( -\frac{\pd}{\pd x^n}, \frac{\pd}{\pd x^1}, \dots, \frac{\pd}{\pd x^{n-1}} \right) \\
&= -dx^1 \wedge \cdots \wedge dx^n \left( \frac{\pd}{\pd x^n}, \frac{\pd}{\pd x^1}, \dots, \frac{\pd}{\pd x^{n-1}} \right) \\
&= (-1)^n dx^1 \wedge \cdots \wedge dx^n \left(\frac{\pd}{\pd x^1}, \dots, \frac{\pd}{\pd x^n} \right) \\
&= (-1)^n,
\end{align*}
where in going from the second line to the third line we made $n - 1$ swaps inside the argument of $dx^1 \wedge \cdots \wedge dx^n$. Thus $(U \cap \pd M, (-1)x^1, x^2, \dots, x^{n-1})$ is a chart on $\pd M$ belonging to the oriented atlas of $\pd M$. We have the following proposition:

\begin{proposition}
If $\{ (U_\alpha, x_\alpha^1, \dots, x_\alpha^n) \}$ is an oriented atlas for $M$, then $\{ (U_\alpha \cap \pd M, (-1)^n x_\alpha^1, \dots, x_\alpha^{n-1}) \}$ is an oriented atlas for $\pd M$ giving it the induced orientation.
\end{proposition}

We defined the orientation on the boundary like this so that we could elegantly state Stokes' theorem as $\int_M d\omega = \int_{\pd M} \omega$, instead of requiring two cases. Next time, we will define integration on manifolds, and state and prove Stokes' theorem. See Lee pages 400-402 for some motivation on this.

\end{document}
  

