\documentclass[11pt]{article}
\usepackage[utf8]{inputenc}
\usepackage{amsmath, amsthm, amssymb, amsfonts, mathtools, tikz-cd, float}
\usepackage[left=2.5cm,right=2.5cm]{geometry}
\usepackage[shortlabels]{enumitem}

\newcommand{\Int}{\mathrm{Int}}
\newcommand{\R}{\mathbb{R}}
\newcommand{\Z}{\mathbb{Z}}
\newcommand{\pd}{\partial}
\renewcommand{\epsilon}{\varepsilon}
\renewcommand{\hat}{\widehat}
\renewcommand{\tilde}{\widetilde}
\newcommand{\supp}{\mathrm{supp}}

\newtheorem{theorem}{Theorem}[section]
\newtheorem{corollary}{Corollary}[theorem]
\newtheorem{lemma}[theorem]{Lemma}
\newtheorem{proposition}{Proposition}[section]

\newtheorem{definition}{Definition}

\pagestyle{myheadings}

\begin{document}

\section{Vector Fields, Integral Flows, and the Lie Derivative (June 11)}

\subsection{Smoothness Criteria for Vector Fields}

We will discuss vector fields in a little more detail and provide a few criteria for a vector field to be $C^\infty$. Note that given a coordinate chart $(U, x^1, \dots, x^n)$ on a manifold $M$, there are functions $a^1, \dots, a^n : U \to \R$ such that for every $p \in U$,
\[
X_p = \sum_i a^i \left. \frac{\pd}{\pd x^i} \right|_p.
\]
We will call the functions $a^1, \dots, a^n$ the components of $X$ in the chart. The first criterion is the one you would expect.

\begin{theorem}
(Smoothness in terms of components) Let $X$ be a section of $TM$. Then $X$ is $C^\infty$ if and only if for every coordinate chart $(U, x^1, \dots, x^n)$ on $M$, the component functions of $X$ in the chart are $C^\infty$ on $U$.
\end{theorem}
\begin{proof}
A coordinate chart $(U, \phi) = (U, x^1, \dots, x^n)$ induces a chart $(TU, \tilde{\phi})$ on $TM$. In these coordinates, if $p \in U$ and if $a^1, \dots, a^n$ are the components of $X$ in $U$, then
\[
\tilde{\phi} \circ X : p \mapsto (x^1(p), \dots, x^n(p), a^1(p), \dots, a^n(p)),
\]
in which it is clear that $X$ is $C^\infty$ if and only if each $a^i$ is $C^\infty$ on $U$.
\end{proof}

We can also think of vector fields as derivations. Let $s : M \to TM$ be a section. If $f \in C^\infty(M)$, define $D_s(f) : M \to \R$ by $D_s(f)(p) := s_p([f])$. It is easy to see that $D_s$ is a derivation on $C^\infty(M)$. With this we may present our second smoothness criterion.

\begin{theorem}
(Smoothness in terms of action on functions as a derivation) Let $s : M \to TM$ be a section. Then $s$ is $C^\infty$ if and only if for every $f \in C^\infty(M)$, the function $D_s(f)$ is $C^\infty$.
\end{theorem}
\begin{proof}
Suppose $s$ is $C^\infty$. Let $(U, x^1, \dots, x^n)$ be a coordinate chart on $M$. If $a^1, \dots, a^n$ are the components of $s$ in this chart, then on $U$ we have
\[
D_s(f) = \sum_i a^i \frac{\pd f}{\pd x^i},
\]
which is certainly $C^\infty$ on $U$. Since this is true for all charts, we have that $D_s(f) \in C^\infty(M)$.

The converse is a simple homework exercise which is done by extending the coordinate functions in any given chart.
\end{proof}

We would like to explore the link between vector fields and derivations some more. Recall our notation: $\Gamma(TM)$ for the smooth sections on $TM$, and $\mathrm{Der}(C^\infty(M))$ for the derivations on $C^\infty(M)$. These sets are both real vector spaces and $C^\infty(M)$-modules. It turns out that our association of a smooth section with a derivation is an isomorphism with respect to both of these structures.

\begin{theorem}
(Smooth sections $\cong$ Derivations) The map $\Phi : \Gamma(TM) \to \mathrm{Der}(C^\infty(M))$ defined by $\Phi(s) = D_s$ is an isomorphism of vector spaces and of modules.
\end{theorem}
\begin{proof}
Checking that $\Phi$ is a homomorphism (with respect to both structures) and injective is a homework exercise. As is the case with vector spaces (and in extreme similarity modules), to be a bijective homomorphism is sufficient for being an isomorphism. We shall only check surjectivity.

Suppose $D \in \mathrm{Der}(C^\infty(M))$. For $p \in M$ let us define $D_p : C_p^\infty(M) \to \R$ by $D_p([f]) = D(\tilde{f})(p)$, where $\tilde{f}$ is a smooth extension of $f$ to all of $M$. (We know we can do this with bump functions.) It is clear that this doesn't depend on the representative of $f$, so we only need to check that it also doesn't depend on the extension of $f$. 

Choose a representative $f : U \to \R$ of the germ. Let $\tilde{f}_1, \tilde{f}_2 \in C^\infty(M)$ be any two extensions of $f$ which both agree with $f$ on some open neighbourhood $V \subseteq U$ of $p$. Let $\rho$ be a bump function at $p$ supported in $V$. Then $\rho \cdot (\tilde{f}_1 - \tilde{f}_2) \equiv 0$ on $M$, so by linearity we have $D(\rho \cdot (\tilde{f}_1 - \tilde{f}_2)) = 0$. By the Leibnitz rule,
\[
0 = D(\rho )(\tilde{f}_1 - \tilde{f}_2) + \rho D(\tilde{f}_1 - \tilde{f}_2).
\]
Evaluating at $p$ gives $D(\tilde{f}_1 - \tilde{f}_2) = 0$, implying that $D(\tilde{f}_1)(p) = D(\tilde{f}_2)(p)$ by linearity. So the function $D_p$ is well defined. 

It is not too hard to see that $D_p \in T_pM$. Define $s : M \to TM$ by $p \mapsto D_p$. Since $D_p \in T_pM$ for each $p \in M$, the map $s$ is a section. We have
\[
D_s(f)(p) = s_p([f]) = D_p([f]) = D(f)(p),
\]
so $D_s = D$. It remains to check that $s$ is a smooth section of $TM$.

Let $(U, \phi) = (U, x^1, \dots, x^n)$ be a chart on $M$ at $p$. Then we have component functions $a^1, \dots, a^n$ of $s$ on $U$. We must first extend the coordinate functions to all of $M$ to use the fact $D_s(x^j) = a^j$. Extend $x^j$ to
$\tilde{x^j} \in C^\infty(M)$ agreeing with $x^j$ on a neighbourhood $\tilde{U}$ of $p$. Then, on $\tilde{U}$,
\[
D_s(\tilde{x^j}) = \left( \sum_i a^i \frac{\pd}{\pd x^i}\right)(\tilde{x^j}) = \sum_i a^i \frac{\pd x^j}{\pd x^i} = a^j,
\] 
so each $a^j$ is $C^\infty$ on $\tilde{U}$. Therefore $s$ is a smooth section, so we can conclude that $\Phi(s)$ makes sense and equals $D$. So $\Phi$ is surjective.
\end{proof}
\begin{corollary}
A section $X$ is $C^\infty$ if and only if $X(f) \in C^\infty(M)$ for every $f \in C^\infty(M)$.
\end{corollary}
We shall hereafter denote by $\mathfrak{X}(M)$ the set of smooth vector fields on $M$. 

\subsection{Integral Flows}

We shall begin the study of ordinary differential equations on manifolds. Everything that happens in $\R^n$ locally should also happen on manifolds, since manifolds are locally modelled by patches of $\R^n$. Therefore it is reasonable to try and generalize differential equations to manifolds. We begin with some definitions.
\begin{definition}
Let $X \in \mathfrak{X}(M)$. An integral curve of $X$ is a $C^\infty$ curve $c : (a,b) \to M$ such that for each $t$, $c'(t) = X_{c(t)}$. We say that the curve starts at $p$ if $c(0) = p$, and we say that it is maximal if its domain may not be extended.
\end{definition}
Let $(U, x^1, \dots, x^n)$ be a chart at $p$. Suppose $c : (a,b) \to M$ is an integral curve for $X$ starting at $p$. Then, in $U$, if $a^1, \dots, a^n$ are the components of $X$, we have
\[
c'(t) = X_{c(t)} = \sum_i (a^i \circ c)(t) \left. \frac{\pd }{\pd x^i} \right|_{c(t)}.
\]
If $\dot{c}^i(t)$ denotes the ordinary calculus derivative of the function $x^i \circ c$ at $t$, then we can write
\[
c'(t) = \sum_i \dot{c}^i(t) \left. \frac{\pd}{\pd x^i} \right|_{c(t)}.
\]
Therefore we have a system of ODEs
\begin{align*}
(x^1 \circ c)'(t) &= (a^1 \circ c)(t) \\
&\vdots \\
(x^n \circ c)'(t) &= (a^n \circ c)(t) \\
c(0) &= p
\end{align*}
Let us now recall some theorems about ODEs.
\begin{theorem}
(Existence and Uniqueness) Let $V \subseteq \R^n$ be open and $f : V \to \R^n$ a $C^\infty$ function. Then the differential equation
\[
\begin{cases} 
\frac{dy}{dt} = f(y) \\
y(0) = p_0
\end{cases}
\]
has a unique $C^\infty$ solution $y : (a(p_0), b(p_0)) \to V$ defined on a maximal open interval.
\end{theorem}
Note that the function $f$ here is basically a vector field. The corresponding section is $V \to TV$, $x \mapsto (x,f(x))$, where we identify $TV$ with $V \times \R^n$. We do not have the luxury of doing this on manifolds. The uniqueness condition simply means that if $z : (-\epsilon_1, \epsilon_2) \to V$ is another solution, then $z$ and $y$ agree on the interval of existence of $z$; the maximality condition ensures that this interval of existence is no larger than that of $y$.

A direct corollary of the existence and uniqueness theorem for ODEs in $\R^n$ is the following:
\begin{corollary}
If $U \subseteq M$ is a coordinate neighbourhood and $p \in U$ and $X \in \mathfrak{X}(U)$, then there is a unique maximal integral curve of $X$ in $U$ starting at $p$.
\end{corollary}

It is natural to ask what happens when we let the initial point vary. We would expect that if the time is fixed and we vary the initial point, the result varies smoothly. This result is, in fact, true.

\begin{theorem}
(Smooth dependence on initial conditions) Let $V \subseteq \R^n$ be open and $f : V \to \R^n$ be $C^\infty$. For each $p_0 \in V$ there is an open neighbourhood $W \subseteq V$ of $p$, an $\epsilon > 0$, and a $C^\infty$ function $y : (-\epsilon, \epsilon) \times W \to V$ such that
\[
\frac{\pd y}{\pd t}(t, q) = f(y(t,q))
\]
and $y(0,q) = q$ for all $(t,q) \in (-\epsilon,\epsilon) \times W$.
\end{theorem}

It follows that if $U$ is a coordinate neighbourhood in $M$ and $X \in \mathfrak{U}$
, then for any $p \in U$ there is an open neighbourhood $W$ of $p$ in $U$, an $\epsilon > 0$, and a $C^\infty$ map $F : (-\epsilon, \epsilon) \times W \to U$ such that for each $q \in W$, the curve $F(t, q)$ is an integral curve of $X$ in $U$ starting at $q$. We will sometimes write $F_t(q)$ to mean $F(t,q)$.

We would like it to be true that $F_t(F_s(q)) = F_{t+s}(q)$, whenever these make sense. If we visualize a curve and a point $q$ on the curve, then this means that moving for $t+s$ time units is the same as moving for $s$ times units and then moving for $t$ time units. Fortunately, this is true; it is a very simple consequence of uniqueness. If $s$ is fixed, then both $F_t(F_s(q))$ and $F_{t+s}(q)$ are integral curves of $X$ starting at $F_s(q)$.

We now make many more definitions.
\begin{definition}
The map $F$ above is called the local flow generated by $X$. The curve $t \mapsto F_t(q)$ is called the flow line. If $F$ is defined on $\R \times M$, then $F$ is called a global flow. A vector field that admits a global flow is said to be complete.
\end{definition}
Not every vector field admits a global flow, as we know from our ODEs class. For example, the ODE $\frac{dx}{dt} = x^2$ with initial condition $x(0) = x_0 \in \R$ with $x_0 \neq 0$ has solution $x(t) = \frac{x_0}{1-tx_0}$, which does not exist everywhere.

If $F$ is a global flow, then for every $t \in \R$ we have $F_t^{-1} = F_{-t}$, which is easy to check. We therefore have a diffeomorphism $F_t : M \to M$ for each $t \in \R$, which can be thought of as sending every point to its position after "flowing for $t$ time units". We generalize this slightly.
\begin{definition}
Let $\mathrm{Diff(M)}$ denote the group of all homomorphisms of a smooth manifold $M$ under composition. A group homomorphism $G : \R \to \mathrm{Diff}(M)$ is called a one-parameter group of diffeomorphisms of $M$.
\end{definition}
So, of course, a global flow is an example of a one-parameter group of diffeomorphisms of a manifold. 

We shall now define what it means to be a local flow independently of any vector field.

\begin{definition}
A local flow about $p$ in an open set $U$ of a manifold is a $C^\infty$ map $F : (-\epsilon, \epsilon) \times W \to V$, where $\epsilon > 0$ and $W \subseteq U$ is an open neighbourhood of $p$, such that
\begin{enumerate}[(i)]
\item
$F_0(q) = q$ for all $q \in W$.
\item
$F_t(F_s(q)) = F_{t+s}(q)$ whenever both sides are defined.
\end{enumerate}
\end{definition}

If we have a local flow $F(t,q)$ as defined above then we may recover the vector field $X$ of which $F$ is a local flow by observing that
\[
F(0,q) = q \qquad \text{ and } \qquad \frac{\pd F}{\pd t}(0, q) = X_{F(0,q)} = X_q.
\]

Let us consider an example. The function $F : \R \times \R^2 \to \R^2$ defined by
\[
F \left( t, \begin{bmatrix}
x \\ y
\end{bmatrix} \right) := \begin{bmatrix}
\cos t & -\sin t \\
\sin t & \cos t
\end{bmatrix}\begin{bmatrix}
x \\ y
\end{bmatrix}
\]
is the global flow on $\R^2$ generated by the vector field 
\[
X = -y \frac{\pd}{\pd x} + x \frac{\pd}{\pd y},
\]
since
\[
\frac{\pd F}{\pd t}(t,(x,y)) =\begin{bmatrix}
0 & -1 \\ 1 & 0
\end{bmatrix}\begin{bmatrix}
x \\ y
\end{bmatrix}= \begin{bmatrix}
-y \\ x
\end{bmatrix} = -y \frac{\pd}{\pd x} + x \frac{\pd}{\pd y}.
\]
\subsection{The Lie Derivative}
We defined smooth functions on manifolds and gave a notion of the directional derivative using tangent vectors. Since we have defined smooth vector fields, can we develop a notion of the derivative of a vector field? Or can we develop a "directional derivative" of a vector field? This is what we attempt to do.

Let $X,Y$ be vector fields. In calculus we define the derivative of a real valued function as
\[
f'(p) = \lim_{t \to 0} \frac{f(p+t) - f(p)}{t},
\]
assuming it exists. We cannot readily generalize this to manifolds because for distinct nearby points $p,q$ in a manifold, the elements of $T_pM$ and $T_qM$ cannot be compared. We can get around this by using the local flow of another vector field $X$ to "transport" $Y_q \in T_qM$ to $T_pM$. 

Let $F$ be the local flow of $X \in \mathfrak{X}(M)$ starting at $p$. Because of the identity $F_t \circ F_s = F_{t+s}$ when they make sense, every $F_t$ is a diffeomorphism onto its image with inverse $F_{-t}$. The following definition therefore makes sense.

\begin{definition}
For $X,Y \in \mathfrak{X}(M)$ and $p \in M$, let $F$ be a local flow of $X$ on a neighbourhood of $p$. Define the Lie derivative of $Y$ with respect to $X$ at $p$ to be the vector
\[
(\mathcal{L}_XY)_p := \lim_{t \to 0} \frac{F_{-t *}(Y_{F_t(p)}) - Y_p}{t},
\]
if the limit exists.
\end{definition}
\begin{figure}[H]
\centering
\includegraphics[width = 0.8\textwidth]{lie_derivative_1.jpg}
\caption{Transporting $Y_{F_t(p)}$ along the flow to $T_pM$ to be compared with $Y_p$.}
\end{figure}

%pushforward of vector fields part may need elaboration
Actually, since each $F_{-t}$ is a diffeomorphism onto its image, we can push vector fields forward and rewrite this as
\[
(\mathcal{L}_XY)_p = \lim_{t \to 0} \frac{(F_{-t *} Y)_p - Y_p}{t}  = \left. \frac{d}{dt} \right|_{t=0} (F_{-t*}Y)_p,
\]
which shows that a sufficient condition for the Lie derivative $(\mathcal{L}_XY)_p$ to exist is that $\{ F_{-t*}Y \}$ be a smooth family of vector fields on $M$.

\begin{theorem}
If $X,Y \in \mathfrak{X}(M)$, then $\mathcal{L}_XY \in \mathfrak{X}(M)$.
\end{theorem}

\begin{figure}[H]
\centering
\includegraphics[width = 0.9\textwidth]{lie_derivative_2.jpg}
\caption{}
\end{figure}
Consider the figure above. Define 
\[
\gamma(t) = (F_t \circ G_t \circ F_{-t} \circ G_{-t})(p),
\]
which we may think of as taking the point $p$ and travelling clockwise around the "square". Do we end up back at the point $p$? The answer is, in general, "no"; however, we can talk about this using the Lie derivative. 

We can also think of this scenario in terms of pushing forward vectors. Considering the diagram, think of a tangent vector $v \in T_pM$. We can push it forward by $G_{-t}$ to get a tangent vector at the "bottom right corner" of the square. Then we can push that forward by $F_{-t}$ to get one at the "bottom left". And that by $G_{t}$. And that by $F_t$. Do we arrive at the same tangent vector $v$? Not always; this difference is something that the Lie derivative measures. (One affirmative case is the origin and the coordinate vector fields on $\R^2$, as one can check). We have a theorem.

\begin{theorem}
\begin{enumerate}
\item $\gamma'(0) = 0$ and $\frac{1}{2} \gamma''(0) = \mathcal{L}_XY|_p$.
\item If $\mathcal{L}_XY \equiv 0$, then $\gamma(t) = p$ for all $t$.
\end{enumerate}
\end{theorem}

Later on we will see that we have
\[
\mathcal{L}_XY = [X,Y] = XY - YX,
\]
so the Lie derivative isn't actually anything new.

\end{document}
  

  