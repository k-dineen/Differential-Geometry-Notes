\documentclass[11pt]{article}
\usepackage[utf8]{inputenc}
\usepackage{amsmath, amsthm, amssymb, amsfonts, mathtools, tikz-cd, float}
\usepackage[left=2.5cm,right=2.5cm]{geometry}
\usepackage[shortlabels]{enumitem}

\newcommand{\Int}{\mathrm{Int}}
\newcommand{\R}{\mathbb{R}}
\newcommand{\Z}{\mathbb{Z}}
\newcommand{\pd}{\partial}
\renewcommand{\epsilon}{\varepsilon}
\renewcommand{\hat}{\widehat}
\renewcommand{\tilde}{\widetilde}
\newcommand{\supp}{\mathrm{supp}}

\newtheorem{theorem}{Theorem}[section]
\newtheorem{corollary}{Corollary}[theorem]
\newtheorem{lemma}[theorem]{Lemma}
\newtheorem{proposition}{Proposition}[section]

\newtheorem{definition}{Definition}

\pagestyle{myheadings}

\begin{document}

\section{Bump Functions and Partitions of Unity (Additional Reading)}

\subsection{Bump Functions}

Define $f : \R \to \R$ by
\[
f(t) = \begin{cases} 
e^{-1/t} & t > 0 \\
0 & t \leq 0
\end{cases}.
\]
This will be the basic function off of which our bump functions will be modelled. The construction of bump functions on manifolds proceeds in four steps. 
\begin{lemma}
The function $f$ defined above is $C^\infty$.
\end{lemma}
\begin{proof}
We claim that for $t > 0$ and $k \geq 0$, there is a polynomial $p_{2k}$ of degree $2k$ such that $f^{(k)}(t) = p_{2k}(1/t)e^{-1/t}$. For $k = 0$ this is obvious, so suppose that this holds true for some $k \geq 0$. Then we have
\begin{align*}
f^{(k+1)}(t) &= \frac{d}{dt} p_{2k} \left( \frac{1}{t} \right) e^{-1/t} \\
&= -\frac{1}{t^2} p_{2k}' \left( \frac{1}{t} \right) e^{-1/t} + \frac{1}{t^2}p_{2k} \left( \frac{1}{t} \right)e^{-1/t} \\
&= \underbrace{\left[ -\frac{1}{t^2} p_{2k}' \left( \frac{1}{t} \right) + \frac{1}{t^2}p_{2k} \left( \frac{1}{t} \right) \right]}_{p_{2(k+1)}(1/t)} e^{-1/t},
\end{align*}
so by induction the claim holds true.

(Finish this. Since $f$ is $C^\infty$ on $\R \setminus 0$, all we need to do is show that each $f^{(k)}(0)$ makes sense and is equal to $0$, by induction.)
\end{proof}

\begin{lemma}
Given real numbers $r_1 < r_2$, there is a $C^\infty$ function $h : \R \to \R$ such that $h^{-1}(1) = (-\infty, r_1]$, $h^{-1}(0) = [r_2, \infty)$, and $0 < h(t) < 1$ for $t \in (r_1, r_2)$.
\end{lemma}
\begin{proof}
Taking $f$ as defined above, define
\[
h(t) = \frac{f(r_2 - t)}{f(r_2 - t) + f(t - r_1)}.
\]
This function is well-defined because $f(r_2 - t) + f(t - r_1) = 0$ if and only if each is zero, which is true if and only if $t \geq r_2$ and $t \leq r_1$, which is clearly impossible. Then $h$ is $C^\infty$. It is clear that $0 < h(t) < 1$ for $t \in (r_1, r_2)$, so we are left with checking the other two conditions.

$h(t) = 0$ if and only if $f(r_2 - t) = 0$, which holds if and only if $r_2 - t \leq 0$, which holds if and only if $t \geq r_2$. So $h^{-1}(0) = [r_2, \infty)$.

$h(t) = 1$ if and only if $f(t - r_1) = 0$, which holds if and only if $t - r_1 \leq 0$, which holds if and only if $t \leq r_1$. So $h^{-1}(1) = (\infty, r_1]$.
\end{proof}

\begin{lemma}
Given real numbers $0 < r_1 < r_2$, there is a $C^\infty$ function $H : \R^n \to \R$ such that $H^{-1}(1) = \overline{B_{r_1}(0)}$, $H^{-1}(0) = \R^n \setminus B_{r_2}(0)$, and $0 < H(x) < 1$ for $x \in B_{r_2}(0) \setminus \overline{B_{r_1}(0)}$.
\end{lemma}
\begin{proof}
With $h$ as in the previous lemma, define $H(x) = h(\|x\|)$. The function $H$ is $C^\infty$ because when $\|x\| < r_1$, it is identically $1$, and it is a composition of $C^\infty$ maps away from the origin. The rest of the lemma is clear.
\end{proof}

Using this, we now work in a coordinate chart to get a bump function on a manifold.

\begin{theorem}
(Existence of bump functions)
Given a smooth manifold $M$, a point $q \in M$, and a neighbourhood $U$ of $q$, there exists a $\rho \in C^\infty(M)$ such that $\rho |_V \equiv 1$ on some neighbourhood $V \subseteq U$ of $q$, and $\supp(\rho) \subseteq U$.
\end{theorem}
\begin{proof}
Choose a coordinate chart $(W, \phi)$ at $q$ such that $\phi(q) = 0$. The set $\phi(W \cap U)$ is an open neighbourhood of the origin, so we can find $0 < r_1 < r_2$ such that 
\[
0 = \phi(q) \in B_{r_1}(0)  \subset B_{r_2}(0) \subset \phi(W \cap U).
\]
This implies that
\[
q \in \phi^{-1}(B_{r_1}(0)) \subset \phi^{-1}(B_{r_2}(0)) \subset W \cap U \subseteq U
.\]
Let $H$ be as in the previous lemma. Define 
\[
\rho(x) = \begin{cases} 
(H \circ \phi)(x) & x \in U \cap W \\
0 & x \not\in U \cap W
\end{cases}.
\]
The function $\rho$ is $C^\infty$ on $U \cap W$ because it is a composition of $C^\infty$ functions on an open set. If $x \not\in U \cap W$, then $x \not\in \phi^{-1}(\overline{B_{r_2}(0)})$, so we can find a neighbourhood of $x$ on which $\rho$ is identically zero. (Finish this.)
\end{proof}

\end{document}
  

  