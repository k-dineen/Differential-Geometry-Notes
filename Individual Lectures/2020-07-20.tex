\documentclass[11pt]{article}
\usepackage[utf8]{inputenc}
\usepackage{amsmath, amsthm, amssymb, amsfonts, mathtools, tikz-cd, float}
\usepackage[left=2.5cm,right=2.5cm]{geometry}
\usepackage[shortlabels]{enumitem}
\usepackage{cancel}

\newcommand{\Int}{\mathrm{Int}}
\newcommand{\R}{\mathbb{R}}
\newcommand{\Z}{\mathbb{Z}}
\newcommand{\pd}{\partial}
\renewcommand{\epsilon}{\varepsilon}
\renewcommand{\hat}{\widehat}
\renewcommand{\tilde}{\widetilde}
\newcommand{\supp}{\mathrm{supp}}

\newtheorem{theorem}{Theorem}[section]
\newtheorem{corollary}{Corollary}[theorem]
\newtheorem{lemma}[theorem]{Lemma}
\newtheorem{proposition}{Proposition}[section]

\newtheorem{definition}{Definition}

\pagestyle{myheadings}

\begin{document}

\section{One Forms (July 20)}

\subsection{Motivation, Einstein Notation}

\begin{definition}
Given a point $p$ on the smooth manifold $M$, the dual space to $T_pM$ is called the cotangent space at $p$ and is denoted by $T_p^*M$. Its elements are called covectors (at $p$).
\end{definition}

Recall that $T_p^*M = \mathrm{Hom}(T_pM, \R)$ is an $n$-dimensional vector space over $\R$.

We define a \emph{covector field} on $M$ to be a choice of a covector at each point, similarly to how a vector field is a choice of a (tangent) vector at each point. Since the cotangent spaces are the duals to the tangent spaces, many familiar constructions regarding the tangent space can be done for the cotangent space. In particular, we will develop a vector bundle called the \emph{cotangent bundle}, and give it a topology and smooth structure so that we can talk about a \emph{smooth covector field}, which we may also call a \emph{(differential) $1$-form}.

We now detour into the world of Einstein notation, whose benefits should become obvious after its definition is given.
\begin{enumerate}
\item
Indices should be placed as follows:
\begin{alignat*}{2}
&\text{vectors: } \, v_i \qquad &&\text{coefficients of vectors: } \, a^i \\
&\text{covectors: } \theta^i \qquad &&\text{coefficients of covectors: } b_i
\end{alignat*}
The main idea is that if the index $i$ appears twice, once up and once down, and if you are summing over that index, then the object in question is basis-independent.

For example, let $v_1, \dots, v_n$ be a basis of $T_pM$ and $\theta^1, \dots, \theta^n$ the dual basis. Then we may write fixed elements $v \in T_pM$, $\theta \in T_p^*M$ as linear combinations $v = \sum a^i v_i$ and $\theta = \sum b_i \theta^i$. Consider the following three objects:
\[
\sum a^i b_i \qquad \sum a^i \theta^i \qquad \sum b_i v_i.
\]
The first object is basis-independent because it is simply $\theta(v)$, and in this one there is a summation with the same index appearing once above and once below: a simple calculation gives
\[
\theta(v) = \sum_i b_i \theta^i \left( \sum_j a^j v_j \right) = \sum_{i,j} a^j b_i \theta^i(v_j) = \sum_{i,j} a^jb_i\delta^j_i = \sum_i a^ib_i.
\]
The other two quantities are not basis-independent, which can be seen by, for example, considering their representations in the bases $-v_1, \dots, -v_n$ and $-\theta^1, \dots, -\theta^n$.

For one more example of a basis-dependent object, consider the inner product $\langle  v,w \rangle = \sum a^i c^i$, where $w = \sum c^i v_i$. This very much depends on the choice of basis, and, as we can see, the index $i$ appears twice above.

\item
If an index appears twice, once up and once down, then the summation symbol is omitted and it is understood that a summation is occurring over all possible values of that index.

For example, in our old notation we would write $v = \sum a^i v_i$, but now we just write $v = a^i v_i$ and it is understood that $i$ is being summed from $1$ to $n$.
\end{enumerate}
For another example, the coordinates $x^1, \dots, x^n$ on $\R^n$ are written with upper indices because they give the coefficients of vectors.

\subsection{The Cotangent Bundle}

The construction of the cotangent bundle is almost identical to that of the tangent bundle, so we will be very brief.

\begin{definition}
The cotangent bundle is the disjoint union $T^*M := \cup T_p^*M$. It comes with a projection map $\pi : T^*M \to M$, sending a covector at $p$ to the point $p$.
\end{definition}

We topologize $T^*M$ as follows: let $(U, \phi) = (U, x^1, \dots, x^n)$ be a coordinate chart on $M$. Let $\{ \lambda_p^1, \dots, \lambda_p^n \} $ be the dual basis to $\{ \left. \frac{\pd}{\pd x^1} \right|_p, \dots, \left. \frac{\pd}{\pd x^i} \right|_p \}$, for each $p \in U$. We define $\tilde{\phi} : T^*U \to \phi(U) \times \R^n$ by $\tilde{\phi}(c_i \lambda_p^i) := (x^1(p), \dots, x^n(p), c_1, \dots, c_n)$. (We are using Einstein notation here.) Then $\tilde{\phi}$ is bijective.

Give to $T^*U$ the unique topology which makes $\tilde{\phi}$ a homeomorphism. Define $\tau \subseteq \mathcal{P}(T^*M)$ by
\[
\tau = \{ A : A \cap T^*U \text{ is open in } U \text{ for all coordinate open sets } U \}.
\]
Then $\tau$ is a topology on $T^*M$. One checks that this makes $T^*M$ into a second-countable Hausdorff topological space.

Consider the collection of all coordinate charts $\{ (T^*U, \tilde{\phi}) \}$ on $T^*M$. This collection makes $T^*M$ into a $2n$-dimensional topological manifold, and can easily be seen to form a $C^\infty$ atlas on $T^*M$, making $T^*M$ into a $2n$-dimensional smooth manifold.

\textbf{Remark: } $TM$ and $T^*M$ are obviously locally diffeomorphic by construction, but are they diffeomorphic? The answer is yes, and one can see this easily after developing the notion of a Riemannian metric.

\subsection{One-Forms}

Having produced a smooth structure on the cotangent bundle, we now proceed with the promised definition of a covector field. 

\begin{definition}
A smooth covector field is a smooth section of the cotangent bundle (i.e. a smooth right inverse of $\pi$). A smooth section of $T^*M$ is also called a (differential) $1$-form.
\end{definition}

Let $f \in C^\infty(M)$. We define a $1$-form $df$ as follows: $(df)_p(v) = v(f)$, where $v \in T_pM$. For each $p$, $(df)_p$ is linear, so $df$ is a $1$-form. Alternatively, $(df)_p(X_p) = X_p([f])$.

\begin{proposition}
Let $f \in C^\infty(M)$. For $p \in M$ and $X_p \in T_pM$, 
\[
f_{*,p}(X_p) = (df)_p(X_p) \left. \frac{d}{dt} \right|_{f(p)}.
\]
\end{proposition}
\begin{proof}
If $f_{*,p}(X_p) = a \left. \frac{d}{dt} \right|_{f(p)}$, then
\[
a = f_{*,p}(X_p)(\mathrm{Id}) = X_p(\mathrm{Id} \circ f) = X_p(f) = (df)_p(X_p).
\]
\end{proof}
Under the usual identification of $T_{f(p)}\R$ with $\R$, this agrees with the previous notion of the differential. We therefore also call $df$ the \emph{differential} of $f$. 

If $(U, x^1, \dots, x^n)$ is a coordinate chart on $M$, then what can we say about $dx^1, \dots, dx^n$? If $p \in U$, then
\[
(dx^i)_p \left( \left. \frac{\pd}{\pd x^j} \right|_p \right) = \left. \frac{\pd x^i}{\pd x^j} \right|_p = \delta^i_j,
\]
implying that $(dx^1)_p, \dots, (dx^n)_p$ is the dual basis to $\left. \frac{\pd}{\pd x^i}\right|_p, \dots, \left. \frac{\pd}{\pd x^n} \right|_p$. Suppose $df = c_i dx^i$ on $U$. We have
\[
c_k = (c_i dx^i) \left( \frac{\pd}{\pd x^k} \right) = df \left( \frac{\pd}{\pd x^k} \right) = \frac{\pd f}{\pd x^k}.
\]
Therefore $df = \frac{\pd f}{\pd x^i} dx^i$ on $U$. In particular, $df$ is a smooth $1$-form as defined earlier, since in local coordinates it is the map
\[
(x^1, \dots, x^n) \mapsto \left(x^1, \dots, x^n, \frac{\pd f}{\pd x^1}, \dots, \frac{\pd f}{\pd x^n} \right).
\]
We state these facts formally as a proposition.
\begin{proposition}
If $f \in C^\infty(M)$, then the $1$-form $df$ defined by $(df)_p(X_p) = X_p(f)$ is a smooth $1$-form on $M$. Moreover, if $x^1, \dots, x^n$ are local coordinates on $M$, then, in these coordinates, $df$ is of the form 
\[
df = \frac{\pd f}{\pd x^i} dx^i.
\]
\end{proposition}

\subsection{Equivalent Notions of Smoothness for 1-forms}

Denote by $\Omega^1(M)$ the space of all smooth $1$-forms on $M$. This becomes a vector space over $\R$ by pointwise addition and scalar multiplication of covectors. It also becomes a $C^\infty(M)$-module as follows: if $\omega \in \Omega^1(M)$ and $f \in C^\infty(M)$, let $f\omega$ be the $1$-form defined by $(f\omega)_p(X_p) = f(p)\omega_p(X_p)$.

Recall the following characterization of smooth vector fields: 
\begin{proposition}
Let $X$ be a section of $TM$. The following are equivalent:
\begin{enumerate}
\item $X : M \to TM$ is $C^\infty$.
\item For any coordinate chart $(U, x^1, \dots, x^n)$, $X = a^i \frac{\pd}{\pd x^i}$ on $U$ for some $a^1, \dots, a^n \in C^\infty(U)$.
\item $X$ takes $C^\infty$ functions to $C^\infty$ functions.
\end{enumerate}
\end{proposition}

We will develop something similar for $1$-forms.

\begin{proposition}
Let $\omega$ be a section of $T^*M$. Then $\omega$ is $C^\infty$ if and only if for every chart $(U, x^1, \dots, x^n)$ on $M$, $\omega = c_i dx^i$ on $U$ for some $c_1, \dots, c_n \in C^\infty(U)$.
\end{proposition}
\begin{proof}
In local coordinates, $\omega$ sends $(x^1, \dots, x^n)$ to $(x^1, \dots, x^n, c_1, \dots, c_n)$, so $\omega$ is smooth on $U$ if and only if the $c_1, \dots, c_n$ are smooth.
\end{proof}
A $1$-form will not take functions to functions, so we cannot directly translate the third smoothness criterion for vector fields into one for covector fields. Instead, a covector field takes vector fields to functions, as we now define.

\begin{definition}
Let $\omega$ be a $1$-form on $M$. If $X$ is a vector field on $M$, define $\omega(X) : M \to \R$ by $\omega(X)(p) = \omega_p(X_p)$.
\end{definition}
Immediately we have the following property, which is a consequence of linearity of $\omega_p$ for each $p$.
\begin{proposition}
Let $\omega$ be a $1$-form on $M$. Then $\omega$ as a map from $\mathfrak{X}(M)$ to functions on $M$ is $C^\infty(M)$-linear (i.e. linear with respect to the module structure on $\mathfrak{X}(M)$).
\end{proposition}
It is surprising that this property is also sufficient for $\omega$ to be a $1$-form, assuming that $\omega$ takes smooth vector fields to smooth functions, which we state and prove in (2) of the following theorem.

\begin{theorem}
\begin{enumerate}
\item
Let $\omega$ be a $1$-form. Then $\omega \in \Omega^1(M)$ if and only if $\omega$ takes smooth vector fields to smooth functions.

\item
Every $C^\infty(M)$-linear map $\omega : \mathfrak{X}(M) \to C^\infty(M)$ is, under the obvious identification, a $1$-form.
\end{enumerate}
\end{theorem}
\begin{proof}
\begin{enumerate}
\item
Suppose $\omega \in \Omega^1(M)$. Given $X \in \mathfrak{X}(M)$, let $(U, x^1, \dots, x^n)$ be a coordinate chart on $M$. Write $X = X^i \frac{\pd}{\pd x^i}$ on $U$ and $\omega = a_i dx^i$ on $U$. Then
\[
\omega(X) = (a_i dx^i)X = a_i dx^i(X) = a_i X^i
\]
on $U$. Since the functions $a_i, X^i$ are all smooth on $U$, $\omega(X)$ is smooth on $U$. Hence $\omega(X) \in C^\infty(M)$.

Conversely, suppose $\omega$ takes smooth vector fields to smooth functions. Let $(U, x^1, \dots, x^n)$ be a coordinate chart on $M$. Write $\omega = a_i dx^i$ on $U$. Extend $\frac{\pd}{\pd x^k}$ to $\tilde{X_k} \in \mathfrak{X}(M)$ using a bump function supported in $U$ which is identically $1$ on some neighbourhood $W_j$ of $p \in U$. Then, on $W_j$,
\[
\omega(\tilde{X}_k) = (a_i dx^i)\left( \frac{\pd}{\pd x^k} \right) = a_k,
\]
implying that $a_k \in C^\infty(W_j)$. If $W = W_1 \cap \cdots \cap W_n$, then $\omega$ is $C^\infty$ on $W$. Therefore $\omega \in \Omega^1(M)$.

\item
Suppose $\omega : \mathfrak{X}(M) \to C^\infty(M)$ is $C^\infty(M)$-linear. We claim that $\omega(X)(p)$ depends only on $X_p$. By linearity we may, without loss of generality, assume that $X_p = 0$. Let $(U, x^1, \dots, x^n)$ be a coordinate chart on $M$ at $p$, and write $X = a^i \frac{\pd}{\pd x^i}$ on $U$. By considering a bump function $\psi$ supported in $U$ with $\psi(p) = 1$, extend each $a^i$ and $\frac{\pd}{\pd x^i}$ to all of $M$ by considering $\psi a^i$ and $\psi \frac{\pd}{\pd x^i}$. Then these extensions are smooth, and
\[
\psi^2 \omega(X) = \omega(\psi^2 X) = \omega \left( \psi a^i \psi \frac{\pd}{\pd x^i} \right) = \psi a^i \omega \left( \psi \frac{\pd}{\pd x^i} \right).
\]
Evaluation at $p$ gives $\omega(X)(p) = 0$.

Now, define, for each $p \in M$, $\omega_p : T_pM \to \R$ by $\omega_p(v) = \omega(X)(p)$, where $X$ is any smooth vector field on $M$ with $X_p = v$. The previous paragraph shows that this is well-defined. Then the $1$-form $\omega : M \to T^*M$ just defined has the same action on smooth vector fields as the $\omega$ we started with.
\end{enumerate}
\end{proof}

By (2) of the preceding theorem, we may identify $\Omega^1(M)$ with the set of all $C^\infty(M)$-linear maps $\mathfrak{X}(M) \to C^\infty(M)$. Continuing with the usual trend of abusing notation the highest degree, we will often consider these sets to be equal.

\subsection{Pullbacks of 1-forms}

Let $F : N \to M$ be $C^\infty$. Denote by $F^{*,p} : T_{F(p)}^*M \to T_p^*N$ the dual map to $F_{*,p}$. (This is the notation that was used in lecture.) If $\theta \in T_{F(p)}^*M$, then $F^{*,p}(\theta) = \theta \circ F_{*,p} \in T_p^*N$. If we have a $1$-form, then we can do this at every point.

Note that while vector fields cannot be pushed forward, in general, covector fields may always be pulled back. Let $\omega$ be a $1$-form on $M$. Define $F^*\omega$ by $(F^*\omega)_p := F^{*,p}(\omega_{F(p)}) = \omega_{F(p)} \circ F_{*,p}$.

We have the following properties of the pullback operation on $1$-forms, whose proofs are straightforward computations.
\begin{proposition}
\begin{enumerate}
\item
$F^*(g\omega) = (F^*g)F^*\omega$.
\item
If $g \in C^\infty(M)$, $F^*(dg) = d(F^*g)$.
\item
$F^* : \Omega^1(M) \to \Omega^1(N)$ is $\R$-linear. (Note that (1) ensures that $F^*$ will not, in general, be $C^\infty(M)$-linear.)
\end{enumerate}
\end{proposition}

\end{document}
  

  