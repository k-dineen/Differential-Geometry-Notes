\documentclass[11pt]{article}
\usepackage[utf8]{inputenc}
\usepackage{amsmath, amsthm, amssymb, amsfonts, mathtools, tikz-cd, float}
\usepackage[left=2.5cm,right=2.5cm]{geometry}
\usepackage[shortlabels]{enumitem}
\usepackage{cancel}

\newcommand{\Int}{\mathrm{Int}}
\newcommand{\R}{\mathbb{R}}
\newcommand{\Z}{\mathbb{Z}}
\newcommand{\pd}{\partial}
\renewcommand{\epsilon}{\varepsilon}
\renewcommand{\hat}{\widehat}
\renewcommand{\tilde}{\widetilde}
\newcommand{\supp}{\mathrm{supp}}
\newcommand{\sgn}{\mathrm{sgn}}

\newtheorem{theorem}{Theorem}[section]
\newtheorem{corollary}{Corollary}[theorem]
\newtheorem{lemma}[theorem]{Lemma}
\newtheorem{proposition}{Proposition}[section]

\newtheorem{definition}{Definition}

\pagestyle{myheadings}

\begin{document}

\section{The Exterior Derivative of a $k$-form (July 29)}

\subsection{Motivating the Local Definition}

\begin{definition}
An antiderivation on a graded algebra $A = \bigoplus_k A^k$ is an $\R$-linear map $D : A \to A$ such that
\[
D(\omega \cdot \tau) = D(\omega) \cdot \tau + (-1)^k \omega \cdot D(\tau) \qquad \text{ whenever } \omega \in A^k.
\]
An element $\omega \in A^k$ is said to be (homogeneous) of degree $k$. The antiderivation $D$ is said to be of degree $m$ if $\deg(D(\omega)) = \deg(\omega) + m$ for all $\omega \in A$.
\end{definition}

Recall that $\Omega^*(M) = \bigoplus_k \Omega^k(M)$. The exterior derivative $d$ on $\Omega^*(M)$ that we wish to define will be an antiderivation of degree $1$.

On $0$-forms, we defined $d$ in a coordinate-independent way as $d : f \mapsto (X \mapsto X(f))$. Alternatively, we could have defined it in each coordinate chart and showed that the definition is independent of the chart. Specifically, if $(U, x^i)$ is a coordinate chart on $M$, we can define $df$ on $U$ as the $1$-form $df = \frac{\pd f}{\pd x^i} dx^i$. To show that this is independent of the coordinate chart, let $(V, y^i)$ be another chart with $U \cap V \neq \emptyset$. Then, on $U \cap V$,
\[
\frac{\pd f}{\pd x^i} dx^i = \frac{\pd f}{\pd x^i}  \frac{\pd x^i}{\pd y^j} dy^j  = \frac{\pd f}{\pd y^j} dy^j,
\]
which shows that the local definition of $df$ is coordinate-independent. We can actually define the exterior derivative $d$ locally in the same manner for forms of higher degree, and show that our definition is independent of the chart we defined it in.

\subsection{Defining $d$ Locally For $1$-forms}

Some time ago, we asked when a $1$-form $\omega$ was expressible as $df$ for some $0$-form $f$. A sufficient condition for \emph{local exactness} is that $\frac{\pd \omega_i}{\pd x^j} - \frac{\pd \omega_j}{\pd x^i} = 0$ on every chart. This condition is also expressible in a coordinate independent manner: 
\[
\frac{\pd \omega_i}{\pd x^j} - \frac{\pd \omega_j}{\pd x^i} = 0 \iff X(\omega(Y)) - Y(\omega(X)) = \omega([X, Y]) \qquad \text{ for all } X, Y \in \mathfrak{X}(U).
\]
The left side in the above equivalence is antisymmetric in $i$ and $j$, so we might hope that if we define
\[
d\omega \coloneqq \sum_{i < j} \left( \frac{\pd \omega_i}{\pd x^j} - \frac{\pd \omega_j}{\pd x^i} \right) dx^i \wedge dx^j,
\]
then this $2$-form would be independent. Note that this is equivalent to saying that $d\omega = d\omega_i \wedge dx^i$. The following lemma shows that this is indeed a coordinate-independent definition.

\begin{lemma}
Let $(U, x^i)$ and $(V, y^i)$ be coordinate charts on $M$ such that $\omega = a_i dx^i = b_i dy^i$ on $U \cap V$. Then $da_i \wedge dx^i = db_i \wedge dy^i$ on $U \cap V$.
\end{lemma}
\begin{proof}
On $U \cap V$,
\begin{align*}
da_i \wedge dx^i &= \frac{\pd a_i}{\pd x^j} dx^j \wedge dx^i \\
&= \frac{\pd}{\pd x^j} \left( \omega \left( \frac{\pd}{\pd x^i} \right) \right) dx^j \wedge dx^i \\
&= \frac{\pd}{\pd x^j} \left( b_k dy^k \left( \frac{\pd}{\pd x^i} \right) \right) dx^j \wedge dx^i \\
&= \frac{\pd}{\pd x^j} \left( b_k  \frac{\pd y^k}{\pd x^i}  \right) dx^j \wedge dx^i \\
&=  \left( \frac{\pd b_k}{\pd x^j} \frac{\pd y^k}{\pd x^i} + b_k \frac{\pd^2 y^k}{\pd x^j \pd x^i}  \right) dx^j \wedge dx^i \\
&=  \frac{\pd b_k}{\pd x^j} \frac{\pd y^k}{\pd x^i} dx^j \wedge dx^i \\
&=  \frac{\pd b_k}{\pd x^j} \frac{\pd y^k}{\pd x^i} \left( \frac{\pd x^j}{\pd y^\ell} dy^\ell \right) \wedge \left( \frac{\pd x^i}{\pd y^m} dy^m \right) \\
&= \left( \frac{\pd b_k}{\pd x^j} \frac{\pd x^j}{\pd y^\ell} \right) \left( \frac{\pd y^k}{\pd x^i} \frac{\pd x^i}{\pd y^m} \right) dy^\ell \wedge dy^m \\
&= \frac{\pd b_k}{\pd y^\ell} \delta^k_m dy^\ell \wedge dy^m \\
&= \frac{\pd b_k}{\pd y^\ell}  dy^\ell \wedge dy^k \\
&= db_k \wedge dy^k
\end{align*}
\end{proof}
Therefore the exterior derivative of a $1$-form is a well-defined $2$-form.

\subsection{Higher-Degree Exterior Derivatives}

\begin{definition}
Let $\omega \in \Omega^k(M)$. Define $d\omega$ locally as follows: if $(U, x^i)$ is a coordinate chart, then define $d\omega$ on $U$ by $d\omega \coloneqq d\omega_I \wedge dx_I$. One shows that this definition is coordinate-independent by a calculation similar to (but more tedious than) the previous one, giving rise to a $(k+1)$-form $d\omega \in \Omega^k(M)$.
\end{definition}

\begin{proposition}
$d : \Omega^k(M) \to \Omega^{k+1}(M)$ is an $\R$-linear map satisfying
\begin{enumerate}[(i)]
\item
$d$ is an antiderivation of degree $1$ on $\Omega^*(M)$.
\item
$d^2 = 0$.
\item
As just defined, $df$ agrees with the differential of a $0$-form $f$ as defined way before this lecture. 
\end{enumerate}
\end{proposition}
\begin{proof}
Exercise!
\end{proof}

\subsection{Outlining Uniqueness}

It turns out that the properties above actually characterize $d$.

\begin{theorem}
There is a unique $\R$-linear map $d : \Omega^k(M) \to \Omega^{k+1}(M)$ satisfying properties (i)-(iii) of the previous proposition.
\end{theorem}
\begin{proof}
We provide an outline of the proof here. A full proof is given in e.g. Tu or Lee. We have shown existence already. For uniqueness, suppose that $D : \Omega^k(M) \to \Omega^{k+1}(M)$ also satisfies those three properties and is $\R$-linear. We proceed in three main steps.
\begin{enumerate}
\item
Show that $D$ is a \emph{local operator}: for any $\omega \in \Omega^k(M)$, $(D\omega)_p$ depends only on $\omega$ in a neighbourhood of $p$.
\item
Given $\omega \in \Omega^k(M)$ and a chart $(U, x^i)$, write $\omega = a_I dx^I$ on $U$. Since $D$ is a local operator, $D|_U : \Omega^k(U) \to \Omega^{k+1}(U)$ defined by $D|_U \eta \coloneqq (D\tilde{\eta})|_U$, where $\tilde{\eta}$ is an extension of $\eta$ to $M$, is well-defined.
\item
One then shows that $D\omega = da_I \wedge dx^I = d\omega$ on $U$, proving uniqueness.
\end{enumerate}
\end{proof}

\begin{theorem}
If $F : N \to M$ is smooth, then $F^*(d\omega) = d(F^*\omega)$ for all $\omega \in \Omega^k(M)$.
\end{theorem}

\end{document}
  

  