\documentclass[11pt]{article}
\usepackage[utf8]{inputenc}
\usepackage{amsmath, amsthm, amssymb, amsfonts, mathtools, tikz-cd, float}
\usepackage[left=2.5cm,right=2.5cm]{geometry}
\usepackage[shortlabels]{enumitem}
\usepackage{cancel}

\newcommand{\Int}{\mathrm{Int}}
\newcommand{\Bd}{\mathrm{Bd}}
\newcommand{\R}{\mathbb{R}}
\renewcommand{\H}{\mathbb{H}}
\newcommand{\Z}{\mathbb{Z}}
\newcommand{\pd}{\partial}
\renewcommand{\epsilon}{\varepsilon}
\renewcommand{\hat}{\widehat}
\renewcommand{\tilde}{\widetilde}
\newcommand{\supp}{\mathrm{supp}}
\newcommand{\sgn}{\mathrm{sgn}}

\newtheorem{theorem}{Theorem}[section]
\newtheorem{corollary}{Corollary}[theorem]
\newtheorem{lemma}[theorem]{Lemma}
\newtheorem{proposition}{Proposition}[section]

\newtheorem{definition}{Definition}

\pagestyle{myheadings}

\begin{document}

\section{Orientation, Manifolds With Boundary (August 7)}

We will develop the notions of orientation for manifolds, and develop manifolds with boundary. These will allow us a more general theory of integration on manifolds. One develops orientation in order to make sense of integration on manifolds, and one develops manifolds with boundary in order to have a suitable theory of integration on manifolds. One may also develop the notion of a manifold-with-corners, but we shall not do so.

\subsection{Orientation on Vector Spaces}

Let us take care of notation, first. A basis for a vector space without any ordering is either written as $v_1, \dots, v_n$ or $\{v_1, \dots, v_n\}$. An ordered basis is written as $(v_1, \dots, v_n)$.

As we did with forms, we will develop things on a vector space first, and then generalize them pointwise to manifolds by doing things on each tangent space.

On $\R$, we have either the "left" orientation or the "right" orientation. If we take the "right" orientation, then
\[
\int_{[a, b]} f = \int_a^b f(x) \, dx,
\]
and if we take the "left" orientation, then
\[
\int_{[a, b]} f = \int_b^a f(x) \, dx.
\]
More precisely, we have positive and negative orientations on $\R$.

On $\R^2$, we have the "clockwise" and "counterclockwise" orientations. The latter is the "usual orientation" of $\R^2$. Similarly, on $\R^3$, we have orientations specified by clockwise and counterclockwise twirls around the $z$-axis. These orientations can be thought of as picking the standard basis vectors in a certain order.

Let us make these notions precise. Let $V$ be an $n$-dimensional vector space. We no longer have a standard basis to work with, so we will develop a notion of orientation by using multiple bases; the orientation on a vector space will be a set of bases, each related to another by an orientation-preserving (i.e. of positive determinant) change of basis matrix.

\begin{definition}
Let $\alpha = (v_1, \dots, v_n)$ and $\beta = ( w_1, \dots, w_n )$ be ordered bases of $V$. We say that $\alpha$ and $\beta$ specify the same orientation if the change of basis matrix from $\alpha$ to $\beta$ has positive determinant. This is obviously an equivalence relation, so it partitions the set of ordered bases of $V$ into two equivalence classes. Each class is called an orientation on $V$.
\end{definition}

Let $\alpha, \beta$ be as in the previous definition, and let $Q$ denote the change of basis matrix from $\alpha$ to $\beta$. Then we have $v_i = Q^j_i w_j$ (Einstein notation), so if $\gamma \in \bigwedge^n(V^*)$,
\[
\gamma(v_1, \dots, v_n) = \det(Q)\gamma(w_1, \dots, w_n).
\]
Thus $\gamma(v_1,\dots,v_n)$ and $\gamma(w_1,\dots,w_n)$ have the same sign if and only if $(v_1,\dots,v_n)$ and $(w_1,\dots,w_n)$ specify the same orientation. We say that the $n$-covector $\gamma$ \emph{specifies the orientation $[(v_1,\dots,v_n)]$} if $\gamma(v_1,\dots,v_n)$. As we just saw, this is a well-defined notion. Thus an $n$-covector specifies an orientation on $V$. Since $\bigwedge^n(V^*)$ is one-dimensional, two $n$-covectors $\gamma$ and $\gamma'$ specify the same orientation if and only if there is a positive $a \in \R$ with $\gamma = a\gamma'$. Taking this to be an equivalence relation on the set of non-zero $n$-covectors on $V$, we see that an orientation of $V$ is also given by an equivalence class of $n$-covectors.

Note that we can also think of an orientation as a choice of component of $\bigwedge^n(V^*)$.

\subsection{Orientation on Manifolds}

We would like to make a "smooth choice" of orientation on each tangent space to $M$. We will call a choice of orientation at each tangent space of $M$ a \emph{pointwise orientation}. We do not want to haphazardly choose orientations of each tangent space, since then a manifold would have too many ($2^{|M|}$, to be precise) orientations.

The most straightforward way to develop the notion of a smooth choice of orientations is to consider a simple case, and then generalize. Consider an embedded $1$-dimensional submanifold $S$ of $\R^n$ (a curve). If we have a non-zero vector field $X$ on $S$, then each $X_p$, for $p \in S$, is a(n ordered) basis of $T_pS$, so it determines an orientation of $T_pS$. Suppose that we have chosen a pointwise orientation on $S$. If, for each $p \in S$, the non-zero tangent vector $X_p$ determines the orientation we chose on $T_pM$, then it would be reasonable to call our pointwise orientation smooth if the vector field $X$ is smooth. This turns out to give us the first notion of orientation of a manifold (note that this did not depend on the fact that $S$ was in $\R^n$: that simply makes it easier to visualize this scenario).

\begin{definition}
An orientation on a manifold $M$ is a pointwise orientation on $M$ such that for all $p \in M$, there is a neighbourhood $U$ of $p$ and a smooth local frame $X_1, \dots, X_n \in \mathfrak{X}(U)$ such that for each $q \in U$, the orientation specified by $(X_{1q}, \dots, X_{nq})$ on $T_qM$ is consistent with the choice of orientation on $T_qM$. 
\end{definition}

Equivalently, an orientation on $M$ is a pointwise orientation on $M$ such that for all $p \in M$, there is a chart $(U, x^1, \dots, x^n)$ near $p$ such that for each $q \in U$, the orientation specified by $\left( \left. \frac{\pd}{\pd x^1} \right|_q, \dots, \left. \frac{\pd}{\pd x^n} \right|_q \right)$ on $T_qM$ is consistent with the choice of orientation on $T_qM$. The proof of this fact will be a homework problem.

This gives rise to the notion of an oriented atlas. An \emph{oriented atlas} on $M$ is an atlas with the property that $\det(D(\psi \circ \phi^{-1})) > 0$ for all transition maps $\psi \circ \phi^{-1}$. So equivalently, an orientation on $M$ is a pointwise orientation admitting an oriented atlas.

Equivalently, an orientation on $M$ is a pointwise orientation on $M$ such that for all $p \in M$, there is a chart $(U, x^1, \dots, x^n)$ near $p$ such that for each $q \in U$, the orientation specified by $(dx^1 \wedge \cdots \wedge dx^n)_q$ on $T_qM$ is consistent with the choice of orientation on $T_qM$.

\begin{definition}
A manifold $M$ is said to be orientable if it admits an orientation. An oriented manifold is an orientable manifold together with a choice of orientation.
\end{definition}
\begin{proposition}
An orientable manifold $M$ admits precisely $2^c$ orientations, where $c$ is the number of components of $M$.
\end{proposition}
\begin{proof}
It suffices to show that a connected manifold $M$ admits precisely two orientations. This is a standard topological argument: construct a locally constant function and argue by connectedness that the function must be constant on the entire topological space. A full proof is given in the book.
\end{proof}

In particular, a connected orientable manifold admits precisely two orientations.

Let $\omega$ be an $n$-form on $M$. If $\omega_p \neq 0$, then $\omega_p$, being a non-zero $n$-covector on $T_pM$, determines an orientation on $T_pM$. Thus a non-vanishing $n$-form $\omega$ on $M$ determines (uniquely) a pointwise orientation on $M$. One would expect that a smooth non-vanishing $n$-form on a manifold determines an orientation of that manifold. This is, in fact, true, and we provide a proof.

\begin{theorem}
A manifold $M$ is orientable if and only if it admits a non-vanishing smooth top-degree form.
\end{theorem}
\begin{proof}
Suppose that $\omega$ is a non-vanishing smooth top degree form on $M$. For each $p \in M$, give $T_pM$ the orientation specified by $\omega_p \in \bigwedge^n(T_p^*M)$. This gives a pointwise orientation on $M$. Given $p \in M$, let $(U, \phi) = (U, x^1, \dots, x^n)$ be a connected coordinate chart at $p$. By connectedness, we have that $\omega \left( \frac{\pd}{\pd x^1}, \dots, \frac{\pd}{\pd x^n} \right)$ is either strictly positive or strictly negative on $U$. Assume without loss of generality that it is a strictly positive function on $U$. Then, for each $q \in U$, $\omega \left( \left. \frac{\pd}{\pd x^1} \right|_q, \dots, \left. \frac{\pd}{\pd x^n} \right|_q \right) > 0$, meaning that the ordered basis $\left( \left. \frac{\pd}{\pd x^1} \right|_q, \dots, \left. \frac{\pd}{\pd x^n} \right|_q \right)$ of $T_qM$ specifies the same orientation as $\omega_q$ did. Therefore $M$ is orientable.

Conversely, suppose that $M$ is orientable. Given $p \in M$, we may find a coordinate chart $(U, x^1, \dots, x^n)$ at $p$ such that for each $q \in U$, the orientation on $T_qM$ coincides with the orientation specified by the $n$-covector $(dx^1 \wedge \cdots \wedge dx^n)_q$. We therefore have a smooth non-vanishing top-degree form defined on an open set of $M$ which gives the orientation there. This is vulnerable to a standard partition of unity argument. Let $\{(U_\alpha, \phi_\alpha)\}$ be an oriented atlas for the oriented manifold $M$. Let $\{\rho_\alpha\}$ be a partition of unity suborindate to this atlas. Define
\[
\omega \coloneqq \sum_\alpha \rho_\alpha dx_\alpha^1 \wedge \cdots \wedge dx_\alpha^n.
\]
It is easy to check that $\omega$ is the desired non-vanishing smooth top-degree form, and moreover, that the orientation $\omega$ specifies on $M$ is the same as the orientation we gave $M$.
\end{proof}

The following corollary of this characterization of orientability provides a wealth of oriented manifolds for us to work with.

\begin{corollary}
Any regular hypersurface in $\R^n$ is orientable. 
\end{corollary}
\begin{proof}
Exercise. (Author will think on it and fill it in, maybe. See exercise 19.11 in Tu.)
\end{proof}

As a corollary, $S^{n-1}$ is orientable (as one would hope), and the Mobius band, being non-orientable, is not a regular hypersurface in $\R^3$.

Given an $n$-manifold $M$, define an equivalence relation on the set of smooth non-vanishing $k$-forms as follows: $\omega \sim \omega'$ if and only if $\omega = f \omega'$ for some strictly positive continuous function $f : M \to \R$. This partitions the non-vanishing smooth top degree forms on $M$ into two equivalence classes, each specifying an orientation on $M$. We have the following correspondences:
\begin{align*}
\text{orientations} &\iff \text{equivalence classes of non-vanishing smooth $k$-forms on $M$} \\
&\iff \text{equivalence classes of oriented atlases on $M$}
\end{align*}

\subsection{Manifolds With Boundary}

As a manifold is locally modelled by $\R^n$, a manifold with boundary is locally modelled by the \emph{half-space} $\H^n = \{ (x^1, \dots, x^n) \in \R^n : x^n \geq 0 \}$ with the subspace topology. Many notions from our original definition of a manifold carry over almost word-for-word, but there is a particularly important notion of interior and boundary for a manifold with boundary.

\begin{definition}
A point $x \in \H^n$ is said to be an interior point if $x^n > 0$, and is said to be a boundary point if $x^n = 0$. The set of interior points is denoted by $(\H^n)^o$, and the set of boundary points is denoted by $\pd \H^n$.
\end{definition}

The set of interior points and boundary points, as just defined, coincide with the topological interior and boundary of $\H^n$, so there is no harm in simply referring to "interior points" and "boundary points" when referring to the "prototype manifold with boundary" $\H^n$. Once we develop the more general manifold with boundary, this will not necessarily be the case. It is important to note that any open neighbourhood of a boundary point of $\H^n$ will not be an open set in $\R^n$.

To establish clear notation, we will denote by $\Int(S)$ the topological interior of a set and $\Bd(S)$ the topological boundary of a set.

We say that a topological space $M$ is \emph{locally-$\H^n$} if every point has an open neighbourhood homeomorphic to an open subset of $\H^n$. With this, we define manifolds with boundary in the continuous category. We will then generalize to smooth manifolds with boundary.

\begin{definition}
A topological $n$-manifold with boundary is a second-countable Hausdorff locally-$\H^n$ space. These homeomorphisms are called (coordinate) charts, as one would expect.
\end{definition}

The standard terminology which applies to coordinates on a smooth manifold, as defined before, also applies the the coordinates on a manifold with boundary, smooth or not.

\begin{definition}
A collection $\{(U_\alpha, \phi_\alpha)\}$ of charts on the topological manifold with boundary $M$ is said to be a \emph{smooth atlas on $M$} if it covers $M$ and if the transition maps (same notion as before) are $C^\infty$ functions on open subsets of $\H^n$. (Here we mean $C^\infty$ in the extended sense.) Maximal atlases are defined as before.

A smooth manifold with boundary is a topological manifold with boundary equipped with a maximal smooth atlas.
\end{definition}

Now we must define the notion of the interior point and the boundary point for a smooth manifold with boundary. We will define them in a seemingly coordinate-dependent way, but it will turn out that our definition is actually coordinate-independent.

\begin{definition}
A point $p$ in the manifold with boundary $M$ is said to be an interior point of $M$ if there is a chart $(U, \phi)$ at $p$ such that $\phi(p) \in (\H^n)^o$, and is said to be a boundary point if $\phi(p) \in \pd \H^n$. These notions are well-defined (coordinate-independent) by the following theorem and its corollary:
\end{definition}

\begin{theorem}
(Smooth invariance of domain) 
Let $U \subseteq \R^n$ be open and $S \subseteq \R^n$ be arbitrary.	Then $S$ is open if there is a diffeomorphism $U \to S$.
\end{theorem}
\begin{corollary}
Let $U, V \subseteq \H^n$ be open and $f : U \to V$ a diffeomorphism. Then $f$ maps interior points to interior points and boundary points to boundary points.
\end{corollary}

Thus we denote by $\pd M$ the boundary points of a manifold with boundary $M$. (We will not really use the notion of an interior point.)

\begin{proposition}
If $M$ is a smooth $n$-manifold with boundary, then $\pd M$ is an embedded codimension $1$ submanifold of $M$ with empty boundary.
\end{proposition}
\begin{proof}
If $(U, x^1, \dots, x^n)$ is a coordinate chart on $M$ with $U \cap \pd M \neq \emptyset$, then $U \cap \pd M = \{ x^n = 0 \}$.
\end{proof}
\begin{corollary}
$\pd^2 = 0$. (Compare with $d^2 = 0$ for forms.)
\end{corollary}

We will eventually see that if $M$ is an oriented manifold with boundary, then $\pd M$ has a natural orientation induced by that of $M$. This will be crucial in developing Stokes' theorem $\int_M d\omega = \int_{\pd M} \omega$.

\end{document}
  

  