\documentclass[11pt]{article}
\usepackage[utf8]{inputenc}
\usepackage{amsmath, amsthm, amssymb, amsfonts, mathtools, tikz-cd, float}
\usepackage[left=2.5cm,right=2.5cm]{geometry}
\usepackage[shortlabels]{enumitem}

\newcommand{\Int}{\mathrm{Int}}
\newcommand{\R}{\mathbb{R}}
\newcommand{\Z}{\mathbb{Z}}
\newcommand{\pd}{\partial}
\renewcommand{\epsilon}{\varepsilon}
\renewcommand{\hat}{\widehat}
\renewcommand{\tilde}{\widetilde}

\newtheorem{theorem}{Theorem}[section]
\newtheorem{corollary}{Corollary}[theorem]
\newtheorem{lemma}[theorem]{Lemma}
\newtheorem{proposition}{Proposition}[section]

\newtheorem{definition}{Definition}

\pagestyle{myheadings}

\begin{document}

\section{Curves, Submanifolds (May 28)}

\subsection{A Local Expression for the Differential}

Let $N, M$ be smooth manifolds and $F : N \to M$ a $C^\infty$ map. For $p \in N$, the differential $F_{*,p} : T_p N \to T_{F(p)}M$ is linear, so if we fix coordinate charts $(U, x^1, \dots, x^n)$ near $p$ and $(V, y^1, \dots, y^m)$ near $F(p)$, then we can speak about the matrix of $F_{*,p}$ relative to the bases $\{ \left. \frac{\pd}{\pd x^i} \right|_p \}$ and  $\{ \left. \frac{\pd}{\pd y^i} \right|_{F(p)} \}$. It turns out that this matrix is precisely the Jacobian of $F$ relative to these two coordinate systems.

Let $A = [a^i_j]$ be the matrix of $F_{*,p}$ relative to the above bases. That is, for $j = 1, \dots, n$,
\[
F_{*,p} \left( \left. \frac{\pd}{\pd x^j} \right|_p \right) = \sum_{i=1}^m a^i_j \left. \frac{\pd}{\pd y^i} \right|_{f(p)}.
\]
Applying $y^i$ to both sides of the above equation gives
\[
a^i_j = \sum_{k=1}^m a^k_j \left. \frac{\pd y^i}{\pd y^k} \right|_{F(p)} = F_{*,p} \left( \left. \frac{\pd}{\pd x^j} \right|_p \right)(y^i) = \left. \frac{\pd F^i}{\pd x^j} \right|_p.
\]
Therefore $A = [\left. \frac{\pd F^i}{\pd x^j} \right|_p]$. We state this fact formally as a proposition.
\begin{proposition}
Let $F : N \to M$ be a $C^\infty$ map and let $p \in N$. Choose coordinate charts $(U, x^1, \dots, x^n)$ near $p$ and $(V, y^1, \dots, y^m)$ near $F(p)$. Then the matrix representation of the linear transformation $F_{*,p} : T_pN \to T_{F(p)}M$ in the bases given by these coordinate charts is the Jacobian $[\left. \frac{\pd F^i}{\pd x^j} \right|_p]$ relative to these coordinate systems.
\end{proposition}
Recall that the inverse function theorem stated that, assuming the hypotheses of the above proposition, $F$ is a local diffeomorphism at $p$ if and only if its Jacobian was nonsingular. The above proposition therefore gives us a "coordinate-free" inverse function theorem.
\begin{theorem}
(Inverse function theorem) Let $F : N \to M$ be a $C^\infty$ map of manifolds of the same dimension and suppose $p \in N$. Then $F$ is a local diffeomorphism at $p$ if and only if its differential $F_{*,p}$ is an isomorphism.
\end{theorem}

\subsection{Curves on Manifolds}

We'd like to be able to relate the abstract tangent space $T_p M$, a set of derivations, to "tangent vectors" of curves in $M$ at $p$.
\begin{definition}
A $C^\infty$ curve in a manifold $M$ is a smooth map $\gamma : (a, b) \to M$. We will usually assume that $0 \in (a, b)$ and that $\gamma(0) = p$.
\end{definition}
How can we discuss that tangent vector? First, consider the case $M = \R^n$. Let $\beta : (a, b) \to \R^n$ be a $C^\infty$ curve with $\beta(0) = p$. Then
\[
\beta'(0) = \left. \frac{d}{dt} \right|_0 \beta,
\]
and so we can think of $\beta'$ as a map $c \mapsto \beta' \cdot c$. 
\begin{definition}
The velocity vector of $\gamma$ at $t_0$ is the differential 
\[
\gamma'(t_0) := \gamma_*\left( \left. \frac{d}{dt} \right|_{t_0} \right).
\]
\end{definition}
Suppose $X_p = \gamma'(0)$ and that $f \in C_p^\infty(M)$. Then
\[
X_p(f) = \gamma'(0)(f) = \gamma_*\left( \left. \frac{d}{dt} \right|_{t_0} \right)(f) = \left. \frac{d}{dt} \right|_0 (f \circ \gamma).
\]
If $M = \R^n$, this is the directional derivative of $f$ at $p$ in the direction $\gamma'(0)$ (this means the standard derivative). Note that the right side of the above equation is independent of the curve $\gamma$.

The following proposition says that every tangent vector is the velocity vector of some curve. Morally, manifolds are locally like $\R^n$, and since velocity vectors of curves are "local things", we can transfer them over to manifolds easily.
\begin{proposition}
For any $X_p \in T_p M$, there is a smooth curve $\gamma : (a,b) \to M$ with $\gamma(0) = p$ and $\gamma'(0) = X_p$.
\end{proposition}
\begin{proof}
Choose a coordinate chart $(U, \phi) = (U, x^1, \dots, x^n)$ near $p$. There are scalars $a^1, \dots, a^n$ such that $X_p = \sum a^i \left. \frac{\pd}{\pd x^i} \right|_p$. Then $\phi_*(X_p) = \sum a^i \left. \frac{\pd}{\pd r^i} \right|_{\phi(p)}$. Define $\beta : (-\epsilon, \epsilon) \to \R^n$ by
\[
\beta(t) = \phi(p) + t(a^1, \dots, a^n),
\]
where $\epsilon > 0$ is small enough so that the entire curve lies in $\phi(U)$. Then $\beta$ is a smooth curve satisfying $\beta(0) = \phi(p)$. There are scalars $b^1, \dots, b^n$ such that $\beta'(0) = \sum b^k \left. \frac{\pd}{\pd r^k} \right|_{\phi(p)}$. Applying $r^i$ to both sides gives
\[
b^i = \sum b^k \left. \frac{\pd r^i}{\pd r^k} \right|_{\phi(p)} = \beta_* \left( \left. \frac{d}{dt} \right|_0 \right) = \left. \frac{d}{dt} \right|_0 (r^i \circ \beta) = a^i,
\]
which implies that $\beta'(0) = \phi_*(X_p) = \sum a^i \left. \frac{\pd}{\pd r^i} \right|_{\phi(p)}$. If $\gamma = \phi^{-1} \circ \beta$, then $\gamma$ is a $C^\infty$ curve in $M$ with $\gamma(0) = p$ and
\[
\gamma'(0) = (\phi^{-1} \circ \beta_*) \left( \left. \frac{d}{dt} \right|_0 \right) = \phi_*^{-1} \left( \beta_* \left( \left. \frac{d}{dt} \right|_0 \right) \right) = X_p.
\]
\end{proof}
Of course, we could have chosen any curve in $\phi(U)$ whose tangent vector is $\phi_*(X_p)$. However, there's no loss in taking the simplest possible one: the line through $\phi(p)$ in the direction $(a^1, \dots, a^n)$ (which is, after some identifications, just the tangent vector $\phi_*(X_p)$).

Consider a smooth map $F : N \to M$. If $X_p \in T_pN$ is equal to $\gamma'(0)$ for some smooth curve $\gamma$ (the previous proposition ensures this is always the case), then
\begin{align*}
F_{*, p}(X_p) &= F_{*, p}(\gamma'(0)) = F_{*, p}\left( \gamma_{*, 0}\left( \left. \frac{d}{dt} \right|_0 \right) \right) = (F \circ \gamma)_{*, 0} \left( \left. \frac{d}{dt} \right|_0 \right) = (F \circ \gamma)'(0).
\end{align*} 
This gives us a way to compute the differential of a smooth map using curves. If we go back to $\R^n$ and $\R^m$, then this is just the directional derivative of $F$ at $\gamma(0) = p$ in the direction $\gamma'(0)$ (identifying tangent vectors with arrows).

\subsection{Immersions and Submersions}

We want a submanifold to be a subset of a smooth manifold that is also a smooth manifold which, in some sense, inherits the smooth structure from the larger manifold. We will make a few definitions.

\begin{definition}
Let $F : N \to M$ be a $C^\infty$ map. We define the rank of $F$ at $p$ to be the rank of the linear map $F_{*, p}$. Equivalently, it is the dimension of $F_{*,p}(T_pN)$, or the rank of the Jacobian of $F$ at $p$ relative to any two charts. (Recall that we showed that this quantity is independent of the charts used.)
\end{definition}

We will mostly be concerned with smooth maps of constant rank. Studying the rank of a smooth map allows us to study smooth manifolds using linear algebra, something we already know very well and that is often very easy to work with.

\begin{definition}
$F$ is an immersion at $p$ if $F_{*,p}$ is injective, and is an immersion if this is the case for all $p \in N$. This definition implies that $n \leq m$, and that if $F$ is an immersion, it has constant rank $n$.
\end{definition}
\begin{definition}
$F$ is a submersion at $p$ if $F_{*, p}$ is surjective, and is a submersion if this is the case for all $p \in N$. This definition implies that $n \geq m$, and that if $F$ is a submersion, it has constant rank $m$.
\end{definition}
The following two examples are the "canonical" immersions and submersions, in the sense that every immersion is locally the canonical immersion, and every submersion is locally the canonical submersion. 

The "canonical immersion" is the map $i : \R^n \to \R^m$, $n < m$, defined by $i(x^1, \dots, x^n) = (x^1, \dots, x^n, 0, \dots, 0)$. This map is clearly $C^\infty$, and its Jacobian relative to the standard coordinates is the matrix
\[
\begin{pmatrix}
I_{n \times n} \\ 0
\end{pmatrix},
\]
which clearly shows $i_{*, p}$ is injective for all $p \in \R^n$. Therefore $i$ is an immersion.

The "canonical submersion" is the map $\pi : \R^n \to \R^m$, $n > m$, defined by $\pi(x^1, \dots, x^n) = (x^1, \dots, x^m)$. This map is clearly $C^\infty$, and its Jacobian relative to the standard coordinates is the matrix
\[
\begin{pmatrix}
I_{m \times m} & 0
\end{pmatrix},
\]
which clearly shows $\pi_{*, p}$ is surjective for all $p \in \R^n$. Therefore $\pi$ is a submersion.

We now state the theorems which say that these are indeed the "canonical" examples of their kind.

\begin{theorem}
(Immersion theorem) Let $F : N \to M$ be an immersion and $p \in N$. There are coordinate charts $(U, \phi)$ near $p$ and $(V, \psi)$ near $F(p)$ such that
\[
\psi \circ F \circ \phi^{-1} : (x^1, \dots, x^n) \mapsto (x^1, \dots, x^n, 0, \dots, 0).
\]
\end{theorem}
\begin{theorem}
(Submersion theorem) Let $F : N \to M$ be an immersion and $p \in N$. There are coordinate charts $(U, \phi)$ near $p$ and $(V, \psi)$ near $F(p)$ such that
\[
\psi \circ F \circ \phi^{-1} : (x^1, \dots, x^n) \mapsto (x^1, \dots, x^m).
\]
\end{theorem}
We will prove both of these later as a corollary of the following theorem, whose proof will be given later.
\begin{theorem}
(Constant rank theorem) Let $F : N \to M$ have constant rank $r$ and $p \in N$. Then there are charts $(U, \phi)$ near $p$ and $(V, \psi)$ near $F(p)$ such that
\[
\psi \circ F \circ \phi^{-1} : (x^1, \dots, x^n) \mapsto (x^1, \dots, x^r, 0, \dots, 0).
\]
\end{theorem}

\begin{definition}
A smooth map $F : N \to M$ is said to be an embedding if it is an immersion and a topological embedding (i.e. a homeomorphism onto its image in the subspace topology).
\end{definition}
We have a few properties of immersions and submersions.

\begin{proposition}
\begin{enumerate}
\item $F : N \to M$ is a local diffeomorphism if and only if it is an immersion and a submersion

\item An immersion is locally injective, and $F$ is an immersion if and only if it is locally an embedding.

\item Submersions are open.
\end{enumerate}
\end{proposition}

%check these
\begin{proof}
Was left as an exercise in class, so here's a solution.
\begin{enumerate}
\item
$F$ is a local diffeomorphism if and only if $F_{*,p}$ is an isomorphism, if and only if $F_{*,p}$ is injective and surjective, if and only if $F$ is an immersion and a submersion at $p$.

\item
Suppose that $F$ is an immersion and that $p \in N$. By the immersion theorem there are charts $(U, \phi)$ near $p$ and $(V, \psi)$ near $F(p)$ such that $\psi \circ F \circ \phi^{-1} : \phi(U \cap F^{-1}(V)) \to \R^m$ takes on the form 
\[
(\psi \circ F \circ \phi^{-1})(x^1, \dots, x^n) = (x^1, \dots, x^n, 0, \dots, 0).
\]
We claim that $F|_{U \cap F^{-1}(V)}$ is injective. Suppose $F(a) = F(b)$ for some $a,b \in U$. Then $(\psi \circ F \circ \phi^{-1})(\phi(a)) = (\psi \circ F \circ \phi^{-1})(\phi(b))$, and so the above equation implies that $\phi(a) = \phi(b)$. Since $\phi$ is bijective, $a = b$. Therefore $F$ is locally injective. Moreover, the above equation implies that on $U \cap F^{-1}(V)$, $F$ is composition of embeddings. Therefore $F$ is a local embedding.

If $F$ is a local embedding, then for each $p \in N$ there is an open neighbourhood $U$ of $p$ such that $F|_U$ is an embedding. Therefore $F|_U$ is an immersion, which implies that $(F|_U)_{*,p}$, and therefore $F_{*,p}$, is injective, so $F$ is an immersion at $p$. This holds for all $p \in N$, so $F$ is an immersion. 
 
\item
Suppose $F$ is a submersion. Let $O \subseteq N$ be open and suppose $F(p) \in F(O)$. Since $p \in O \subseteq N$, there are, by the submersion theorem, coordinate charts $(U, \phi)$ at $p$ and $(V, \psi)$ at $F(p)$ such that $\psi \circ F \circ \phi^{-1} : \phi(U \cap F^{-1}(V)) \to \R^m$ takes on the form 
\[
(\psi \circ F \circ \phi^{-1})(x^1, \dots, x^n) = (x^1, \dots, x^m),
\]
where $n \geq m$. This implies that on $U \cap F^{-1}(V)$, $F$ is a composition of open mappings (the "canonical submersion" $\pi$ is certainly open). Moreover, $U \cap F^{-1}(V) \cap O$ is an open neighbourhood of $p$, and so we can find an open neighbourhood $W$ of $p$ contained in $U \cap F^{-1}(V) \cap O$. Then $F(W)$ is an open neighbourhood of $F(p)$ contained in $F(O)$, which implies that $F(O)$ is open. 
\end{enumerate}
\end{proof}

\subsection{Submanifolds}

\begin{proposition}
Let $M$ be a $k$-dimensional manifold in $\R^n$, in the MAT257 sense. Then the inclusion map $i : M \to \R^n$ is an embedding.
\end{proposition}
\begin{proof}
That the inclusion map is a topological embedding is obvious. Given a coordinate map $\phi : U \to V \cap M$, the pair $(V \cap M, \phi^{-1})$ is a coordinate chart on $M$. Then $i \circ \phi$ is simply $\phi$, which is injective since $D\phi(q)$ is injective. Therefore $i$ is an immersion.
\end{proof}
Therefore the smooth structure on $M$ is, in some sense, "inherited" from the ambient space $\R^n$. That is, if $F : \R^n \to \R$ is $C^\infty$, then $F|_M$ is a $C^\infty$ map of smooth manifolds. (The local converse is true.)

\begin{definition}
$S \subseteq M$ is a(n embedded) submanifold of the smooth manifold $M$ if it is a smooth manifold such that the inclusion map $i : S \to M$ is an embedding.
\end{definition}
The following proposition is immediate.
\begin{proposition}
If $F : N \to M$ is a $C^\infty$ map between manifolds and $S \subseteq N$ is a submanifold as we just defined it, then $F|_S : S \to M$ is $C^\infty$.
\end{proposition}
We list some examples of submanifolds.
\begin{enumerate}
\item
Every "MAT257 manifold" in $\R^n$ is a submanifold of $\R^n$.

\item
Every open subset $U \subseteq M$ of a smooth manifold is a submanifold, since the inclusion $i : U \to M$ is an embedding. 

In fact, the only submanifolds of $M$ with the same dimension as $M$ are the open sets. This was left as an exercise in class, so here's a solution. Suppose $S \subseteq M$ is a submanifold of $M$ with $\dim S = \dim M$. Then the inclusion map $i : S \to M$ is an embedding. For $p \in S$, the differential $i_{*,p} : T_pS \to T_pM$ is an injective linear map between two vector spaces of the same dimension, so it is invertible. By the inverse function theorem, $i$ is a local diffeomorphism at $p$, which implies that $p$ is contained in some open subset of $M$ contained in $S$. Therefore $S$ is open.

\item
The graph $\Gamma_f$ of $f(x) = |x|$ defined on $\R$ is a smooth manifold, but it is not a submanifold of $\R^2$. Consider the atlas $\{(\Gamma_f, \pi)\}$, where $\pi : (x,f(x)) \mapsto x$. The inclusion map $i : \Gamma_f \to \R^2$ is not $C^\infty$, since $(i \circ \pi^{-1})(x) = (x, |x|)$.

\item Embeddings give rise to submanifolds in the following sense.
\begin{proposition}
Let $F : N \to M$ be an embedding. Then there is a unique smooth structure on $F(N)$ such that $F(N)$ is a submanifold of $M$ and that $F : N \to F(N)$ a diffeomorphism.
\end{proposition}
\begin{proof}
Let $\mathcal{A}$ be an atlas on $N$. Define $\mathcal{A}' = \{ (F(U), \phi \circ F^{-1}) : (U, \phi) \in \mathcal{A} \}$. Each set $F(U)$ is open in $F(N)$, and each $\phi \circ F^{-1} : F(U) \to \phi(U)$ is a homeomorphism, because $F$ is a homeomorphism onto its image. If $(U_1, \phi_1 \circ F^{-1}), (U_2, \phi_2 \circ F^{-1}) \in \mathcal{A}'$, then
\[
(\phi_1 \circ F^{-1}) \circ (\phi_2 \circ F^{-1})^{-1} = \phi_1 \circ \phi_2^{-1}
\]
is $C^\infty$. Therefore $\mathcal{A}'$ is a $C^\infty$ atlas on $F(N)$, making it a smooth manifold of the same dimension as $N$. 

Consider a chart $(W, \sigma)$ on $N$ and a chart $(F(U), \phi \circ F^{-1})$ on $F(N)$. Then
\[
(\phi \circ F^{-1}) \circ F \circ \sigma^{-1} = \phi \circ \sigma^{-1}
\]
is $C^\infty$ because $\phi$ and $\sigma$ are both coordinate systems on $N$, and
\[
\sigma \circ F^{-1} \circ (\phi \circ F^{-1})^{-1} = \sigma \circ F^{-1} \circ F \circ \phi^{-1} = \sigma \circ \phi^{-1}
\]
is $C^\infty$ for the same reason. Therefore $F : N \to F(N)$ is a diffeomorphism. That the smooth structure corresponding to $\mathcal{A}'$ is the unique one on $F(N)$ with respect to which $F : N \to F(N)$ is a diffeomorphism is clear.

The inclusion map $i : S \hookrightarrow M$ is the composition of a diffeomorphism and an embedding: $S \xrightarrow{F^{-1}} N \xrightarrow{F} M$, and so it is an embedding itself. Therefore $F(N)$ is an embedded submanifold.
\end{proof}

\item
Let $U \subseteq N$ be an open subset of a smooth manifold and $F : U \to M$ be a $C^\infty$ map into a smooth manifold $M$. Then $\Gamma_f = \{ (x, f(x)) : x \in U \}$ is a submanifold of $N \times M$. This can be proved by defining $F : U \to N \times M$ by $F(x) = (x, f(x))$ and showing that this is an embedding.
\end{enumerate}

\subsection{Regular Submanifolds}

\begin{definition}
Suppose $M$ is an $n$-dimensional manifold. $S \subseteq M$ is a regular submanifold of dimension $k$ if for each $p \in S$, there exists a chart $(U, x^1, \dots, x^n)$ of $M$ near $p$ such that $U \cap S$ is defined by the vanishing of $n - k$ of the coordinate functions; we may as well assume it is defined by the vanishing of the last $n-k$ coordinates. That is,
\[
U \cap S = \{ q \in S : x^{k+1}(q) = \cdots = x^n(q) = 0 \}.
\]
Such a coordinate chart is called an adapted chart relative to $S$.
\end{definition}
If $(U, \phi) = (U, x^1, \dots, x^n)$ is an adapted chart relative to $S$, define $\phi_S : U \cap S \to \R^k$ by $\phi_S(q) = (x^1(q), \dots, x^k(q))$. The pair $(U \cap S, \phi_S)$ is a coordinate chart on $S$ in the subspace topology. (That is, $\phi_S = \pi \circ \phi|_S$.)

If $\{ (U \cap S, \phi_S) \}$ is a collection of adapted charts relative to $S$ covering $S$, it is not hard to see that they form a $C^\infty$ atlas on $S$, making $S$ a manifold of dimension $k$. $S$ is said to have \emph{codimension $n-k$ in $M$}.

Note that the definition of a submanifold we gave in which the inclusion was required to be a smooth embedding (an "embedded submanifold") is equivalent to the definition of a regular manifold. We state this as a theorem - without proof for now.
\begin{theorem}
$S \subseteq M$ is a regular submanifold if and only if it is an embedded submanifold.
\end{theorem}
\begin{theorem}
(Whitney embedding theorem) Any smooth manifold of dimension $n$ can be embedded in $\R^{2n}$.
\end{theorem}
The Klein bottle is an example of a manifold of dimension $2$ which cannot be embedded in $\R^3$, but can be embedded in $\R^4$.
\end{document}
  

  